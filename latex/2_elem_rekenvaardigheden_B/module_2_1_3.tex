\subsection{Verloop van functies}

\subsubsection{Absolute en relatieve extrema}


\begin{definitie}
	Een functie $f$ heeft een \textbf{absoluut maximum} $f(x_{0})$
in het punt $x_{0}\in\textrm{dom}\:f$, als voor alle $x\in\textrm{dom}\:f$
geldt dat $f(x_{0})\geqslant f(x)$

Een functie $f$ heeft een \textbf{absoluut minimum} $f(x_{0})$
in het punt $x_{0}\in\textrm{dom}\:f$, als voor alle $x\in\textrm{dom}\:f$
geldt dat $f(x_{0})\leqslant f(x)$

Een functie $f$ heeft een \textbf{relatief of lokaal maximum}
in het punt $x_{0}$, indien er een open interval $I$ bestaat rond
het punt $x_{0}$ zodat voor alle $x$ in het interval $I$ geldt
dat $f(x_{0})\geqslant f(x)$

Een functie $f$ heeft een \textbf{relatief of lokaal minimum}
in het punt $x_{0}$, indien er een open interval $I$ bestaat rond
het punt $x_{0}$ zodat voor alle $x$ in het interval $I$ geldt
dat $f(x_{0})\leqslant f(x)$

\end{definitie}


\begin{voorbeeld}
\begin{figure}[H]
	\centering          
	
\begin{tikzpicture}[scale=1,cap=round]

% Styles
\tikzstyle{axes}=[]
\tikzstyle help lines=[color=blue!50,very thin,dotted]


%%%%%%%%%%%%%%%%%%%%%%%%%%%%%%%%
%		GRID
%%%%%%%%%%%%%%%%%%%%%%%%%%%%%%%%

\draw[style=help lines,step=1cm] (-3.9,-5.9) grid (3.9,5.9);



%%%%%%%%%%%%%%%%%%%%%%%%%%%%%%%%
%		ASSENSTELSEL
%%%%%%%%%%%%%%%%%%%%%%%%%%%%%%%%

\draw[->] (-5,0) -- (5,0) node[right] {$x$};
\draw[->] (0,-7) -- (0,7) node[left]{$y$};

%\draw[fill,cyan](1,1)circle [radius=0.025];

%\draw[red,cap=rect, loosely dashed, ultra thick, domain=-2:2] plot (\x, {(\x*\x-1)+0.05}) node[above,yshift=-.7cm, right]{};

%%%%%%%%%%%%%%%%%%%%%%%%%%%%%%%%
%legende
%%%%%%%%%%%%%%%%%%%%%%%%%%%%%%%%
%\tkzDefPoint(0.5,3.5){A}
%\tkzDefPoint(1,3.5){B}
%\tkzLabelPoint[right,xshift=+0.1cm](B){${\color{cyan}f(x)=|x^2-1|}$}
%\tkzDrawSegment[cyan,ultra thick](A,B)

%\tkzDefPoint(0.5,3.2){C}
%\tkzDefPoint(1,3.2){D}
%\tkzLabelPoint[right,xshift=+0.1cm](D){${\color{red}e(x)=x^2-1}$}
%\tkzDrawSegment[red,cap=rect, loosely dashed, ultra thick](C,D)


%%%%%%%%%%%%%%%%%%%%%%%%%%%%%%%%
%getallen op de x-as en lijntjes
%%%%%%%%%%%%%%%%%%%%%%%%%%%%%%%%   
\foreach \x/\xtext in {-4,-3,-2,-1,1,2,3,4}
	\draw[xshift=\x cm] (0pt,1pt) -- (0pt,0pt) node[below,fill=white]
	{$\xtext$};,3
	
%getallen op de y-as en lijntjes  
%BEGIN LUS
\foreach \y/\ytext in {-6,-5,-4,-3,-2,-1,1,2,3,4,5,6}
	\draw[yshift=\y cm] (1pt,0pt) -- (0pt,0pt) node[left,fill=white]
	{$\ytext$}; %EINDE LUS



%%%%%%%%%%%%%%%%%%%%%%%%%%%%%%%%
%		GRAFIEKEN
%%%%%%%%%%%%%%%%%%%%%%%%%%%%%%%%
%\draw[cyan,cap=rect,thick, domain=-6:6] plot (\x, \x) node[above, right]{${\color{cyan}y=x}$};

\draw[cyan,cap=rect,ultra thick, domain=-1.2:3.5] plot (\x, {
	pow(\x,3)-3*pow(\x,2)		% <- plaats het functievoorschrift hier
}) node[above]{$f(x)=x^3-3x^2$};

\draw[opacity=0] (-1.2,-5) --(-1,-3) node[opacity=1,blue, midway,sloped]{stijgen};  
\draw[opacity=0] (0.5,-1) --(1.3,-3) node[opacity=1,blue, midway,sloped,above]{dalen};  
\draw[opacity=0] (3,1) --(3.5,5) node[opacity=1,blue, midway,sloped,above]{stijgen};  


%node[blue]{stijgen} 
%\draw[cyan,cap=rect,ultra thick, domain=2.25:6] plot (\x, {(\x-2)^(-1)}) node[above,yshift=+0.5cm,left]{$\color{cyan} y=\frac{1}{x-2}$};


%\draw[cyan,cap=rect,ultra thick, domain=-7:1.9] plot (\x, {exp{\x}}) node[above, right]{${\color{cyan}y=\exp{x}}$};

%%%%%%%%%%%%%%%%%%%%%%%%%%%%%%%%
%		MARKERINGEN
%%%%%%%%%%%%%%%%%%%%%%%%%%%%%%%%
%verticale lijn
%\draw[-o,line width=4,teal, cap=rect,opacity=0.3] (0,-4) -- (0,0.25) node[right] {};
%\draw[line width=4,teal, cap=rect,opacity=0.3] (0,0) -- (0,4.2) node[right] {bld $f$ = $\mathbb{R}_0$};
%horizontale lijn
\draw[line width=4,blue, cap=rect,opacity=0.3] (-1,0) -- (1,0) node[near start,above,xshift=-1.1cm,opacity=1] {maximum in (0,0)};

 
\draw[line width=4,blue, cap=rect,opacity=0.3] (1,-4) -- (3,-4) node[below,yshift=-.3cm,opacity=1] {minium in (2,-4)};
\end{tikzpicture}

	\caption{voorbeeld 1}
	\label{fig:verloop_vb1}	
\end{figure}


Voor de functie in gegeven in bovenstaande figuur zijn $x_{1},\:x_{2},\:x_{3},\:x_{4}$ relatieve of lokale extrema. Hiervan zijn $x_{2}$ en $x_{3}$ absolute extrema.

$x_{1},\:x_{3}$ zijn relatieve of lokale maxima.
Hiervan is $x_{3}$ een absoluut maximum.

$x_{2},\:x_{4}$ zijn relatieve of lokale minima. Hiervan
is $x_{2}$ een absoluut minimum.

\end{voorbeeld}

\subsubsection{Stijgen, dalen en extrema}

De eerste afgeleide speelt een belangrijke rol bij het onderzoek
naar het verloop van een functie. Men zal dus het tekenverloop van
deze afgeleide onderzoeken.




\begin{definitie}
	Als de eerste afgeleide $f^{'}(x)>0$ is in een interval,
dan zal deze functie in dat interval \textbf{stijgen}.

Als de eerste afgeleide $f^{'}(x)<0$ is in een interval,
dan zal deze functie in dat interval \textbf{dalen}.

De functie zal een relatief of lokaal extremum (maximum
of minimum) bereiken waar de kromme overgaat van een stijgende functie
naar een dalende functie of omgekeerd.

Een functie $f(x)$ heeft in het punt $(x_{0},y_{0})$ een
extremum als in dat punt aan de volgende voorwaarden voldaan is:

\begin{itemize}
\item als $f^{'}(x_{0})=0$ en $f^{''}(x_{0})>0$ zal het
extremum een minimum zijn, en
\item als $f^{'}(x_{0})=0$ en $f^{''}(x_{0})<0$ zal het
extremum een maximum zijn.
\end{itemize}

\end{definitie}

\begin{opmerking}
	\ \\
	\begin{itemize}
\item Opmerking 1: in een extremum bezit de functie een horizontale
raaklijn (aangezien $f^{'}(x_{0})=0$ is).
\item Opmerking 2: indien $f^{'}(x_{0})=0$ en ook $f^{''}(x_{0})=0$
werkt deze methode niet om na te gaan of er in $x_{0}$ een extremum
is. In dat geval gaan we naar de derde afgeleide kijken. Als $f^{'''}(x_{0})\neq0$
dan is het punt $x_{0}$ een buigpunt. Indien ook de derde afgeleide
nul is moeten we naar de eerstvolgende afgeleide gaan kijken die niet
nul is. Stel dat deze van de n$^{de}$ orde is, dus $f^{(n)}(x_{0})\neq0$
. Als dan $n$ even is, is er in $x_{0}$ een lokaal minimum als $f^{(n)}(x_{0})>0$
, en een lokaal maximum als $f^{(n)}(x_{0})<0$ . Is de n$^{de}$
oneven dan is $x_{0}$ terug een buigpunt. 

Tip: als je dit moeilijk
kan onthouden, denk dan aan de eenvoudige functies $f(x)=x^{2}$ en
$f(x)=x^{3}$ . De functie $x^{2}$ heeft een minimum in nul, terwijl
$x^{3}$ in de oorsprong een buigpunt heeft.
\end{itemize}

\end{opmerking}



\begin{voorbeeld}
	Onderzoek het stijgen en dalen van volgende functie $f$ met voorschrift: $f(x)=x^{3}-3x^{2}$ in onderstaande figuur.


\begin{figure}[H]
	\centering          
	
\begin{tikzpicture}[scale=1,cap=round]

% Styles
\tikzstyle{axes}=[]
\tikzstyle help lines=[color=blue!50,very thin,dotted]


%%%%%%%%%%%%%%%%%%%%%%%%%%%%%%%%
%		GRID
%%%%%%%%%%%%%%%%%%%%%%%%%%%%%%%%

\draw[style=help lines,step=1cm] (-3.9,-5.9) grid (3.9,5.9);



%%%%%%%%%%%%%%%%%%%%%%%%%%%%%%%%
%		ASSENSTELSEL
%%%%%%%%%%%%%%%%%%%%%%%%%%%%%%%%

\draw[->] (-5,0) -- (5,0) node[right] {$x$};
\draw[->] (0,-7) -- (0,7) node[left]{$y$};

%\draw[fill,cyan](1,1)circle [radius=0.025];

%\draw[red,cap=rect, loosely dashed, ultra thick, domain=-2:2] plot (\x, {(\x*\x-1)+0.05}) node[above,yshift=-.7cm, right]{};

%%%%%%%%%%%%%%%%%%%%%%%%%%%%%%%%
%legende
%%%%%%%%%%%%%%%%%%%%%%%%%%%%%%%%
%\tkzDefPoint(0.5,3.5){A}
%\tkzDefPoint(1,3.5){B}
%\tkzLabelPoint[right,xshift=+0.1cm](B){${\color{cyan}f(x)=|x^2-1|}$}
%\tkzDrawSegment[cyan,ultra thick](A,B)

%\tkzDefPoint(0.5,3.2){C}
%\tkzDefPoint(1,3.2){D}
%\tkzLabelPoint[right,xshift=+0.1cm](D){${\color{red}e(x)=x^2-1}$}
%\tkzDrawSegment[red,cap=rect, loosely dashed, ultra thick](C,D)


%%%%%%%%%%%%%%%%%%%%%%%%%%%%%%%%
%getallen op de x-as en lijntjes
%%%%%%%%%%%%%%%%%%%%%%%%%%%%%%%%   
\foreach \x/\xtext in {-4,-3,-2,-1,1,2,3,4}
	\draw[xshift=\x cm] (0pt,1pt) -- (0pt,0pt) node[below,fill=white]
	{$\xtext$};,3
	
%getallen op de y-as en lijntjes  
%BEGIN LUS
\foreach \y/\ytext in {-6,-5,-4,-3,-2,-1,1,2,3,4,5,6}
	\draw[yshift=\y cm] (1pt,0pt) -- (0pt,0pt) node[left,fill=white]
	{$\ytext$}; %EINDE LUS



%%%%%%%%%%%%%%%%%%%%%%%%%%%%%%%%
%		GRAFIEKEN
%%%%%%%%%%%%%%%%%%%%%%%%%%%%%%%%
%\draw[cyan,cap=rect,thick, domain=-6:6] plot (\x, \x) node[above, right]{${\color{cyan}y=x}$};

\draw[cyan,cap=rect,ultra thick, domain=-1.2:3.5] plot (\x, {
	pow(\x,3)-3*pow(\x,2)		% <- plaats het functievoorschrift hier
}) node[above]{$f(x)=x^3-3x^2$};

\draw[opacity=0] (-1.2,-5) --(-1,-3) node[opacity=1,blue, midway,sloped]{stijgen};  
\draw[opacity=0] (0.5,-1) --(1.3,-3) node[opacity=1,blue, midway,sloped,above]{dalen};  
\draw[opacity=0] (3,1) --(3.5,5) node[opacity=1,blue, midway,sloped,above]{stijgen};  


%node[blue]{stijgen} 
%\draw[cyan,cap=rect,ultra thick, domain=2.25:6] plot (\x, {(\x-2)^(-1)}) node[above,yshift=+0.5cm,left]{$\color{cyan} y=\frac{1}{x-2}$};


%\draw[cyan,cap=rect,ultra thick, domain=-7:1.9] plot (\x, {exp{\x}}) node[above, right]{${\color{cyan}y=\exp{x}}$};

%%%%%%%%%%%%%%%%%%%%%%%%%%%%%%%%
%		MARKERINGEN
%%%%%%%%%%%%%%%%%%%%%%%%%%%%%%%%
%verticale lijn
%\draw[-o,line width=4,teal, cap=rect,opacity=0.3] (0,-4) -- (0,0.25) node[right] {};
%\draw[line width=4,teal, cap=rect,opacity=0.3] (0,0) -- (0,4.2) node[right] {bld $f$ = $\mathbb{R}_0$};
%horizontale lijn
\draw[line width=4,blue, cap=rect,opacity=0.3] (-1,0) -- (1,0) node[near start,above,xshift=-1.1cm,opacity=1] {maximum in (0,0)};

 
\draw[line width=4,blue, cap=rect,opacity=0.3] (1,-4) -- (3,-4) node[below,yshift=-.3cm,opacity=1] {minium in (2,-4)};
\end{tikzpicture}

	\caption{Voorbeeld stijgen en dalen}
	\label{fig:verloop_stijgen_en_dalen}	
\end{figure}
De eerste afgeleide $f^{'}(x)=3x^{2}-6x$ is nul als $3x^{2}-6x=3x(x-2)=0$, dus als $x=0$ of als $x=2$.

Om de $y$-co\"ordinaat van deze extrema punten te vinden
vullen we $x=0$ en $x=2$ in in de functie $f(x)$. Dit geeft: $f(0)=0$
en $f(2)=-4$.

De functie $f$ zal in de punten $(0,0)$ en $(2,-4)$ een
extremum bezitten. In het punt $(0,0)$ is dit een maximum en in het
punt $(2,-4)$ een minimum. De raaklijn is in die punten ook telkens
horizontaal.

Aangezien de eerste afgeleide een tweedegraadsfunctie is,
zullen we het tekenverloop van de tweedegraadsfunctie toepassen, zie onderstaande tabel.

%\begin{table}
%	\centering
\begin{center}
	\begin{tabular}{c||c|c|c|c|c}
	$x$ &  & 0 &  & 2 & \tabularnewline
	\hline 
	$f^{'}(x)=3x(x-2)$ & + & 0 & - & 0 & + \\
	\hline 
	$f(x)$ &  & 0 &  & -4 & \\
	& $\nearrow$ & max & $\searrow$ & min & $\nearrow$ \\
\end{tabular}
\end{center}


\end{voorbeeld}

\subsubsection{Convex, concaaf en buigpunten}

Ook de tweede afgeleide speelt een belangrijke rol bij het
onderzoek naar het verloop van een functie. Men zal dus ook het tekenverloop
van de tweede afgeleide onderzoeken.


\begin{definitie}
	Als de tweede afgeleide $f^{''}(x)>0$ is in een interval,
dan zal de functie in dat interval \textbf{hol of concaaf} zijn (symbool:
$\bigcup$).

Als de tweede afgeleide $f^{''}(x)<0$ is in een interval,
dan zal de functie in dat interval \textbf{bol of convex} zijn (symbool:
$\bigcap$).

Punten waar de kromme overgaat van convex naar concaaf of
omgekeerd, noemt men \textbf{buigpunten}. In een buigpunt verandert
dus de tweede afgeleide van teken. Een functie $f(x)$ heeft in het
punt $(x_{0},y_{0})$ een buigpunt als in dat punt aan de volgende
voorwaarden voldaan zijn: $f^{''}(x_{0})=0$ en $f^{'''}(x_{0})\neq0$

\end{definitie}


\begin{opmerking}
	\ \\
	\begin{itemize}
	\item Opmerking 1: in een buigpunt snijdt de raaklijn de kromme,
	deze raaklijn noemt men de buigraaklijn.
	
	\item Opmerking 2: aangezien de kromming in een buigpunt verandert,
	is een buigpunt nooit een extremum.
\end{itemize}

\end{opmerking}


\begin{voorbeeld}
	Onderzoek het hol en bol zijn van
de functie $f$ met voorschrift $f(x)=x^{3}-3x^{2}$, zie onderstaande figuur.


\begin{figure}[H]
	\centering          
	\tikzsetfigurename{Fig_module_2_1_3_verloop_stijgen_dalen}

\begin{tikzpicture}[scale=1,cap=round]

% Styles
\tikzstyle{axes}=[]
\tikzstyle help lines=[color=blue!50,very thin,dotted]


%%%%%%%%%%%%%%%%%%%%%%%%%%%%%%%%
%		GRID
%%%%%%%%%%%%%%%%%%%%%%%%%%%%%%%%

\draw[style=help lines,step=1cm] (-3.9,-5.9) grid (3.9,5.9);



%%%%%%%%%%%%%%%%%%%%%%%%%%%%%%%%
%		ASSENSTELSEL
%%%%%%%%%%%%%%%%%%%%%%%%%%%%%%%%

\draw[->] (-5,0) -- (5,0) node[right] {$x$};
\draw[->] (0,-7) -- (0,7) node[left]{$y$};

%\draw[fill,cyan](1,1)circle [radius=0.025];

%\draw[red,cap=rect, loosely dashed, ultra thick, domain=-2:2] plot (\x, {(\x*\x-1)+0.05}) node[above,yshift=-.7cm, right]{};

%%%%%%%%%%%%%%%%%%%%%%%%%%%%%%%%
%legende
%%%%%%%%%%%%%%%%%%%%%%%%%%%%%%%%
%\tkzDefPoint(0.5,3.5){A}
%\tkzDefPoint(1,3.5){B}
%\tkzLabelPoint[right,xshift=+0.1cm](B){${\color{cyan}f(x)=|x^2-1|}$}
%\tkzDrawSegment[cyan,ultra thick](A,B)

%\tkzDefPoint(0.5,3.2){C}
%\tkzDefPoint(1,3.2){D}
%\tkzLabelPoint[right,xshift=+0.1cm](D){${\color{red}e(x)=x^2-1}$}
%\tkzDrawSegment[red,cap=rect, loosely dashed, ultra thick](C,D)


%%%%%%%%%%%%%%%%%%%%%%%%%%%%%%%%
%getallen op de x-as en lijntjes
%%%%%%%%%%%%%%%%%%%%%%%%%%%%%%%%   
\foreach \x/\xtext in {-4,-3,-2,-1,1,2,3,4}
	\draw[xshift=\x cm] (0pt,1pt) -- (0pt,0pt) node[below,fill=white]
	{$\xtext$};,3
	
%getallen op de y-as en lijntjes  
%BEGIN LUS
\foreach \y/\ytext in {-6,-5,-4,-3,-2,-1,1,2,3,4,5,6}
	\draw[yshift=\y cm] (1pt,0pt) -- (0pt,0pt) node[left,fill=white]
	{$\ytext$}; %EINDE LUS



%%%%%%%%%%%%%%%%%%%%%%%%%%%%%%%%
%		GRAFIEKEN
%%%%%%%%%%%%%%%%%%%%%%%%%%%%%%%%
%\draw[cyan,cap=rect,thick, domain=-6:6] plot (\x, \x) node[above, right]{${\color{cyan}y=x}$};

\draw[cyan,cap=rect,ultra thick, domain=-1.2:3.5] plot (\x, {
	pow(\x,3)-3*pow(\x,2)		% <- plaats het functievoorschrift hier
}) node[above]{$f(x)=x^3-3x^2$};

 
%node[blue]{stijgen} 
%\draw[cyan,cap=rect,ultra thick, domain=2.25:6] plot (\x, {(\x-2)^(-1)}) node[above,yshift=+0.5cm,left]{$\color{cyan} y=\frac{1}{x-2}$};


%\draw[cyan,cap=rect,ultra thick, domain=-7:1.9] plot (\x, {exp{\x}}) node[above, right]{${\color{cyan}y=\exp{x}}$};

%%%%%%%%%%%%%%%%%%%%%%%%%%%%%%%%
%		MARKERINGEN
%%%%%%%%%%%%%%%%%%%%%%%%%%%%%%%%
%verticale lijn
%\draw[-o,line width=4,teal, cap=rect,opacity=0.3] (0,-4) -- (0,0.25) node[right] {};
%\draw[line width=4,teal, cap=rect,opacity=0.3] (0,0) -- (0,4.2) node[right] {bld $f$ = $\mathbb{R}_0$};
%horizontale lijn

 \draw[white,fill=blue,opacity=.5] (1,-2) circle [radius=.1]   node[blue, above,xshift=-1.1cm,opacity=1] {buigpunt in $(1,-2)$};


 
\draw[teal,cap=rect,line width=4, opacity=.5, domain=.5:1.5] plot (\x, {
	-2-3*(\x-1)		% <- plaats het functievoorschrift hier
}) node[opacity=1,above]{};


\draw[] (1.5,1.5) node[blue] {hol of concaaf};
\draw[] (1.5,-5) node[blue] {bol of convex};

\end{tikzpicture}
 
	\caption{Voorbeeld convex, concaaf en buigpunten}
	\label{fig:verloop_convex_concaaf_buigpunten}	
\end{figure}


De eerste afgeleide is $f^{'}(x)=3x^{2}-6x$ 

De tweede afgeleide is $f^{''}(x)=6x-6$ en is nul als $6x-6=6(x-1)=0$
, en dus als $x=1$.

Om de $y$-co\"ordinaat van dit buigpunt te vinden vullen
we $x=1$ in in de functie $f(x)$. Dit geeft: $f(1)=-2$.

Aangezien de tweede afgeleide een eerstegraadsfunctie is,
zullen we het tekenverloop van de eerstegraadsfunctie toepassen, zie onderstaande tabel.

%\begin{table}
%	\centering
\begin{center}
	\begin{tabular}{c||c|c|c}
	$x$ &  & 1 & \tabularnewline
	\hline 
	$f^{''}(x)=6(x-1)$ & - & 0 & +\\
	\hline 
	$f(x)$ & $\bigcap$ & -2 & $\bigcup$\\
	&  & buigpunt & \\
\end{tabular}
\end{center}
%\caption{Voorbeeld convex, concaaf en buigpunten}
%\label{tab:convex}
%\end{table}

\end{voorbeeld}
\begin{voorbeeld}
	Wat kan je zeggen over de functie $f$ met voorschrift
$f(x)=x^{3}$ in het punt waar $x=0$ is?

De eerste afgeleide is $f^{'}(x)=3x^{2}$ en is nul als
$x=0$. Dus $x=0$ kan een extremum zijn.

De tweede afgeleide is $f^{''}(x)=6x$ en is nul als $x=0$.
We kunnen niks zeggen; we moeten naar de eerstvolgende afgeleide gaan
kijken die niet nul is.

De derde afgeleide is $f^{'''}(x)=6$. De derde orde afgeleide
(n=3 is oneven) is niet nul, dus het punt $x=0$ is een buigpunt (en
geen extremum).

\end{voorbeeld}

\begin{voorbeeld}
	Zoek de buigpunten van de functie $f$ met voorschrift
$f(x)=\cos(x)$.

De eerste afgeleide is $f^{'}(x)=-\sin(x)$. De nulpunten
van $\sin x$ zijn $\{x=n\pi\:\textrm{met}\:n\in\mathbb{Z}\}$. In
deze punten kan $f(x)=\cos(x)$ een extremum bereiken.

De tweede afgeleide is $f^{''}(x)=-\cos(x)$. De nulpunten
van $\cos(x)$ zijn $\{x=\frac{\pi}{2}+n\pi\:\textrm{met}\:n\in\mathbb{Z}\}$.
In deze punten verandert de cosinus en dus ook de tweede afgeleide
van teken, dus dit zijn buigpunten. M.a.w. de nulpunten van de functie
$f(x)=\cos(x)$ zijn tevens de buigpunten (dit geldt ook voor de sinus).

\end{voorbeeld}