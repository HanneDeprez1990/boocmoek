
%\DeclareMathOperator{\tri}{tri} \DeclareMathOperator{\rect}{rect}
%\DeclareMathOperator{\sgn}{sgn} \DeclareMathOperator{\ramp}{ramp}
%\DeclareMathOperator{\sinc}{sinc}


\subsection{Bijzondere functies}
%\begin{itemize}
%\item Wat is de grafiek van de absolute waarde functies?
%\item Wat betekent de functie sgn(x)?
%\end{itemize}

\subsubsection{De absolute waarde}

In Module 1: Elementaire vaardigheden A heb je reeds de
definitie gezien van de absolute waarde:
% TODO referentie automatisch!

\begin{equation*}
\left|x\right|=\begin{cases}
-x & \textrm{als}\:x<0\\
x & \textrm{als}\:x\geq0
\end{cases}
\end{equation*}

en de bijhorende grafiek:

\begin{figure}[H]
	\centering
		\tikzsetfigurename{Fig_module_2_1_14_Absolute_value}

\begin{center}
	\begin{tikzpicture}[scale=1,cap=round]
	
	% Styles
	\tikzstyle{axes}=[]
	\tikzstyle help lines=[color=blue!50,very thin,dotted]
	
	% grid
	\draw[style=help lines,step=1cm] (-4.9,-1.9) grid (4.9,4.9);
	
	\draw[->] (-5,0) -- (5,0) node[right] {$x$};
	\draw[->] (0,-2) -- (0,5) node[above] {$y$};
	
	%\draw[fill,cyan](1,1)circle [radius=0.025];
	%FUNCTIEVOORSCHRIFTEN
	
	
	\draw[red,cap=rect,line width=1, opacity=1, domain=-4:0,samples=100] plot (\x, {
		-\x	% <- plaats het functievoorschrift hier	
	}) node[opacity=1,above,right]{};
	%-------------------------------------------
	
	
	
	\draw[red,cap=rect,line width=1, opacity=1, domain=0:4,samples=100] plot (\x, {
		\x	% <- plaats het functievoorschrift hier	
	}) node[opacity=1,above,right]{$f(x)=|x|$};
	%-------------------------------------------
	
	
	%legende
	
	
	
	%getallen op de x-as en lijntjes   
	\foreach \x/\xtext in {-4,-3,-2,-1,0,1,2,3,4}
	\draw[xshift=\x cm] (0pt,1pt) -- (0pt,0pt) node[below,fill=white]
	{$\xtext$};,3
	
	%getallen op de y-as en lijntjes  
	%BEGIN LUS
	\foreach \y/\ytext in {-1,1,2,3,4}
	\draw[yshift=\y cm] (1pt,0pt) -- (0pt,0pt) node[left,fill=white]
	{$\ytext$}; %EINDE LUS
	
	
	
	\end{tikzpicture}
\end{center}


\end{figure}



We hebben er toen ook op gewezen dat je voorzichtig moet
zijn met formules van het type $\sqrt{x^{2}}$.

Passen we dit toe op een functie die horizontaal verschoven
is: $y=\sqrt{(x-1)^{2}}=\left|x-1\right|$.

Als deze functie zou gegeven zijn als de veelterm $x^{2}-2x+1$,
dan loont het dus de moeite om dit te herschrijven als een volledig
kwadraat: ${\displaystyle x^{2}-2x+1=\left(x-1\right)^{2}}$.


\subsubsection{De signum functie}

De sign of signum functie $\textrm{sgn}(x)$ is een eenvoudige wiskundige
functie, die eigenlijk het teken van het argument aangeeft:

\begin{equation*}
\textrm{sgn}(x)=\begin{cases}
-1 & \textrm{als}\:x<0\\
0 & \textrm{\textrm{als}}\:x=0\\
+1 & \textrm{\textrm{als}}\:x>0
\end{cases}
\end{equation*}

De grafiek van de signum functie:



\begin{figure}[H]
	\centering
	
		\tikzsetfigurename{Fig_module_2_1_14_Signum_function}

\begin{center}
	\begin{tikzpicture}[scale=1,cap=round]
	
	% Styles
	\tikzstyle{axes}=[]
	\tikzstyle help lines=[color=blue!50,very thin,dotted]
	
	% grid
	\draw[style=help lines,step=1cm] (-4.9,-1.9) grid (4.9,1.9);
	
	\draw[->] (-5,0) -- (5,0) node[right] {$x$};
	\draw[->] (0,-2) -- (0,2) node[above] {$y$};
	
	\draw[fill,red](0,0)circle [radius=0.05];
	\draw[red](0,1)circle [radius=0.05];
	\draw[red](0,-1)circle [radius=0.025];
	
	
		
	%getallen op de x-as en lijntjes   
	\foreach \x/\xtext in {-4,-3,-2,-1,0,1,2,3,4}
	\draw[xshift=\x cm] (0pt,1pt) -- (0pt,0pt) node[below,fill=white]
	{$\xtext$};,3
	
	%getallen op de y-as en lijntjes  
	%BEGIN LUS
	\foreach \y/\ytext in {-1,1}
	\draw[yshift=\y cm] (1pt,0pt) -- (0pt,0pt) node[left,fill=white]
	{$\ytext$}; %EINDE LUS
	
	
	
	\draw[red,cap=rect,line width=1, opacity=1, domain=-4:0,samples=100] plot (\x, {
		-1	% <- plaats het functievoorschrift hier	
	}) node[opacity=1,above,right]{};
	%-------------------------------------------
	
	
	
	\draw[red,cap=rect,line width=1, opacity=1, domain=0:4,samples=100] plot (\x, {
		1	% <- plaats het functievoorschrift hier	
	}) node[opacity=1,above,right]{$f(x)=|x|$};
	%-------------------------------------------
	
	
	%legende
	
	

	
	\end{tikzpicture}
\end{center}


\end{figure}

