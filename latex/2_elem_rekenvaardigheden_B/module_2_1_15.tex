%\DeclareMathOperator{\tri}{tri} \DeclareMathOperator{\rect}{rect}
%\DeclareMathOperator{\sgn}{sgn} \DeclareMathOperator{\ramp}{ramp}
%\DeclareMathOperator{\sinc}{sinc}


\subsection{Verschuiven en herschalen}

%\begin{itemize}
%\item Wat gebeurt er met de grafiek van een functie als we bij het argument
%en/of bij de functie zelf, een getal optellen?
%\item Wat gebeurt er met de grafiek van een functie als we het argument
%en/of de functie zelf vermenigvuldigen met een getal?
%\end{itemize}

\subsubsection{Verschuiven}

\noindent Wanneer we de grafiek van $y=f(x)$ kennen, kunnen we met
verschuivingen (of translaties) de grafiek van $y=f(x+a)$ en $y=f(x)+a$
daaruit afleiden (met $a\in\mathbb{R}$).

\medskip{}


\noindent De grafiek van $y=f(x+a)$ wordt verkregen uit de grafiek
van $y=f(x)$ door alle punten op de grafiek met een afstand van a
eenheden horizontaal te verschuiven (naar links als $a>0$, en naar
rechts als $a<0$).

\noindent Het lijkt misschien een beetje tegenstrijdig dat de grafiek
van een functie naar rechts verschuift als je bij het argument $x$
een negatief getal optelt. Maar kijk eens naar de sinus functie $f_{1}(x)=\sin(x)$
in onderstaand voorbeeld. We weten dat de sinus functie onder andere
een nulpunt heeft als het argument nul is, dus voor $x=0$. Als we
van $x$ de waarde $\frac{\pi}{2}$ aftrekken, dan bekomen we terug
datzelfde nulpunt als het argument nul is. Uit $x-\frac{\pi}{2}=0$
volgt dat dit het geval zal zijn als $x=+\frac{\pi}{2}$. Met andere
woorden, de functie $f_{2}(x)=\sin(x-\frac{\pi}{2})$ heeft dezelfde
grafiek als de functie $f_{1}(x)=\sin(x)$ maar is $\frac{\pi}{2}$
eenheden naar rechts verschoven.\medskip{}


\begin{figure}
	\centering
	\begin{subfigure}{.48\linewidth}
	\includegraphics[height=5cm]{\string"2_elem_rekenvaardigheden_B/inputs/Verschuiven - horizontaal - naar links\string".eps}
	\end{subfigure}
	\begin{subfigure}{.48\linewidth}
	 \includegraphics[height=5cm]{\string"2_elem_rekenvaardigheden_B/inputs/Verschuiven - horizontaal - naar rechts\string".eps} 
	\end{subfigure}
\end{figure}

\medskip{}


\noindent De grafiek van $y=f(x)+a$ wordt verkregen uit de grafiek
van $y=f(x)$ door alle punten op de grafiek met een afstand van a
eenheden verticaal te verschuiven (naar beneden als $a<0$, en naar
boven als $a>0$).

\noindent \medskip{}


%\begin{tabular}{ccc}
%\includegraphics[height=5cm]{\string"2_elem_rekenvaardigheden_B/inputs/Verschuiven - verticaal - naar omlaag\string".eps}  &  & \includegraphics[height=5cm]{\string"2_elem_rekenvaardigheden_B/inputs/Verschuiven - verticaal -  naar omhoog\string".eps} \tabularnewline
%\end{tabular}\medskip{}


Een voorbeeld: hoe ziet de grafiek eruit van de functie $y=x^{2}+2x+3$
?

We herschrijven de functie als: $y=\left(x^{2}+2x+1\right)+2=\left(x+1\right)^{2}+2$

We herkennen hierin de basisfunctie $y=x^{2}$. De gegeven functie
stelt dus een parabool voor die over 1 eenheid naar links en 2 eenheden
naar boven is verschoven.


\subsubsection{Herschalen}

\noindent Wanneer we de grafiek van $y=f(x)$ kennen, kunnen we met
herschalen (of vermenigvuldigen) de grafiek van $y=f(ax)$ en $y=af(x)$
daaruit afleiden (met $a\in\mathbb{R}$).

\medskip{}


\noindent De grafiek van $y=f(ax)$ wordt verkregen uit de grafiek
van $y=f(x)$ door, bij gelijkblijvende $y$-co\"ordinaten, alle $x$-co\"ordinaten
van de punten op de grafiek met een factor $\frac{1}{a}$ te vermenigvuldigen
(openrekken als $0<a<1$, en inkrimpen als $a>1$).

\noindent Voor $a=2$ bijvoorbeeld krimpt de grafiek in elkaar. Het
is alsof de functie dubbel zo snel verloopt.

\noindent \medskip{}


\begin{tabular}{ccc}
\includegraphics[height=5cm]{\string"2_elem_rekenvaardigheden_B/inputs/Herschalen - horizontaal - openrekken\string".eps}  &  & \includegraphics[height=5cm]{\string"2_elem_rekenvaardigheden_B/inputs/Herschalen - horizontaal - inkrimpen\string".eps} \tabularnewline
\end{tabular}

\noindent \medskip{}


\noindent Een bijzonder geval treedt op als $a=-1$: de grafiek van
$g(x)=f(-x)$ kan uit de grafiek van $f(x)$ worden verkregen door
van alle punten op de grafiek de $x$-co\"ordinaten met -1 te vermenigvuldigen,
dus door de grafiek van $f(x)$ te spiegelen ten opzichte van de $Y$-as.
Op die manier kan je bijvoorbeeld heel eenvoudig afleiden dat de grafiek
van de functie $g(x)=\sqrt{-x}$ bestaat en het spiegelbeeld is van
$f(x)=\sqrt{x}$. Terwijl de functie $f(x)$ enkel gedefinieerd is
voor alle positieve re\"ele getallen, is de functie $g(x)$ dit enkel
voor alle negatieve re\"ele getallen.

\medskip{}


\noindent De grafiek van $y=af(x)$ wordt verkregen uit de grafiek
van $y=f(x)$ door, bij gelijkblijvende $x$-co\"ordinaten, alle $y$-co\"ordinaten
van de punten op de grafiek met een factor $a$ te vermenigvuldigen
(inkrimpen als $0<a<1$, en openrekken als $a>1$).

\noindent \medskip{}


\begin{tabular}{ccc}
\includegraphics[height=5cm]{\string"2_elem_rekenvaardigheden_B/inputs/Herschalen - verticaal - inkrimpen\string".eps}  &  & \includegraphics[height=5cm]{\string"2_elem_rekenvaardigheden_B/inputs/Herschalen - verticaal - openrekken\string".eps} \tabularnewline
\end{tabular}\medskip{}

