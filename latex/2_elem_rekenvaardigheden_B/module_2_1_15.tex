%\DeclareMathOperator{\tri}{tri} \DeclareMathOperator{\rect}{rect}
%\DeclareMathOperator{\sgn}{sgn} \DeclareMathOperator{\ramp}{ramp}
%\DeclareMathOperator{\sinc}{sinc}


\subsection{Verschuiven en herschalen}

%\begin{itemize}
%\item Wat gebeurt er met de grafiek van een functie als we bij het argument
%en/of bij de functie zelf, een getal optellen?
%\item Wat gebeurt er met de grafiek van een functie als we het argument
%en/of de functie zelf vermenigvuldigen met een getal?
%\end{itemize}

\subsubsection{Verschuiven}

Wanneer we de grafiek van $y=f(x)$ kennen, kunnen we met
verschuivingen (of translaties) de grafiek van $y=f(x+a)$ en $y=f(x)+a$
daaruit afleiden (met $a\in\mathbb{R}$).

De grafiek van $y=f(x+a)$ wordt verkregen uit de grafiek
van $y=f(x)$ door alle punten op de grafiek met een afstand van $a$
eenheden horizontaal te verschuiven (naar links als $a>0$, en naar
rechts als $a<0$).

Het lijkt misschien een beetje tegenstrijdig dat de grafiek
van een functie naar rechts verschuift als je bij het argument $x$
een negatief getal optelt. Maar kijk eens naar de sinus functie $f_{1}(x)=\sin(x)$
in onderstaand voorbeeld. We weten dat de sinus functie onder andere
een nulpunt heeft als het argument nul is, dus voor $x=0$. Als we
van $x$ de waarde $\frac{\pi}{2}$ aftrekken, dan bekomen we terug
datzelfde nulpunt als het argument nul is. Uit $x-\frac{\pi}{2}=0$
volgt dat dit het geval zal zijn als $x=+\frac{\pi}{2}$. Met andere
woorden, de functie $f_{2}(x)=\sin(x-\frac{\pi}{2})$ heeft dezelfde
grafiek als de functie $f_{1}(x)=\sin(x)$ maar is $\frac{\pi}{2}$
eenheden naar rechts verschoven.


\begin{figure}[H]
\begin{center}
	\input{2_elem_rekenvaardigheden_B/Fig_module_2_1_15_verschuiven_en_verschalen_1}
\end{center}
\end{figure}

\begin{figure}[H]
\begin{center}
	\input{2_elem_rekenvaardigheden_B/Fig_module_2_1_15_verschuiven_en_verschalen_2}
\end{center}
\end{figure}



De grafiek van $y=f(x)+a$ wordt verkregen uit de grafiek
van $y=f(x)$ door alle punten op de grafiek met een afstand van $a$
eenheden verticaal te verschuiven (naar beneden als $a<0$, en naar
boven als $a>0$).


\begin{figure}[H]
	\input{2_elem_rekenvaardigheden_B/Fig_module_2_1_15_verschuiven_en_verschalen_3}
\end{figure}


\begin{figure}[H]
	\input{2_elem_rekenvaardigheden_B/Fig_module_2_1_15_verschuiven_en_verschalen_4}
\end{figure}


\begin{voorbeeld}
hoe ziet de grafiek eruit van de functie $y=x^{2}+2x+3$?

We herschrijven de functie als: $y=\left(x^{2}+2x+1\right)+2=\left(x+1\right)^{2}+2$

We herkennen hierin de basisfunctie $y=x^{2}$. De gegeven functie
stelt dus een parabool voor die over 1 eenheid naar links en 2 eenheden
naar boven is verschoven.
\end{voorbeeld}


\subsubsection{Herschalen}

Wanneer we de grafiek van $y=f(x)$ kennen, kunnen we met
herschalen (of vermenigvuldigen) de grafiek van $y=f(ax)$ en $y=af(x)$
daaruit afleiden (met $a\in\mathbb{R}$).

De grafiek van $y=f(ax)$ wordt verkregen uit de grafiek
van $y=f(x)$ door, bij gelijkblijvende $y$-co\"ordinaten, alle $x$-co\"ordinaten
van de punten op de grafiek met een factor $\frac{1}{a}$ te vermenigvuldigen
(openrekken als $0<a<1$, en inkrimpen als $a>1$).

Voor $a=2$ bijvoorbeeld krimpt de grafiek in elkaar. Het
is alsof de functie dubbel zo snel verloopt.


\begin{figure}[H]
	
%TODO polynoombenadering uitrekenen > zie cursus Algebra

\begin{tikzpicture}[scale=1,cap=round]

% Styles
\tikzstyle{axes}=[]
\tikzstyle help lines=[color=blue!50,very thin,dotted]


%%%%%%%%%%%%%%%%%%%%%%%%%%%%%%%%
%		GRID
%%%%%%%%%%%%%%%%%%%%%%%%%%%%%%%%

\draw[style=help lines,step=1cm] (-6.9,-3.9) grid (6.9,3.9);



%%%%%%%%%%%%%%%%%%%%%%%%%%%%%%%%
%		ASSENSTELSEL
%%%%%%%%%%%%%%%%%%%%%%%%%%%%%%%%

\draw[->] (-7,0) -- (7,0) node[right] {$x$};
\draw[->] (0,-4) -- (0,4) node[left]{$y$};

%\draw[fill,cyan](1,1)circle [radius=0.025];

%\draw[red,cap=rect, loosely dashed, ultra thick, domain=-2:2] plot (\x, {(\x*\x-1)+0.05}) node[above,yshift=-.7cm, right]{};

%%%%%%%%%%%%%%%%%%%%%%%%%%%%%%%%
%legende
%%%%%%%%%%%%%%%%%%%%%%%%%%%%%%%%
%\tkzDefPoint(0.5,3.5){A}
%\tkzDefPoint(1,3.5){B}
%\tkzLabelPoint[right,xshift=+0.1cm](B){${\color{cyan}f(x)=|x^2-1|}$}
%\tkzDrawSegment[cyan,ultra thick](A,B)

%\tkzDefPoint(0.5,3.2){C}
%\tkzDefPoint(1,3.2){D}
%\tkzLabelPoint[right,xshift=+0.1cm](D){${\color{red}e(x)=x^2-1}$}
%\tkzDrawSegment[red,cap=rect, loosely dashed, ultra thick](C,D)


%%%%%%%%%%%%%%%%%%%%%%%%%%%%%%%%
%getallen op de x-as en lijntjes
%%%%%%%%%%%%%%%%%%%%%%%%%%%%%%%%   
\foreach \x/\xtext in {-6,-5,-4,-3,-2,-1,1,2,3,4,5,6}
	\draw[xshift=\x cm] (0pt,1pt) -- (0pt,0pt) node[below,fill=white]
	{$\xtext$};,3
	
%getallen op de y-as en lijntjes  
%BEGIN LUS
\foreach \y/\ytext in {-3,-2,-1,1,2,3}
	\draw[yshift=\y cm] (1pt,0pt) -- (0pt,0pt) node[left,fill=white]
	{$\ytext$}; %EINDE LUS



%%%%%%%%%%%%%%%%%%%%%%%%%%%%%%%%
%		GRAFIEKEN
%%%%%%%%%%%%%%%%%%%%%%%%%%%%%%%%


\draw[teal,cap=rect,line width=2, opacity=0.8, domain=-7:7,samples=100] plot (\x, {
	sin(\x r)	% <- plaats het functievoorschrift hier	
}) node[opacity=1,,pos=0,xshift=+6cm,yshift=-1.5cm]{$f_1(x)=\sin{x}$};
%-------------------------------------------
\draw[red,cap=rect,line width=1, opacity=1, domain=-7:7,samples=100] plot (\x, {
	(sin(0.5*(\x r))	% <- plaats het functievoorschrift hier	
}) node[opacity=1,,pos=0,xshift=-5cm,yshift=+1.5cm]{$f_2(x)=0.5\sin{x}+2$};
%-------------------------------------------



%\draw[cyan,cap=rect,thick, domain=-6:6] plot (\x, \x) node[above, right]{${\color{cyan}y=x}$};v

%\draw[cyan,cap=rect,ultra thick, domain=-6:1.75] plot (\x, {(\x-2)^(-1)}) node[above,right]{};


%\draw[cyan,cap=rect,ultra thick, domain=2.25:6] plot (\x, {(\x-2)^(-1)}) node[above,yshift=+0.5cm,left]{$\color{cyan} y=\frac{1}{x-2}$};


%\draw[cyan,cap=rect,ultra thick, domain=-7:1.9] plot (\x, {exp{\x}}) node[above, right]{${\color{cyan}y=\exp{x}}$};

%%%%%%%%%%%%%%%%%%%%%%%%%%%%%%%%
%		MARKERINGEN
%%%%%%%%%%%%%%%%%%%%%%%%%%%%%%%%
%verticale lijn
%\draw[-o,line width=4,teal, cap=rect,opacity=0.3] (0,-4) -- (0,0.25) node[right] {};
%\draw[line width=4,teal, cap=rect,opacity=0.3] (0,0) -- (0,4.2) node[right] {bld $f$ = $\mathbb{R}_0$};
%horizontale lijn
%\draw[arrows=-o,line width=4,blue, cap=rect,opacity=0.3] (-7,0) -- (2.25,0) node[right] {};
%\draw[line width=4,blue, cap=rect,opacity=0.3] (2.25,0) -- (7,0) node[below,yshift=-0.3cm] {dom$f$ = $\mathbb{R}  \setminus 2 $};
 
\end{tikzpicture}

\end{figure}
  
\begin{figure}[H]
	\input{2_elem_rekenvaardigheden_B/Fig_module_2_1_15_verschuiven_en_verschalen_6}
\end{figure}



Een bijzonder geval treedt op als $a=-1$: de grafiek van
$g(x)=f(-x)$ kan uit de grafiek van $f(x)$ worden verkregen door
van alle punten op de grafiek de $x$-co\"ordinaten met -1 te vermenigvuldigen,
dus door de grafiek van $f(x)$ te spiegelen ten opzichte van de $y$-as.
Op die manier kan je bijvoorbeeld heel eenvoudig afleiden dat de grafiek
van de functie $g(x)=\sqrt{-x}$ bestaat en het spiegelbeeld is van
$f(x)=\sqrt{x}$. Terwijl de functie $f(x)$ enkel gedefinieerd is
voor alle positieve re\"ele getallen, is de functie $g(x)$ dit enkel
voor alle negatieve re\"ele getallen.


De grafiek van $y=af(x)$ wordt verkregen uit de grafiek
van $y=f(x)$ door, bij gelijkblijvende $x$-co\"ordinaten, alle $y$-co\"ordinaten
van de punten op de grafiek met een factor $a$ te vermenigvuldigen
(inkrimpen als $0<a<1$, en openrekken als $a>1$).

%TODO figuur aanpassen
\begin{figure}[H]
	\input{2_elem_rekenvaardigheden_B/Fig_module_2_1_15_verschuiven_en_verschalen_7}
\end{figure}

\begin{figure}[H]
	\input{2_elem_rekenvaardigheden_B/Fig_module_2_1_15_verschuiven_en_verschalen_8}
\end{figure}




%\begin{figure}[H]
%	\begin{subfigure}{.5\linewidth}
%	\includegraphics[height=5cm]{\string"2_elem_rekenvaardigheden_B/inputs/Herschalen - verticaal - i%nkrimpen\string".eps}
%	\end{subfigure}
%	\begin{subfigure}{.5\linewidth}
%	\includegraphics[height=5cm]{\string"2_elem_rekenvaardigheden_B/inputs/Herschalen - verticaal - o%penrekken\string".eps}
%	\end{subfigure}	
%\end{figure}

