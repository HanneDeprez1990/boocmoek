\subsection{Irrationale functies}
%\maketitle
%\begin{itemize}
%\item Wat zijn irrationale functies?
%\item Wanneer hebben ze (verticale, horizontale en/of schuine) asymptoten?
%\end{itemize}
%Dit zijn vergelijkingen waarin wortelvormen voorkomen, eventueel met
%een breuk. Deze wortelvormen kan je wegwerken door beide leden van
%de vergelijking te verheffen tot een gepaste macht.

\begin{voorbeeld}
	$f(x)=\sqrt{x^{2}+2x+4}$, $f(x)=2\cdot \sqrt[3]{x^{5}-3x^{2}+7}-x-1$,
$f(x)=\frac{x-2}{\sqrt{10-x^{2}}}$
\end{voorbeeld}

\textbf{Grafische voorstelling}

De grafiek van de elementaire wortelfuncties $y=\sqrt[n]{x}$
is hieronder weergegeven. Denk eraan dat (machts)wortels ook als macht
kunnen geschreven worden: $y=\sqrt[n]{x}=x^{\frac{1}{n}}$.

\gewonefiguur{width=10cm}{2_elem_rekenvaardigheden_B/inputs/machtsw1}

Om het domein (de toegelaten waarden voor $x$) te bepalen moeten
we rekening houden met zowel de \emph{bestaansvoorwaarden} als de
\emph{kwadrateringsvoorwaarde}.

\begin{itemize}
	\item Bestaansvoorwaarde(n): de uitdrukking onder een even machtswortel
moet steeds positief zijn!! En niet vergeten, de eventuele noemer
mag niet nul worden.
\item Kwadrateringsvoorwaarde: we spreken af dat een even machtswortel uit
een uitdrukking steeds positief is.
\end{itemize}

Het domein van een irrationale functie valt samen met de intervallen,
waar de vorm onder de vierkantswortel niet negatief is en waar de
uitdrukking van een even machtsfunctie steeds een positief resultaat
oplevert.

\begin{voorbeeld}
	\begin{equation*}
f(x)=\sqrt{x^{2}-4}
\end{equation*}

\gewonefiguur{width=7cm}{2_elem_rekenvaardigheden_B/inputs/machtsw2}

Om het domein van de functie $f(x)$ te bepalen moet er aan de bestaansvoorwaarde(n)
voldaan zijn. In dit geval mag de functie onder het wortelteken niet
negatief worden. We werken dit verder uit:

De bestaansvoorwaarde: $\begin{array}{cccl}
x^{2}-4\geq0 &\iff & x^{2}\geq4\\
& \iff & x\geq2 \textrm{ en } x\leq-2
\end{array}$

De kwadrateringsvoorwaarde is reeds voldaan (want er staat infeite
$f(x)=+\sqrt{x^{2}-4}$).

Welke waarden van $x$ voldoen hier nu aan? $\textrm{dom}f(x)$ is voor $x\in]-\infty,-2]\bigcup[2,+\infty[$.

\end{voorbeeld}

\begin{voorbeeld}
\begin{equation*}
f(x)=x+\sqrt{5x+2}-1
\end{equation*}

\gewonefiguur{width=7cm}{2_elem_rekenvaardigheden_B/inputs/vbirrat1}

Om het domein van de functie $f(x)$ te bepalen moet er aan de bestaansvoorwaarde(n)
voldaan zijn. In dit geval mag de functie onder het wortelteken niet
negatief worden. We werken dit verder uit:

De bestaansvoorwaarde:$\begin{array}{cclccc}
 & 5x+2\geq0
 & \iff & x & \geq & -\frac{2}{5}\\
\textrm{dus} & & \iff & x & \in & [-\frac{2}{5},+\infty[
\end{array}$

Nu moeten we nog de kwadrateringsvoorwaarde controleren, m.a.w. volgens
onze gemaakte afspraak is een even machtswortel uit een uitdrukking
steeds positief is:

De bestaansvoorwaarde:
$\begin{array}{cccclcc}
 & x+\sqrt{5x+2}=1 & \iff & \sqrt{5x+2} & = & 1-x & \geq0\\
 & & \iff & 1-x & \geq & 0\\
\textrm{dus} & & \iff & x & \leq & 1\\
\textrm{of} & & \iff & x & \in & ]-\infty,1]
\end{array}$

De mogelijke waarden voor $x$ moeten aan beide voorwaarden voldoen.
Dit betekent: $\textrm{dom}f(x)$ is voor $x\in[-\frac{2}{5},1]$

\end{voorbeeld}

\textbf{Nulpunten}

De vergelijking wordt $f(x)=0$.

De wortelvormen kan je wegwerken door beide leden van de vergelijking
te verheffen tot een gepaste macht. Bepaal vervolgens de nulpunten
van de functie. Wanneer een breuk voorkomt in het functievoorschrift,
bepaal dan ook de nulpunten van de noemer, deze zijn de polen.

\textbf{Asymptoten}

\begin{enumerate}
	\item De rechte $x=a$ is een \textbf{verticale asymptoot} (VA) van de
irrationale functie $f(x)$ als en slechts als $a$ een nulpunt is
van de noemer en geen nulpunt van de teller.

\begin{equation*}
\lim_{\overset{x\rightarrow a}{<}}f(x)=\pm\infty\quad\textrm{of}\quad \lim_{\overset{x\rightarrow a}{>}}f(x)=\pm\infty \Rightarrow\:x=a\:\textrm{is een VA}
\end{equation*}

\item Een irrationale functie $f(x)$ heeft een \textbf{horizontale asymptoot}
(HA) als en slechts als de graad van de teller $\le$ graad
van de noemer.
\begin{equation*}
\lim_{x\to\pm\infty}f(x)=b \Rightarrow\:y=b\:\textrm{is een HA}
\end{equation*}

\item Een irrationale functie $f(x)$ heeft een \textbf{schuine asymptoot}
(SA) als en slechts als de graad van de teller = graad van de noemer
+1.

\begin{equation*}
m=\lim_{x\to\pm\infty}\frac{f(x)}{x}\quad\textrm{en}\quad 
q= \lim_{x\to\pm\infty}\left[f(x)-mx\right]
\Rightarrow\:y=mx+q\;\textrm{is een SA}
\end{equation*}

\end{enumerate}


\textbf{Tekenverloop}

Wanneer je de wortels hebt weggewerkt door de vergelijking te verheffen
tot een gepaste macht, zal je een veeltermfunctie, een kwadratische
of een lineaire functie bekomen. Pas het desbetreffend tekenonderzoek
toe.


\begin{voorbeeld}
	Bespreek de irrationale functie $f(x)=\frac{\sqrt{4-x^{2}}}{x-1}$

\underline{Domein}

\begin{eqnarray}
\textrm{dom}f(x)=[-2,1[\bigcup]1,2]
\end{eqnarray}

want de bestaansvoorwaarden zijn:  
\begin{equation*}
\begin{array}{ccclcc}
& 4-x^{2}\geq0 & \iff & x^{2}\leq4\\
\textrm{dus} & & \iff & x\leq2\;\textrm{en}\:x\geq-2\\
\textrm{of} & & \iff & x\in[-2,2]
\end{array}
\end{equation*}

en:
\begin{equation*}
x-1\neq0 \textrm{ dus } x\neq1
\end{equation*}

De kwadrateringsvoorwaarde is hier niet van toepassing (vanwege de
noemer kan $f(x)$ ook negatief worden).

\underline{Nulpunten}

Stap 1: Bepaal de nulpunten van de teller, deze zijn de nulpunten
van de functie $f(x)$.

Dus 
\begin{equation*}
\begin{array}{ccc}
f(x)=0 & \iff & \frac{\sqrt{4-x^{2}}}{x-1} = 0\\
& \iff & \sqrt{4-x^{2}} = 0\\
& \iff & 4-x^{2} = 0\\
& \iff & x^{2} = 4\\
& \iff & x = \pm2
\end{array}
\end{equation*}

Stap 2: Bepaal eventueel de nulpunten van de noemer, deze zijn de
polen van de functie $f(x)$. Bij elke pool hoort een verticale asymptoot.

Het nulpunt van de noemer is $x=1$ (dit is dan tevens de vergelijking
van de verticale asymptoot).

\underline{Asymptoten}

Stap 3: Ga na of de functie een verticale asymptoot bezit. 

De rechte $x=1$ is een verticale asymptoot (VA) van de functie $f(x)$
aangezien $1$ een nulpunt is van de noemer en geen nulpunt van de
teller is.

Hoe verloopt de functie $f(x)$ in de buurt van deze asymptoot:

\begin{equation*}
\textrm{LL}:\lim_{\overset{x\rightarrow1}{<}}\left(\frac{\sqrt{4-x^{2}}}{x-1}\right)=-\infty\quad\textrm{en}\quad \textrm{RL}:\lim_{\overset{x\rightarrow1}{>}}\left(\frac{\sqrt{4-x^{2}}}{x-1}\right)=+\infty \Rightarrow x=1 \text{ is een VA}
\end{equation*}


Stap 4: Ga na of de functie een horizontale asymptoot bezit.

De functie heeft geen horizontale asymptoot, want $x$ die nadert
naar $+\infty$ of $-\infty$ behoort niet tot het domein.

Stap 5: Ga na of de functie een schuine asymptoot heeft.

De functie heeft geen schuine asymptoot, want $x$ die nadert naar
$+\infty$ of $-\infty$ behoort niet tot het domein.

\underline{Tekenverloop}

Stap 6:
\begin{itemize}
\item Schrijf in \'e\'en tabel bovenaan alle nulpunten en alle polen
	in stijgende volgorde. 
\item Onderzoek het teken voor elke factor van $f(x)$. 
\item Het teken van $f(x)$ is dan het product van deze tekens.
\end{itemize}


\begin{tabel}{Voorbeeld irrationale functies: tekenverloop.}
\begin{tabular}{c|cccccccc}
$x$ &  & $-2$ &  & $1$ &  & $2$ &  & $\longrightarrow\mathbb{R}$\\
\hline  
$\sqrt{4-x^{2}}$ & / & 0 & + &  & + & 0 & / & \\
$x-1$ & - & - & - & 0 & + & + & + & \\
\hline 
$f(x)=\frac{\sqrt{4-x^{2}}}{x-1}$ & / & 0 & - & $\mid$ & + & 0 & / & \\
\end{tabular}
\label{tab:irrattk}
\end{tabel}

\underline{Grafiek}

\gewonefiguur{width=7cm}{2_elem_rekenvaardigheden_B/inputs/vbirrat2}

\end{voorbeeld}

\begin{ftonthoud}
	De grafiek en het tekenverloop van een irrationale functie bepaal
je als volgt:

Het domein van een irrationale functie valt samen met de intervallen,
waar de vorm onder de vierkantswortel niet negatief (BVW) is en waar
de uitdrukking van een even machtsfunctie steeds een positief resultaat
oplevert (KVW).

Stap 1: Bepaal de nulpunten van de teller, dit zijn de nulpunten van
de functie $f(x)$.

Stap 2: Bepaal eventueel de nulpunten van de noemer, dit zijn de polen
van de functie $f(x)$.

Stap 3: De rechte $x=a$ is een \textbf{verticale asymptoot} (VA)
van de irrationale functie $f(x)$ als en slechts als $a$ een nulpunt
is van de noemer en maar geen nulpunt van de teller.

\begin{equation*}
\lim_{\overset{x\rightarrow a}{<}}f(x)=\pm\infty\quad\textrm{of}\quad \lim_{\overset{x\rightarrow a}{>}}f(x)=\pm\infty\quad\Rightarrow\:x=a\:\textrm{is een VA}
\end{equation*}

Stap 4: Een irrationale functie $f(x)$ heeft een \textbf{horizontale
asymptoot} (HA) als en slechts als de graad van de teller \ensuremath{\le}
graad van de noemer.
\begin{equation*}
 \lim_{x\to\pm\infty}f(x)=b\quad\Rightarrow\:y=b\:\textrm{is een HA}
\end{equation*}


Stap 5: Een irrationale functie $f(x)$ heeft een \textbf{schuine
asymptoot} (SA) als en slechts als de graad van de teller = graad
van de noemer +1.

\begin{equation*}
m=\lim_{x\to\pm\infty}\frac{f(x)}{x}\quad\textrm{en}\quad q= \lim_{x\to\pm\infty}\left[f(x)-mx\right] \Rightarrow\:y=mx+q\:\textrm{is een SA}
\end{equation*}


Stap 6: Bepaal het tekenverloop

\begin{itemize}
\item Schrijf in \'e\'en tabel bovenaan alle nulpunten en alle polen in stijgende
volgorde.

\item Hou rekening met de bestaansvoorwaarde en de kwadrateringsvoorwaarde.

\item Onderzoek het teken voor elke factor van $f(x)$.

\item Het teken van $f(x)$ is dan het product van deze tekens.
\end{itemize}

\end{ftonthoud}