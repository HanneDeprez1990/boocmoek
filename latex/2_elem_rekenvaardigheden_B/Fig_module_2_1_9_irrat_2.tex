\begin{center}
\begin{tikzpicture}[scale=1,cap=round]

% Styles
\tikzstyle{axes}=[]
\tikzstyle help lines=[color=blue!50,very thin,dotted]

% grid
\draw[style=help lines,step=1cm] (-3.9,-5.9) grid (3.9,5.9);

\draw[->] (-4,0) -- (4,0) node[right] {$x$};
\draw[->] (0,-6) -- (0,6) node[above] {$y$};

%\draw[fill,cyan](1,1)circle [radius=0.025];
%FUNCTIEVOORSCHRIFTEN



%\draw[teal,cap=rect,line width=1, opacity=1, domain=-2:2] plot (\x, {
%	pow(\x,2)  		% <- plaats het functievoorschrift hier
%}) node[right,opacity=1]{$h(x)=x^2$};

%\draw[red,cap=rect,line width=1, opacity=1, domain=-1.5:1.5] plot (\x, {
%	pow(\x,4)  		% <- plaats het functievoorschrift hier
%}%) node[opacity=1,above,xshift=-3.5cm]{$p(x)=x^4$};

\draw[blue,cap=rect,line width=1, opacity=1, domain=-2:0.65,samples=100] plot (\x, {
	pow(4-\x*\x,1/2)/(\x-1)	% <- plaats het functievoorschrift hier	
}) node[opacity=1,above,left]{};
%-------------------------------------------

\draw[blue,cap=rect,line width=1, opacity=1, domain=1.3:2,samples=100] plot (\x, {
	pow(4-\x*\x,1/2)/(\x-1)	% <- plaats het functievoorschrift hier	
}) node[opacity=1,above,right,yshift=+3cm]{$f(x)=\frac{\sqrt{4-x^2}}{x-1}
	$};
%-------------------------------------------


%\draw[orange,cap=rect,line width=1, opacity=1, domain=-1.4:1.4] plot (\x, {
%	pow(\x,5)  		% <- plaats het functievoorschrift hier
%}) node[right,opacity=1]{$g(x)=x^5$};



%\draw[cyan,cap=rect,ultra thick, domain=1:2] plot (\x, {\x*\x-1}) node[above, right]{};
%\draw[red,cap=rect, loosely dashed, ultra thick, domain=-2:2] plot (\x, {(\x*\x-1)+0.05}) node[above,yshift=-.7cm, right]{};

%legende



%getallen op de x-as en lijntjes   
\foreach \x/\xtext in {-2,-1,0,1,2}
	\draw[xshift=\x cm] (0pt,1pt) -- (0pt,0pt) node[below,fill=white]
	{$\xtext$};,3
	
%getallen op de y-as en lijntjes  
%BEGIN LUS
\foreach \y/\ytext in {-6,-5,-4,-3,-2,-1,1,2,3,4,5,6}
	\draw[yshift=\y cm] (1pt,0pt) -- (0pt,0pt) node[left,fill=white]
	{$\ytext$}; %EINDE LUS


%%%%%%%%%%%%%%%%%%%%%%%%%%%%%%%
%		MARKERINGEN
%%%%%%%%%%%%%%%%%%%%%%%%%%%%%%%%
%verticale lijn
\draw[line width=1,red, cap=rect,opacity=1] (1,-6) -- (1,6) node[right] {};
%\draw[line width=4,teal, cap=rect,opacity=0.3] (0,0) -- (0,4.2) node[right] {bld $f$ = $\mathbb{R}_0$};
%horizontale lijn
%\draw[line width=4,blue, cap=rect,opacity=0.3] (-1,0) -- (1,0) node[near start,above,xshift=-1.1cm,opacity=1] {maximum in (0,0)};


%\draw[line width=4,blue, cap=rect,opacity=0.3] (1,-4) -- (3,-4) node[below,yshift=-.3cm,opacity=1] {minium in (2,-4)};
\end{tikzpicture}
\end{center}

