\subsection{Re\"ele functies}

\subsubsection{Definitie, notatie, functievoorschrift}

\begin{definitie}
	
Een functie is een verband: een functie $f$  associeert bij elk gegeven getal $x$ hoogstens \'e\'en ander getal (de functiewaarde $f(x)$). De functie $f$ kan een functievoorschrift hebben om volgens een bepaalde vaste regel van zo'n $x$ de bijhorende functiewaarde $f(x)$ te berekenen.

\end{definitie}

Er zijn functies met een speciale naam, zoals de sinus-functie: $\sin$ en functies met speciale symbolen, zoals de wortel-functie: $\sqrt{\ }$. Het functievoorschrift van deze functies is resp. $f(x)=\sin(x)$ en $f(x)=\sqrt{x}$.

\subsubsection{Grafische voorstelling}

\begin{definitie}
	Een functie $f$  kan altijd grafisch worden weergeven. De vergelijking $f=f(x)$  geeft het verband tussen de $x$-waarden op de \emph{horizontale} as en de functiewaarden $f(x)$ op de \emph{verticale} as.
\end{definitie}

De letter $y$ wordt de \emph{afhankelijke variabele}, de
\emph{beeldwaarde} (of kortweg het \emph{beeld}), de \emph{output}
of ook nog de \emph{functiewaarde} genoemd. 
De letter $x$ is hier
de \emph{onafhankelijke variabele}, de \emph{input} of ook wel het
\emph{argument} genoemd. 

Voorbeeld: het verband tussen de temperatuur in graden
Celsius \textdegree$C$ ($x$) en graden Fahrenheit \textdegree$F$ ($y$)
wordt gegeven door volgende functie: $y=1,8x+32$

%Merk op dat bij een functie voor elke $y$-waarde hoogstens
%één $x$-waarde mag bestaan. M.a.w. met niet elke $x$-waarde hoeft
%een beeldwaarde $y$ overeen te komen, maar als er een beeld is, mag
%er slechts één beeld zijn. Als er meerdere beelden bestaan spreken
%we van een \emph{relatie}.


\subsubsection{Domein en beeld}


\begin{definitie}
	Het \emph{domein} (of \emph{definitiegebied}) van een functie
$f$ is de verzameling van alle getallen $\in \mathbb{R}$ ($x$-waarden) die en beeld
een beeld ($y$-waarde) hebben. Het domein is dus de
verzameling van alle getallen $\in \mathbb{R}$ die acceptabel zijn als input.
\end{definitie}
\begin{notatie}
$\textrm{dom}f$ of $\textrm{def}f$.
\end{notatie}



\begin{voorbeeld}
	Voor de functie $y=\sqrt{x}$ mag $x$ geen
negatief getal zijn, dus $\textrm{dom}f=\mathbb{R}^{+}=\left[0,+\infty\right[$
\end{voorbeeld}

\begin{voorbeeld}
	De functie $y=\frac{1}{x-1}$ heeft geen beeld
voor $x=1$, dus $\textrm{dom}f=\mathbb{R}\setminus\left\{ 1\right\} $

\end{voorbeeld}

Tips bij het bepalen van het domein:
\begin{itemize}
\item Een even machtswortel kan je enkel nemen van positieve getallen $\in \mathbb{R}$.
\item Een oneven machtswortel kan je van elk getal $\in \mathbb{R}$ nemen.
\item Delen door nul mag niet!
\end{itemize}

\begin{definitie}
	Het \emph{beeld} (of \emph{bereik}) van een functie $f$
is de verzameling van alle getallen $\in \mathbb{R}$ welke het beeld (de functiewaarde
$y$) van de functie kunnen zijn. Het beeld is dus de verzameling
van alle getallen $\in \mathbb{R}$ die kunnen optreden als output. Notatie: $\textrm{bld}f$.
\end{definitie}

Merk op dat bij een functie voor elke beeldwaarde hoogstens 1 $x$-waarde mag bestaan.

Bij een vergelijking waarbij er meerdere beelden bestaan voor een $x$-waarde, spreken we van een relatie.

\begin{ftonthoud}
\begin{itemize}
\item Het domein lees je af op de horizontale $x$-as (in onderstaande voorbeelden aangeduid met een groene lijn).
\item Het beeld lees je af op de verticale $y$-as (in onderstaande voorbeelden aangeduid met een blauwe lijn).
\end{itemize}
\end{ftonthoud}


\begin{voorbeeld}
Bij de functie $f(x)=1-x^{2}$ kan je voor geen
enkele waarde van $x$ een beeldwaarde bekomen die groter is dan 1 (omdat $1-x^2 \le 1$),
dus $\textrm{bld}f=\left]-\infty,1\right]$
\end{voorbeeld}

De vergelijking $y^2=x$ kunnen we niet zien als een functie in $x$, omdat zowel $y=1$ als $y=-1$ zouden horen bij $x=1$. Dit is dus een voorbeeld van een relatie (maar geen functie)!

\subsubsection{Nulpunten}

\begin{definitie}
	Een \emph{nulwaarde} van de functie $f$ is een getal $a$ waarvoor geldt dat $f(a)=0$. Het punt $(a,0)$ noemen we een nulpunt.
\end{definitie}

Het berekenen van de nulwaarden van een
functie $f$ vereist het oplossen van de vergelijking $f(x)=0$.
Grafisch gezien zijn de nulpunten $(a,0)$ de snijpunten van de grafiek
van de functie $f$ met de horizontale as (de $X$-as).

\begin{voorbeeld}
	Voor de tweedegraadsfunctie met voorschrift $f(x)=ax^2+bx+c$ komt dit neer op $ax^2+bx+c=0$ oplossen en kunnen we dus de methode van de discriminant toepassen om de nulwaarden te vinden.
\end{voorbeeld}

\subsubsection{Snijpunten met de assen}

De snijpunten van de functie $f$ met de $x$-as vinden
we door het oplossen van het stelsel
\begin{equation*}
\begin{cases}
y=f(x)\\
y=0
\end{cases}
\end{equation*}


 of dus de vergelijking $f(x)=0$. Het snijpunt van de functie $y=f(x)$ met de $y$-as vinden
we door het oplossen van het stelsel 
$\begin{cases}
y=f(x)\\
x=0
\end{cases}$ of dus het berekenen van $f(0)$.


%\subsubsection{Grafische voorstelling}
%
%Een functie $y=f(x)$ kan altijd grafisch worden weergeven.
%Voor een functie $y=f(x)$ staan de $x$-waarden op de horizontale
%as en de functiewaarden $f(x)$ op de verticale as.
%
%Het domein lees je af op de horizontale $X$-as (in onderstaande
%voorbeelden aangeduid met een groene lijn) 
%
%Het beeld lees je af op de verticale $Y$-as (in onderstaande
%voorbeelden aangeduid met een blauwe lijn)


\subsubsection{Even en oneven functies}


\begin{definitie}
	Een functie $f(x)$ noemt men \emph{even} als $f(-x)=f(x)$.
\\
Een even functie heeft de $y$-as als as van symmetrie.


Een functie $f(x)$ noemt men \emph{oneven} als $f(-x)=-f(x)$.
\\
Een oneven functie heeft de oorsprong als middelpunt van symmetrie.

\end{definitie}

Als voor een functie $f(-x)$ niet gelijk is aan $f(x)$
of $-f(x)$ dan is de functie noch even, noch oneven. Dit hoeft niet
te betekenen dat deze functie niet ergens anders symmetrisch zou kunnen
zijn.




\begin{voorbeeld}
	De functie $f$ met voorschrift $f(x)=x^2$ is een eenvoudig voorbeeld van een even functie. Voor elke  $x \in \mathbb{R}$ geldt immers dat 

\begin{equation*}
f(-x)=(-x)^2=x^2=f(x)
\end{equation*}

De functie $g$ met voorschrift $g(x)=x^3$ is dan weer een eenvoudig voorbeeld van een oneven functie. Voor elke $x \in \mathbb{R}$ geldt immers dat

\begin{equation*}
g(-x)=(-x)^3=-x^3=g(x)
\end{equation*}

Een functie $h$ met voorschrift $h(x)=x+1$ is noch een even functie, noch een oneven functie. Er zijn namelijk $x$-waarden waarvoor geen van de verbanden gelden:

\begin{eqnarray*}
h(1) = 2 &\text{ maar }& h(-1) = 0 \\
h(2) = 3 &\text{ maar }& h(-2) = -1 
\end{eqnarray*}

\end{voorbeeld}

\subsubsection{Snijpunten van twee functies}

Als je voor twee verschillende functies $f$ en $g$ de
eventuele snijpunten van hun grafiek zoekt, dan moet je de vergelijking
$f(x)=g(x)$ oplossen. Zo vind je de $x$-co\"ordinaten van de snijpunten.


Invullen van deze gevonden $x$-co\"ordinaten in één van de twee functievoorschriften
geeft dan de bijhorende $y$-co\"ordinaten.




\begin{voorbeeld}
	Om te bepalen waar de snijpunten van de grafieken van $f(x)=x^2$ en $g(x)=3x-2$ liggen, onderzoeken we bijgevolg

\begin{equation*}
x^2 = 3x-2 \iff x^2-3x+2=0
\end{equation*}

wat snel opgelost kan worden met de methode van de discriminant:

\begin{equation*}
D = b^2 - 4ac = 9-8=1
\end{equation*}

zodat

\begin{equation*}
x_1 = \frac{3+1}{2} \text{ en } x_2 = \frac{3-1}{2} = 1.
\end{equation*}


De snijpunten van de grafieken van $f$ en $g$ hebben dus als $x$-waarden 1 en 2. Ook de beeldwaarden van deze snijpunten kunnen we nu snel berekenen: $f(1)=1$ en $f(2)=4$.

\end{voorbeeld}

%Opmerking 1: eigenlijk hebben we dit reeds toegepast om
%de nulpunten van een functie te vinden (= snijpunten met de $X$-as).
%Hierbij is dan de functie $g(x)=0$.
%
%
%
%
%Opmerking 2: een vaak voorkomend probleem is het vinden
%van de (twee) nulpunten van de tweedegraadsvergelijking $f(x)=ax^{2}+bx+c$
%of m.a.w. zoek de snijpunten van deze parabool met de $X$-as. De
%oplossingen $x_{1}$ en $x_{2}$ van de vergelijking $f(x)=ax^{2}+bx+c=0$
%kunnen snel gevonden worden via de formules $x_{1}=\frac{-b+\sqrt{b^{2}-4ac}}{2a}$
%en $x_{2}=\frac{-b-\sqrt{b^{2}-4ac}}{2a}$


