
%\DeclareMathOperator{\tri}{tri} \DeclareMathOperator{\rect}{rect}
%\DeclareMathOperator{\sgn}{sgn} \DeclareMathOperator{\ramp}{ramp}
%\DeclareMathOperator{\sinc}{sinc}


\subsection{Goniometrische functies}
%\begin{itemize}
%\item Wat is een periode?
%\item Wat is een goniometrische functie?
%\item Wat zijn de grafieken van sin, cos, tan en cot? 
%\end{itemize}

\subsubsection{De periode}

Veel fenomenen in ons dagelijks leven herhalen zich, bijvoorbeeld
je hartslag, de slingerbeweging van een staande klok, ... Ook in technische
wetenschappen komen zichzelf herhalende patronen vaak voor, denk maar
aan de trillingen in een gebouw, of het principe van wisselstroom.
Deze zichzelf herhalende fenomenen worden voorgesteld door periodieke
functies.


We noemen een functie periodiek als er een getal $\textrm{T}>0$
bestaat zodat

\begin{equation*}
f(x)=f(x+\textrm{T})=f(x+2\textrm{T})=\ldots \text{ en ook } f(x)=f(x-\textrm{T})=f(x-2\textrm{T})=\ldots
\end{equation*}

en dat voor alle $x$. Het kleinste getal $T$ met deze eigenschap
noemen we \textbf{de periode}.

We kunnen dit ook iets korter noteren: $f(x)=f(x+k\textrm{T})$
met $k\in\mathbb{Z}$ en dit $\forall x\in\mathbb{R}$.


Een periodieke functie met periode $\textrm{T}$ herhaalt zich dus
telkens na een \textquoteleft afstand\textquoteright{} $\textrm{T}$.
Je kan de periode ook zien als de \textquoteleft breedte\textquoteright{}
van het zich herhalende stuk. Een voorbeeld:




\begin{figure}[H]
	\centering
			\tikzsetfigurename{Fig_module_2_1_12_periodieke}

\begin{tikzpicture}[xscale=.5,cap=round]

% Styles
\tikzstyle{axes}=[]
\tikzstyle help lines=[color=blue!50,very thin,dotted]


%%%%%%%%%%%%%%%%%%%%%%%%%%%%%%%%
%		GRID
%%%%%%%%%%%%%%%%%%%%%%%%%%%%%%%%

\draw[style=help lines,step=1cm] (-14.9,-1.9) grid (14.9,1.9);



%%%%%%%%%%%%%%%%%%%%%%%%%%%%%%%%
%		ASSENSTELSEL
%%%%%%%%%%%%%%%%%%%%%%%%%%%%%%%%

\draw[->] (-15,0) -- (15,0) node[right] {$x$};
\draw[->] (0,-2) -- (0,2) node[left]{$y$};

%\draw[fill,cyan](1,1)circle [radius=0.025];

%\draw[red,cap=rect, loosely dashed, ultra thick, domain=-2:2] plot (\x, {(\x*\x-1)+0.05}) node[above,yshift=-.7cm, right]{};

%%%%%%%%%%%%%%%%%%%%%%%%%%%%%%%%
%legende
%%%%%%%%%%%%%%%%%%%%%%%%%%%%%%%%
%\tkzDefPoint(0.5,3.5){A}
%\tkzDefPoint(1,3.5){B}
%\tkzLabelPoint[right,xshift=+0.1cm](B){${\color{cyan}f(x)=|x^2-1|}$}
%\tkzDrawSegment[cyan,ultra thick](A,B)

%\tkzDefPoint(0.5,3.2){C}
%\tkzDefPoint(1,3.2){D}
%\tkzLabelPoint[right,xshift=+0.1cm](D){${\color{red}e(x)=x^2-1}$}
%\tkzDrawSegment[red,cap=rect, loosely dashed, ultra thick](C,D)


%%%%%%%%%%%%%%%%%%%%%%%%%%%%%%%%
%getallen op de x-as en lijntjes
%%%%%%%%%%%%%%%%%%%%%%%%%%%%%%%%   
\foreach \x/\xtext in {0}
	\draw[xshift=\x cm] (0pt,1pt) -- (0pt,0pt) node[below,fill=white]
	{$\xtext$};
	
%getallen op de y-as en lijntjes  
%BEGIN LUS
\foreach \y/\ytext in {}
	\draw[yshift=\y cm] (1pt,0pt) -- (0pt,0pt) node[left,fill=white]
	{$\ytext$}; %EINDE LUS



%%%%%%%%%%%%%%%%%%%%%%%%%%%%%%%%
%		GRAFIEKEN
%%%%%%%%%%%%%%%%%%%%%%%%%%%%%%%%


\draw[teal,cap=rect,line width=2, opacity=0.8, domain=-15:15,samples=500] plot (\x, {
%	sin(\x r) -2*	cos((\x +pi/4) r) % <- plaats het functievoorschrift hier	
0.4+	cos( (\x r) )  + 0.5*sin(2* (\x r) )% + 1.3 * cos(   (\x r) )  % <- plaats het functievoorschrift hier	
}) node[opacity=1,,pos=0,xshift=+6cm,yshift=+1.5cm]{};
%-------------------------------------------






%\draw[cyan,cap=rect,thick, domain=-6:6] plot (\x, \x) node[above, right]{${\color{cyan}y=x}$};v

%\draw[cyan,cap=rect,ultra thick, domain=-6:1.75] plot (\x, {(\x-2)^(-1)}) node[above,right]{};


%\draw[cyan,cap=rect,ultra thick, domain=2.25:6] plot (\x, {(\x-2)^(-1)}) node[above,yshift=+0.5cm,left]{$\color{cyan} y=\frac{1}{x-2}$};


%\draw[cyan,cap=rect,ultra thick, domain=-7:1.9] plot (\x, {exp{\x}}) node[above, right]{${\color{cyan}y=\exp{x}}$};

%%%%%%%%%%%%%%%%%%%%%%%%%%%%%%%%
%		MARKERINGEN
%%%%%%%%%%%%%%%%%%%%%%%%%%%%%%%%
%verticale lijn
%\draw[-o,line width=4,teal, cap=rect,opacity=0.3] (0,-4) -- (0,0.25) node[right] {};
%\draw[line width=4,teal, cap=rect,opacity=0.3] (0,0) -- (0,4.2) node[right] {bld $f$ = $\mathbb{R}_0$};
%horizontale lijn
%\draw[arrows=-o,line width=4,blue, cap=rect,opacity=0.3] (-7,0) -- (2.25,0) node[right] {};
%\draw[line width=4,blue, cap=rect,opacity=0.3] (2.25,0) -- (7,0) node[below,yshift=-0.3cm] {dom$f$ = $\mathbb{R}  \setminus 2 $};
 
\end{tikzpicture}

\end{figure}

Deze functie herhaalt zichzelf met periode $2\pi$.

\subsubsection{Goniometrische functies}

Goniometrische functies zijn functies opgebouwd uit de basisfuncties
$\sin$, $\cos$ en $\tan$.

Goniometrische functies gebruiken \emph{radialen} als argument,
en geen graden. Omdat deze basisfuncties zo belangrijk zijn, zetten
we ze even op een rijtje met bijhorende grafieken. 

\begin{itemize}
\item{Sinusfunctie}

$f(x)=\sin x$


\begin{figure}[H]
	\centering
			\tikzsetfigurename{Fig_module_2_1_12_sin}


\begin{tikzpicture}[scale=1,cap=round]

% Styles
\tikzstyle{axes}=[]
\tikzstyle help lines=[color=blue!50,very thin,dotted]


%%%%%%%%%%%%%%%%%%%%%%%%%%%%%%%%
%		GRID
%%%%%%%%%%%%%%%%%%%%%%%%%%%%%%%%

\draw[style=help lines,step=1cm] (-6.9,-1.9) grid (6.9,1.9);



%%%%%%%%%%%%%%%%%%%%%%%%%%%%%%%%
%		ASSENSTELSEL
%%%%%%%%%%%%%%%%%%%%%%%%%%%%%%%%

\draw[->] (-7,0) -- (7,0) node[right] {$x$};
\draw[->] (0,-2) -- (0,2) node[left]{$y$};

%\draw[fill,cyan](1,1)circle [radius=0.025];

%\draw[red,cap=rect, loosely dashed, ultra thick, domain=-2:2] plot (\x, {(\x*\x-1)+0.05}) node[above,yshift=-.7cm, right]{};

%%%%%%%%%%%%%%%%%%%%%%%%%%%%%%%%
%legende
%%%%%%%%%%%%%%%%%%%%%%%%%%%%%%%%
%\tkzDefPoint(0.5,3.5){A}
%\tkzDefPoint(1,3.5){B}
%\tkzLabelPoint[right,xshift=+0.1cm](B){${\color{cyan}f(x)=|x^2-1|}$}
%\tkzDrawSegment[cyan,ultra thick](A,B)

%\tkzDefPoint(0.5,3.2){C}
%\tkzDefPoint(1,3.2){D}
%\tkzLabelPoint[right,xshift=+0.1cm](D){${\color{red}e(x)=x^2-1}$}
%\tkzDrawSegment[red,cap=rect, loosely dashed, ultra thick](C,D)


%%%%%%%%%%%%%%%%%%%%%%%%%%%%%%%%
%getallen op de x-as en lijntjes
%%%%%%%%%%%%%%%%%%%%%%%%%%%%%%%%   
\foreach \x/\xtext in {-6,-5,-4,-3,-2,-1,1,2,3,4,5,6}
	\draw[xshift=\x cm] (0pt,1pt) -- (0pt,0pt) node[below,fill=white]
	{$\xtext$};,3
	
%getallen op de y-as en lijntjes  
%BEGIN LUS
\foreach \y/\ytext in {-2,-1,1,2}
	\draw[yshift=\y cm] (1pt,0pt) -- (0pt,0pt) node[left,fill=white]
	{$\ytext$}; %EINDE LUS



%%%%%%%%%%%%%%%%%%%%%%%%%%%%%%%%
%		GRAFIEKEN
%%%%%%%%%%%%%%%%%%%%%%%%%%%%%%%%


\draw[teal,cap=rect,line width=2, opacity=0.8, domain=-7:7,samples=100] plot (\x, {
	sin(\x r)	% <- plaats het functievoorschrift hier	
}) node[opacity=1,,pos=0,xshift=+6cm,yshift=+1.5cm]{$f_1(x)=\sin{x}$};
%-------------------------------------------
%\draw[red,cap=rect,line width=1, opacity=1, domain=-7:7,samples=100] plot (\x, {
%	sin((\x +pi/2) r)	% <- plaats het functievoorschrift hier	
%}) node[opacity=1,,pos=0,xshift=-6cm,yshift=+1.5cm]{$f_2(x)=\sin{x+\frac{\pi}{2}}$};
%-------------------------------------------



%\draw[cyan,cap=rect,thick, domain=-6:6] plot (\x, \x) node[above, right]{${\color{cyan}y=x}$};v

%\draw[cyan,cap=rect,ultra thick, domain=-6:1.75] plot (\x, {(\x-2)^(-1)}) node[above,right]{};


%\draw[cyan,cap=rect,ultra thick, domain=2.25:6] plot (\x, {(\x-2)^(-1)}) node[above,yshift=+0.5cm,left]{$\color{cyan} y=\frac{1}{x-2}$};


%\draw[cyan,cap=rect,ultra thick, domain=-7:1.9] plot (\x, {exp{\x}}) node[above, right]{${\color{cyan}y=\exp{x}}$};

%%%%%%%%%%%%%%%%%%%%%%%%%%%%%%%%
%		MARKERINGEN
%%%%%%%%%%%%%%%%%%%%%%%%%%%%%%%%
%verticale lijn
%\draw[-o,line width=4,teal, cap=rect,opacity=0.3] (0,-4) -- (0,0.25) node[right] {};
%\draw[line width=4,teal, cap=rect,opacity=0.3] (0,0) -- (0,4.2) node[right] {bld $f$ = $\mathbb{R}_0$};
%horizontale lijn
%\draw[arrows=-o,line width=4,blue, cap=rect,opacity=0.3] (-7,0) -- (2.25,0) node[right] {};
%\draw[line width=4,blue, cap=rect,opacity=0.3] (2.25,0) -- (7,0) node[below,yshift=-0.3cm] {dom$f$ = $\mathbb{R}  \setminus 2 $};
 
\end{tikzpicture}

\end{figure}

\begin{itemize}
	\item De sinusfunctie kan je voor elke $x$ berekenen. 
	\item De sinusfunctie ligt altijd tussen -1 en 1.
	\item De periode is $2\pi$. 
	\item De sinusfunctie is een oneven functie want $\sin(-x)=-\sin(x)$
\end{itemize}

\item{Cosinusfunctie}

$f(x)=\cos x$

%TODO
%\gewonefiguur{width=7cm}{2_elem_rekenvaardigheden_B/inputs/cos}

\begin{figure}[H]
	\centering
			\tikzsetfigurename{Fig_module_2_1_12_cos}



\begin{tikzpicture}[scale=1,cap=round]

% Styles
\tikzstyle{axes}=[]
\tikzstyle help lines=[color=blue!50,very thin,dotted]


%%%%%%%%%%%%%%%%%%%%%%%%%%%%%%%%
%		GRID
%%%%%%%%%%%%%%%%%%%%%%%%%%%%%%%%

\draw[style=help lines,step=1cm] (-6.9,-1.9) grid (6.9,1.9);



%%%%%%%%%%%%%%%%%%%%%%%%%%%%%%%%
%		ASSENSTELSEL
%%%%%%%%%%%%%%%%%%%%%%%%%%%%%%%%

\draw[->] (-7,0) -- (7,0) node[right] {$x$};
\draw[->] (0,-2) -- (0,2) node[left]{$y$};

%\draw[fill,cyan](1,1)circle [radius=0.025];

%\draw[red,cap=rect, loosely dashed, ultra thick, domain=-2:2] plot (\x, {(\x*\x-1)+0.05}) node[above,yshift=-.7cm, right]{};

%%%%%%%%%%%%%%%%%%%%%%%%%%%%%%%%
%legende
%%%%%%%%%%%%%%%%%%%%%%%%%%%%%%%%
%\tkzDefPoint(0.5,3.5){A}
%\tkzDefPoint(1,3.5){B}
%\tkzLabelPoint[right,xshift=+0.1cm](B){${\color{cyan}f(x)=|x^2-1|}$}
%\tkzDrawSegment[cyan,ultra thick](A,B)

%\tkzDefPoint(0.5,3.2){C}
%\tkzDefPoint(1,3.2){D}
%\tkzLabelPoint[right,xshift=+0.1cm](D){${\color{red}e(x)=x^2-1}$}
%\tkzDrawSegment[red,cap=rect, loosely dashed, ultra thick](C,D)


%%%%%%%%%%%%%%%%%%%%%%%%%%%%%%%%
%getallen op de x-as en lijntjes
%%%%%%%%%%%%%%%%%%%%%%%%%%%%%%%%   
\foreach \x/\xtext in {-6,-5,-4,-3,-2,-1,1,2,3,4,5,6}
	\draw[xshift=\x cm] (0pt,1pt) -- (0pt,0pt) node[below,fill=white]
	{$\xtext$};,3
	
%getallen op de y-as en lijntjes  
%BEGIN LUS
\foreach \y/\ytext in {-1,0,1}
	\draw[yshift=\y cm] (1pt,0pt) -- (0pt,0pt) node[left,fill=white]
	{$\ytext$}; %EINDE LUS



%%%%%%%%%%%%%%%%%%%%%%%%%%%%%%%%
%		GRAFIEKEN
%%%%%%%%%%%%%%%%%%%%%%%%%%%%%%%%


\draw[teal,cap=rect,line width=2, opacity=0.8, domain=-7:7,samples=100] plot (\x, {
	cos(\x r)	% <- plaats het functievoorschrift hier	
}) node[opacity=1,,pos=0,xshift=+6cm,yshift=+1.5cm]{$f_1(x)=\cos{x}$};
%-------------------------------------------
%\draw[red,cap=rect,line width=1, opacity=1, domain=-7:7,samples=100] plot (\x, {
%	sin((\x +pi/2) r)	% <- plaats het functievoorschrift hier	
%}) node[opacity=1,,pos=0,xshift=-6cm,yshift=+1.5cm]{$f_2(x)=\sin{x+\frac{\pi}{2}}$};
%-------------------------------------------



%\draw[cyan,cap=rect,thick, domain=-6:6] plot (\x, \x) node[above, right]{${\color{cyan}y=x}$};v

%\draw[cyan,cap=rect,ultra thick, domain=-6:1.75] plot (\x, {(\x-2)^(-1)}) node[above,right]{};


%\draw[cyan,cap=rect,ultra thick, domain=2.25:6] plot (\x, {(\x-2)^(-1)}) node[above,yshift=+0.5cm,left]{$\color{cyan} y=\frac{1}{x-2}$};


%\draw[cyan,cap=rect,ultra thick, domain=-7:1.9] plot (\x, {exp{\x}}) node[above, right]{${\color{cyan}y=\exp{x}}$};

%%%%%%%%%%%%%%%%%%%%%%%%%%%%%%%%
%		MARKERINGEN
%%%%%%%%%%%%%%%%%%%%%%%%%%%%%%%%
%verticale lijn
%\draw[-o,line width=4,teal, cap=rect,opacity=0.3] (0,-4) -- (0,0.25) node[right] {};
%\draw[line width=4,teal, cap=rect,opacity=0.3] (0,0) -- (0,4.2) node[right] {bld $f$ = $\mathbb{R}_0$};
%horizontale lijn
%\draw[arrows=-o,line width=4,blue, cap=rect,opacity=0.3] (-7,0) -- (2.25,0) node[right] {};
%\draw[line width=4,blue, cap=rect,opacity=0.3] (2.25,0) -- (7,0) node[below,yshift=-0.3cm] {dom$f$ = $\mathbb{R}  \setminus 2 $};
 
\end{tikzpicture}

\end{figure}

\begin{itemize}
	\item De cosinusfunctie kan je voor elke $x$ berekenen. 
	\item De cosinusfunctie ligt altijd tussen -1 en 1.
	\item De periode is $2\pi$. 
	\item De cosinusfunctie is een even functie want $\cos(-x)=\cos(x)$
\end{itemize}

\item{Tangensfunctie}

\begin{equation*}
f(x)=\tan x=\textrm{tg}x=\frac{\sin x}{\cos x}
\end{equation*}

%TODO
%\gewonefiguur{width=5cm}{2_elem_rekenvaardigheden_B/inputs/tan}

\begin{figure}[H]
	\centering
			\tikzsetfigurename{Fig_module_2_1_12_tan}

\begin{tikzpicture}[scale=1,cap=round]

% Styles
\tikzstyle{axes}=[]
\tikzstyle help lines=[color=blue!50,very thin,dotted]


%%%%%%%%%%%%%%%%%%%%%%%%%%%%%%%%
%		GRID
%%%%%%%%%%%%%%%%%%%%%%%%%%%%%%%%

\draw[style=help lines,step=1cm] (-7.9,-6.9) grid (7.9,6.9);



%%%%%%%%%%%%%%%%%%%%%%%%%%%%%%%%
%		ASSENSTELSEL
%%%%%%%%%%%%%%%%%%%%%%%%%%%%%%%%

\draw[->] (-8,0) -- (8,0) node[right] {$x$};
\draw[->] (0,-7) -- (0,7) node[left]{$y$};

%\draw[fill,cyan](1,1)circle [radius=0.025];

%\draw[red,cap=rect, loosely dashed, ultra thick, domain=-2:2] plot (\x, {(\x*\x-1)+0.05}) node[above,yshift=-.7cm, right]{};

%%%%%%%%%%%%%%%%%%%%%%%%%%%%%%%%
%legende
%%%%%%%%%%%%%%%%%%%%%%%%%%%%%%%%
%\tkzDefPoint(0.5,3.5){A}
%\tkzDefPoint(1,3.5){B}
%\tkzLabelPoint[right,xshift=+0.1cm](B){${\color{cyan}f(x)=|x^2-1|}$}
%\tkzDrawSegment[cyan,ultra thick](A,B)

%\tkzDefPoint(0.5,3.2){C}
%\tkzDefPoint(1,3.2){D}
%\tkzLabelPoint[right,xshift=+0.1cm](D){${\color{red}e(x)=x^2-1}$}
%\tkzDrawSegment[red,cap=rect, loosely dashed, ultra thick](C,D)


%%%%%%%%%%%%%%%%%%%%%%%%%%%%%%%%
%getallen op de x-as en lijntjes
%%%%%%%%%%%%%%%%%%%%%%%%%%%%%%%%   
\foreach \x/\xtext in {-6,-5,-4,3,-2,-1,0,1,2,3,4,5,6}
	\draw[xshift=\x cm] (0pt,1pt) -- (0pt,0pt) node[below,fill=white]
	{$\xtext$};,3
	
%getallen op de y-as en lijntjes  
%BEGIN LUS
\foreach \y/\ytext in {-6,-5,-4,-3,-2,-1,1,2,3,4,5,6}
	\draw[yshift=\y cm] (1pt,0pt) -- (0pt,0pt) node[left,fill=white]
	{$\ytext$}; %EINDE LUS



%%%%%%%%%%%%%%%%%%%%%%%%%%%%%%%%
%		GRAFIEKEN
%%%%%%%%%%%%%%%%%%%%%%%%%%%%%%%%


\draw[teal,cap=rect,line width=2, opacity=0.8, domain=3*pi/2+0.15:5*pi/2-0.15,samples=500] plot (\x, {
	tan(\x r)	% <- plaats het functievoorschrift hier	
}) ;




\draw[teal,cap=rect,line width=2, opacity=0.8, domain=pi/2+0.15:3*pi/2-0.15,samples=500] plot (\x, {
	tan(\x r)	% <- plaats het functievoorschrift hier	
}) ;


\draw[teal,cap=rect,line width=2, opacity=0.8, domain=-pi/2+0.15:pi/2-0.15,samples=500] plot (\x, {
	tan(\x r)	% <- plaats het functievoorschrift hier	
}) node[opacity=1,,pos=0,xshift=3cm,yshift=4cm]{$f_1(x)=\tan{x}$};


\draw[teal,cap=rect,line width=2, opacity=0.8, domain=-3*pi/2+0.15:-pi/2-0.15,samples=500] plot (\x, {
	tan(\x r))% <- plaats het functievoorschrift hier	
}) ;



\draw[teal,cap=rect,line width=2, opacity=0.8, domain=-5*pi/2+0.15:-3*pi/2-0.15,samples=500] plot (\x, {
	tan(\x r))% <- plaats het functievoorschrift hier	
}) ;




%-------------------------------------------
%\draw[red,cap=rect,line width=1, opacity=1, domain=-7:7,samples=100] plot (\x, {
%	sin((\x +pi/2) r)	% <- plaats het functievoorschrift hier	
%}) node[opacity=1,,pos=0,xshift=-6cm,yshift=+1.5cm]{$f_2(x)=\sin{x+\frac{\pi}{2}}$};
%-------------------------------------------



%\draw[cyan,cap=rect,thick, domain=-6:6] plot (\x, \x) node[above, right]{${\color{cyan}y=x}$};v

%\draw[cyan,cap=rect,ultra thick, domain=-6:1.75] plot (\x, {(\x-2)^(-1)}) node[above,right]{};


%\draw[cyan,cap=rect,ultra thick, domain=2.25:6] plot (\x, {(\x-2)^(-1)}) node[above,yshift=+0.5cm,left]{$\color{cyan} y=\frac{1}{x-2}$};


%\draw[cyan,cap=rect,ultra thick, domain=-7:1.9] plot (\x, {exp{\x}}) node[above, right]{${\color{cyan}y=\exp{x}}$};

%%%%%%%%%%%%%%%%%%%%%%%%%%%%%%%%
%		MARKERINGEN
%%%%%%%%%%%%%%%%%%%%%%%%%%%%%%%%
%verticale lijn
%\draw[-o,line width=4,teal, cap=rect,opacity=0.3] (0,-4) -- (0,0.25) node[right] {};
%\draw[line width=4,teal, cap=rect,opacity=0.3] (0,0) -- (0,4.2) node[right] {bld $f$ = $\mathbb{R}_0$};
%horizontale lijn
%\draw[arrows=-o,line width=4,blue, cap=rect,opacity=0.3] (-7,0) -- (2.25,0) node[right] {};
%\draw[line width=4,blue, cap=rect,opacity=0.3] (2.25,0) -- (7,0) node[below,yshift=-0.3cm] {dom$f$ = $\mathbb{R}  \setminus 2 $};
 
\end{tikzpicture}

\end{figure}


\begin{itemize}
	\item De tangensfunctie kan je niet berekenen voor $\frac{\pi}{2}$, $-\frac{\pi}{2}$,
	$\frac{\pi}{2}+\pi$, $\ldots$ Dus kortweg als $x=\frac{\pi}{2}+k\pi$
	met $k\in\mathbb{Z}$. In deze punten zou de noemer immers nul worden;
	hier heeft de tangens verticale asymptoten.
	\item De tangensfunctie kan oneindig groot $(+\infty)$ en oneindig klein
	$(-\infty)$ worden.
	\item De periode is $\pi$.
	\item De tangensfunctie is een oneven functie want $\tan(-x)=-\tan(x)$
\end{itemize}

\item{Cotangensfunctie}

\begin{equation*}
f(x)=\cot x=\textrm{cotg}x=\frac{\cos x}{\sin x}
\end{equation*}




\begin{figure}[H]
	\centering
			\tikzsetfigurename{Fig_module_2_1_12_cot}


\begin{tikzpicture}[scale=1,cap=round]

% Styles
\tikzstyle{axes}=[]
\tikzstyle help lines=[color=blue!50,very thin,dotted]


%%%%%%%%%%%%%%%%%%%%%%%%%%%%%%%%
%		GRID
%%%%%%%%%%%%%%%%%%%%%%%%%%%%%%%%

\draw[style=help lines,step=1cm] (-8.9,-6.9) grid (8.9,6.9);



%%%%%%%%%%%%%%%%%%%%%%%%%%%%%%%%
%		ASSENSTELSEL
%%%%%%%%%%%%%%%%%%%%%%%%%%%%%%%%

\draw[->] (-9,0) -- (9,0) node[right] {$x$};
\draw[->] (0,-7) -- (0,7) node[left]{$y$};

%\draw[fill,cyan](1,1)circle [radius=0.025];

%\draw[red,cap=rect, loosely dashed, ultra thick, domain=-2:2] plot (\x, {(\x*\x-1)+0.05}) node[above,yshift=-.7cm, right]{};

%%%%%%%%%%%%%%%%%%%%%%%%%%%%%%%%
%legende
%%%%%%%%%%%%%%%%%%%%%%%%%%%%%%%%
%\tkzDefPoint(0.5,3.5){A}
%\tkzDefPoint(1,3.5){B}
%\tkzLabelPoint[right,xshift=+0.1cm](B){${\color{cyan}f(x)=|x^2-1|}$}
%\tkzDrawSegment[cyan,ultra thick](A,B)

%\tkzDefPoint(0.5,3.2){C}
%\tkzDefPoint(1,3.2){D}
%\tkzLabelPoint[right,xshift=+0.1cm](D){${\color{red}e(x)=x^2-1}$}
%\tkzDrawSegment[red,cap=rect, loosely dashed, ultra thick](C,D)


%%%%%%%%%%%%%%%%%%%%%%%%%%%%%%%%
%getallen op de x-as en lijntjes
%%%%%%%%%%%%%%%%%%%%%%%%%%%%%%%%   
\foreach \x/\xtext in {-8,-7,-6,-5,-4,-3,-2,-1,0,1,2,3,4,5,6,7,8}
\draw[xshift=\x cm] (0pt,1pt) -- (0pt,0pt) node[below,fill=white]
{$\xtext$};,3

%getallen op de y-as en lijntjes  
%BEGIN LUS
\foreach \y/\ytext in {-6,-5,-4,-3,-2,-1,1,2,3,4,5,6}
\draw[yshift=\y cm] (1pt,0pt) -- (0pt,0pt) node[left,fill=white]
{$\ytext$}; %EINDE LUS



%%%%%%%%%%%%%%%%%%%%%%%%%%%%%%%%
%		GRAFIEKEN
%%%%%%%%%%%%%%%%%%%%%%%%%%%%%%%%


\draw[teal,cap=rect,line width=2, opacity=0.8, domain=2*pi+0.15:3*pi-0.15,samples=500] plot (\x, {
	cot(\x r)	% <- plaats het functievoorschrift hier	
}) ;

\draw[teal,cap=rect,line width=2, opacity=0.8, domain=pi+0.15:2*pi-0.15,samples=500] plot (\x, {
	cot(\x r)	% <- plaats het functievoorschrift hier	
}) ;


\draw[teal,cap=rect,line width=2, opacity=0.8, domain=0.15:pi-0.15,samples=500] plot (\x, {
	cot(\x r)	% <- plaats het functievoorschrift hier	
}) node[opacity=1,,pos=0,xshift=2cm,yshift=4cm]{$f_1(x)=\cot{x}$};

\draw[teal,cap=rect,line width=2, opacity=0.8, domain=-pi+0.15:-0.15,samples=500] plot (\x, {
	cot(\x r))% <- plaats het functievoorschrift hier	
}) ;





\draw[teal,cap=rect,line width=2, opacity=0.8, domain=-2*pi+0.15:-pi-0.15,samples=500] plot (\x, {
	cot(\x r))% <- plaats het functievoorschrift hier	
}) ;



\draw[teal,cap=rect,line width=2, opacity=0.8, domain=-3*pi+0.15:-2*pi-0.15,samples=500] plot (\x, {
	cot(\x r))% <- plaats het functievoorschrift hier	
}) ;




%-------------------------------------------
%\draw[red,cap=rect,line width=1, opacity=1, domain=-7:7,samples=100] plot (\x, {
%	sin((\x +pi/2) r)	% <- plaats het functievoorschrift hier	
%}) node[opacity=1,,pos=0,xshift=-6cm,yshift=+1.5cm]{$f_2(x)=\sin{x+\frac{\pi}{2}}$};
%-------------------------------------------



%\draw[cyan,cap=rect,thick, domain=-6:6] plot (\x, \x) node[above, right]{${\color{cyan}y=x}$};v

%\draw[cyan,cap=rect,ultra thick, domain=-6:1.75] plot (\x, {(\x-2)^(-1)}) node[above,right]{};


%\draw[cyan,cap=rect,ultra thick, domain=2.25:6] plot (\x, {(\x-2)^(-1)}) node[above,yshift=+0.5cm,left]{$\color{cyan} y=\frac{1}{x-2}$};


%\draw[cyan,cap=rect,ultra thick, domain=-7:1.9] plot (\x, {exp{\x}}) node[above, right]{${\color{cyan}y=\exp{x}}$};

%%%%%%%%%%%%%%%%%%%%%%%%%%%%%%%%
%		MARKERINGEN
%%%%%%%%%%%%%%%%%%%%%%%%%%%%%%%%
%verticale lijn
%\draw[-o,line width=4,teal, cap=rect,opacity=0.3] (0,-4) -- (0,0.25) node[right] {};
%\draw[line width=4,teal, cap=rect,opacity=0.3] (0,0) -- (0,4.2) node[right] {bld $f$ = $\mathbb{R}_0$};
%horizontale lijn
%\draw[arrows=-o,line width=4,blue, cap=rect,opacity=0.3] (-7,0) -- (2.25,0) node[right] {};
%\draw[line width=4,blue, cap=rect,opacity=0.3] (2.25,0) -- (7,0) node[below,yshift=-0.3cm] {dom$f$ = $\mathbb{R}  \setminus 2 $};

\end{tikzpicture}

\end{figure}


\begin{itemize}
	\item De cotangensfunctie kan je niet berekenen voor $0$, $\pi$, $-\pi$,
	$2\pi$ $\ldots$ Dus kortweg als $x=k\pi$ met $k\in\mathbb{Z}$.
	In deze punten zou de noemer immers nul worden; hier heeft de cotangens
	verticale asymptoten.
	\item De cotangensfunctie kan oneindig groot $(+\infty)$ en oneindig klein
	$(-\infty)$ worden.
	\item De periode is $\pi$.
	\item De cotangensfunctie is een oneven functie want $\cot(-x)=-\cot(x)$
\end{itemize}

\end{itemize}

