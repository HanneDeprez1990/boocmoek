
		\tikzsetfigurename{Fig_module_2_2_1_Continuvb4}
\begin{center}
	\begin{tikzpicture}[scale=0.7,cap=round]
	
	% Styles
	\tikzstyle{axes}=[]
	\tikzstyle help lines=[color=blue!50,very thin,dotted]
	
	% grid
	\draw[style=help lines,step=1cm] (-3.9,-1.9) grid (5.9,7.9);
	
	\draw[->] (-4,0) -- (6,0) node[right] {$x$};
	\draw[->] (0,-2) -- (0,8) node[above] {$y$};
	
	%\draw[fill,cyan](1,1)circle [radius=0.025];
	%FUNCTIEVOORSCHRIFTEN
	
	

	\draw[dashed] (2,-1.5)--(2,7.5);
	
	
	%getallen op de x-as en lijntjes   
	\foreach \x/\xtext in {-3,-2,-1,0,1,2,3,4,5}
	\draw[xshift=\x cm] (0pt,1pt) -- (0pt,0pt) node[below,fill=white]
	{$\xtext$};
	
	%getallen op de y-as en lijntjes  
	%BEGIN LUS
	\foreach \y/\ytext in {-1,1,2,3,4,5,6,7}
	\draw[yshift=\y cm] (1pt,0pt) -- (0pt,0pt) node[left,fill=white]
	{$\ytext$}; %EINDE LUS
	
	
%	\draw[teal] (1,5) circle[radius=0.1] node[right]{};
	
	\draw[teal,cap=rect,line width=1, opacity=1, domain=-4:1.63] plot (\x, {
 pow(\x-2,-2 )% <- plaats het functievoorschrift hier
	}) node[right,opacity=1]{};
	
		\draw[teal,cap=rect,line width=1, opacity=1, domain=2.37:4] plot (\x, {
		pow(\x-2,-2 )% <- plaats het functievoorschrift hier
	}) node[right,opacity=1]{$f(x)=\frac{1}{(x-2)^2}$};
	
	%legende
	
	
	
	
	
	
	\end{tikzpicture}
\end{center}

