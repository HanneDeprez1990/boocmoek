\tikzsetfigurename{Fig_module_2_1_9_machtswortel_2}

\begin{center}
\begin{tikzpicture}[scale=0.7,cap=round]

% Styles
\tikzstyle{axes}=[]
\tikzstyle help lines=[color=blue!50,very thin,dotted]

% grid
\draw[style=help lines,step=1cm] (-7.9,-1.9) grid (7.9,7.9);

\draw[->] (-8,0) -- (8,0) node[right] {$x$};
\draw[->] (0,-2) -- (0,8) node[above] {$y$};

%\draw[fill,cyan](1,1)circle [radius=0.025];
%FUNCTIEVOORSCHRIFTEN


\draw[blue,cap=rect,line width=1, opacity=1, domain=-7:-2,samples=100] plot (\x, {
	pow(pow(\x,2)-4,1/2)	% <- plaats het functievoorschrift hier	
}) node[opacity=1,above,left]{};
%-------------------------------------------




\draw[blue,cap=rect,line width=1, opacity=1, domain=2:7,samples=100] plot (\x, {
	pow(pow(\x,2)-4,1/2)	% <- plaats het functievoorschrift hier	
}) node[opacity=1,above]{$f(x)=\sqrt{x^2-4}$};
%-------------------------------------------

%legende



%getallen op de x-as en lijntjes   
\foreach \x/\xtext in {-7,-6,-5,-4,-3,-2,-1,0,1,2,3,4,5,6,7}
	\draw[xshift=\x cm] (0pt,1pt) -- (0pt,0pt) node[below,fill=white]
	{$\xtext$};,3
	
%getallen op de y-as en lijntjes  
%BEGIN LUS
\foreach \y/\ytext in {-1,0,1,2,3,4,5,6,7}
	\draw[yshift=\y cm] (1pt,0pt) -- (0pt,0pt) node[left,fill=white]
	{$\ytext$}; %EINDE LUS



\end{tikzpicture}
\end{center}

