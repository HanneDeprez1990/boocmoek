
%\DeclareMathOperator{\tri}{tri} \DeclareMathOperator{\rect}{rect}
%\DeclareMathOperator{\sgn}{sgn} \DeclareMathOperator{\ramp}{ramp}
%\DeclareMathOperator{\sinc}{sinc}

\subsection{Logaritmische functies}

%\begin{itemize}
%\item Wat is een logaritmische functie?
%\item Hoe ziet het functievoorschrift van een logaritmische functie eruit?
%\item Welke rekenregels voor logaritmen zijn er?
%\item Hoe lossen we logaritmische vergelijkingen op?
%\item Hoe kan je een logaritmische functie grafisch voorstellen?
%\item Hoe bepaal je het tekenverloop van een logaritmische functie? 
%\end{itemize}

\subsubsection{Definities}

\noindent De logaritmische functie wordt gedefinieerd als de inverse
van de exponenti\"ele functie.

\noindent De logaritmische functie is van de vorm:

\noindent \vspace{0.2cm}


\fbox{%
\begin{minipage}[c]{10cm}%
${\displaystyle y=\log_{a}\left(x\right)=\sideset{^{a}}{\left(x\right)}\log\quad\Longleftrightarrow\quad a^{y}=x}$
met ${\displaystyle a\in\mathbb{R}_{0}^{+}\setminus\left\{ 1\right\} }$
en ${\displaystyle x\in\mathbb{R}_{0}^{+}}$%
\end{minipage}}\vspace{0.5cm}


\noindent $a$ noemen we het \textbf{grondtal} , en moet strikt positief
en verschillend van 1 zijn.

\noindent $x$ is een strikt positief re\"eel getal.

\medskip{}


\noindent Om de waarde van y te vinden, stel je jezelf de vraag: ``tot
welke macht moet ik het grondtal $a$ verheffen om $x$ uit te komen?''\medskip{}


\begin{figure}[h]
linker figuur ${\displaystyle y=\log_{10}\left(x\right)=\log(x)}$
en $y=10^{x}$ 

rechter figuur: ${\displaystyle y=\log_{e}\left(x\right)=\ln(x)}$
en $y=e^{x}$ 

\includegraphics[scale=0.5]{\string"2_elem_rekenvaardigheden_B/inputs/log en exp\string".eps}
\end{figure}


\medskip{}


\noindent ${\displaystyle \log_{a}(x)}$ is de inverse functie van
de functie ${\displaystyle a^{x}}$; beide functies zijn elkaar spiegelbeeld
t.o.v. de eerste bissectrice ( $y=x$ ).

\noindent ${\displaystyle \ln(x)}$ is de inverse functie van de functie
${\displaystyle e^{x}}$; beide functies zijn elkaar spiegelbeeld
t.o.v. de eerste bissectrice ($y=x$ ).\medskip{}


Voorbeelden: 

${\displaystyle \log_{10}\left(100\right)}=2$ (want ${\displaystyle 10^{2}=100}$)

${\displaystyle \log_{2}8=3}$

${\displaystyle \log_{4}16=2}$

${\displaystyle \log_{3}9=2}$

${\displaystyle \log_{3}\sqrt{3}=\log_{3}3^{\frac{1}{2}}=\frac{1}{2}}$

${\displaystyle \log_{5}\frac{1}{5}=\log_{5}5^{-1}=-1}$


\subsubsection{Bijzondere gevallen}
\begin{itemize}
\item De \textbf{tiendelige of Briggse logaritme} is de logaritme met grondtal
10. In de notatie wordt vaak het grondtal 10 weggelaten: ${\displaystyle y=\log_{10}\left(x\right)}=\log\left(x\right)$
\item De \textbf{natuurlijke of Neperiaanse logaritme} is de logaritme met
grondtal e (e=2,718281828) en wordt genoteerd als ${\displaystyle y=\log_{e}}(x)$
. Ook hier wordt het grondtal e weggelaten, en gebruikt men de specifieke
notatie: ${\displaystyle y=\ln(x)}$
\item ${\displaystyle \log_{a}\left(a\right)=1}$ (want ${\displaystyle a^{1}=a}$)
\item ${\displaystyle \log_{a}\left(1\right)=0}$ (want ${\displaystyle a^{0}=1}$)
\item ${\displaystyle \log_{a}\left(0\right)}$ en ${\displaystyle \log_{a}\left(-...\right)}$
bestaan niet!
\end{itemize}

\subsubsection{Rekenregels}

\begin{tabular}{|l|}
\hline 
logaritmen\tabularnewline
\hline 
\hline 
${\displaystyle \log\left(x.y\right)=\log x+\log y}$\tabularnewline
\hline 
${\displaystyle \log\left(\frac{x}{y}\right)=\log x-\log y}$\tabularnewline
\hline 
${\displaystyle \log\left(x^{n}\right)=n\log x}$\tabularnewline
\hline 
${\displaystyle \log\left(\sqrt[n]{x}\right)=\frac{1}{n}\log x}$\tabularnewline
\hline 
${\displaystyle \log_{a}\left(b\right)=\frac{1}{\log_{b}\left(a\right)}}$\tabularnewline
\hline 
\end{tabular}\medskip{}


Opmerkingen:
\begin{itemize}
\item voor de eenvoud hebben we enkele keren het grondtal weggelaten. 
\item laat je niet verleiden, er bestaat geen eenvoudige formule voor ${\displaystyle \log\left(x+y\right)=\ldots}$
\item ${\displaystyle \log\left(x^{n}\right)=\log\left(x.x.x.\:\ldots\:x\right)=\log x+\log x+\:\ldots\:+\log x=n\log x}$
Toch niet zo moeilijk h/'e!?
\item om een logaritme met grondtal a om te zetten naar een logaritme met
grondtal c maken we gebruiken van de volgende eigenschap: ${\displaystyle \log_{a}x=\frac{\log_{c}x}{\log_{c}a}}$
Hiermee vind je trouwens ook gemakkelijk het laatste rekenregeltje.
Stel, je moet uitrekenen hoeveel ${\displaystyle \log_{2}8}$ is door
over te gaan op een ander grondtal (bijvoorbeeld 10, wat immers op
je rekentoestel staat). Dan is ${\displaystyle \log_{2}8=\frac{\log_{10}8}{\log_{10}2}=\frac{\log8}{\log2}=\frac{0.9031}{0.3010}=3.000}$
(en dat hadden we natuurlijk ook al lang uit het hoofd uitgerekend
via een andere eigenschap).
\item soms is het nodig om de variabele $x$ of een getal, op een andere
manier te schrijven. Zowel de notatie ${\displaystyle x=10^{log_{10}(x)}}$
als ${\displaystyle x=\log_{10}10^{x}}$ worden regelmatig gebruikt.
Hetzelfde geldt voor ${\displaystyle x=e^{log_{e}(x)}=e^{\ln x}}$
en ${\displaystyle x=\log_{e}e^{x}=\ln e^{x}}$ . Het idee hierachter
is dat de logaritmische en exponenti\"ele functies elkaars inverse zijn,
waardoor ze elkaar opheffen als ze na elkaar worden toegepast op $x$.
Maar let wel op met bijvoorbeeld ${\displaystyle \sqrt{x^{2}}\neq x}$
maar wel ${\displaystyle \sqrt{x^{2}}=\left|x\right|}$(zie ook Module
1 - Absolute waarde).
\end{itemize}

\subsubsection{Oplossen van logaritmische vergelijkingen}

\begin{itemize}
\item{Algemeen}

In sommige opgaves kom je logaritmische vergelijkingen tegen. Dit
zijn vergelijkingen waar de onbekende voorkomt in een logaritmische
functie. Om logaritmische vergelijkingen vlot te kunnen oplossen,
maak je best gebruik van onderstaand stappenplan:\medskip{}


Stap 1: Noteer de vergelijking in haar standaardvorm: ${\displaystyle \log_{a}\left(f(x)\right)=c}$
of ${\displaystyle \log_{a}\left(f(x)\right)=\log_{a}\left(g(x)\right)}$
.

Stap 2: Laat op beide leden van de vergelijking een geschikte exponenti\"ele
functie inwerken.

Stap 3: Controleer of de bekomen oplossingen voldoen aan de voorwaarden:
\begin{itemize}
	\item het grondtal moet strikt positief zijn, en verschillend van 1 zijn,
	en
	\item de ${\displaystyle \log_{a}\left(f(x)\right)}$ bestaat enkel als
	${\displaystyle f(x)>0}$ . 
\end{itemize}
Stap 4: Stop de uitkomst in de opgave en controleer.


\item{Voorbeeld1: ${\displaystyle \log_{a}\left(f(x)\right)=\log_{a}\left(g(x)\right)}$ }

Los op: ${\displaystyle \log_{3}(x+4)+\log_{3}(x-2)=2\log_{3}x}$

\[
{\displaystyle \begin{split}{\displaystyle \log_{3}(x+4)+\log_{3}(x-2)=2\log_{3}x} &  &  &  & \text{ (stap 1)}\\
	\Longleftrightarrow & \log_{3}\left[\left(x+4\right)\left(x-2\right)\right] & = & \log_{3}x^{2}\\
	\Longleftrightarrow & 3^{\log_{3}\left[\left(x+4\right)\left(x-2\right)\right]} & = & 3^{\log_{3}x^{2}} & \text{ (stap 2)}\\
	\Longleftrightarrow & \left(x+4\right)\left(x-2\right) & = & x^{2}\\
	\Longleftrightarrow & x^{2}-2x+4x-8 & = & x^{2}\\
	\Longleftrightarrow & 2x-8 & = & 0\\
	\Longleftrightarrow & x & = & 4
	\end{split}
}
\]


In stap 3 controleren we de voorwaarden voor $x=4$:

1) Voor ${\displaystyle \log_{3}(x+4)}$ moet gelden: $x+4>0$ $\Longleftrightarrow$
$x>-4$? ok.

2) Voor ${\displaystyle \log_{3}(x-2)}$ moet gelden: $x-2>0$ $\Longleftrightarrow$
$x>2$? ok.

3) Voor ${\displaystyle 2\log_{3}x}$ moet gelden: $x>0$ ? ok.

Besluit: de oplossing $x=4$ voldoet aan de voorwaarden, en is dus
een oplossing van de vergelijking.

Stap 4: controleer je oplossing: ${\displaystyle \log_{3}(4+4)+\log_{3}(4-2)=2\log_{3}4}$?
ok.

Opmerking: stap 2 kan je ook in gedachten doen, m.a.w. hoef je niet
op te schrijven.


\item{Voorbeeld2: ${\displaystyle \log_{a}\left(f(x)\right)=c}$ }

Los op: ${\displaystyle \log_{(x+3)}(x+5)=2}$

\[
{\displaystyle \begin{split}{\displaystyle \log_{(x+3)}(x+5)=2} &  &  &  & \text{ (stap 1)}\\
	\Longleftrightarrow & (x+3)^{2} & = & x+5 & \textrm{definitie:}\:{\displaystyle y=\log_{a}x\Leftrightarrow a^{y}=x}\\
	\Longleftrightarrow & x^{2}+6x+9 & = & x+5 & \text{}\\
	\Longleftrightarrow & x^{2}+5x+4 & = & 0
	\end{split}
}
\]


${\displaystyle x_{1\textrm{en}2}=\frac{-5\pm\sqrt{5^{2}-4.1.4}}{2.1}=\frac{-5\pm3}{2}\Longleftrightarrow x_{1}=-1}$
en ${\displaystyle x_{2}=-4}$

In stap 3 controleren we de voorwaarden voor $x_{1}$ en $x_{2}$:

1) Is $x+3>0$ $\Longleftrightarrow$ $x>-3$? 

2) Is $x+5>0$ $\Longleftrightarrow$ $x>-5$? 

Besluit: ${\displaystyle x_{1}=-1}$ voldoet aan de voorwaarden, maar
${\displaystyle x_{2}=-4}$ niet. Dus enkel ${\displaystyle x_{1}=-1}$
is een oplossing van de vergelijking.

Stap 4: controleer je oplossing: ${\displaystyle \log_{(-1+3)}(-1+5)=2}$?
ok. Maar ${\displaystyle \log_{(-4+3)}(-4+5)=2}$ gaat niet.


\item{Voorbeeld3: het grondtal is onbekend}

Los op: ${\displaystyle \log_{a}250=3+\log_{a}2}$

\[
{\displaystyle \begin{split}{\displaystyle \log_{a}250=3+\log_{a}2}\\
	\Longleftrightarrow & {\displaystyle \log_{a}250-\log_{a}2} & = & 3\\
	\Longleftrightarrow & {\displaystyle \log_{a}\left(\frac{250}{2}\right)} & = & 3 & \text{}\\
	\Longleftrightarrow & {\displaystyle \log_{a}125} & = & 3\\
	\Longleftrightarrow & a^{3} & = & 125\\
	\Longleftrightarrow & a & = & \sqrt[3]{125}=5
	\end{split}
}
\]


Alternatieve aanpak:

\[
{\displaystyle \begin{split}{\displaystyle \log_{a}250=3+\log_{a}2}\\
	\Longleftrightarrow & {\displaystyle \log_{a}250} & = & 3\log_{a}a+\log_{a}2 & (\textrm{infeite}\:\textrm{is}\:\log_{a}a=1)\\
	\Longleftrightarrow & {\displaystyle \log_{a}250} & = & \log_{a}a^{3}+\log_{a}2 & \text{}\\
	\Longleftrightarrow & {\displaystyle \log_{a}250} & = & \log_{a}\left(a^{3}.2\right)\\
	\Longleftrightarrow & 250 & = & 2a^{3}\\
	\Longleftrightarrow & a & = & \sqrt[3]{\frac{250}{2}}=5
	\end{split}
}
\]



\item{Voorbeeld4: rekenen met zeer grote en kleine getallen}

Uit hoeveel cijfers bestaat het getal ${\displaystyle 1995^{1995}}$
?

Stel ${\displaystyle x=1995^{1995}}$ 

Dan is: 
\[
{\displaystyle \begin{split}\log_{10}x & = & \log_{10}\left(1995^{1995}\right)\\
	& = & 1995.\log_{10}\left(1995\right)\\
	& = & 6583,386\ldots &  & \text{}
	\end{split}
}
\]


\[
{\displaystyle 6583<\log_{10}x<6584}
\]


\[
{\displaystyle \log_{10}10^{6583}<\log_{10}x<\log_{10}10^{6584}}
\]


\[
{\displaystyle 10^{6583}<x<10^{6584}}
\]


Besluit: $x$ is een geheel getal dat bestaat uit $6584$ cijfers

\end{itemize}


\subsubsection{De logaritmische functie}

\begin{itemize}
\item{Algemeen}

\noindent \textbf{Grafische voorstelling}
\begin{itemize}
	\item Het domein van de logaritmische functie is $\mathbb{R}_{0}^{+}$,
	dus alle strikt positieve getallen, de grafiek ligt rechts van de
	Y-as.
	\item Het beeld van de logaritmische functie is ${\displaystyle \mathbb{R}}$.
	\item De punten $(1,0)$ en $(a,1)$ behoren steeds tot de logaritmische
	functie ${\displaystyle f(x)=\log_{a}x}$.
	\item Als het grondtal $a>1$ is het een stijgende functie.
	\item Als het grondtal $0<a<1$ is het een dalende functie.
	\item De grafieken ${\displaystyle y=\log_{a}x}$ en ${\displaystyle y=\log_{\frac{1}{a}}x}$
	zijn elkaars spiegelbeeld t.o.v. de X-as.
\end{itemize}
\begin{figure}
	\includegraphics[scale=0.5]{\string"2_elem_rekenvaardigheden_B/inputs/de logaritmische functie\string".eps}
\end{figure}


\noindent \textbf{Nulpunten}
\begin{itemize}
	\item De logaritmische functie heeft 1 nulpunt: het punt (1,0) is het enige
	snijpunt met de X-as. 
	\item Er zijn geen snijpunten met de Y-as. De Y-as is de verticale asymptoot.
\end{itemize}
\noindent \textbf{Tekenverloop}

\begin{tabular}{|c||c|c|c|c|c|}
	\hline 
	$x$ & 0 &  & 1 &  & ${\displaystyle \longrightarrow\mathbb{R}}$\tabularnewline
	\hline 
	\hline 
	${\displaystyle \log_{a}x}$ met $0<a<1$ & $/$ & + & 0 & - & \multicolumn{1}{c|}{}\tabularnewline
	\hline 
	${\displaystyle \log_{a}x}$ met $a>1$ & $/$ & - & 0 & + & \multicolumn{1}{c|}{}\tabularnewline
	\hline 
\end{tabular}


\item{Voorbeeld: ${\displaystyle y=\log_{2}x}$}

\textbf{Grafische voorstelling}: (stap1) het grondtal is 2, en $2>0$,
dus krijgen we een stijgende functie. Het punt $(a,1)$ is hier dus
$(2,1)$ en behoort tot de functie ${\displaystyle y=\log_{2}x}$.\medskip{}


\noindent \textbf{Nulpunten}: (stap2) er is altijd /'e/'en nulpunt; de
X-as wordt altijd gesneden in het punt $(1,0)$. De Y-as wordt nooit
gesneden. De Y-as is de verticale asymptoot.\medskip{}


\noindent \textbf{Tekenverloop}: (stap3) %
\begin{tabular}{|c||c|c|c|c|c|}
	\hline 
	$x$ & 0 &  & 1 &  & ${\displaystyle \longrightarrow\mathbb{R}}$\tabularnewline
	\hline 
	\hline 
	${\displaystyle \log_{2}x}$ met $a=2>1$ & $/$ & - & 0 & + & \multicolumn{1}{c|}{}\tabularnewline
	\hline 
\end{tabular}

\noindent \textbf{Grafiek}: (stap4) 
\begin{figure}
	\includegraphics[scale=0.5]{\string"2_elem_rekenvaardigheden_B/inputs/voorbeeld logaritmische functie\string".eps}
\end{figure}
\end{itemize}

