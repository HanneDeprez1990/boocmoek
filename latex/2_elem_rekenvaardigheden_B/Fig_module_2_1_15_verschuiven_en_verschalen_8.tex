
%TODO polynoombenadering uitrekenen > zie cursus Algebra

\begin{tikzpicture}[scale=1,cap=round]

% Styles
\tikzstyle{axes}=[]
\tikzstyle help lines=[color=blue!50,very thin,dotted]


%%%%%%%%%%%%%%%%%%%%%%%%%%%%%%%%
%		GRID
%%%%%%%%%%%%%%%%%%%%%%%%%%%%%%%%

\draw[style=help lines,step=1cm] (-6.9,-3.9) grid (6.9,3.9);



%%%%%%%%%%%%%%%%%%%%%%%%%%%%%%%%
%		ASSENSTELSEL
%%%%%%%%%%%%%%%%%%%%%%%%%%%%%%%%

\draw[->] (-7,0) -- (7,0) node[right] {$x$};
\draw[->] (0,-4) -- (0,4) node[left]{$y$};

%\draw[fill,cyan](1,1)circle [radius=0.025];

%\draw[red,cap=rect, loosely dashed, ultra thick, domain=-2:2] plot (\x, {(\x*\x-1)+0.05}) node[above,yshift=-.7cm, right]{};

%%%%%%%%%%%%%%%%%%%%%%%%%%%%%%%%
%legende
%%%%%%%%%%%%%%%%%%%%%%%%%%%%%%%%
%\tkzDefPoint(0.5,3.5){A}
%\tkzDefPoint(1,3.5){B}
%\tkzLabelPoint[right,xshift=+0.1cm](B){${\color{cyan}f(x)=|x^2-1|}$}
%\tkzDrawSegment[cyan,ultra thick](A,B)

%\tkzDefPoint(0.5,3.2){C}
%\tkzDefPoint(1,3.2){D}
%\tkzLabelPoint[right,xshift=+0.1cm](D){${\color{red}e(x)=x^2-1}$}
%\tkzDrawSegment[red,cap=rect, loosely dashed, ultra thick](C,D)


%%%%%%%%%%%%%%%%%%%%%%%%%%%%%%%%
%getallen op de x-as en lijntjes
%%%%%%%%%%%%%%%%%%%%%%%%%%%%%%%%   
\foreach \x/\xtext in {-6,-5,-4,-3,-2,-1,1,2,3,4,5,6}
	\draw[xshift=\x cm] (0pt,1pt) -- (0pt,0pt) node[below,fill=white]
	{$\xtext$};,3
	
%getallen op de y-as en lijntjes  
%BEGIN LUS
\foreach \y/\ytext in {-3,-2,-1,1,2,3}
	\draw[yshift=\y cm] (1pt,0pt) -- (0pt,0pt) node[left,fill=white]
	{$\ytext$}; %EINDE LUS



%%%%%%%%%%%%%%%%%%%%%%%%%%%%%%%%
%		GRAFIEKEN
%%%%%%%%%%%%%%%%%%%%%%%%%%%%%%%%


\draw[teal,cap=rect,line width=2, opacity=0.8, domain=-7:7,samples=100] plot (\x, {
	sin(\x r)	% <- plaats het functievoorschrift hier	
}) node[opacity=1,,pos=0,xshift=+6cm,yshift=-1.5cm]{$f_1(x)=\sin{x}$};
%-------------------------------------------
\draw[red,cap=rect,line width=1, opacity=1, domain=-7:7,samples=100] plot (\x, {
	3*(sin((\x r))	% <- plaats het functievoorschrift hier	
}) node[opacity=1,,pos=0,xshift=-5cm,yshift=+1.5cm]{$f_2(x)=3\sin{x}$};
%-------------------------------------------



%\draw[cyan,cap=rect,thick, domain=-6:6] plot (\x, \x) node[above, right]{${\color{cyan}y=x}$};v

%\draw[cyan,cap=rect,ultra thick, domain=-6:1.75] plot (\x, {(\x-2)^(-1)}) node[above,right]{};


%\draw[cyan,cap=rect,ultra thick, domain=2.25:6] plot (\x, {(\x-2)^(-1)}) node[above,yshift=+0.5cm,left]{$\color{cyan} y=\frac{1}{x-2}$};


%\draw[cyan,cap=rect,ultra thick, domain=-7:1.9] plot (\x, {exp{\x}}) node[above, right]{${\color{cyan}y=\exp{x}}$};

%%%%%%%%%%%%%%%%%%%%%%%%%%%%%%%%
%		MARKERINGEN
%%%%%%%%%%%%%%%%%%%%%%%%%%%%%%%%
%verticale lijn
%\draw[-o,line width=4,teal, cap=rect,opacity=0.3] (0,-4) -- (0,0.25) node[right] {};
%\draw[line width=4,teal, cap=rect,opacity=0.3] (0,0) -- (0,4.2) node[right] {bld $f$ = $\mathbb{R}_0$};
%horizontale lijn
%\draw[arrows=-o,line width=4,blue, cap=rect,opacity=0.3] (-7,0) -- (2.25,0) node[right] {};
%\draw[line width=4,blue, cap=rect,opacity=0.3] (2.25,0) -- (7,0) node[below,yshift=-0.3cm] {dom$f$ = $\mathbb{R}  \setminus 2 $};
 
\end{tikzpicture}
