\tikzsetfigurename{Fig_module_2_1_8_vb_rat1}
\begin{center}
\begin{tikzpicture}[xscale=1,yscale=0.5,cap=round]

% Styles
\tikzstyle{axes}=[]
\tikzstyle help lines=[color=blue!50,very thin,dotted]

% grid
\draw[style=help lines,step=1cm] (-7.9,-2.9) grid (7.9,9.9);

\draw[->] (-8,0) -- (8,0) node[right] {$x$};
\draw[->] (0,-3) -- (0,10) node[above] {$y$};

\draw[-,thin] (-1.41,-3)--(-1.41,10);

\draw[-,thin] (1.41,-3)--(1.41,10);


\draw[-,thin] (-7,3)--(7,3);
%\draw[fill,cyan](1,1)circle [radius=0.025];



%getallen op de x-as en lijntjes   
\foreach \x/\xtext in {-7,-6,-5,-4,-3,-2,-1,1,2,3,4,5,6,7}
\draw[xshift=\x cm] (0pt,1pt) -- (0pt,0pt) node[below,fill=white]
{$\xtext$};,3

%getallen op de y-as en lijntjes  
%BEGIN LUS
\foreach \y/\ytext in {-1,0,1,2,3,4,5,6,7,8,9}
\draw[yshift=\y cm] (1pt,0pt) -- (0pt,0pt) node[left,fill=white]
{$\ytext$}; %EINDE LUS



%FUNCTIEVOORSCHRIFTEN




\draw[blue,cap=rect,line width=1, opacity=1, domain=-7:-1.7,samples=100] plot (\x, {
	( 3*(pow(\x,2))) /(	pow(\x,2)-2)	% <- plaats het functievoorschrift hier	
}) node[opacity=1,above]{$$};
%-------------------------------------------


%---------------------------------------

\draw[blue,cap=rect,line width=1, opacity=1, domain=-1:1,samples=100] plot (\x, {
	( 3*(pow(\x,2))) /(	pow(\x,2)-2)	% <- plaats het functievoorschrift hier	
}) node[opacity=1,above]{$$};
%-------------------------------------------



\draw[blue,cap=rect,line width=1, opacity=1, domain=1.7:7,samples=100] plot (\x, {
	( 3*(pow(\x,2))) /(	pow(\x,2)-2)	% <- plaats het functievoorschrift hier	
}) node[opacity=1,above]{$$};
%-------------------------------------------


%legende





\end{tikzpicture}
\end{center}

