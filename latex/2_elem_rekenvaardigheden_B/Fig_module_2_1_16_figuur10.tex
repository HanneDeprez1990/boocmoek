\begin{center}
	\begin{tikzpicture}[yscale=2,cap=round]
	
	% Styles
	\tikzstyle{axes}=[]
	\tikzstyle help lines=[color=blue!50,very thin,dotted]
	
	% grid
%	\draw[style=help lines,step=1cm] (-4.9,-1.9) grid (4.9,4.9);
	
	\draw[->] (-1,0) -- (8,0) node[right] {$x$};
	\draw[->] (0,-.5) -- (0,2) node[above] {$y$};
	
	%\draw[fill,cyan](1,1)circle [radius=0.025];
	%FUNCTIEVOORSCHRIFTEN
	
	

	
	
	
	\draw[red,cap=rect,line width=1, opacity=1, domain=1:6,samples=100] plot (\x, {
	-1* exp(pow(0.5,\x))+2	% <- plaats het functievoorschrift hier	
	}) node[opacity=1,above,right]{$y=f(x)$};
	%-------------------------------------------
	
	
	\draw[dotted] (4,0)--(4,0.95); 
	\draw[dotted] (0,0.95)--(4,0.95); 
	
	\draw[] (4,0) node[below]{$x_0$};
	\draw[] (0,0.95) node[left]{$y_0$};
	
	\draw[] (4,1) node[above]{$P$};
	
	\draw[] (-0.2,0) node[left,below]{$0$};
	
	

	%legende
	
	
	
	%getallen op de x-as en lijntjes   
%	\foreach \x/\xtext in {-4,-3,-2,-1,0,1,2,3,4}
%	\draw[xshift=\x cm] (0pt,1pt) -- (0pt,0pt) node[below,fill=white]
%	{$\xtext$};,3
	
	%getallen op de y-as en lijntjes  
	%BEGIN LUS
%	\foreach \y/\ytext in {-1,1,2,3,4}
%	\draw[yshift=\y cm] (1pt,0pt) -- (0pt,0pt) node[left,fill=white]
%	{$\ytext$}; %EINDE LUS
	
	
	
	\end{tikzpicture}
\end{center}

