\subsection*{Inleiding}

Stel dat je een computerspel speelt en je op een beeldscherm, of eventueel door een virtual reality headset, naar een door een computer gegenereerd landschap kijkt. Kijk je in deze wereld een beetje naar links of naar rechts dan berekent de computer hoe het landschap eruit ziet in je nieuwe kijkrichting en je ziet de virtuele wereld als het ware rond je gezichtspunt roteren. Misschien heb je er nog niet bij stilgestaan maar bij dit soort zaken wordt stevig gebruikt gemaakt van wiskunde.\\
We kunnen een idee krijgen van hoe dit in zijn werk gaat door ons in eerste instantie te beperken tot een twee dimensionale situatie. In de figuur is een punt $P$ voorgesteld waarvan de positie in het vlak is vastgelegd door een plaatsvector $\vec{r}$. Deze plaatsvector kan voorgesteld worden ten opzichte van verschillende basissen. 

\begin{figure}[h]
	\begin{center}
		\includegraphics[scale=0.5]{5_vglen_ongelijkheden_stelsels_matrices/inputs/matrices-fig-1}
	\end{center}
\end{figure}

In de figuur worden twee orthonormale basissen (m.a.w. basissen van loodrecht op elkaar staande \'{e}\'{e}nheidsvectoren) gebruikt waarbij de basis $\{\vec{e}_u ,\vec{e}_v \}$ over een hoek $\theta$ is geroteerd ten opzichte van de basis $\{\vec{e}_x ,\vec{e}_y \}$. \\  De plaatsvector van het punt $P$ wordt ten opzichte van de basis $\{\vec{e}_x ,\vec{e}_y \}$ geschreven als

\[  \vec{r}=x \vec{e}_x + y \vec{e}_y \] 

en ten opzichte van de basis $\{\vec{e}_u ,\vec{e}_v \}$ als 

\[  \vec{r}=u \vec{e}_u + v \vec{e}_v \] 

Om te beschrijven wat er gebeurt bij een rotatie van ons referentiestelsel over een hoek $\theta$ moeten we het verband vinden tussen de co\"{o}rdinaten van het punt $P$ in de twee referentiestelsels, met andere woorden we moeten het verband vinden tussen de componenten van $\vec{r}$ ten opzichte van de basis $\{\vec{e}_u ,\vec{e}_v \}$ en ten opzichte van de basis $\{\vec{e}_x ,\vec{e}_y \}$.\\ 

Door te projecteren kunnen we de componenten van de \'{e}\'{e}nheidsvectoren van de basis $\{\vec{e}_u ,\vec{e}_v \}$ schrijven ten opzichte van de basis $\{\vec{e}_x ,\vec{e}_y \}$.

\[  
\left \{ \begin{array}{l} 
\vec{e}_u =\cos \theta \vec{e}_x + \sin \theta \vec{e}_y \\
\vec{e}_v =-\sin \theta \vec{e}_x + \cos \theta \vec{e}_y
\end{array}  \right.
\]

Omgekeerd kunnen we de componenten van de basisvectoren $\{\vec{e}_x ,\vec{e}_y \}$ ten opzichte van de basis $\{\vec{e}_u ,\vec{e}_v \}$ neerschrijven als volgt:

\[  
\left \{ \begin{array}{l} 
\vec{e}_x =\cos \theta \vec{e}_u - \sin \theta \vec{e}_v \\
\vec{e}_y =\sin \theta \vec{e}_u + \cos \theta \vec{e}_v
\end{array}  \right.
\]

Door substitutie in $\vec{r}=u \vec{e}_u + v \vec{e}_v$ vinden we

\[ \vec{r}=u(\cos \theta \vec{e}_x + \sin \theta \vec{e}_y) + v(-\sin \theta \vec{e}_x + \cos \theta \vec{e}_y)  \]

ofwel

\[ \vec{r}=(u \cos \theta -v \sin \theta)\vec{e}_x + (u \sin \theta + v \cos \theta)\vec{e}_y \]

De co\"{o}rdinaten van het punt $P$ in het oorspronkelijke referentiestelsel worden dus als volgt geschreven in functie van de co\"{o}rdinaten van het punt $P$ in het nieuwe referentiestelsel:

\[   
\left\{ \begin{array}{l}
x= u \cos \theta -v \sin \theta \\
y= u \sin \theta + v \cos \theta \end{array} \right.
\]

Op dezelfde manier vindt men door substitutie de co\"{o}rdinaten van het punt $P$ in het nieuwe referentiestelsel, uitgedrukt in functie van de co\"{o}rdinaten van $P$ in het oude referentiestelsel:

\[   
\left\{ \begin{array}{l}
u= x \cos \theta + y \sin \theta \\
v= -x \sin \theta + y \cos \theta \end{array} \right.
\]

Deze uitdrukkingen kan men ook op een alternatieve, elegante manier neerschrijven met behulp van getallenschema's. De co\"{o}rdinaten in een referentiestelsel schrijft met in een kolom, een {\bf kolommatrix} en het verband tussen de co\"{o}rdinaten in beide referentiestelsels drukt met uit met een {\bf co\"{o}rdinatentransformatiematrix}.

\[
\left( \begin{array}{l} u \\ v \end{array} \right)= \left( \begin{array}{rr} \cos \theta & \sin \theta \\ -\sin \theta & \cos \theta \end{array} \right) \left( \begin{array}{l} x \\ y \end{array} \right) 
\]

Een soortgelijke, maar verschillende situatie wordt in de volgende figuur voorgesteld. Een vector $\vec{r}=x \vec{e}_x + y \vec{e}_y$ roteert over een hoek $-\theta$ (dus gemeten in wijzerszin), na deze rotatie wordt de vector, die we nu $\vec{r'}$ noemen, gegeven door $\vec{r'}$=$x' \vec{e}_x + y' \vec{e}_y$. 

\begin{figure}[h]
	\begin{center}
		\includegraphics[scale=0.5]{5_vglen_ongelijkheden_stelsels_matrices/inputs/matrices-fig-2}
	\end{center}
\end{figure}

De componenten van $\vec{r'}$ worden nu uitgedrukt in functie van de componenten van $\vec{r}$ met

\[
\left( \begin{array}{l} x' \\ y' \end{array} \right)= \left( \begin{array}{rr} \cos \theta & \sin \theta \\ -\sin \theta & \cos \theta \end{array} \right) \left( \begin{array}{l} x \\ y \end{array} \right) 
\]

In de volgende figuur is de rotatie van een vector $\vec{r}$ over een hoek $\theta$ in tegenwijzerszin voorgesteld.

\begin{figure}[h]
	\begin{center}
		\includegraphics[scale=0.5]{5_vglen_ongelijkheden_stelsels_matrices/inputs/matrices-fig-3}
	\end{center}
\end{figure}

De transformatie voor deze rotatie vinden we door in de bovenstaande uitdrukking $\theta$ te vervangen door $-\theta$. We vinden dan:

\[
\left( \begin{array}{l} x' \\ y' \end{array} \right)= \left( \begin{array}{rr} \cos \theta & -\sin \theta \\ \sin \theta & \cos \theta \end{array} \right) \left( \begin{array}{l} x \\ y \end{array} \right) 
\]

De matrix

\[
\left( \begin{array}{rr} \cos \theta & -\sin \theta \\ \sin \theta & \cos \theta \end{array} \right)
\]

wordt de rotatiematrix voor rotatie over een hoek $\theta$ genoemd.\\

Matrices zijn een handige manier om veranderingen van referentiestelsel, co\"{o}rdinatentransformaties, te beschrijven.\\
Daarnaast is er een hele familie bewerkingen, lineaire transformaties, waarbij een vector via een matrixbewerking wordt omgezet in een nieuwe vector. Typisch voorbeeld is het roteren van een vector zoals hierboven beschreven.\\

In dit hoofdstuk zullen we een aantal definities en eigenschappen van matrices bespreken en zullen we rekenwerk met matrices inoefenen met het oog op praktische toepassingen. Alle eigenschappen die besproken worden kunnen wiskundig bewezen worden maar dat gaan we in dit hoofdstuk dus niet doen.\\

\newpage

\subsection{Definities}

%\begin{itemize}
%	\item Wat is een matrix?
%    \item Welke bewerkingen zijn er mogelijk met matrices?
%    \item Wat hebben matrices te maken met het oplossen van stelsels van vergelijkingen?
%\end{itemize}

\subsubsection{Definitie van een matrix}

\begin{definitie}
	Een $m \times n$ matrix $A$ is een rechthoekig getallenschema waarin re\"{e}le en/of complexe getallen in $m$ rijen en $n$ kolommen zijn gerangschikt. Het getal op de $i$-de rij en $j$-de kolom wordt symbolisch weergegeven als $a_{ij}$.

\end{definitie}
\[
A= \left( \begin{matrix}
a_{11} & a_{12} & a_{13} & \ldots & a_{1n} \\
a_{21} & a_{22} & a_{23} & \ldots & a_{2n} \\
\vdots & \vdots & \vdots & \ddots & \vdots \\
a_{m1} & a_{m2} & a_{m3} & \ldots & a_{mn}
\end{matrix} \right)
\]

\subsubsection{Enkele veel voorkomende speciale matrices}

\begin{itemize}
\item{Een rijmatrix}

Een matrix $A$ met $1$ rij en $n$ kolommen wordt een $1xn$ matrix of rijmatrix genoemd.

\[
A= \left( \begin{matrix}
a_{11} & a_{12} & a_{13} & \ldots & a_{1n}
\end{matrix} \right)
\]

\item{Een kolommatrix}

Een matrix $A$ met $m$ rijen en $1$ kolom wordt een $mx1$ matrix of kolommatrix genoemd.

\[
A= \left( \begin{matrix}
a_{11} \\
a_{21} \\
a_{31} \\
\vdots \\
a_{m1}
\end{matrix} \right)
\]

\item{Een vierkante matrix}

Een matrix met evenveel rijen als kolommen wordt een vierkante matrix genoemd.\\ De onderstaande matrix $A$ is een $n \times n$ matrix.

\[
A= \left( \begin{matrix}
a_{11} & a_{12} & \ldots & a_{1n} \\
a_{21} & a_{22} & \ldots & a_{2n} \\
\vdots & \vdots & \ddots & \vdots \\
a_{n1} & a_{n2} & \ldots & a_{nn}
\end{matrix} \right)
\]

\item{Een diagonaalmatrix}

Een vierkante matrix $A$ waarin alle getallen die niet op de diagonaal liggen nul zijn ($a_{ij}=0$ als $i \neq j$) en minstens \'{e}\'{e}n getal op de diagonaal verschillend is van nul (voor minstens $1$ getal $a_{ii}$ geldt $a_{ii} \neq 0$) noemt men een diagonaalmatrix.

\[
A= \left( \begin{matrix}
a_{11} & 0 & 0 & 0 & \ldots & 0 \\
0  & a_{22} & 0 & 0 & \ldots & 0 \\
0 & 0 & a_{33} & 0 & \ldots & 0 \\
0 & 0 & 0 & \ddots &  & 0 \\
\vdots & \vdots & \vdots &  & \ddots & \vdots \\
0 & 0 & 0 & 0 & \ldots & a_{nn}
\end{matrix} \right)
\]

\item{Een driehoeksmatrix}

Een driehoeksmatrix is een vierkante matrix waarbij alle elementen onder de diagonaal nul zijn. De overige elementen zijn niet allemaal nul.

\[
A= \left( \begin{matrix}
a_{11} & a_{12} & a_{13} & a_{14} & \ldots & a_{1n} \\
0  & a_{22} & a_{23} & a_{24} & \ldots & a_{2n} \\
0 & 0 & a_{33} & a_{34} & \ldots & a_{3n} \\
0 & 0 & 0 & \ddots &  & a_{4n} \\
\vdots & \vdots & \vdots &  & \ddots & \vdots \\
0 & 0 & 0 & 0 & \ldots & a_{nn}
\end{matrix} \right)
\]

\item{Een symmetrische matrix}

Een vierkante matrix $A$ waarbij de diagonaal een spiegellijn vormt (m.a.w. voor alle $a_{ij} \in A$ geldt $a_{ij}=a_{ji}$) is een symmetrische matrix.

\[
A= \left( \begin{matrix}
a_{11} & a_{12} & \ldots & a_{1n} \\
a_{12} & a_{22} & \ldots & a_{2n} \\
\vdots & \vdots & \ddots & \vdots \\
a_{1n} & a_{2n} & \ldots & a_{nn}
\end{matrix} \right)
\]

\item{Een \'{e}\'{e}nheidsmatrix}

Een diagonaal matrix waarbij alle getallen op de diagonaal gelijk zijn aan \'{e}\'{e}n (alle $a_{ii}=1$) is een \'{e}\'{e}nheidsmatrix $I$.

\[
I= \left( \begin{matrix}
1 & 0 & 0 & 0 & \ldots & 0 \\
0 & 1 & 0 & 0 & \ldots & 0 \\
0 & 0 & 1 & 0 & \ldots & 0 \\
0 & 0 & 0 & \ddots &  & 0 \\
\vdots & \vdots & \vdots &  & \ddots & \vdots \\
0 & 0 & 0 & 0 & \ldots & 1
\end{matrix} \right)
\]

\item{Een nulmatrix}

Een $m \times n$ matrix waarin alle $a_{ij}=0$ is een nulmatrix $O$. 

\[
O= \left( \begin{matrix}
0 & 0 & 0 & 0 & \ldots & 0 \\
0 & 0 & 0 & 0 & \ldots & 0 \\
0 & 0 & 0 & 0 & \ldots & 0 \\
0 & 0 & 0 & \ddots &  & 0 \\
\vdots & \vdots & \vdots &  & \ddots & \vdots \\
0 & 0 & 0 & 0 & \ldots & 0
\end{matrix} \right)
\]

\end{itemize}


\begin{opmerking}
	De nulmatrix is niet noodzakelijk een vierkante matrix.

\[  
\left( \begin{matrix}
0 & 0 & 0 & 0 & 0\end{matrix} \right) , \left( \begin{matrix} 
0 & 0 \\
0 & 0 \end{matrix} \right) , \left( \begin{matrix} 
0 & 0 & 0 \\
0 & 0 & 0 \\
0 & 0 & 0 \\
0 & 0 & 0 \end{matrix} \right) , \ldots                                            
\] 

worden allemaal nulmatrix genoemd.\\

\end{opmerking}

\subsection{Bewerkingen met matrices}

\subsubsection{Transponeren van een matrix}

De getransponeerde $A^t$ van een matrix $A$ vindt men door de rijen en kolommen van $A$ om te wisselen.\\

\begin{voorbeeld}
	

\[ \begin{array}{ll}
A= \left( \begin{matrix}
1 & 3 & 0 & 2 \\
7 & 0 & 0 & 8 \\
5 & 4 & 1 & 3
\end{matrix} \right) &
A^t = \left( \begin{matrix}
1 & 7 & 5 \\
3 & 0 & 4 \\
0 & 0 & 1 \\
2 & 8 & 3
\end{matrix}
\right)
\end{array}
\]

\end{voorbeeld}
Voor een symmetrische matrix $B$ geldt dat $B^t =B$.\\

\begin{voorbeeld}
	

\[ \begin{array}{ll}
B= \left( \begin{matrix}
1 & 3 & 0  \\
3 & 0 & 4  \\
0 & 4 & 1 
\end{matrix} \right) &
B^t = \left( \begin{matrix}
1 & 3 & 0  \\
3 & 0 & 4  \\
0 & 4 & 1 
\end{matrix}
\right)
\end{array}
\]

\end{voorbeeld}

De getransponeerde van een rijmatrix is een kolommatrix en omgekeerd.\\


\begin{voorbeeld}
	\[ \begin{array}{ll}
C= \left( \begin{matrix}
1 & 84 & -1  
\end{matrix} \right) &
C^t = \left( \begin{matrix}
1 \\
84 \\
-1
\end{matrix}
\right)
\end{array}
\]

\end{voorbeeld}

\subsubsection{Product van een matrix met een getal}
	
Het product van een $m \times n$ matrix $A$ met een re\"{e}el of complex getal $\lambda$ is een nieuwe $m \times n$ matrix $B$ die men vindt door elk element $a_{ij}$ van de matrix te vermenigvuldigen met $\lambda$.
	
\[
\begin{array}{ll}
A= \left( \begin{matrix}
a_{11} & a_{12} & a_{13} & \ldots & a_{1n} \\
a_{21} & a_{22} & a_{23} & \ldots & a_{2n} \\
\vdots & \vdots & \vdots & \ddots & \vdots \\
a_{m1} & a_{m2} & a_{m3} & \ldots & a_{mn}
\end{matrix} \right) &
B=\lambda A=\left( \begin{matrix}
\lambda a_{11} & \lambda_{12} & \lambda a_{13} & \ldots & \lambda a_{1n} \\
\lambda a_{21} & \lambda a_{22} & \lambda a_{23} & \ldots & \lambda a_{2n} \\
\vdots & \vdots & \vdots & \ddots & \vdots \\
\lambda a_{m1} & \lambda a_{m2} & \lambda a_{m3} & \ldots & \lambda a_{mn}
\end{matrix} \right)
\end{array}
\]
	
\subsubsection{Optellen van matrices}

Het optellen van twee matrices $A$ en $B$ is alleen gedefini\"{e}erd als het aantal rijen van beide matrices gelijk is en als het aantal kolommen van beide matrices gelijk is. In dat geval is de som een matrix $C$ met hetzelfde aantal rijen en kolommen als $A$ en $B$ die men als volgt vindt:

\[ 
\begin{array}{ll}
A= \left( \begin{matrix}
a_{11} & a_{12} & a_{13} & \ldots & a_{1n} \\
a_{21} & a_{22} & a_{23} & \ldots & a_{2n} \\
\vdots & \vdots & \vdots & \ddots & \vdots \\
a_{m1} & a_{m2} & a_{m3} & \ldots & a_{mn}
\end{matrix} \right) &
B= \left( \begin{matrix}
b_{11} & b_{12} & b_{13} & \ldots & b_{1n} \\
b_{21} & b_{22} & b_{23} & \ldots & b_{2n} \\
\vdots & \vdots & \vdots & \ddots & \vdots \\
b_{m1} & b_{m2} & b_{m3} & \ldots & b_{mn}
\end{matrix} \right)
\end{array}
\]

\[
C=A+B=\left( \begin{matrix}
a_{11}+b_{11} & a_{12}+b_{12} & a_{13}+b_{13} & \ldots & a_{1n}+b_{1n} \\
a_{21}+b_{21} & a_{22}+b_{22} & a_{23}+b_{23} & \ldots & a_{2n}+b_{2n} \\
\vdots & \vdots & \vdots & \ddots & \vdots \\
a_{m1}+b_{m1} & a_{m2}+b_{m2} & a_{m3}+b_{m3} & \ldots & a_{mn}+b_{mn}
\end{matrix} \right)
\]


\begin{voorbeeld}
	\[
\begin{array}{lll}
A= \left( \begin{matrix}
1 & 4 & 8 \\
2 & 3 & 1
\end{matrix} \right) &
B= \left( \begin{matrix}
6 & 2 & -2 \\
1 & 1 & -1 
\end{matrix} \right) &
A+B=\left( \begin{matrix}
7 & 6 & 6 \\
3 & 4 & 0
\end{matrix} \right)
\end{array}
\]

\end{voorbeeld}	

\begin{voorbeeld}
	\[ 
\begin{array}{ll}
A= \left( \begin{matrix}
1 & 4 & 8 \\
2 & 3 & 1
\end{matrix} \right) &
C= \left( \begin{matrix}
6 & 2 \\
1 & 1  
\end{matrix} \right)
\end{array}
\]
De som $A+C$ is niet gedefini\"{e}erd.
\end{voorbeeld}


\subsubsection{Vermenigvuldiging van twee matrices}

Het product van een matrix $A$ met een matrix $B$ is alleen gedefini\"{e}erd als het aantal kolommen van de eerste matrix $A$ gelijk is aan het aantal rijen van de tweede matrix $B$.\\

In dat geval geldt dat het product van de $m \times n$ matrix $A$ met de $nxp$ matrix $B$ een nieuwe matrix $C=AB$ is met $m$-rijen en $p$-kolommen.\\
Elk element $c_{ij}$ van de productmatrix $C$ wordt gegeven door:

\[ c_{ij}=\sum\limits_{k=1}^{n} a_{ik}b_{kj}    \]


m.a.w. $c_{ij}$ wordt gevonden door het eerste element van de $i$-de rij van $A$ te vermenigvuldigen met het eerste element van de $j$-de kolom van $B$, het tweede element van de $i$-de rij van $A$ te vermenigvuldigen met het tweede element van de $j$-de kolom van $B$, het derde element van de $i$-de rij van $A$ te vermenigvuldigen met het derde element van de $j$-de kolom van $B$, enz... en vervolgens al deze producten op te tellen.\\


\begin{voorbeeld}
		Bereken de producten $C=AB$ en $D=BA$.
	\[
	\begin{array}{ll}
		A=\left( \begin{matrix}
			1 & 0 & 2 & 4 \\
			0 & 2 & 3 & 5 \end{matrix} \right) &
		B=\left( \begin{matrix}
			10 \\ 12 \\ 0 \\ 0
		\end{matrix} \right)
	\end{array}	
	\]
	Het aantal kolommen van $A$ is gelijk aan het aantal rijen van $B$, het product $AB$ is dus gedefini\"{e}erd en is een $2x1$ matrix $C$
	\[
	C=AB=\left( \begin{matrix} 
	1\cdot 10+0\cdot 12+2\cdot 0+4\cdot 0 \\
	0\cdot 10+2\cdot 12+3\cdot 0+5\cdot 0 \end{matrix} \right)=
	\left( \begin{matrix}
	10 \\ 24
	\end{matrix} \right)
	\]
	Het aantal kolommen van $B$ is verschillend van het aantal rijen van $A$. Dit betekent dat het product $D=BA$ niet gedefini\"{e}erd is.
	
\end{voorbeeld}

\begin{voorbeeld}
		Bereken de producten $C=AB$ en $D=BA$.
	\[
	\begin{array}{ll}
	A=\left( \begin{matrix}
	1 & 0 & 1 \\
	\end{matrix} \right) &
	B=\left( \begin{matrix}
	1 \\ 0 \\ 1
	\end{matrix} \right)
	\end{array}	
	\]
	Het aantal kolommen van $A$ is gelijk aan het aantal rijen van $B$, het product $AB$ is dus gedefini\"{e}erd en is een $1x1$ matrix $C$
	\[ C=AB=\left( \begin{matrix} 1\cdot 1+0\cdot 0+1\cdot 1 \end{matrix} \right)=\left( \begin{matrix} 2 \end{matrix} \right)=2 \]
	Het aantal kolommen van $B$ is gelijk aan het aantal rijen van $A$, het product $BA$ is dus gedefini\"{e}erd en is een $3x3$ matrix $D$
	\[
	D=BA=\left( \begin{matrix} 1\cdot 1 & 0\cdot 0 & 1\cdot 1 \\ 0\cdot 1 & 0\cdot 0 & 0\cdot 1 \\  1\cdot 1 & 0\cdot 0 & 1\cdot 1 \end{matrix} \right)=\left( \begin{matrix} 1 & 0 & 1 \\ 0 & 0 & 0 \\  1 & 0 & 1 \end{matrix} \right)
	\]
\end{voorbeeld}


\begin{opmerking}
	\ \\
	\begin{itemize}
\item Bij het vermenigvuldigen van matrices geldt dus over het algemeen {\bf NIET} dat $AB=BA$
\item Het is zelfs {\bf NIET} zo dat uit het feit dat het product van twee matrices $C=AB$ bestaat volgt dat het product $BA$ ook bestaat.
\end{itemize}
\end{opmerking}
\subsection{Determinant}

\subsubsection{Inleiding}

De determinant van een matrix is een getal dat via een welbepaalde procedure uit de elementen van een vierkante matrix kan berekend worden.\\
De determinant van een matrix is alleen {\bf gedefini\"{e}erd voor vierkante matrices}.\\

\subsubsection{Determinant van een 1x1 matrix}

De determinant van een $1x1$ matrix $A$ is het enige element van de matrix.

\[ \begin{array}{ll}
A=\left( \begin{matrix} a_{11} \end{matrix} \right) & det A = |a_{11}|=a_{11} 
\end{array}
\]

\subsubsection{Determinant van een 2x2 matrix}

De determinant van een $2x2$ matrix $A$ wordt als volgt berekend:

\[ \begin{array}{ll}
A=\left( \begin{matrix} a_{11} & a_{12} \\ a_{21} & a_{22}  \end{matrix} \right) & det A = \left| \begin{matrix} a_{11} & a_{12} \\ a_{21} & a_{22}  \end{matrix} \right| = a_{11}a_{22}-a_{21}a_{12}   
\end{array}
\]

\subsubsection{Minor van een element van een vierkante matrix}

De minor van een element $a_{ij}$ van een vierkante matrix $A$ is het getal dat men bekomt door in de matrix $A$ de $i$-de rij en de $j$-de kolom te schrappen, en van de overblijvende matrix de determinant te berekenen.\\

\begin{voorbeeld}
	

We berekenen de minor van het element $a_{13}$ van de matrix $A=\left( \begin{matrix} a_{11} & a_{12} & a_{13} \\ a_{21} & a_{22} & a_{23} \\ a_{31} & a_{32} & a_{33} \end{matrix} \right)$\\

We schrappen de eerste rij en de derde kolom van $A$ en berekenen de determinant van de overblijvende matrix.

\[ minor(a_{13})=det \left( \begin{matrix} a_{21} & a_{22} \\ a_{31} & a_{32} \end{matrix} \right) = a_{21}a_{32}-a_{31}a_{22} \]
\end{voorbeeld}

\subsubsection{Cofactor van een element van een vierkante matrix}

De cofactor van een element $a_{ij}$ van een vierkante matrix $A$ is het getal dat men bekomt door de minor van $a_{ij}$ te vermenigvuldigen met $(-1)^{i+j}$\\

\[ \text{cofactor}(a_{ij})=(-1)^{i+j} minor(a_{ij}) \]

\begin{voorbeeld}
	
We berekenen de cofactor van het element $a_{ij}$ van de bovenstaande matrix $A$\\

\[ \text{cofactor}(a_{ij})=(-1)^{1+3} minor(a_{13})=minor(a_{13})=a_{21}a_{32}-a_{31}a_{22} \] 

\end{voorbeeld}

\begin{voorbeeld}
	Bereken de cofactor van het element $a_{23}=8$ van de matrix $A=\left( \begin{matrix} 1 & 2 & 0 \\ 0 & 4 & 8 \\ 2 & 5 & 8 \end{matrix} \right)$\\

We schrappen de tweede rij en de derde kolom en berekenen de determinant van de resterende vierkante matrix.\\

\[ \text{cofactor}(a_{23})=(-1)^{2+3} (1.5-2.2) = (-1).1 =-1 \]

Merk op dat in de berekening van de cofactor van $a_{23}$ de waarde van $a_{23}$, in dit geval $8$, geen enkele rol speelt.

\end{voorbeeld}

\subsubsection{Determinant van een $n \times n$ matrix}

Om de determinant van een willekeurige $n \times n$ matrix te berekenen maken we gebruik van de ontwikkelingsformule van Laplace. Deze formule kan toegepast worden op eender welke rij of kolom van een $n \times n$ matrix, het resultaat is steeds de determinant van de matrix.\\

Toegepast op de $i$-de rij van de matrix noemt men dit {\bf de ontwikkeling van de determinant naar de $i$-de rij}:

\[ det A=\sum\limits_{k=1}^{n} a_{ik}\text{cofactor}(a_{ik}) \] 


Toegepast op de $j$-de kolom noemt men dit {\bf de ontwikkeling van de determinant naar de $j$-de kolom}:

	
\[ det A=\sum\limits_{k=1}^{n} a_{kj}\text{cofactor}(a_{kj}) \]


\begin{voorbeeld}
	Bereken de determinant van de volgende matrix:

\[ A=\left( \begin{matrix}
	1 & 2 & 0 \\
	2 & 1 & 3 \\
	3 & 0 & 4 \end{matrix} \right)
	\]
	
	Oplossing: we passen ontwikkeling naar de eerste rij toe.
	
	\[
	det A= 1 (-1)^{1+1} det \left( \begin{matrix} 
	1 & 3 \\ 0 & 4 \end{matrix} \right) + 2 (-1)^{1+2} det \left( \begin{matrix}
	2 & 3 \\ 3 & 4 \end{matrix} \right) + 0 (-1)^{1+3} det \left( \begin{matrix}
	2 & 1 \\ 3 & 0 \end{matrix} \right)
	\]
	\[ det A=(4-2(8-9))=6 \] 

\end{voorbeeld}	
\begin{voorbeeld}
	Bereken de determinant van de volgende matrix:

	 \[ B=\left( \begin{matrix}
	1 & 5 & 9 & -5 \\
	\pi & 0 & 3 & 11 \\
	0 & \pi^2 & 4 & -3
	\end{matrix} \right)
	\]
	
	Oplossing: de matrix $B$ is geen vierkante matrix, de determinant bestaat niet.
\end{voorbeeld}	
\begin{voorbeeld}
	Bereken de determinant van de volgende matrix:
	 \[ C=\left( \begin{matrix}
	\pi & \pi^2 & \pi^3 & \pi^4 \\
	0   &  3    &   0   &   6   \\
	0   &  0    &   4   &   7   \\
	0   &  0    &   0   &   1  
	\end{matrix} \right)
	\]
	
	Oplossing: we passen ontwikkeling naar de eerste kolom toe.
	
	\[ det C= \pi (-1)^{1+1} det \left( \begin{matrix}
		3    &   0   &   6   \\
		0    &   4   &   7   \\
		0    &   0   &   1  
	\end{matrix} \right) \]
	
	Op de overblijvende determinant passen we opnieuw ontwikkeling naar de eerste kolom toe.
	
	\[ det C= \pi 3 (-1)^{1+1} det \left( \begin{matrix}
	4 & 7 \\ 0 & 1 
	\end{matrix} \right) \]
	
	\[ det C = \pi\cdot 3\cdot 4\cdot 1 = 12\pi \]
	
	Er geldt algemeen dat {\bf de determinant van een driehoeksmatrix het product van de diagonaalelementen is}. 
\end{voorbeeld}		


\subsubsection{Eigenschappen van determinanten}

Alle matrices die in deze paragraaf voorkomen worden ondersteld vierkante matrices te zijn. 


\begin{eigenschap}
		De determinant van de getransponeerde van een matrix is hetzelfde als de determinant van de oorspronkelijke matrix.
	
	\[ det A^{t}=det A \]
	\end{eigenschap}
	
\begin{eigenschap}
		Als men twee rijen (of twee kolommen) van plaats wisselt in een matrix dat verandert de determinant van teken. 
	\end{eigenschap}
	
	
	\begin{voorbeeld}
		
	\[ \begin{array}{ll} A=\left( \begin{matrix}
	1 & 0 & 0 \\ 0 & 1 & 0 \\ 1 & 0 & 1
	\end{matrix} \right) & det A = 1 \end{array} \]
	Wissel de eerste en tweede rij van plaats:
	\[ \begin{array}{lll} A'=\left( \begin{matrix}
	0 & 1 & 0 \\ 1 & 0 & 0 \\ 1 & 0 & 1
	\end{matrix} \right) & det A'=1(-1)^{1+2}det \left( \begin{matrix} 1 & 0 \\ 1 & 1 \end{matrix} \right) & det A'=-1 \end{array} \]
	
	\end{voorbeeld}
\begin{eigenschap}
		Als de elementen van een rij (of kolom) van een matrix vermenigvuldigt worden met een getal $\lambda$ dan wordt de determinant van de matrix vermenigvuldigt met $\lambda$.
	\end{eigenschap}
	

	\begin{voorbeeld}
		
	\[ \begin{array}{ll} B=\left( \begin{matrix}
	3 & 2 \\ 2 & 4 
	\end{matrix} \right) & det B=12-4=8 \end{array} \]
	We vermenigvuldigen de eerste rij met $\lambda=2$
	\[ \begin{array}{ll} B'=\left( \begin{matrix}
	6 & 4 \\ 2 & 4
	\end{matrix} \right) & det B'=24-8=16 \end{array} \]
	
	\end{voorbeeld}

\begin{eigenschap}
		Als men de elementen van een rij (kolom) van een matrix vermenigvuldigt met een getal $\lambda \neq 0$ en deze vervolgens optelt bij de elementen van een andere rij (kolom) dan verandert de waarde van de determinant niet. 
	
	\end{eigenschap}
	
	\begin{voorbeeld}
		
	We vertrekken opnieuw van de matrix $B$ uit het vorige voorbeeld. We vermenigvuldigen nu de eerste rij met $2$ en tellen deze rij op bij de tweede rij.
	
	\[ \begin{array}{ll} B''=\left( \begin{matrix}
	3 & 2 \\ 8 & 8 
	\end{matrix} \right) & det B''=24-16=8 \end{array} \]
	
	\end{voorbeeld}

\begin{eigenschap}
		De determinant van een matrix met een nulrij (nulkolom) is nul.
	\end{eigenschap}
	
	\begin{voorbeeld}
		
	
	\[ C=\left( \begin{matrix}
	\pi & \pi+1 & \pi+2 \\ 0 & 0 & 0 \\ \pi-1 & \pi & \pi+1
	\end{matrix} \right) \]
	De determinant ontwikkelen naar de tweede rij geeft onmiddellijk $det C=0$
	
	\end{voorbeeld}
\begin{voorbeeld}
		Als twee rijen (kolommen) van een matrix gelijk zijn dan is de determinant nul.
	\end{voorbeeld}
	
	\begin{voorbeeld}
		
	
	\[ \begin{array}{ll} D=\left( \begin{matrix}
	1 & 1 & 8 \\ 3 & 3 & 7 \\ 2 & 2 & 5
	\end{matrix} \right) & det D=1 det \left( \begin{matrix} 
	3 & 7 \\ 2 & 5 \end{matrix} \right) -1 det \left( \begin{matrix}
	3 & 7 \\ 2 & 5 \end{matrix} \right) + 8 det \left( \begin{matrix} 3 & 3 \\ 2 & 2 \end{matrix} \right) \end{array} \]
	\[ det D = 0 + 8 (6-6) = 0 \]
	
	
	\end{voorbeeld}	

\begin{eigenschap}
		Als twee rijen (kolommen) van een matrix evenredig zijn met elkaar dan is de determinant nul. 
	\end{eigenschap}
	
	\begin{voorbeeld}
		de tweede rij is drie maal de eerste rij.
	
	\[ \begin{array}{ll} E=\left( \begin{matrix}
	5 & 10 \\ 15 & 30 
	\end{matrix} \right) & det E=150-150=0 \end{array} \]
	
	\end{voorbeeld}
\begin{eigenschap}
		Als een rij (kolom) van een matrix een lineaire combinatie is van andere rijen (kolommen) van de matrix dan is de determinant nul.
	\end{eigenschap}
	
	\begin{voorbeeld}
		De tweede rij is de derde rij waarbij twee keer de eerste rij is opgeteld. We ontwikkelen naar de derde rij. 
	
	\[ \begin{array}{ll} F=\left( \begin{matrix}
	1 & 2 & 3 \\ 6 & 4 & 10 \\ 4 & 0 & 4 
	\end{matrix} \right) & det F = 4 det \left( \begin{matrix} 2 & 3 \\ 4 & 10 \end{matrix} \right) + 4 det \left( \begin{matrix} 1 & 2 \\ 6 & 4 \end{matrix} \right) \end{array} \]
	\[ det F =4(20-12)+4(4-12)=4(8-8)=0 \]
	
	\end{voorbeeld}
	\begin{eigenschap}
		Als $A$ en $B$ beide $n \times n$ matrices zijn dan geldt:
	\[ det(AB)=(det A) (det B) \]
	
	\end{eigenschap}
	\begin{voorbeeld}
		
	\[ \begin{array}{ll} A=\left( \begin{matrix}
	2 & 3 \\ 4 & 10
	\end{matrix} \right) & B=\left( \begin{matrix}
	1 & 2 \\ 6 & 4
	\end{matrix} \right) \end{array} \]
	
	\[ \begin{array}{lll} det A = 8 & det B = -8 & (det A)(det B)=-64 \end{array} \]
	
	\[ \begin{array}{ll} AB=\left( \begin{matrix}
	2+18 & 4+12 \\ 4+60 & 8+40 
	\end{matrix} \right)=\left( \begin{matrix}
	20 & 16 \\ 64 & 48 
	\end{matrix} \right) & det(AB)=20.48-64.16 \end{array} \]
	\[ det(AB)=16(20.3-64)=16(-4)=-64 \]
	
	\end{voorbeeld}

\subsection{Inverse van een matrix}

\subsubsection{Reguliere en singuliere matrix}

Een vierkante matrix $A$ waarvoor geldt $det A \neq 0$ is een reguliere matrix.

Een vierkante matrix $A$ waarvoor geldt $det A =0$ is een singuliere matrix.

\subsubsection{Inverse van een matrix}

Als er voor een vierkante matrix $A$ een vierkante matrix $X$ bestaat waarvoor geldt\\ $AX=XA=I$, met $I$ de \'{e}\'{e}nheidsmatrix dan is de matrix $X$ de inverse van $A$.\\
Notatie:
\[ X=A^{-1} \]

\begin{opmerking}
	\ \\
\begin{itemize}
	\item Opdat $X$ de inverse van $A$ zou zijn moet dus aan {\bf beide} uitdrukkingen $AX=I$ en $XA=I$ voldaan zijn.
	\item Als de inverse van $A$ bestaat dan geldt $(A^{-1})^{-1}=A$
	\item Als de inverse van $A$ bestaat dan zegt men dat $A$ inverteerbaar is.
	\item Men kan aantonen dat voor een inverteerbare matrix $A$ geldt $det A \neq 0$, met andere woorden "$A$ is inverteerbaar" $\Leftrightarrow$ "$A$ is regulier". 
\end{itemize}
\end{opmerking}


\begin{voorbeeld}
	De inverse van de matrix $A=\left( \begin{matrix} 1 & 1 \\ -1 & 1 \end{matrix} \right)$ is de matrix $B=\left( \begin{matrix} \frac{1}{2} & -\frac{1}{2} \\ \frac{1}{2} & \frac{1}{2} \end{matrix} \right)$ \\
Controle:\\
\[ AB=\left( \begin{matrix} 1 & 1 \\ -1 & 1 \end{matrix} \right) \left( \begin{matrix} \frac{1}{2} & -\frac{1}{2} \\ \frac{1}{2} & \frac{1}{2} \end{matrix} \right)= \left( \begin{matrix} 1 & 0 \\ 0 & 1 \end{matrix} \right)    \]

en

\[ BA=\left( \begin{matrix} \frac{1}{2} & -\frac{1}{2} \\ \frac{1}{2} & \frac{1}{2} \end{matrix} \right) \left( \begin{matrix} 1 & 1 \\ -1 & 1 \end{matrix} \right) =  \left( \begin{matrix} 1 & 0 \\ 0 & 1 \end{matrix} \right)    \]

\end{voorbeeld}

\subsubsection{Adjunctmatrix van een vierkante matrix}

De adjunctmatrix van een vierkante matrix $A$ is de matrix die men bekomt door elk element van $A$ te vervangen door zijn cofactor en vervolgens de matrix te transponeren.\\

Dus, met $a_{ij}$ de elementen van de $n \times n$ matrix $A$ geldt:


\[ adj{A}=\left( \begin{matrix}
\text{cofactor}(a_{11}) & \text{cofactor}(a_{12}) & \text{cofactor}(a_{13}) & \ldots & \text{cofactor}(a_{1n}) \\
\text{cofactor}(a_{21}) & \text{cofactor}(a_{22}) & \text{cofactor}(a_{23}) & \ldots & \text{cofactor}(a_{2n}) \\ 
\text{cofactor}(a_{31}) & \text{cofactor}(a_{32}) & \text{cofactor}(a_{33}) & \ldots & \text{cofactor}(a_{3n}) \\
\vdots & \vdots &  \vdots & \ddots & \vdots \\
\text{cofactor}(a_{n1}) & \text{cofactor}(a_{n2}) & \text{cofactor}(a_{n3}) & \ldots & \text{cofactor}(a_{nn})
\end{matrix} \right)^{t}           \]


Het is mogelijk om aan te tonen dat voor een reguliere vierkante matrix $A$ geldt:


	\[  A^{-1}=\frac{adj(A)}{det A}    \]


\begin{voorbeeld}
	

De vierkante matrix $A= \left( \begin{matrix} 1 & 8 & 0 \\ 1 & 7 & 2 \\ 0 & 3 & 1 \end{matrix} \right)$.\\

We gaan na of deze matrix regulier is (m.a.w. of er een inverse matrix $A^{-1}$ bestaat.\\
  
\[ det A= 1 det \left( \begin{matrix} 7 & 2 \\ 3 & 1 \end{matrix} \right) -8 det \left( \begin{matrix} 1 & 2 \\ 0 & 1 \end{matrix} \right) = -7   \]

Dus $det A \neq 0$ zodat $A$ is regulier, $A^{-1}$ bestaat.\\

\[  A^{-1}= \frac{adj(A)}{det A} = \frac{1}{-7} \left( \begin{matrix}
1 & -1 & 3 \\
-8 & 1 & -3 \\
16 & -2 & -1 
\end{matrix} \right)^{t}                   \]

De inverse van $A$ is dus

\[ A^{-1}= \left( \begin{matrix}
-\frac{1}{7} & \frac{8}{7} & -\frac{16}{7} \\
\frac{1}{7} & -\frac{1}{7} & \frac{2}{7} \\
-\frac{3}{7} & \frac{3}{7} & \frac{1}{7}
\end{matrix}   \right)
\]

\end{voorbeeld}

\subsection{De rang van een matrix}

\subsubsection{Onderdeterminant van p-de orde van een matrix}

Een onderdeterminant van $p$-de orde van een $m \times n$ matrix $A$ is een getal dat men bekomt door:
\begin{itemize}
	\item $m-p$ rijen en $n-p$ kolommen van de matrix $A$ te schrappen
	\item de determinant van de overblijvende vierkante matrix te berekenen
\end{itemize}


\begin{opmerking}
	\ \\
	\begin{itemize}
	\item onderdeterminanten zijn dus ook voor niet-vierkante matrices gedefini\"{e}erd
	\item er staat {\bf een} onderdeterminant... naargelang welke rijen en kolommen geschrapt worden zal je een andere onderdeterminant vinden
\end{itemize}
\end{opmerking}

\begin{voorbeeld}
	

Beschouw de $3x4$ matrix $A= \left( \begin{matrix} 1 & 2 & 0 & 0 \\ 0 & 3 & 4 & 0 \\ 0 & 0 & 5 & 6 \end{matrix} \right)$ \\
We berekenen een onderdeterminant van $3$-de orde door $3-3=0$ rijen te schrappen en $4-3=1$ kolom te schrappen en de determinant van de overblijvende vierkante matrix te berekenen.\\
\begin{itemize}
	\item Schrappen van de vierde kolom levert als onderdeterminant van $3$-orde:
	\[ det \left( \begin{matrix} 1 & 2 & 0 \\ 0 & 3 & 4 \\ 0 & 0 & 5 \end{matrix} \right)=15 \]
	\item Schrappen van de derde kolom levert als onderdeterminant van $3$-de orde:
	\[ det \left( \begin{matrix} 1 & 2 & 0 \\ 0 & 3 & 0 \\ 0 & 0 & 6 \end{matrix} \right)=18 \]
\end{itemize}

We berekenen een onderdeterminant van $1$-ste orde door $3-1=2$ rijen en $4-1=3$ kolommen te schrappen en de determinant van de overblijvende matrix te berekenen.\\
\begin{itemize}
	\item Schrap de eerste twee rijen en de eerste drie kolommen. De onderdeierminant van $1$-ste orde is
	\[  det \left( \begin{matrix} 6 \end{matrix} \right) =6 \]
	\item Schrap de eerste twee rijen en de laatste drie kolommen. De onderdeterminant van $1$-ste orde is
	\[  det \left( \begin{matrix} 0 \end{matrix} \right) =0 \]
\end{itemize}

\end{voorbeeld}
\subsubsection{Rang van een matrix}

De rang van een $m \times n$ matrix $A$ is nu gedefini\"{e}erd als de hoogste orde $r$ van alle van nul verschillende onderdeterminanten van $A$.\\

Er bestaan verschillende notaties voor de rang van een matrix maar de meest voorkomende zijn:\\ 

\[ rang(A)=r, Rg(A)=r  \quad \textrm{of} \quad Rank(A)=r \]


\begin{opmerking}
	\ \\
	\begin{itemize}
	\item De rang van een $m \times n$ matrix $A$ is nooit groter dan het minimum van $m$ en $n$. 
	\[ rang(A) \leq min(m,n)  \]
	Bij het berekenen van de onderdeterminanten kan men immers nooit minder dan $0$ rijen of kolommen schrappen...
	\item Voor een nulmatrix geldt: $rang(O)=0$
	\item Voor een vierkante $n \times n$ matrix $A$ geldt:
		\begin{itemize}
			\item as $A$ regulier is (m.a.w. $det A \neq 0$) dan geldt $rang(A)=n$\\
			De van nul verschillende determinant van $A$ is immers de onderdeterminant van orde $n$.
			\item als $A$ singulier is (m.a.w. $det A =0$ dan geldt $rang(A)<n$
		\end{itemize}
\end{itemize}

\end{opmerking}

\begin{voorbeeld}
	

We bepalen de rang van $A=\left( \begin{matrix}
1 & 2 & 3 & 0\\
0 & 1 & 0 & 7\\
2 & 5 & 6 & 7
\end{matrix} \right)$ 

De matrix $A$ is een $3x4$ matrix. We weten dus op voorhand dat $rang(A)\leq 3$\\

We berekenen nu de onderdeterminanten, startend van onderdeterminanten van orde $3$, tot we een onderdeterminant tegenkomen die verschillend is van nul. De orde van deze onderdeterminant is volgens de definitie de rang van de matrix.\\

\begin{itemize}
	\item Onderdeterminanten van orde $3$.\\
	We schrappen $0$ rijen en $1$ kolom.\\
	$det \left( \begin{matrix} 1 & 2 & 3\\ 0 & 1 & 0\\ 2 & 5 & 6 \end{matrix} \right)= 0$, $det \left( \begin{matrix} 1 & 2 & 0\\ 0 & 1 & 7\\ 2 & 5 & 7 \end{matrix} \right)= 0$, $det \left( \begin{matrix} 1 & 3 & 0\\ 0 & 0 & 7\\ 2 & 6 & 7 \end{matrix} \right) =0$,  $det \left( \begin{matrix} 2 & 3 & 0\\ 1 & 0 & 7\\  5 & 6 & 7 \end{matrix} \right) =0$
	De rang van de matrix zal dus kleiner zijn dan $3$.
	\item Onderdeterminanten van orde $2$.\\
	We schrappen $1$ rij en $2$ kolommen.\\
	Door de derde rij en de twee laatste kolommen te schrappen vinden we: 
	\[ det \left( \begin{matrix} 1 & 2 \\ 0 & 1 \end{matrix} \right) = 1 \neq 0 \]
	We vinden dus dat
	\[ rang(A)=2 \] 
\end{itemize}

\end{voorbeeld}


\subsection{Elementaire omvormingen van een matrix}

\subsubsection{Definitie}

	
\begin{definitie}
	Elementaire omvormingen van een matrix zijn omvormingen die de rang van een matrix niet veranderen. Er zijn drie elementaire omvormingen die men kan toepassen op de rijen of op de kolommen van een matrix.
\begin{itemize}
	\item Twee rijen (of twee kolommen) van plaats wisselen.
	\item Alle elementen van een rij (of kolom) vermenigvuldigen met een getal $\lambda \neq 0$ .
	\item Alle elementen van een rij (of kolom) vermenigvuldigen met een getal $\lambda \neq 0$ en deze dan optellen bij de elementen van een andere rij (kolom).
\end{itemize}

\end{definitie}


\begin{opmerking}
De rang van een matrix wordt bepaald door na te gaan of  determinanten verschillen zijn van nul of niet. De omvormingen die hierboven staan kunnen de waarde van een determinant wel veranderen maar hebben geen effect op het al dan niet nul zijn van een determinant (zie: eigenschappen van determinanten).
\end{opmerking}

\subsubsection{De rang van een matrix bepalen met elementaire omvormingen}

\begin{itemize}

\item{Echelonvorm van een matrix}

Een matrix is in echelonvorm als
\begin{itemize}
	\item alle eventuele nulrijen van de matrix onderaan de matrix staan
	\item de eerste rij begint met een element dat niet nul is, de tweede rij begint met een nul gevolgd door een element dat niet nul is, de derde rij begint met twee nullen gevolgd door een element dat niet nul is, enz... tot ofwel de nulrijen ofwel het einde van de matrix wordt bereikt.
\end{itemize}

Een $m \times n$ matrix in echelonvorm:\\

\[ \left( \begin{matrix}
b_{11} & b_{12} & \ldots & \ldots & b_{1r} & b_{1,r+1} & b_{1,r+2} & \ldots & b_{1n} \\
0 & b_{22} & \ldots & \ldots & b_{2r} & b_{2,r+1} & b_{2,r+2} & \ldots & b_{2n} \\
0 & 0 & \ddots & \ldots & b_{3r} & b_{3,r+1} & b_{3,r+2} & \ldots & b_{3n} \\
\vdots & \vdots & \vdots & \ddots & \vdots & \vdots & \vdots & \vdots & \vdots \\
\vdots & \vdots & \vdots & \ldots & b_{rr} & b_{r,r+1} & b_{r,r+2} & \ldots & b_{rn} \\
0 & 0 & \ldots & \ldots & 0 & 0 & 0 & \ldots & 0\\
\vdots & \vdots & \vdots & \vdots & \vdots & \vdots & \vdots & \vdots & \vdots \\
0 & 0 & \ldots & \ldots & 0 & 0 & \ldots & \ldots & 0
\end{matrix} \right)  \] \\

met alle $b_{ii} \neq 0$ ($i=1..r$) de elementen $b_{ij}$ met $i \neq j$ kunnen zowel gelijk als verschillend van nul zijn.\\


\item{Praktische manier om de rang van een matrix te bepalen}

De rang van een $m \times n$ matrix $A$ bepalen door onderdeterminanten te berekenen blijkt in de praktijk nogal lastig en tijdrovend te zijn. Een veel handiger manier is gebaseerd op elementaire omvormingen.\\

Men gaat als volgt te werk:

\begin{itemize}
	\item zet de matrix $A$ met behulp van elementaire omvormingen om in een nieuwe matrix $A'$ die in echelonvorm staat\\
	
	\[ A'= \left( \begin{matrix}
	b_{11} & b_{12} & \ldots & \ldots & b_{1r} & b_{1,r+1} & b_{1,r+2} & \ldots & b_{1n} \\
	0 & b_{22} & \ldots & \ldots & b_{2r} & b_{2,r+1} & b_{2,r+2} & \ldots & b_{2n} \\
	0 & 0 & \ddots & \ldots & b_{3r} & b_{3,r+1} & b_{3,r+2} & \ldots & b_{3n} \\
	\vdots & \vdots & \vdots & \ddots & \vdots & \vdots & \vdots & \vdots & \vdots \\
	\vdots & \vdots & \vdots & \ldots & b_{rr} & b_{r,r+1} & b_{r,r+2} & \ldots & b_{rn} \\
	0 & 0 & \ldots & \ldots & 0 & 0 & 0 & \ldots & 0\\
	\vdots & \vdots & \vdots & \vdots & \vdots & \vdots & \vdots & \vdots & \vdots \\
	0 & 0 & \ldots & \ldots & 0 & 0 & \ldots & \ldots & 0
	\end{matrix} \right) \]
	
	\item Het aantal niet-nulrijen van $A'$ is dan gelijk aan de rang van de matrix $A$
	
		\[ rang(A)=r=\textrm{aantal niet nulrijen van} \quad A'  \]
\end{itemize}

\end{itemize}

{\bf Verklaring:}\\

Elementaire omvormingen veranderen de rang van een matrix niet. Er geldt dus $rang(A)=rang(A')$.\\
De matrix $A'$ heeft minstens \'{e}\'{e}n onderdeterminant van orde $r$ want alle $b_{ii} \neq 0$ ($i=1..r$) zodat de determinant van de $rxr$ driehoeksmatrix, waarvan de $b_{ii}$ de diagonaal vormen, niet nul is.\\
Alle onderdeterminanten van een grotere orde dan $r$ van $A'$ zijn zeker nul, ze bevatten immers altijd minstens \'{e}\'{e}n nulrij.\\  


\begin{voorbeeld}
	Bereken de rang van $A=\left( \begin{matrix}
1 & 2 & 3 & 0 \\
0 & 1 & 2 & 3 \\
1 & 3 & 5 & 3 \\
2 & 0 & 1 & 1
\end{matrix} \right) $\\

We gaan de matrix nu stap voor stap, met elementaire opvormingen, omzetten in een echelonmatrix. Hierbij gaan we van links naar rechts door de matrix: eerst maken we alle elementen onder het eerste element van de eerste rij nul, vervolgens maken we alle elementen onder het tweede element van de tweede rij nul, enz... tot we de echelonvorm bekomen.\\

\begin{itemize}
	\item stap 1: rij 3 - rij 1 en rij 4 -2 rij 1 
	\[ A'=\left( \begin{matrix}
	1 & 2 & 3 & 0 \\
	0 & 1 & 2 & 3 \\
	0 & 1 & 2 & 3 \\
	0 & -4 & -5 & 1 \end{matrix} \right) \] 
	\item stap 2: rij 3 - rij 2 en rij 4 + 4 rij 2
	\[ A'=\left( \begin{matrix}
	1 & 2 & 3 & 0 \\
	0 & 1 & 2 & 3 \\
	0 & 0 & 0 & 0 \\
	0 & 0 & 3 & 13 \end{matrix} \right) \]
	\item stap 3: wissel rij 3 en rij 4 
	\[ A'=\left( \begin{matrix}
	1 & 2 & 3 & 0 \\
	0 & 1 & 2 & 3 \\
	0 & 0 & 3 & 13 \\
	0 & 0 & 0 & 0 \end{matrix} \right) \]
\end{itemize}

De resulterende $4x4$ matrix $A'$ staat in echelonvorm en heeft drie niet-nulrijen: 
\[ rang(A)=rang(A')=3 \]

\end{voorbeeld}

\subsection{Praktische berekening van de inverse van een matrix}

Om na te gaan of een vierkante matrix regulier is moeten we de determinant berekenen. Voor een iets grotere matrix kan dit behoorlijk tijdrovend zijn. Als blijkt dat de matrix regulier is, kan men de inverse vinden door de adjunctmatrix te berekenen. Hiervoor moeten opnieuw talrijke determinanten worden uitgerekend. Dit maakt dat het berekenen van de inverse van een matrix op deze manier enorm veel tijd en moeite kost, in die mate zelfs dat het voor grotere matrices, zelfs met een computer, onpraktisch wordt.\\

Er is echter een methode ontwikkeld die toelaat om, met elementaire omvormingen, na te gaan of een matrix inverteerbaar is en eventueel de inverse te vinden. Deze methode staat bekend als de methode van {\bf Gauss-Jordan}. 
We geven hier deze methode zonder bewijs.\\

Methode van Gauss-Jordan om na te gaan of een vierkante $n \times n$ matrix $A$ inverteerbaar is en om eventueel de inverse van $A$ te vinden:\\

\begin{itemize}
	\item Stap 1: Schrijf een nieuwe $nx2n$ matrix met links de matrix $A$ en rechts de \'{e}\'{e}nheidsmatrix $I$
	\[ \left( \begin{matrix}
	a_{11} & \ldots & \ldots & \ldots & a_{1n} & 1 & 0 & \ldots & \ldots & 0 \\
	\vdots &  &  &  & \vdots  & 0 & 1 &  &  &  0 \\
	\vdots &  &  &  & \vdots & \vdots &  & \ddots &  & \vdots\\
	\vdots &  &  &  & \vdots & \vdots &  &  & \ddots & \vdots \\
	a_{n1} & \ldots & \ldots & \ldots & a_{nn} & 0 & \ldots & \ldots & \ldots & 1  \\
	\end{matrix} \right) \] 
	\item Stap 2: Probeer de matrix $A$, door opeenvolgnede elementaire omvormingen, om te zetten in de \'{e}\'{e}nheidsmatrix $I$ en pas tegelijkertijd dezelfde elementaire omvormingen toe op $I$. Om geen rekenfouten te maken is het aan te raden je te beperken tot elementaire rij-omvormingen.\\
	Er zijn nu twee mogelijkheden:
	\begin{enumerate}
		\item Na een aantal omvormingen wordt de $n \times n$ matrix $A$ omgezet in een matrix met een nulrij. Dus $rang(A)<n$, $detA=0$ en $A$ is singulier. De inverse van $A$ bestaat niet.
		\item Na een aantal elementaire omvormingen is $A$ omgezet in de \'{e}\'{e}nheidsmatrix $I$. Dit betekent dat $rang(A)=n$, $detA \neq 0$ en $A$ is regulier (inverteerbaar). De \'{e}\'{e}nheidsmatrix $I$ is dan door de elementaire omvormingen omgezet in een nieuwe $n \times n$ matrix $B=A^{-1}$
		\[  \left( \begin{matrix}
		1 & 0 & \ldots & \ldots & 0 & b_{11} & \ldots & \ldots & \ldots & b_{1n} \\
		0 & 1 &  &  &  0 & \vdots & & & & \vdots \\
		\vdots &  & \ddots & & \vdots & \vdots &  &  &  & \vdots \\
		\vdots &  &  & \ddots & \vdots & \vdots & & & & \vdots \\
		0 & \ldots & \ldots & \ldots & 1 & b_{n1} & \ldots & \ldots & \ldots & b_{nn} \end{matrix} \right) 
		\]
	\end{enumerate}
\end{itemize}


\begin{voorbeeld}
	We gebruiken de methode van Gauss-Jordan om na te gaan of de vierkante matrix $A$ inverteerbaar is. Als dat zo is bepalen we meteen de inverse van $A$.

\[ A=\left( \begin{matrix}
1 & 7 & 0 \\
0 & 6 & 1 \\
2 & 5 & 8
\end{matrix} \right)   \]

We schrijven de $3x3$ matrix $A$ in een $3x6$ matrix door de \'{e}\'{e}nheidsmatrix rechts bij te voegen.

\[ \left( \begin{matrix}
1 & 7 & 0 & 1 & 0 & 0 \\
0 & 6 & 1 & 0 & 1 & 0 \\
2 & 5 & 8 & 0 & 0 & 1 \end{matrix} \right) 
\] 

We passen nu elementaire rij-omvormingen toe waarbij we proberen $A$ om te zetten in de \'{e}\'{e}nheidsmatrix. Tegelijkertijd passen we dezelfde rij-omvormingen toe op de \'{e}\'{e}nheidsmatrix $I$.\\
We gaan hierbij systematisch te werk waarbij we in $A$, van links naar rechts, de elementen onder de diagonaal omzetten in nullen. 

\begin{itemize}
	\item Stap 1: rij 3 - 2 rij 1 
	\[ \left( \begin{matrix}
	1 & 7 & 0 & 1 & 0 & 0 \\
	0 & 6 & 1 & 0 & 1 & 0 \\
	0 & -9 & 8 & -2 & 0 & 1 \end{matrix} \right) 
	\] 
	\item Stap 2: rij 3 + 3/2 rij 2 
	\[ \left( \begin{matrix}
	1 & 7 & 0 & 1 & 0 & 0 \\
	0 & 6 & 1 & 0 & 1 & 0 \\
	0 & 0 & \frac{19}{2} & -2 & \frac{3}{2} & 1 \end{matrix} \right) 
	\] 
	\item Stap 3: 1/6 rij 2 en 2/19 rij 3 
	\[ \left( \begin{matrix}
	1 & 7 & 0 & 1 & 0 & 0 \\
	0 & 1 & \frac{1}{6} & 0 & \frac{1}{6} & 0 \\
	0 & 0 & 1 & \frac{-4}{19} & \frac{3}{19} & \frac{2}{19} \end{matrix} \right) 
	\]
\end{itemize}

Pas als alle elementen onder de diagonaal nul gemaakt zijn bekijken we de elementen boven de diagonaal. We maken opnieuw gebruik van elementaire rij-omvormingen om deze elementen om te zetten in nullen. Deze keer gaan we systematisch van rechts naar links door de matrix.

\begin{itemize}
	\item Stap 4: rij 2 - 1/6 rij 3
	\[ \left( \begin{matrix}
	1 & 7 & 0 & 1 & 0 & 0 \\
	0 & 1 & 0 & \frac{2}{57} & \frac{8}{57} & \frac{-1}{57} \\
	0 & 0 & 1 & \frac{-4}{19} & \frac{3}{19} & \frac{2}{19} \end{matrix} \right) 
	\]
	\item Stap 5: rij 1 - 7 rij 2
	\[ \left( \begin{matrix}
	1 & 0 & 0 & \frac{43}{57} & \frac{-56}{57} & \frac{7}{57} \\
	0 & 1 & 0 & \frac{2}{57} & \frac{8}{57} & \frac{-1}{57} \\
	0 & 0 & 1 & \frac{-4}{19} & \frac{3}{19} & \frac{2}{19} \end{matrix} \right) 
	\]
\end{itemize}

De linkerhelft van de $3x6$ matrix, $A$ is omgezet in de \'{e}\'{e}nheidsmatrix. De matrix $A$ is dus inverteerbaar en tegelijkertijd is de rechterkant van de $3x6$ matrix omgezet in $B=A^{-1}$

\[  A^{-1}=  \left( \begin{matrix}
\frac{43}{57} & \frac{-56}{57} & \frac{7}{57} \\
\frac{2}{57} & \frac{8}{57} & \frac{-1}{57} \\
\frac{-4}{19} & \frac{3}{19} & \frac{2}{19} \end{matrix} \right) 
\]

\end{voorbeeld}
\subsection{Methode van Gauss voor het oplossen van een stelsel van vergelijkingen}

\subsubsection{Stelsels van lineaire vergelijkingen in matrixnotatie}

Een stelsel van lineaire vergelijkingen kan ook geschreven worden in matrixvorm. Neem een stelsel van $m$ vergelijkingen in $n$ onbekenden $x_1$, $x_2$,...,$x_n$:

\[ 
\left\{ \begin{array}{l}
a_{11} x_1 + a_{12} x_2 + ... + a_{1n} x_n = c_1 \\
a_{21} x_1 + a_{22} x_2 + ... + a_{2n} x_n = c_2 \\
\vdots \\ \vdots \\
a_{m1} x_1 + a_{m2} x_2 + ... + a_{mn} x_n = c_m
\end{array}
\right.
\]

We merken op dat het {\bf niet} noodzakelijk is dat het aantal vergelijkingen $m$ gelijk is aan het aantal onbekenden $n$.\\

Door alle co\"{e}ffici\"{e}nten $a_{ij}$ in een $m \times n$ matrix $A=\left( \begin{matrix} a_{11} & \ldots & \ldots & \ldots & a_{1n} \\ \vdots & & & & \vdots \\ \vdots & & & & \vdots \\ \vdots & & & & \vdots \\ a_{m1} & \ldots & \ldots & \ldots & a_{mn} \end{matrix} \right) $ te schrijven, de onbekenden in een $nx1$ kolommatrix $X=\left( \begin{matrix} x_1 \\ \vdots \\ \vdots \\ x_n \end{matrix} \right) $ te plaatsen en de rechterleden van de vergelijkingen in een $mx1$ kolommatrix $C=\left( \begin{matrix} c_1 \\ \vdots \\ \vdots \\ \vdots \\ c_m \end{matrix} \right) $ te schrijven, kunnen we het stelsel van vergelijkingen beschouwen als een matrixvermenigvuldiging $AX=C$:

\[ 
\left( \begin{matrix} a_{11} & \ldots & \ldots & \ldots & a_{1n} \\ \vdots & & & & \vdots \\ \vdots & & & & \vdots \\ \vdots & & & & \vdots \\ a_{m1} & \ldots & \ldots & \ldots & a_{mn} \end{matrix} \right) \left( \begin{matrix} x_1 \\ \vdots \\ \vdots \\ x_n \end{matrix} \right) = \left( \begin{matrix} c_1 \\ \vdots \\ \vdots \\ \vdots \\ c_m \end{matrix} \right)
\]

In plaats van de matrixvermenigvuldiging expliciet op te schrijven wordt het stelsel in matrixvorm dikwijls verkort genoteerd met behulp van de {\bf verhoogde co\"{e}ffici\"{e}ntenmatrix} $\left( \begin{array}{c|c} A & C \end{array} \right)$ : 

\[
\left( 
\begin{array}{ccccc|c}
a_{11} & \ldots & \ldots & \ldots & a_{1n} & c_1 \\ \vdots & & & & \vdots & \vdots \\ \vdots & & & & \vdots & \vdots \\ \vdots & & & & \vdots & \vdots \\ a_{m1} & \ldots & \ldots & \ldots & a_{mn} & c_m
\end{array} 
\right)
\]

\subsubsection{Equivalentietransformaties van een stelsel lineaire vergelijkingen}

Een equivalentietransformatie van een stelsel van vergelijkingen is een omvorming van het stelsel tot een nieuw stelsel van lineaire vergelijkingen dat nog steeds dezelfde oplossingen heeft als het oorspronkelijke stelsel. 

\begin{framed}
Er zijn drie equivalentietransformaties:

\begin{itemize}
	\item Twee vergelijkingen van het stelsel van plaats verwisselen
	\item Een vergelijking vermenigvuldigen met een getal $\lambda \neq 0$
	\item Een vergelijking vermenigvuldigen met een getal $\lambda \neq 0$ en het resultaat optellen bij een andere vergelijking.
\end{itemize}
\end{framed}

Het is niet moeilijk in te zien dat de eerste twee inderdaad de oplossingen van het stelsel niet veranderen. De derde transformatie komt neer op de substitutiemethode voor het oplossen van een stelsel van vergelijkingen die we in het hoofdstuk over vergelijkingen besproken hebben. 

Schrijven we nu een stelsel van lineaire vergelijkingen eerst in matrixvorm\\ $AX=C$ en vervolgens in verkorte vorm met de verhoogde co\"{e}ffici\"{e}ntenmatrix $\left( \begin{array}{c|c} A & C \end{array} \right)$. {\bf Het toepassen van de equivalentietransformaties op het stelsel komt dan neer op het toepassen van elementaire rij-omvormingen op de verhoogde co\"{e}ffici\"{e}ntenmatrix}. 

Hierop is de methode van Gauss voor het oplossen van een stelsel van vergelijkingen gebaseerd.

\subsubsection{Methode van Gauss}

We schrijven een stelsel van $m$ vergelijkingen in $n$ onbekenden 
\[ 
\left\{ \begin{array}{l}
a_{11} x_1 + a_{12} x_2 + ... + a_{1n} x_n = c_1 \\
a_{21} x_1 + a_{22} x_2 + ... + a_{2n} x_n = c_2 \\
\vdots \\ \vdots \\
a_{m1} x_1 + a_{m2} x_2 + ... + a_{mn} x_n = c_m
\end{array}
\right.
\]
eerst in matrixvorm $AX=C$ en vervolgens in de verkorte notatie $\left( \begin{array}{c|c} A & C \end{array} \right)$:
\[
\left( 
\begin{array}{ccccc|c}
a_{11} & \ldots & \ldots & \ldots & a_{1n} & c_1 \\ \vdots & & & & \vdots & \vdots \\ \vdots & & & & \vdots & \vdots \\ \vdots & & & & \vdots & \vdots \\ a_{m1} & \ldots & \ldots & \ldots & a_{mn} & c_m
\end{array} 
\right)
\]
We maken dan gebruik van elementaire rij-omvormingen (equivalentietransformaties) om de matrix om te zetten in echelonvorm $\left( \begin{array}{c|c} A' & C' \end{array} \right)$. Omdat we alleen elementaire omvormingen gebruiken geldt dan steeds $rang(A)=rang(A')$ en $rang \left( \begin{array}{c|c} A & C \end{array} \right) = rang \left( \begin{array}{c|c} A' & C' \end{array} \right)$.\\
In de meest algemene schrijfwijze heeft de echelonmatrix de vorm
\[ \left( \begin{array}{c|c} A' & C' \end{array} \right) = 
\left(
\begin{array}{cccccccccc|c}
a'_{11} & a'_{12} & \ldots & \ldots & a'_{1r} & a'_{1,r+1} & \ldots & \ldots & \ldots & a'_{1n} & c'_{1} \\
0 & a'_{22} & \ldots & \ldots & a'_{2r} & a'_{2,r+1} & \ldots & \ldots & \ldots & a'_{2n} & c'_{2} \\
\vdots & 0 & \ddots &  & \vdots & \vdots & & & & \vdots & \vdots \\
\vdots & \vdots & & \ddots & \vdots & \vdots & & & & \vdots & \vdots \\
\vdots & \vdots & & & a'_{rr} & a'_{r,r+1} & \ldots & \ldots & \ldots & a'_{rn} & c'_{r} \\
0 & \ldots & \ldots & \ldots & 0 & 0 & \ldots & \ldots & \ldots & 0 & c'_{r+1} \\
\vdots & & & & \vdots & \vdots & & & & \vdots & \vdots \\
\vdots & & & & \vdots & \vdots & & & & \vdots & \vdots \\
0 & \ldots & \ldots & \ldots & 0 & 0 & \ldots & \ldots & \ldots & 0 & c'_{m} 
\end{array} 
\right) 
\]

Merk op dat de rang van $A$ het aantal niet-nulrijen van $A'$ is: $rang(A)=r$ \\

Er kunnen zich nu drie verschillende situaties voordoen.

\begin{ftonthoud} 
	{\bf Geval 1:} voor minstens \'{e}\'{e}n van de $m-r$ laatste vergelijkingen (rijen) geldt $c'_{i} \neq 0$.\\ Dit betekent dat in het stelsel minstens \'{e}\'{e}n vergelijking voorkomt van de vorm $0 \neq 0$. Het is natuurlijk onmogelijk om aan deze vergelijking te voldoen: {\bf het stelsel van vergelijkingen heeft geen oplossingen}.\\
	Merk op dat in deze situatie geldt: $rang(A) \neq rang \left( \begin{array}{c|c} A' & C' \end{array} \right)$ \\
	
	{\bf Geval 2:} voor alle $m-r$ laatste rijen geldt $c'_{i}=0$ (m.a.w. $rang(A) = rang \left( \begin{array}{c|c} A' & C' \end{array} \right)$ ) en bovendien geldt $rang(A)=n$ met $n$ het aantal onbekenden. \\
	De echelonmatrix heeft dan de vorm:
	\[
	\left( \begin{array}{c|c} A' & C' \end{array} \right) = 
	\left(
	\begin{array}{ccccc|c}
	a'_{11} & a'_{12} & \ldots & \ldots & a'_{1n} & c'_{1} \\
	0 & a'_{22} & \ldots & \ldots & a'_{2n} & c'_{2} \\
	\vdots & 0 & \ddots &  & \vdots & \vdots \\
	\vdots & \vdots & & \ddots & \vdots & \vdots \\
	\vdots & \vdots & & & a'_{nn} & c'_{n} \\
	0 & & & & 0 & 0 \\
	\vdots & & & & \vdots & \vdots \\
	0 & & & & 0 & 0
	\end{array} 
	\right) 
	\]
	{\bf In dit geval heeft het stelsel van vergelijkingen juist \'{e}\'{e}n oplossing} die men vindt door de vergelijkingen \'{e}\'{e}n voor \'{e}\'{e}n op te lossen, beginnend met de laatste vergelijking. De oplossing van de laatste vergelijking $a'_{nn} x_{n}=c'_{n}$ wordt dan in de voorlaatste gesubstitueerd, enz... \\
	Dit geeft een unieke oplossing $x_{1}, x_{2}, ... , x_{n}$ \\
	
	{\bf Geval 3:} voor alle $m-r$ laatste rijen geldt $c'_{i}=0$ (m.a.w. $rang(A) = rang \left( \begin{array}{c|c} A' & C' \end{array} \right)$ ) en $rang(A)<n$ met $n$ het aantal onbekenden. \\
	De echelonmatrix heeft nu de vorm:
	\[
	\left( \begin{array}{c|c} A' & C' \end{array} \right) = 
	\left(
	\begin{array}{cccccccccc|c}
	a'_{11} & a'_{12} & \ldots & \ldots & a'_{1r} & a'_{1,r+1} & \ldots & \ldots & \ldots & a'_{1n} & c'_{1} \\
	0 & a'_{22} & \ldots & \ldots & a'_{2r} & a'_{2,r+1} & \ldots & \ldots & \ldots & a'_{2n} & c'_{2} \\
	\vdots & 0 & \ddots &  & \vdots & \vdots & & & & \vdots & \vdots \\
	\vdots & \vdots & & \ddots & \vdots & \vdots & & & & \vdots & \vdots \\
	\vdots & \vdots & & & a'_{rr} & a'_{r,r+1} & \ldots & \ldots & \ldots & a'_{rn} & c'_{r} \\
	0 & \ldots & \ldots & \ldots & 0 & 0 & \ldots & \ldots & \ldots & 0 & 0 \\
	\vdots & & & & \vdots & \vdots & & & & \vdots & \vdots \\
	\vdots & & & & \vdots & \vdots & & & & \vdots & \vdots \\
	0 & \ldots & \ldots & \ldots & 0 & 0 & \ldots & \ldots & \ldots & 0 & 0 
	\end{array} \right)
	\]
	{\bf Het stelsel is oplosbaar en heeft een oneindig aantal oplossingen.} \\
	Men vindt de oplossingen door de vergelijkingen in het stelsel, van onder naar boven, \'{e}\'{e}n voor \'{e}\'{e}n op te lossen. De oplossing van de $r$-de vergelijking substitueert men in de $r-1$-ste  vergelijking, enz... \\
	De $r$-de vergelijking kan men oplossen naar de onbekende $x_{r}$ door de onbekenden $x_{r+1}$,...,$x_{n}$ als parameters te kiezen. 
\end{ftonthoud}	

\begin{voorbeeld}
	We controleren met de methode van Gauss of het volgende stelsel oplosbaar is en we zoeken de eventuele oplossing(en.)

	\[ \left\{ 
	\begin{array}{l}
		x_1 -3x_2 +5x_3  = 0 \\
		2x_1 + x_2 -10x_3 = 2 \\
		-x_1 + 2x_2 = 1 \end{array} 
	\right. \]
	We schrijven de verhoogde co\"{e}ffici\"{e}ntenmatrix:
	\[ \left(
	\begin{array}{ccc|c}
	1 & -3 & 5 & 0 \\
	2 & 1 & -10 & 2 \\
	-1 & 2 & 0 & 1 \end{array} \right) 
	\]
	We gebruiken elementaire rijomvormingen om de matrix in echelonvorm te brengen. \\
	
	rij 2 - 2 rij 1 en rij 3 + rij 1 
	\[ \left(
	\begin{array}{ccc|c}
	1 & -3 & 5 & 0 \\
	0 & 7 & -20 & 2 \\
	0 & -1 & 5 & 1 \end{array} \right) 
	\]
	Verwissel rij 2 en rij 3
	\[ \left(
	\begin{array}{ccc|c}
	1 & -3 & 5 & 0 \\
	0 & -1 & 5 & 1 \\
	0 & 7 & -20 & 2  \end{array} \right) 
	\]
	rij 3 + 7 rij 2
	\[ \left(
	\begin{array}{ccc|c}
	1 & -3 & 5 & 0 \\
	0 & -1 & 5 & 1 \\
	0 & 0 & 15 & 9  \end{array} \right) 
	\]
	We lezen af op de matrix dat $rang(A)=rang \left( \begin{array}{c|c} A & C \end{array} \right)$ (het stelsel is oplosbaar) en $rang(A)=3$, het aantal onbekenden. Het stelsel heeft dus juist \'{e}\'{e}n oplossing.\\
	We lossen het stelsel van onder naar boven op: \\
	$x_3 =\frac{9}{15}=\frac{3}{5}$, substitutie in de voorlaatste vergelijking geeft $-x_2 + 5\frac{3}{5} = 1$, dus $x_2 =2$, substitutie in de eerste vergelijking geeft $x_1 -6+3=0$, dus $x_1 =3$.\\
	Het stelsel heeft als oplossing $x_1 =3, x_2 =2, x_3= \frac{3}{5}$\\ 
	
	\end{voorbeeld}

\begin{voorbeeld}
		We controleren met de methode van Gauss of het volgende stelsel oplosbaar is en we zoeken de eventuele oplossing(en.)
\[ \left\{ 
	\begin{array}{l}
	x_1 -3x_2 +5x_3  = 0 \\
	2x_1 + x_2 -10x_3 = 2 \\
	-3x_1 + 9x_2 -15x_3= 0 \end{array} 
	\right. \]
	We schrijven de verhoogde co\"{e}ffici\"{e}ntenmatrix:
	\[ \left(
	\begin{array}{ccc|c}
	1 & -3 & 5 & 0 \\
	2 & 1 & -10 & 2 \\
	-3 & 9 & -15 & 0 \end{array} \right) 
	\]
	We gebruiken elementaire rijomvormingen om de matrix in echelonvorm te brengen. \\
	
	rij 2 -2 rij 1 en rij 3 + 3 rij 1 
	\[ \left(
	\begin{array}{ccc|c}
	1 & -3 & 5 & 0 \\
	0 & 7 & -20 & 2 \\
	0 & 0 & 0 & 0
	\end{array} \right)
	\]
	We lezen af dat $rang(A)=rang \left( \begin{array}{c|c} A & C \end{array} \right)$ (het stelsel is oplosbaar) en $rang(A)=2$, dit is kleiner dan het aantal onbekenden. 
	In deze situatie heeft het stelsel een oneindig aantal oplossingen. \\
	We lossen de tweede vergelijking op naar $x_2$ waarbij we $x_3$ gelijk stellen aan een willekeurig te kiezen parameter $x_3 = \lambda \in \mathbb{R}$. Dit geeft $x_2 = \frac{2+20 \lambda}{7}$. Substitutie van $x_2$ en $x_3$ in de eerste vergelijking geeft $x_1 =\frac{25 \lambda + 6}{7}$. \\
	Aangezien $x_1, x_2, x_3$ telkens een andere oplossing geeft als men een andere waarde voor $\lambda$ kiest heeft men hier inderdaad een oneindig aantal oplossingen.


\end{voorbeeld} 

\begin{opmerking}
	\ \\
	\begin{itemize}
	\item We herhalen nog eens dat het niet noodzakelijk is dat het aantal vergelijkingen gelijk is aan het aantal onbekenden.
	\item Laten we een homogeen stelsel van vergelijkingen (een stelsel waarbij alle rechterleden nul zijn) beschouwen:
	\[ 
	\left\{ \begin{array}{l}
	a_{11} x_1 + a_{12} x_2 + ... + a_{1n} x_n = 0 \\
	a_{21} x_1 + a_{22} x_2 + ... + a_{2n} x_n = 0 \\
	\vdots \\ \vdots \\
	a_{m1} x_1 + a_{m2} x_2 + ... + a_{mn} x_n = 0
	\end{array}
	\right.
	\]
	Dergelijk stelsel is altijd oplosbaar want in dit geval geldt $rang(A) = rang \left( \begin{array}{c|c} A & C \end{array} \right)$ \\
	Het stelsel kan \'{e}\'{e}n of oneindig veel oplossingen hebben maar $x_1 =0$, $x_2 =0$, $...$ , $x_n =0$ is altijd een oplossing van een homogeen stelsel.
\end{itemize}
\end{opmerking}

\subsection{De regel van Cramer}

\subsubsection{Inleiding}

We bekijken hier het speciale geval van een niet homogeen stelsel van lineaire vergelijkingen waarbij er evenveel vergelijkingen als onbekenden zijn. 
\[ 
\left\{ \begin{array}{l}
a_{11} x_1 + a_{12} x_2 + ... + a_{1n} x_n = c_1 \\
a_{21} x_1 + a_{22} x_2 + ... + a_{2n} x_n = c_2 \\
\vdots \\ \vdots \\
a_{n1} x_1 + a_{n2} x_2 + ... + a_{nn} x_n = c_n
\end{array}
\right.
\]
In matrixnotatie wordt dit
\[ 
\left( \begin{matrix} a_{11} & \ldots & \ldots & \ldots & a_{1n} \\ \vdots & & & & \vdots \\ \vdots & & & & \vdots \\ \vdots & & & & \vdots \\ a_{n1} & \ldots & \ldots & \ldots & a_{nn} \end{matrix} \right) \left( \begin{matrix} x_1 \\ \vdots \\ \vdots \\ \vdots \\ x_n \end{matrix} \right) = \left( \begin{matrix} c_1 \\ \vdots \\ \vdots \\ \vdots \\ c_n \end{matrix} \right)
\]
en met de verhoogde co\"{e}ffici\"{e}ntenmatrix
\[
\left( \begin{array}{c|c} A & C \end{array} \right) =
\left( 
\begin{array}{ccccc|c}
a_{11} & \ldots & \ldots & \ldots & a_{1n} & c_1 \\ \vdots & & & & \vdots & \vdots \\ \vdots & & & & \vdots & \vdots \\ \vdots & & & & \vdots & \vdots \\ a_{n1} & \ldots & \ldots & \ldots & a_{nn} & c_n
\end{array} 
\right)
\]
Volgens de methode van Gauss heeft een dergelijk $n \times n$ stelsel juist \'{e}\'{e}n oplossing als en slechts als $rang(A)=n$ met $n$ het aantal onbekenden.\\
Omdat $A$ een vierkante $n \times n$ matrix is geldt
\[ rang(A)=n \Leftrightarrow det A \neq 0 \Leftrightarrow A \quad \textrm{is inverteerbaar} \]
De regel van Cramer is een manier om, onafhankelijk van de methode van Gauss, de unieke oplossing van een dergelijk $n \times n$ stelsel te vinden. \\

Pas op! De regel van Cramer is op geen enkele andere situatie toepasbaar...

\subsubsection{De regel van Cramer}

Stel dat je een niet homogeen stelsel van $n$ lineaire vergelijkingen in $n$ onbekenden $x_1$, $...$ , $x_n$ hebt waarbij de co\"{e}ffici\"{e}ntenmatrix $A$ regulier is.\\
In matrixnotatie wordt dit 
\[ AX=C \quad \textrm{met} \quad det A \neq 0  \]
Door de matrixvergelijking links te vermenigvuldigen met de inverse van $A$ vinden we
\[ A^{-1}AX=A^{-1}C \]
of
\[ X=A^{-1}C \]
Dit kan je ook schrijven als
\[ X=\frac{1}{det A}adj(A) C \]
of
\[
\left( \begin{matrix} x_1 \\ \vdots \\ \vdots \\ \vdots \\ x_n \end{matrix} \right) = \frac{1}{det A} \left( \begin{matrix} \text{cofactor}(a_{11}) & \text{cofactor}(a_{21}) & \ldots & \ldots & \text{cofactor}(a_{n1}) \\ \text{cofactor}(a_{12}) & \vdots & & & \vdots \\ \vdots & & & & \vdots \\ \vdots & & & & \vdots \\ \text{cofactor}(a_{1n}) & \ldots & \ldots & \ldots & \text{cofactor}(a_{nn}) \end{matrix} \right) \left( \begin{matrix} c_1 \\ \vdots \\ \vdots \\ \vdots \\ c_n \end{matrix} \right)
\]
Dit geeft voor elke $x_i$ met $i=1...n$ de volgende uitdrukking
\[
x_i = \frac{1}{det A} (c_1 \text{cofactor}(a_{1i}) + c_2 \text{cofactor}(a_{2i}) + ... + c_n \text{cofactor}(a_{ni})) 
\]
ofwel
\[
x_i = \frac{det A_{i}}{det A}
\]
met $i=1...n$ en $A_{i}$ de co\"{e}ffici\"{e}ntenmatrix $A$ waarin de $i$-de kolom vervangen is door de kolommatrix $C$. \\

\begin{voorbeeld}
	

Neem het volgende stelsel van $3$ vergelijkingen in $3$ onbekenden:
\[
\left\{ \begin{array}{c} 2x_1 +3x_3 =1 \\ x_2 + x_3 =4 \\ -x_1 + x_2 =3 \end{array}  
\right.
\]
We controleren of de co\"{e}ffici\"{e}ntenmatrix $A=\left( \begin{matrix} 2 & 0 & 3 \\ 0 & 1 & 1 \\ -1 & 1 & 0 \end{matrix} \right)$ regulier is.
\[ det \left( \begin{matrix} 2 & 0 & 3 \\ 0 & 1 & 1 \\ -1 & 1 & 0 \end{matrix} \right) = 1 \neq 0 \]
Dus $A$ is regulier, $A^{-1}$ bestaat.\\ 

We berekenen de unieke oplossing van het stelsel:
\[  x_1 =\frac{det A_1}{det A}=det \left( \begin{matrix} 1 & 0 & 3 \\ 4 & 1 & 1 \\ 3 & 1 & 0 \end{matrix} \right) = 2 \]
\[  x_2 =\frac{det A_2}{det A}=det \left( \begin{matrix} 2 & 1 & 3 \\ 0 & 4 & 1 \\ -1 & 3 & 0 \end{matrix} \right) = 5 \] 
\[  x_3 =\frac{det A_3}{det A}=det \left( \begin{matrix} 2 & 0 & 1 \\ 0 & 1 & 4 \\ -1 & 1 & 3 \end{matrix} \right) = -1 \]                     


\end{voorbeeld}