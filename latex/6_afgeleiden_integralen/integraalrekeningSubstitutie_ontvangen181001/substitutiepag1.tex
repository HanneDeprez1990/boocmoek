\documentclass{article}
\usepackage{amsrefs}
\usepackage{amssymb}
\usepackage{enumerate}
\usepackage{amsmath}
\usepackage{amsthm}
\usepackage{graphicx}
\usepackage{amssymb,latexsym}
\usepackage{graphicx}

\begin{document}

Aan de hand van voorbeelden zie je hoe je de kettingregel voor het afleiden kunt gebruiken om onbepaalde integralen op te lossen.
Dit gebruik van de kettingregel is de basis voor de methode van substitutie bij het berekenen van onbepaalde integralen.\\

De onbepaalde integraal $\int (x-2)^3dx$ is heel eenvoudig om uit te rekenen omdat het de onbepaalde integraal is van een veeltermfunctie.
Je moet dan wel eerst de 3-de macht uitrekenen:
\[
(x-2)^3=x^3-6x^2+12x-8 \text { .}
\]
Je bekomt dan
\[
\int (x-2)^3dx=\int (x^3-6x^2+12x-8)dx = \frac{x^4}{4}-2x^3+6x^2-8x+C \text { .}
\]

In principe kun je op dezelfde manier $\int (x-2)^{300}dx$ uitrekenen.
Maar het uitschrijven van $(x-2)^{300}$ is een hele klus.
Je leest nu hoe je door de kettingregel van het afleiden te gebruiken deze klus kunt ontwijken.\\

De kettingregel voor het afleiden is
\[
D(g(f(x)))=Dg(f(x)).Df(x) \text { .}
\]
Omdat per definitie
\[
\int D(g(f(x)))=g(f(x))+C
\]
volgt hieruit dat
\[
\int Dg(f(x))Df(x)dx =g(f(x))+C \text { .}
\]

\noindent (Het lijkt mij het beste dat dit; samen met het voorgaande; voorgedaan wordt op een filmpje)

Als je $g(x)=\frac{x^{301}}{301}$ neemt dan is $Dg(x)=x^{300}$.
Neem je dan $f(x)=x-2$ dan is $Dg(f(x))=(x-2)^{300}$.
Omdat $Df(x)=1$ bekom met deze functies $f$ en $g$
\[
\int (x-2)^{300}dx = \int Dg(f(x))Df(x)dx = g(f(x))+C=\frac{(x-2)^{301}}{301}+C \text { .}
\]

Nog een voorbeeldje: $\int \sqrt[3]{x+7} dx$.

Een functie $g(x)$ waarvoor $Dg(x)=\sqrt[3]{x}$ behoort tot $\int \sqrt[3]{x} dx=\int x^{1/3}dx=\frac{x^{4/3}}{4/3}+C$.
Hieruit volgt dat $g(x)=\frac{3 \sqrt[3]{x^4}}{4}$ een functie is waarvoor $Dg(x)=\sqrt[3 ]{x}$.
Neem je $f(x)=x+7$ dan is $Df(x)=1$ en dan bekom je met deze functies $f$ en $g$
\[
\int \sqrt[3]{x+7}dx=\int Dg(f(x))Df(x)dx = g(f(x))+C=\frac{3\sqrt[3]{(x+7)^4}}{4}+C \text { .}
\]

Nog een laatste voorbeeldje in dit inleidend deel: $\int \sin(2x-5)dx$.

Een functie $g(x)$ waarvoor $Dg(x)=sin(x)$ is $g(x)=-cos(x)$.
Neem je $f(x)=2x-5$ dan is $Df(x)=2$.
Schrijf daarom de integraal als volgt:
\[
\int \sin(2x-5)dx=\int \sin(2x-5).\frac{1}{2}.2dx=\frac{1}{2} \int \sin(2x-5)2dx \text { .}
\]
Je bekomt met die functies $f$ en $g$ dat
\[
\int \sin(2x-5)2dx = \int Dg(f(x)).Df(x)dx=g(f(x))+C=-\cos(2x-5)+C \text { .}
\]
\vspace{2mm}

Je merkt aan deze voorbeelden dat de kettingregel van het afleiden kan helpen om onbepaalde integralen op te lossen.
Deze werkwijze wordt duidelijker als we dit herleiden tot de methode van substitutie.
Om deze methode te verantwoorden gebruiken we differentialen die we op de volgende pagina eerst defini\"eren.
Je kunt los van de definitie van een differentiaal hier ook louter formeel mee rekenen.




\end{document}