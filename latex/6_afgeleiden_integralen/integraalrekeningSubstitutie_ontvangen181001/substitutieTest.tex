\documentclass{article}
\usepackage{amsrefs}
\usepackage{amssymb}
\usepackage{enumerate}
\usepackage{amsmath}
\usepackage{amsthm}
\usepackage{graphicx}
\usepackage{amssymb,latexsym}
\usepackage{graphicx}

\newtheorem{opgave}{Opgave}
\newtheorem*{oplossing}{Oplossing}

\title{Test Integraalrekening Substitutiemethode}
\date { }

\begin{document}


\maketitle \noindent

\begin{opgave}
Welke substitutie gebruik je om $\int \frac{x^5}{\sqrt{1-x^3}}dx$ te herleiden tot een integraal die oplosbaar is zonder nogmaals substitutie te moeten toepassen?

\vspace{2mm}
\begin{enumerate}[a]
\item $t=x^3$
\item $t=x^5$
\item $t=1-x^3$
\end{enumerate}
\end{opgave}

 \begin{oplossing}
c
\end{oplossing}

\begin{opgave}
Welke substitutie gebruik je om $\int x^4 \sin (x^5)dx$ te herleiden tot een integraal die oplosbaar is zonder nogmaals substitutie te moeten toepassen?

\begin{enumerate}[a]
\item $t=x^4$
\item $t=x^5$
\item $t=\sin(x^5)$
\end{enumerate}
\end{opgave}

\begin{oplossing}
b
\end{oplossing}

\begin{opgave}
Welke substitutie gebruik je om $\int \frac{\cos x}{1-\sin x}dx$ te herleiden tot een integraal die oplosbaar is zonder nogmaals substitutie te moeten toepassen?

\begin{enumerate}[a]
\item $u=\sin x$
\item $u=\cos x$
\item $u=1-\sin x$
\item $u=\frac{1}{\sin x}$
\end{enumerate}

\end{opgave}

\begin{oplossing}
c
\end{oplossing}

\begin{opgave}
Welke substitutie gebruik je om $\int \frac{x^5}{1+x^4}dx$ te herleiden tot een integraal die oplosbaar is zonder nogmaals substitutie te moeten toepassen?

\begin{enumerate}[a]
\item $v=1+x^4$
\item $v=x^2$
\item $v=x^4$
\item $v=\sqrt{1+x^4}$
\end{enumerate}

\end{opgave}

\begin{oplossing}
b
\end{oplossing}

\begin{opgave}
Als je voor het oplossen van $\int x^3\sqrt {1-x^2}dx$ substitutie $u=1-x^2$ gebruikt, welke integraal bekom je dan?

\begin{enumerate}[a]
\item $-\frac{1}{2} \int u \sqrt{u} du$
\item $-\frac{1}{2} \int (1-u)\sqrt{u} du$
\item $-2\int (1-u) \sqrt{u}du$
\item $-2 \int u \sqrt{u}du$
\end{enumerate}
\end{opgave}

\begin{oplossing}
b
\end{oplossing}

\begin{opgave}
Als je voor het oplossen van $\int \frac{e^{2\arctan x}}{1+x^2}dx$ substitutie $u=\arctan x$ gebruikt, welke integraal bekom je dan?

\begin{enumerate}[a]
\item $2 \int udu$
\item $\int e^{2u} du$
\item $\frac{1}{2} \int e^{2u}du$
\item $2 \int e^udu$
\end{enumerate}

\end{opgave}

\begin{oplossing}
b
\end{oplossing}

\begin{opgave}
$\int \frac{xdx}{\sqrt[3]{1-x^2}}=a\sqrt[3]{(1-x^2)^b}+C$ Wat zijn de waarden van $a$ en $b$?\vspace{3mm}

\noindent $a=\cdots $, $b= \cdots$\vspace{1mm}

\noindent Een aantal van deze getallen kunnen 0 en/of 1 zijn. Je mag enkel gehele getallen of breuken van gehele getallen ingeven en je moet zoveel mogelijk vereenvoudigen. Bij een breuk die negatief is plaats je het minteken in de teller.

\begin{oplossing}
$a=\frac{-3}{4}$; $b=2$
\end{oplossing}


\end{opgave}

\begin{opgave}
$\int x^2 \cos (4x^3)dx=a.f(bx^3)+C$ Staat $f$ voor de functie $\sin$ of $\cos$? Wat zijn de waarden van $a$ en $b$?\vspace{3mm}

\noindent $f \text { is } \cdots $; $a=\cdots $; $b=\cdots $.\vspace{1mm}

\noindent Een aantal van deze getallen kunnen 0 en/of 1 zijn. Je mag enkel gehele getallen of breuken van gehele getallen ingeven en je moet zoveel mogelijk vereenvoudigen. Bij een breuk die negatief is plaats je het minteken in de teller.
\end{opgave}

\begin{oplossing}
$f \text { is } \sin$; $a=\frac{1}{12}$; $b=4$
\end{oplossing}

\begin{opgave}
$\int x^2 \sqrt[3]{1+5x}dx=a\sqrt[3]{(1+5x)^d}+b\sqrt[3]{(1+5x)^e}+c\sqrt[3]{(1+5x)^f}+C$ Wat zijn de waarden van $a$,$b$,$c$,$d$,$e$ en $f$ met $d>e>f$?\vspace{3mm}

\noindent $a= \cdots$; $b= \cdots$; $c= \cdots$; $d= \cdots$; $e=\cdots$ ; $f= \cdots$\vspace{1mm}

\noindent Een aantal van deze getallen kunnen 0 en/of 1 zijn. Je mag enkel gehele getallen of breuken van gehele getallen ingeven en je moet zoveel mogelijk vereenvoudigen. Bij een breuk die negatief is plaats je het minteken in de teller.
\end{opgave}

\begin{oplossing}
$a=\frac{3}{1250}$; $b=\frac{-6}{875}$; $c=\frac{3}{500}$; $d=10$; $e=7$; $f=4$
\end{oplossing}

\begin{opgave}
$\int \frac{x^3dx}{\sqrt{1-x^2}}=a\sqrt{(1-x^2)^c}+b\sqrt{(1-x^2)^d}+C$ Wat zijn de waarden van $a$; $b$; $c$ en $d$ met $c<d$?\vspace{3mm}

\noindent $a= \cdots$; $b= \cdots$; $c= \cdots$; $d= \cdots$ \vspace{1mm}

\noindent Een aantal van deze getallen kunnen 0 en/of 1 zijn. Je mag enkel gehele getallen of breuken van gehele getallen ingeven en je moet zoveel mogelijk vereenvoudigen. Bij een breuk die negatief is plaats je het minteken in de teller.
\end{opgave}

\begin{oplossing}
$a=-1$; $b=\frac{1}{3}$; $c=1$; $d=3$
\end{oplossing}

\begin{opgave}
Om $\int ^2_0 \frac{dx}{\sqrt[3]{1+4x}}$ te berekenen gebruik je de substitutie $t=1+4x$.
Je bekomt in de veranderlijke $t$ als integraal $\frac{1}{4} \int ^b_a \frac{dt}{\sqrt[3]{t}}$. Wat zijn $a$ en $b$?\vspace{3mm}

\noindent $a= \cdots$; $b=\cdots$\vspace{1mm}

\noindent Een aantal van deze getallen kunnen 0 en/of 1 zijn. Je mag enkel gehele getallen of breuken van gehele getallen ingeven en je moet zoveel mogelijk vereenvoudigen. Bij een breuk die negatief is plaats je het minteken in de teller.
\end{opgave}

\begin{oplossing}
$a=1$; $b=9$
\end{oplossing}

\begin{opgave}
Om $\int ^2_1 \frac{xdx}{1+x^4}$ te berekenen gebruik je de substitutie $u=x^2$.
Je bekomt in de veranderlijke $u$ als integraal $\frac{1}{2} \int ^b_a \frac{du}{1+u^2}$. Wat zijn $a$ en $b$?\vspace{3mm}

\noindent $a= \cdots$; $b=\cdots$\vspace{1mm}

\noindent Een aantal van deze getallen kunnen 0 en/of 1 zijn. Je mag enkel gehele getallen of breuken van gehele getallen ingeven en je moet zoveel mogelijk vereenvoudigen. Bij een breuk die negatief is plaats je het minteken in de teller.
\end{opgave}

\begin{oplossing}
$a=1$; $b=4$
\end{oplossing}















\end{document}