\documentclass{article}
\usepackage{amsrefs}
\usepackage{amssymb}
\usepackage{enumerate}
\usepackage{amsmath}
\usepackage{amsthm}
\usepackage{graphicx}
\usepackage{amssymb,latexsym}
\usepackage{graphicx}

\begin{document}

\begin{enumerate}

\item

Stel je $t=1-x^3$ dan is $dt=-3x^2dx$.
Je splitst $x^5$ op in het product $x^3.x^2$.
De factor $x^2$ gebruik je om $x^2dx$ te vervangen door $-\frac{dt}{3}$.
De factor $x^3$ vervang je door $1-t$.
Je bekomt
\[
-\frac{1}{3}\int \frac{(1-t)dt}{\sqrt{t}}
\]
en deze integraal kun je door te splitsen in twee machten van $t$ eenvoudig oplossen.\\

Stel je $t=x^3$ dan bekom je
\[
\frac{1}{3}\int \frac{(1-t)dt}{\sqrt{1-t}}
\]
en dan ga je nogmaals een substitutie $ u=1-t$ nodig hebben.\\

Stel je $t=x^5$ dan bekom je $dt=5x^4dx$.
Je moet in de teller dan $x^5$ opsplitsen in het product $x.x^4$.
De factor $x^4$ gebruik je om $x^4dx$ te vervangen door $\frac{dt}{5}$.
De factor $x$ moet je dan vervangen door $\sqrt[5]{x}$.
In de noemer vervang je $1-x^3$ door $1-\left( \sqrt[5]{t}  \right)^3$.
Je bekomt
\[
\frac{1}{5} \int \frac{\sqrt[5]{t}dt}{\sqrt {1-\left(\sqrt[5]{t}  \right)^3}}
\]
wat niet zonder substitutie oplosbaar is.

\item

Stel je $t=x^5$ dan bekom je $dt=5x^4dx$.
Je bekomt dan de integraal
\[
\frac{1}{5} \int sin (t)dt
\]
waarvan je de oplossing kent.\\

Stel je $t=x^4$ dan is $dt=4x^3dx$.
Je gaat dan $x^4$ opsplitsen in het product $x.x^3$.
De factor $x^3$ gebruik je om $x^3dx$ te vervangen door $\frac {dt}{4}$.
Je vervangt dan $\sin \left( x^5 \right)$ door$\sin \left( t.\sqrt[4]{t} \right)$.
Je bekomt
\[
\frac{1}{4} \int \sqrt[4]{t}\sin \left( t.\sqrt[4]{t} \right)dt
\]
en deze integraal is niet oplosbaar zonder substitutie te gebruiken.\\

Stel je $t=\sin \left( x^5 \right)$ dan is $dt=\cos \left( x^5 \right)5x^4 dx$.
Schrijf je $\sin \left( x^5 \right)$ als $\tan \left( x^5 \right).\cos \left( x^5 \right)$ dan bekom je als integraal $\frac{1}{5} \int \tan (t)dt$.
Alhoewel je deze integraal in veel formularia kunt tegenkomen staat deze integraal niet in het lijstje van de bijzondere integralen (dat je bekomt door de belangrijkste functies af te leiden).
Om deze integraal te herleiden naar zulke bijzondere integraal heb je substitutie nodig.

\item

Stel je $u=1-\sin x$ dan is $du=-\cos x dx$.
Je bekomt dan de integraal
\[
-\int \frac{du}{u}
\]
welke een basisintegraal is.\\

Stel je $u=\sin x$ dan is $du=\cos x dx$.
Je bekomt dan de integraal
\[
-\int \frac {du}{1-u}
\]
waarvoor je nogmaals substitutie moet toepassen.\\

Stel je $u=\cos x$ dan bekom je $du=-\sin x dx$ Het is onmogelijk om hiermee een eenvoudig voorschrift te vinden voor de integraal naar $u$.\\

Stel je $u=\frac{1}{\sin x}$ dan bekom je $du =-\frac {\ cos x}{\sin ^2 x}dx$
Je herschrijft $\frac{\cos x}{1- \sin x}$ als het product $\frac{\sin^2 x}{1- \sin x}.\frac {cos x}{\sin ^2 x}$.
De tweede factor gebruik je om $\frac{\cos x}{\sin ^2 x}dx$ te vervangen door $-du$.
In de eerste factor vervang je $\sin x$ door $\frac{k1}{u}$ en na vereenvoudigen bekom je $\frac {1}{u^2-u}$.
Je bekomt de integraal
\[
-\int \frac {du}{u^2-u}
\]
waarvoor nog bijzondere integratietechnieken gebruikt moeten worden om te herleiden tot een basisintegraal.

\item

Stel je $v=x^2$ dan is $dv=2xdx$.
De teller $x^5$ splits je op in $\left( x^2) \right)^2.x$.
De tweede factor gebruik je om $xdx$ te vervangen door $\frac{dv}{2}$.
De eerste factor vervang je door $v^2$ en de noemer $1+x^4$ vervang je door $1+v^2$.
Je bekomt de integraal
\[
\frac{1}{2} \int \frac{t^2dt}{1+t^2} \text {.}
\]
Schrijf je de teller $t^2$ als $(1+t^2)-1$ dan splits je de integraal op in twee integralen die je kent:
\[
\frac{1}{2} \left( \int dt - \int \frac{dt}{1+t^2} \right) \text { .}
\]\\

Stel je $v=1+x^4$ dan bekom je $dv=4x^3dx$.
De teller splits je dan op in een product $x^5=x^2.x^3$.
De tweede factor gebruik je om $x^3dx$ te vervangen door $\frac{dv}{4}$.
De eerste factor vervang je door $\sqrt{v-1}$.
Je bekomt dan de integraal
\[
\frac{1}{4} \int \frac{\sqrt {v-1}}{v}dv
\]
die je niet kunt oplossen zonder substitutie te gebruiken.\\

Stel je $v=x^4$ dan bekom je $dv=4x^3dx$.
Met een soortgelijke opsplitsing van de teller als een produkt bekom je de integraal
\[
\frac{1}{4} \int \frac{\sqrt{v}}{1+v}dv
\]
die je eveneens niet kunt oplossen zonder substitutie te gebruiken.\\

Stel je $v=\sqrt {1+x^4}$ dan is $dv=\frac {1}{2\sqrt{1+x^4}}.4x^3dx$.
Schrijf $\frac{x^5}{1+x^4}$ als het product $\left( \frac{x^2}{\sqrt{1+x^4}}  \right).\left(  \frac{x^3}{\sqrt{1+x^4}} \right)$.
De tweede factor gebruik je om $\frac{x^3dx}{\sqrt{1+x^4}}$ te vervangen door $\frac{dv}{2}$.
In de eerste factor vervang je de teller $x^2$ door $\sqrt{v^2-1}$ en dus de eerste factor in zijn geheel door $\frac{\sqrt{v^2-1}}{v}$.
Je bekomt de integraal
\[
\frac{1}{2} \int \frac{\sqrt{v^2-1}}{v}dv
\]
die je niet kunt oplossen zonder opnieuw substitutie te gebruiken.

\item
Je bekomt $du=-2xdx$.
Je splitst $x^3$ op in een product $x^2.x$.
Je gebruikt de tweede factor om $xdx$ te vervangen door $-\frac{du}{2}$.
De factor $x^2$ vervang je door $1-u$.
Je bekomt de integraal
\[
-\frac{1}{2} \int (1-u) \sqrt {u}du \text { .}
\]

\item
Je bekomt $du=\frac{1}{1+x^2}dx$.
Je bekomt dan onmiddelijk de integraal $\int e^{2u}du$.

\item
Je gebruikt de substitutie $t=1-x^2$.
Dan is $dt=-2xdx$.
Met de variabele $t$ bekom je dan
\[
-\frac{1}{2} \int \frac {dt}{\sqrt[3]{t}}=-\frac{1}{2}\int t^{-1/3}dt=-\frac{1}{2}\frac {t^{2/3}}{2/3}+C \text { .}
\]
Je bekomt als oplossing $-\frac{3}{4}\sqrt[3]{\left( 1-x^2  \right)^2}+C$.

\item
Je gebruikt de substitutie $t=4x^3$.
Dan is $dt=12x^2dx$.
Met de variabele $t$ bekom je dan
\[
\frac{1}{12} \int \cos tdt=\frac{1}{12} \sin t+C \text { .}
\]
Je bekomt als oplossing $\frac{1}{12} \sin \left( 4x^3 \right)+C$.

\item
Je gebruikt de substitutie $t=1+5x$.
Dan is $dt=5dx$ en je gaat $dx$ vervangen door $\frac{dt}{5}$.
Omdat $x=\frac {t-1}{5}$ vervang je $x^2$ door $\frac {(t-1)^2}{25}=\frac{1}{25}\left(  t^2-2t+1 \right)$.
Je bekomt als integraal in de veranderlijke $t$:
\[
\frac{1}{125}\int \left(  t^2-2t+1 \right)\sqrt[3]{t}dt=\frac{1}{125}\int \left(  t^{7/3}-2t^{4/3}+t^{1/3}  \right)dt=
\]
\[
=\frac{1}{125}\left(  \frac{t^{10/3}}{10/3}-2 \frac{t^{7/3}}{7/3}+\frac{t^{4/3}}{4/3}  \right) +C \text { .}
\]
Je bekomt als oplossing
\[
\frac{3}{1250}\sqrt[3]{(1+5x)^{10}}-\frac{6}{875}\sqrt[3]{(1+5x)^7}+\frac{3}{500}\sqrt[3]{(1+5x)^4}+C \text { .}
\]

\item
Je gebruikt de substitutie $t=1-x^2$.
Dan is $dt=-2xdx$.
Je splitst de teller $x^3$ op in een product $x^2.x$.
Je gebruikt de tweede factor om $xdx$ te vervangen door $-\frac{dt}{2}$.
De eerste factor $x^2$ vervang je door $1-t$.
Met de variabele $t$ bekom je dan
\[
-\frac {1}{2} \int \frac {(1-t)dt}{\sqrt {t}}=-\frac{1}{2} \int \left( t^{-1/2}-t^{1/2}  \right)dt=-\frac{1}{2} \left( \frac{t^{1/2}}{1/2}-\frac{t^{3/2}}{3/2}  \right)+C \text { .}
\]
Je bekomt als oplossing $-\sqrt {1-x^2} +\frac{1}{3} \sqrt{\left( 1-x^2  \right)^3}+C$.

\item
Als $x=0$ dan is $t=1+4.0=1$. Dus $a=1$.

\noindent Als $x=2$ dan is $t=1+4.2=9$. Dus $b=9$.

\item
Als $x=1$ dan is $u=1^2=1$. Dus $a=1$.

\noindent Als $x=2$ dan is $u=2^2=4$. Dus $b=4$.


\end{enumerate}

\end{document}