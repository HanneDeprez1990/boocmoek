\documentclass{article}
\usepackage{amsrefs}
\usepackage{amssymb}
\usepackage{amsmath}
\usepackage{amsthm}
\usepackage{graphicx}

\DeclareMathOperator{\dom}{dom}
\newtheorem*{definitie}{Definitie} \newtheorem*{notatie}{Notatie} \newtheorem*{voorbeeld}{Voorbeeld} \newtheorem*{eigenschap}{Eigenschap}
\newtheorem{opgave}{Opgave}
\newtheorem*{oplossing}{Oplossing}

\title{Test Integraalrekenen}
\date { }

\begin{document}

\maketitle \noindent

\begin{opgave}
$\int (6x^3-9x^2+11x+3)dx=ax^4+bx^3+cx^2+dx+C$. Wat zijn de waarden van $a$, $b$, $c$, en $d$?\vspace{2mm}

\noindent $a= \cdots$, $b=\cdots$, $c=\cdots$, $d=\cdots$\vspace{1mm}

\noindent Je mag enkel gehele getallen of breuken van gehele getallen ingeven en je moet zoveel mogelijk vereenvoudigen. Bij een negatieve breuk schrijf je het - teken voor de breuk. Een aantal van deze getallen kunnen 0 en/of 1 zijn.
\end{opgave}

\begin{oplossing}
$a=\frac{2}{3}$, $b=-3$, $c=\frac{11}{2}$, $d=3$
\end{oplossing}

\begin{opgave}
$\int (2x-1)^2 dx = ax^4+bx^3+cx^2+dx+C$. Wat zijn de waarden van $a$, $b$, $c$, en $d$?\vspace{2mm}

\noindent $a= \cdots$, $b=\cdots$, $c=\cdots$, $d=\cdots$, \vspace{1mm}

\noindent Je mag enkel gehele getallen of breuken van gehele getallen ingeven en je moet zoveel mogelijk vereenvoudigen. Bij een negatieve breuk schrijf je het - teken voor de breuk. Een aantal van deze getallen kunnen 0 en/of 1 zijn.
\end{opgave}

\begin{oplossing}
$a=0$, $b=\frac{4}{3}$, $c=-2$, $d=1$
\end{oplossing}

\begin{opgave}
$\int \sqrt[7]{5x^4}dx = a \sqrt[b]{x^c} +C$.
Hierin zijn $b$ en $c$ gehele getallen.

\noindent Welke van de volgende uitdrukkingen is $a$?

\begin{enumerate}
\item $\frac{11}{7} \sqrt[7]{5}$
\item $\frac{7}{11}$
\item $\frac{7}{11} \sqrt[7]{5}$
\item $\frac{7}{4}$
\item $\frac{7}{4}\sqrt[7]{5}$
\end{enumerate}

\noindent Wat zijn de waarden van $b$ en $c$?

\noindent $b=\cdots$, $c=\cdots$ \vspace{1mm}


\end{opgave}

\begin{oplossing}
Antwoord voor a is mogeljkheid 3

\noindent $b=7$, $c=11$
\end{oplossing}

\begin{opgave}
Als $\int f(x)dx = 3x^3+2x^2+C$ dan is $f(x)=ax^4+bx^3+cx^2+dx+e$. Wat zijn de waarden van $a$, $b$, $c$, $d$ en $e$?\vspace{2mm}

\noindent $a= \cdots$, $b=\cdots$, $c=\cdots$, $d=\cdots$, $e=\cdots$ \vspace{1mm}

\noindent Je mag enkel gehele getallen of breuken van gehele getallen ingeven en je moet zoveel mogelijk vereenvoudigen. Bij een negatieve breuk schrijf je het - teken voor de breuk. Een aantal van deze getallen kunnen 0 en/of 1 zijn.
\end{opgave}

\begin{oplossing}
$a=0$, $b=0$, $c=9$, $d=4$, $e=0$
\end{oplossing}

\begin{opgave}
Juist of fout?

\[
\int x \sin x dx =-\frac{x^2}{2} \cos x +C
\]
\end{opgave}

\begin{oplossing}
fout
\end{oplossing}

\begin{opgave}
Juist of fout?

\[
\int (e^{lnx}+x)dx = x^2 +C
\]
\end{opgave}

\begin{oplossing}
juist
\end{oplossing}

\begin{opgave}
Juist of fout?

Als $\int f(x)dx=F(x)+C$ dan is $\int (F(x)+xf(x))dx = xF(x)+C$.
\end{opgave}

\begin{oplossing}
juist
\end{oplossing}

\begin{opgave}
$\int ^5_2 (3x^2+1)dx = \cdots$\vspace{1mm}

\noindent Je mag enkel een geheel getal of breuk van gehele getallen ingeven en je moet zoveel mogelijk vereenvoudigen. Bij een negatieve breuk schrijf je het - teken voor de breuk.
\end{opgave}

\begin{oplossing}
120
\end{oplossing}

\begin{opgave}
$\int ^{27}_{8} \frac{dx}{\sqrt[3]{x}} = \cdots $\vspace{1mm}

\noindent Je mag enkel een geheel getal of breuk van gehele getallen ingeven en je moet zoveel mogelijk vereenvoudigen. Bij een negatieve breuk schrijf je het - teken voor de breuk.
\end{opgave}

\begin{oplossing}
$\frac{15}{2}$
\end{oplossing}

\begin{opgave}
Welke van de volgende getallen is de waarde van $\int^{\frac{\pi}{4}}_{\frac{\pi}{6}} \frac{dx}{1+x^2}$?

\begin{enumerate}
\item $\frac{\pi}{24}+\frac{2}{\pi}$
\item $\frac{\sqrt{3}-1}{\sqrt {3}}$
\item $\ln \left( 1+\frac{\pi ^2}{16} \right)- \ln \left( 1+ \frac{\pi ^2}{36} \right)$
\item $\frac{2}{3}$
\item $\frac{16}{16+\pi ^2}-\frac{36}{36+\pi ^2}$
\end{enumerate}
\end{opgave}

\begin{oplossing}
Antwoord 2
\end{oplossing}

\begin{opgave}
Als $\int ^4_3 f(x)dx = 7$ en $\int ^9_3 f(x)dx = 20$ dan is $\int ^4_9 3f(x) dx = \cdots$.\vspace{1cm}

\noindent Je mag enkel een geheel getal of breuk van gehele getallen ingeven en je moet zoveel mogelijk vereenvoudigen. Bij een negatieve breuk schrijf je het - teken voor de breuk.
\end{opgave}

\begin{oplossing}
-39
\end{oplossing}

\begin{opgave}
De oppervlakte van het vlakdeel gelegen tussen de rechten met vergelijking $y=2x+1$ en $y=-3x+6$ en de vertikale rechte $x=3$ is gelijk aan $\cdots$.\vspace{1cm}

\noindent Je mag enkel een geheel getal of breuk van gehele getallen ingeven en je moet zoveel mogelijk vereenvoudigen. Bij een negatieve breuk schrijf je het - teken voor de breuk.
\end{opgave}

\begin{oplossing}
10
\end{oplossing}

\begin{opgave}
Wat is de oppervlakte van het deel van het vlak begrensd door de grafiek van de functies $y=x^2+1$ en $y=19-x^2$.

\noindent Je mag enkel een geheel getal of breuk van gehele getallen ingeven en je moet zoveel mogelijk vereenvoudigen. Bij een negatieve breuk schrijf je het - teken voor de breuk.
\end{opgave}

\begin{oplossing}
72
\end{oplossing}

\end{document}