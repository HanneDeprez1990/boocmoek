\documentclass{article}
\usepackage{amsrefs}
\usepackage{amssymb}
\usepackage{amsmath}
\usepackage{amsthm}
\usepackage{graphicx}

\DeclareMathOperator{\unit}{u} \DeclareMathOperator{\comb}{comb}
\DeclareMathOperator{\tri}{tri} \DeclareMathOperator{\rect}{rect}
\DeclareMathOperator{\sgn}{sgn} \DeclareMathOperator{\ramp}{ramp}
\DeclareMathOperator{\sinc}{sinc}

\title{Primitieve functies en onbepaalde integralen}
\date { }

\begin{document}

\maketitle \noindent

Uit de Actimathcursus overnemen wat binnen 1.1.1 op bladzijde 3 staat behalve de laatste zin (over het begrip integreerbaar).\\

Daarnna wat binnen 1.1.2 staat (maar geen aparte titel) alles voor het kader uitgezonderde de laatste zin die vervangen wordt door de volgende zin.

\noindent Omgekeerd kan men aantonen dat functies die dezelfde afgeleide hebben op een interval $I$ een constant verschil op dat interval $I$ hebben.\\

In het kader zelf (en dat mag dan ook in een kader staan) de eerste zin maar de tweede zin veranderen in de volgende zin.

\noindent Als $F(x)$ een primitieve functie van f(x) is dan schrijft men

\[
\int f(x)dx = F(x)+C
\]\\

En daarna die 4 "notaties" (ik zou eerder over terminologie spreken).




\end{document}