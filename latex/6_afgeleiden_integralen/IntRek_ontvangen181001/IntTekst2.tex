\documentclass{article}
\usepackage{amsrefs}
\usepackage{amssymb}
\usepackage{amsmath}
\usepackage{amsthm}
\usepackage{graphicx}

\newtheorem*{voorbeeld}{Voorbeeld}

\title{Bijzonderde onbepaalde integralen}
\date { }

\begin{document}

\maketitle \noindent

\noindent Iedere afgeleide van een bijzondere functie geeft aanleiding tot een bijzondere onbepaalde integraal.

\begin{voorbeeld} Omdat $D(\sin x)=\cos x$ is $\int \cos x dx = \sin x +C$.
\end{voorbeeld}

\begin{voorbeeld} Omdat $Dx^n=nx^{n-1}$ is $\int nx^{n-1}dx = x^n +C$.

\noindent Als je dit wat herwerkt bekom je $D \left( \frac{x^{m+1}}{m+1} \right) = x^m$ en dus $\int x^mdx=\frac{x^{m+1}}{m+1} + C$.
Deze onbepaalde integraal geldt voor alle $m \neq -1$.
Die laatste voorwaarde komt door de noemer $m+1$ die niet 0 mag zijn.
\end{voorbeeld}

\begin{voorbeeld} Omdat $D(\ln\vert x \vert)=\frac{1}{x}$ is $\int \frac{dx}{x} = \ln \vert x \vert +C$.

\noindent Vanwege $\frac{1}{x} = x^{-1}$ staat hier ook een uitkomst voor $\int x^m dx$ als $m=-1$, namelijk $\int x^{-1} dx = \ln \vert x \vert +C$.
\end{voorbeeld}

\noindent Op deze wijze ontstaat de volgende lijst van bijzondere onbepaalde integralen.

\noindent Hierop volgt de lijst uit de Actimath-cursus in 1.3.


\end{document}