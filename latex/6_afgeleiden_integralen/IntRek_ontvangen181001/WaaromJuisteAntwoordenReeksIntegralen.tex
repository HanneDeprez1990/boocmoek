\documentclass{article}
\usepackage{amsrefs}
\usepackage{amssymb}
\usepackage{enumerate}
\usepackage{amsmath}
\usepackage{amsthm}
\usepackage{graphicx}
\usepackage{amssymb,latexsym}
\usepackage{graphicx}

\begin{document}

\begin{enumerate}

\item

\[
\int \left(  6x^3-9x^2+11x+3  \right)dx = 6 \int x^3dx -9\int x^2dx +11 \int xdx +3 \int dx =
\]
\[
=6 \frac{x^4}{4}-9\frac{x^3}{3}+11\frac{x^2}{2}+3x+C=\frac{3}{2}x^4+(-3)x^3+\frac{11}{2}x^2+3x+C
\]

Vergelijken met $ax^4+bx^3+cx^2+dx+C$ geeft

$a=\frac{3}{2}$; $b=-3$; $c=\frac{11}{2}$; $d=2$.

\item

\[ 
\int (2x-1)^2dx = \int \left(  4x^2-4x+1 \right)dx=4 \int x^2dx - 4 \int xdx +\int dx=
\]
\[
=4 \frac{x^3}{3}-4\frac{x^2}{2}+x+C=\frac{4}{3}x^3+(-2)x^2+x+C
\]

Vergelijken met $ax^4+bx^3+cx^2+dx+C$ geeft

$a=0$; $b=\frac{4}{3}$; $c=-2$; $d=1$.

\item

\[
\int \sqrt[7]{5x^4}dx=\int \sqrt[7]{5}.\sqrt[7]{x^4}dx=\sqrt[7]{5} \int x^{4/7}dx=
\]
\[
=\sqrt[7]{5}.\frac{x^{11/7}}{11/7}+C=\frac{7}{11}\sqrt[7]{5}\sqrt[7]{x^{11}}+C
\]

Vergelijkgen met $a\sqrt[b]{x^c}+C$ geeft

$a=\frac{7}{11}\sqrt[7]{5}$; $b=7$; $c=11$.

\item

$\int f(x)dx=3x^3+2x^2+C$ betekent dat de functie $y=3x^3+2x^2$ een primitieve functie is van $f$.
Dus
\[
f(x)=D \left(  3x^3+2x^2 \right)=9x^2+4x \text { .}
\]

Vergelijken met $ax^4+bx^3+cx^2+dx+e$ geeft

$a=b=e=0$; $c=9$; $d=4$.

\item

Als $\int x \sin x dx=-\frac{x^2}{2} \cos x +C$ dan zou $y=-\frac{x^2}{2}\cos x$ een primitieve functie zijn van $y=x \sin x$, dus dan zou $D\left(  -\frac{x^2}{2}\cos x \right)=x \sin x$.

Uit de productregel voor de afgeleide volgt
\[
D\left(  -\frac{x^2}{2}\cos x \right)=-\left( D \left(  \frac{x^2}{2} \right)\cos x + \frac{x^2}{2} D (\cos x)   \right)=-\left( \frac{2x}{2}\cos x-\frac{x^2}{2} \sin x   \right)=-x\cos x-\frac{x^2}{2} \sin x
\]

Maar $x \sin x \neq -x\cos x-\frac{x^2}{2} \sin x$.
Bijvoorbeeld als $x=\frac{\pi}{2}$ dan is
\[
\frac{\pi}{2}\sin \left(  \frac{\pi}{2} \right)=\frac{\pi}{2}
\]
\[
-\frac{\pi}{2}\cos \left(  \frac{\pi}{2} \right)-\frac{\left( \frac{\pi}{2}  \right)^2}{2}\sin \left( \frac{\pi}{2}  \right)=-\frac{\pi ^2}{8}
\]
en $\frac{\pi}{2} \neq -\frac{\pi ^3}{8}$.

Bij het kiezen van het antwoord JA gebruik je allicht dat de integraal van een product het product van integralen is.
Immers $\int xdx=\frac{x^2}{2}+C$ en $\sin x dx=-\cos x+C$.

Maar dat is geen correcte rekenregel voor het integreren.

\item

Denk eraan dat per definitie van de functie $\ln$ geldt $e^{\ln x}=x$.
Dus $\int \left( e^{\ln x}+x \right) dx=\int 2xdx=x^2+C$.

\item

Uit $\int f(x)dx=F(x)+C$ volgt dat $DF(x)=f(x)$.

Dan is $D(xF(x))=Dx.F(x)+x.DF(x)=F(x)+xf(x)$.

Dus $xF(x)$ is een primitieve functie van $F(x)+xf(x)$ en dus $\int \left( F(x)+xf(x)  \right)dx=xF(x)+C$.

\item

$\int \left( 3x^2+1 \right) dx = 3\frac{x^3}{3}+x+C=x^3+x+C$.

Dus $\int _2^5 \left( 3x^2+1  \right)dx=\left(  x^3+x  \right)\vert ^5_2 =\left( 5^3+5  \right)-\left(  2^3+2 \right)=(125+5)-(8+2)=120$.

\item

$\int \frac{dx}{\sqrt[3]{x}} = \int x^{-1/3}dx=\frac{x^{2/3}}{2/3}+C=\frac{3\sqrt[3]{x^2}}{2}+C$.

Dus $\int_8^{27} \frac{dx}{\sqrt[3]{x}} =\frac{3\sqrt[3]{x^2}}{2} \vert_8^{27}=\frac{3}{2} \left(  \sqrt[3]{27^2}-\sqrt[3]{8^2} \right)=\frac{3}{2} (9-4)=\frac{15}{2}$.

\item

$\int \frac{dx}{1+x^2} = \arctan x+C$.

Dus $\int_{\pi /6}^{\pi /4}  \frac{dx}{1+x^2}=\arctan x \vert _{\pi /6}^{\pi /4}=\arctan \left( \frac{\pi}{4}  \right) - \arctan \left( \frac{\pi}{6}  \right)=1-\frac{1}{\sqrt{3}}=\frac{\sqrt{3}-1}{\sqrt{3}}$.

\item

\[
\int^4_9 3f(x)dx=3\left(  \int^4_9 f(x)dx \right)=3\left(  \int^3_9 f(x)dx + \int^4_3 f(x)dx  \right)=
\]
\[
=3\left( - \int^9_3 f(x)dx + \int^4_3 f(x)dx  \right)=3 (-20+ 7)=3.(-13)=-39 \text { .}
\]
 
\item

Een tekening geeft het volgende:

\begin{figure}[h]
\begin{center}
\includegraphics[height=8 cm]{Opgave12ReeksInt.pdf}
\caption{figuur bij opgave 12}
\end{center}
\end{figure}

Om de $x$-co\"ordinaat van het punt $P$ te vinden los je de vergelijking $2x+1=-3x+6$ op.
Je bekomt hieruit $5x=5$, dus $x=1$.

De oppervlakte is daarom
\[
\int^3_1 \left( (2x+1)-(-3x+6)  \right)dx
\]
(hoogste grafiek - laagste grafiek).

\[
\int ^3_1 (5x-5)dx=\left( \frac{5x^2}{2}-5x  \right) \vert ^3_1=
\]
\[
= \left(  \frac{5.9}{2}-5.2  \right)-\left(  \frac{5.1}{2}-5.1 \right)=\frac{45}{2}-15-\frac{5}{2}+5=10
\]

\item

Een tekening geeft het volgende:

\begin{figure}[h]
\begin{center}
\includegraphics[height=8 cm]{Opgave13ReeksInt.pdf}
\caption{figuur bij opgave 13}
\end{center}
\end{figure}

De $x$-co\"ordinaten van de snijpunten van de twee grafieken vind je door het oplossen van de vergelijking $x^2+1=19-x^2$.
Je bekomt $2x^2=18$, dus $x^2=9$ en dat geeft de oplossingen -3 en 3.

De oppervlakte is daarom
\[
\int_{-3}^3 \left(  \left( 19-x^2 \right) - \left( x^2+1 \right)  \right)dx
\]
(hoogste grafiek)-(laagste grafiek).

\[
\int_{-3}^3 \left(  -2x^2+18 \right)dx = \left(   -\frac{2x^3}{3}+18x \right) \vert ^3_{-3}=
\]
\[
=\left( -\frac{2.3^3}{3}+18.3  \right)- \left( -\frac{2.(-3)^3}{3}+18.(-3)  \right)=(-18+18.3)+(-18+18.3)=72
\]

\end{enumerate}

\end{document}