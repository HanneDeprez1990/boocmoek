\documentclass{article}
\usepackage{amsrefs}
\usepackage{amssymb}
\usepackage{enumerate}
\usepackage{amsmath}
\usepackage{amsthm}
\usepackage{graphicx}
\usepackage{amssymb,latexsym}
\usepackage{graphicx}

\begin{document}

Een staaf heeft de vorm van een lange cilinder met een doorsnede met oppervlakte 1 en met lengte $L$.
Indien de samenstelling van het materiaal waaruit die staaf vervaardigd is overal hetzelfde is dan heet die staaf homogeen.
De massa van de staaf kun je dan vinden als de je massa $m$ kent van zulke staaf met lengte 1; de massa van de staaf is dan gelijk aan $m.L$.

Veronderstel nu dat de staaf niet homogeen is maar dat je op iedere plaats wel weet wat de massa van zulke staaf met lengte 1 zou zijn als de staaf wel homogeen zou zijn zoals op die betreffende plaats. Dit is de massadichtheid $\rho$ op die plaats.
Je stelt de staaf voor op een geijkte rechte met 0 in het ene uiteinde en $L$ in het andere (zie figuur \ref{figuur11.1Wis1}.
\begin{figure}[h]
\begin{center}
\includegraphics[height=30 mm]{figuur11_1Wis1.pdf}
\caption{Voorstelling van de staaf } \label{figuur11.1Wis1}
\end{center}
\end{figure}
Iedere $x \in [0;L]$ komt overeen met een plaats op de staaf en we noteren $\rho (x)$ voor de massadichtheid op die plaats van de staaf.
Je bekomt een functie $\rho : [0;L] \subset \mathbb{R} \rightarrow \mathbb{R} : x \rightarrow \rho (x)$.

Je veronderstelt dat de samenstelling van het materiaal van de staaf geleidelijk verandert.
Wiskundig vertaal je dat in: je veronderstelt dat $\rho$ een continue functie is.

Je wil de massa van de volledige staaf vinden uit deze functie $\rho$.
Daartoe beschouw je een tweede functie $m : [0;L] \subset \mathbb{R} \rightarrow \mathbb{R} : x \rightarrow m(x)$.
Hierbij is $m(x)$ de massa van de staaf als je de staaf zou doorknippen op de plaats die overeenkomt met $x$ en je neemt het stuk waar de plaats die overeenkomt met 0 bij behoort.
Je zoekt dan $m(L)$; dit is de massa van de volledige staaf.

Indien je de staaf doorknipt op de plaatsen die overeenkomen met $x$ en $x+\Delta x$ (met $\Delta x>0$) dan is de massa van dat stukje tussen die twee doorgeknipte plaatsen $m(x+\Delta x)-m(x)$.
Dit is de different $\Delta m(x)$.
Als je $\Delta x$ heel klein neemt dan kun je dat stukje staaf opvatten als ongeveer homogeen met massadichtheid $\rho (x)$.
Omdat de lengte van dat stukje $\Delta x$ is bekom je als massa van dat stukje ongeveer $\rho (x).\Delta x$.
Dit is de eerste graadsveelterm met constante term die een goede benadering is voor $\Delta m(x)$ als $\Delta x$ klein is.
Het is dus de differentiaal van de functie $m$ in $x$, je besluit dus $dm(x)=\rho (x).\Delta x$.
Omdat per definitie $dm(x)=Dm(x).\Delta x$ (zie Definitie \ref{definitie10.6Wis1}) bekom je $Dm(x)=\rho (x)$.

Wiskundig bekom je het volgende besluit.
Je zoekt de waarde $m(L)$ van een functie $m$ op $[0;L]$ waarvan je weet dat $m(0)=0$ en $Dm(x)=\rho (x)$ voor alle $x \in [0;L]$.
In het bijzonder heb je een functie $m$ nodig waarvan de afgeleide $\rho$ gegeven is.

\end{document}