\documentclass{article}
\usepackage{amsrefs}
\usepackage{amssymb}
\usepackage{amsmath}
\usepackage{amsthm}
\usepackage{graphicx}

\newtheorem*{voorbeeld}{Voorbeeld} \newtheorem*{eigenschap}{Eigenschap} \newtheorem*{opmerking}{Opmerking}

\title{Berekenen van bepaalde integralen}
\date { }

\begin{document}

\maketitle \noindent

\noindent Uit het voorgaande weten we dat uit $\int f(x)dx=F(x)+C$ volgt dat $\int^b_a f(x)dx=F(b)-F(a)$ (onder voorwaarde dat $f$ continu is op een interval dat $a$ en $b$ bevat).

\noindent Bovenste zin uit de Actimath cursus blz 18 in 2.1

\begin{voorbeeld}
Voorbeelden uit de Actimath cursus blz 18 in 2.1
\end{voorbeeld}

\begin{opmerking}
De opmerking die in de Acctimath cursus na de voorbeelden blz 18 in 2.1 staat.
\end{opmerking}

\begin{eigenschap} Bepaalde integraal van een som

$\boxed { \int^b_a (f(x)+g(x))dx = \int^b_a f(x)dx + \int^b_a g(x)dx }$
\end{eigenschap}

\begin{eigenschap} Bepaalde integraal van een functie vermenigvuldigd met een getal

$\boxed { \int^b_a c.f(x)dx = c \int^b_a f(x)dx}$
\end{eigenschap}

\begin{eigenschap} Opsplitsing van grenzen van een bepaalde integraal

$\boxed { \int^b_a f(x) dx = \int^c_a f(x)dx + \int^b_c f(x)dx }$
\end{eigenschap}

\noindent Uit de defintie volgen ook nog de volgende twee rekenregels:

$\boxed { \int^a_a f(x) dx = 0}$

$\boxed { \int^b_a f(x) dx = - \int^a_b f(x) dx }$

\begin{voorbeeld}
Voorbeeld 1 uit de Actimath cursus blz 19
\end{voorbeeld}




\end{document}