\documentclass{article}
\usepackage{amsrefs}
\usepackage{amssymb}
\usepackage{amsmath}
\usepackage{amsthm}
\usepackage{graphicx}

\DeclareMathOperator{\dom}{dom}
\newtheorem*{definitie}{Definitie} \newtheorem*{notatie}{Notatie} \newtheorem*{voorbeeld}{Voorbeeld} \newtheorem*{eigenschap}{Eigenschap}
\newtheorem{opgave}{Opgave}
\newtheorem*{oplossing}{Oplossing}

\title{Test Differentiaalrekenen}
\date { }

\begin{document}

\maketitle \noindent

\begin{opgave}
Als $y=7x^3-2x^2+4x-11$ dan is $Dy=ax^4+bx^3+cx^2+dx+e$. Wat zijn de waarden van $a$, $b$, $c$, $d$ en $e$?\vspace{2mm}

\noindent $a= \cdots$, $b=\cdots$, $c=\cdots$, $d=\cdots$, $e=\cdots$\vspace{1mm}

\noindent Een aantal van deze getallen kunnen 0 en/of 1 zijn.
\end{opgave}

\begin{oplossing}
$a=0$, $b=0$, $c=21$, $d=-4$, $e=4$
\end{oplossing}

\begin{opgave}
Als $y=15\sqrt[5]{x^3}$ dan is $Dy=\frac{a}{b\sqrt[c]{x^d}}$. Wat zijn de waarden van $a$, $b$, $c$, en $d$?\vspace{2mm}

\noindent $a= \cdots$, $b=\cdots$, $c=\cdots$, $d=\cdots$, \vspace{1mm}

\noindent Een aantal van deze getallen kunnen 0 en/of 1 zijn. Je mag enkel gehele getallen ingeven en je moet zoveel mogelijk vereenvoudigen.
\end{opgave}

\begin{oplossing}
$a=9$, $b=1$, $c=5$, $d=2$
\end{oplossing}

\begin{opgave}
Als $y=\sqrt[3]{x^7}\cos x$ dan is $Dy=a\sqrt[b]{x^c}\sin x+d\sqrt[e]{x^f}\cos x$. Wat zijn de waarden van $a$, $b$, $c$, $d$, $e$ en $f$?\vspace{2mm}

\noindent $a= \cdots$, $b=\cdots$, $c=\cdots$, $d=\cdots$, $e=\cdots$, $f=\cdots$ \vspace{1mm}

\noindent Een aantal van deze getallen kunnen 0 en/of 1 zijn. Je mag enkel gehele getallen of breuken van gehele getallen ingeven en je moet zoveel mogelijk vereenvoudigen.
\end{opgave}

\begin{oplossing}
$a=-1$, $b=3$, $c=7$, $d=\frac{7}{3}$, $e=3$, $f=4$
\end{oplossing}

\begin{opgave}
Als $y=\frac{2x^2-5x+3}{x-2}$ dan is $Dy=\frac{ax^3+bx^2+cx+d}{ex^3+fx^2+gx+h}$. Wat zijn de waarden van $a$, $b$, $c$, $d$, $e$, $f$, $g$ en $h$?\vspace{2mm}

\noindent $a= \cdots$, $b=\cdots$, $c=\cdots$, $d=\cdots$, $e=\cdots$, $f=\cdots$, $g=\cdots$, $h=\cdots$ \vspace{1mm}

\noindent Een aantal van deze getallen kunnen 0 en/of 1 zijn. Je mag enkel gehele getallen ingeven en je moet zoveel mogelijk vereenvoudigen.
\end{opgave}

\begin{oplossing}
$a=0$, $b=0$, $c=-4$, $d=7$, $e=0$, $f=1$, $g=-4$, $h=4$
\end{oplossing}

\begin{opgave}
Juist of fout?

\[
D(5^{3x})=9x5^{3x-1}
\]
\end{opgave}

\begin{oplossing}
fout
\end{oplossing}

\begin{opgave}
Juist of fout?

\[
D(Bgtan(\sqrt{x}))=\frac{1}{2(1+x)\sqrt{x}}
\]
\end{opgave}

\begin{oplossing}
juist
\end{oplossing}

\begin{opgave}
Juist of fout?

\[
D(\ln (\sin (\ln x)))=\frac{\cot (\ln x)}{x}
\]
\end{opgave}

\begin{oplossing}
juist
\end{oplossing}

\begin{opgave}
De vergelijking van de raaklijn aan de grafiek $G$ van $y=x^3+4x^2-3$ in het punt $P$ op $G$ waarvoor $x=-2$ is gelijk aan $y=ax+b$.
Wat zijn de waarden van $a$ en $b$? \vspace{2mm}

\noindent $a=\cdots$, $b=\cdots$.\vspace{1mm}

\noindent Een aantal van deze getallen kunnen 0 en/of 1 zijn. Je mag enkel gehele getallen of breuken van gehele getallen ingeven en je moet zoveel mogelijk vereenvoudigen.
\end{opgave}

\begin{oplossing}
$a=-4$, $b=-3$
\end{oplossing}

\begin{opgave}
Schrijf in de vorm van een interval de verzameling van alle $x\in \mathbb{R}$ waarvoor de functie $y=-x^3-x^2+5x+7$ strikt stijgend is in $x$.\vspace{1mm}

\noindent De grenzen van dit interval mag je enkel schrijven als gehele getallen of breuken van gehele getallen en je moet zoveel mogelijk vereenvoudigen.


\end{opgave}

\begin{oplossing}
$]-\frac{5}{3},1[$
\end{oplossing}

\begin{opgave}
Voor welke waarde van $x$ heeft de functie $y=5x^3+16x^2-7x+8$ een relatief minimum in $x$?\vspace{2mm}

\noindent De oplossing mag je enkel schrijven als een geheel getal of een breuk van gehele getallen en moet je zoveel mogelijk vereenvoudigen.
\end{opgave}

\begin{oplossing}
$\frac{1}{5}$
\end{oplossing}


\end{document}