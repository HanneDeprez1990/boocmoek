\documentclass{article}
\usepackage{amsrefs}
\usepackage{amssymb}
\usepackage{amsmath}
\usepackage{amsthm}
\usepackage{graphicx}

\DeclareMathOperator{\unit}{u} \DeclareMathOperator{\comb}{comb}
\DeclareMathOperator{\tri}{tri} \DeclareMathOperator{\rect}{rect}
\DeclareMathOperator{\sgn}{sgn} \DeclareMathOperator{\ramp}{ramp}
\DeclareMathOperator{\sinc}{sinc}

\title{Inleidend voorbeeld}
\date { }

\begin{document}

\maketitle \noindent


\noindent Een object wordt de hoogte in geschoten.
volgende functie $x(t)$ geeft na $t$ seconden de hoogte van het object uitgedrukt in meter

\[
x(t)=1,5+50 t -4,9 t^2
\]

\noindent Met welke snelheid beweegt dat object naar boven 2 seconden na het schot?

\section{Eerste benadering}

Na 2 seconden is de hoogte $x(2)=1,5+50.2-4,9.2^2=81,9$ meter.\\
Na 3 seconden is de hoogte $x(3)=1,5+50.3-4,9.3^2=107,4$ meter.\\
Gedurende 1 seconde is het object $107,4-81,9=25,5$ meter verhoogd.
Dit is een gemiddelde snelheid van 25,5 m/s. \vspace{5mm}

\noindent Maar die snelheid neemt tijdens het verhogen wel geleidelijk af.
We bekomen een betere benadering voor de snelheid 2 seconden na het schot door het tijdsverschil kleiner te nemen.

\section{Tweede benadering}

Na 2,5 seconden is de hoogte  $x(2,5)=1,5+50.2,5-4,9.2,5^2=95,875$ meter.\\
Gedurende 0,5 seconden is het object $95,875-81,9=13,825$ meter verhoogd.
Dit is een gemiddelde snelheid van $\frac{13,825}{0,5}=27,6$ m/s.

\section{Derde benadering}

Na 2,1 seconden is de hoogte $x(2,1)=1,5+50.2,1-4,9.2,1^2=84,891$ meter.\\
Gedurende 0,1 seconden is het object $84,891-81,9=2,991$ meter verhoogd.
Dit is een gemiddelde snelheid van $\frac{2,991}{0,1}=29,91$ m/s.

\section{Alle benaderingen}

Voor een tijdsverschil $\Delta t>0$ is de hoogte op het tijdstip $2+\Delta t$

\[
x(2+\Delta t) = 1,5+50 (2+ \Delta t) - 4,9 (2+ \Delta t)^2 = 81,9+30,4 \Delta t - 4,9 \Delta t^2
\]

\noindent Gedurende $\Delta t$ seconden is het object $30,4 \Delta t -4,9 \Delta t^2$ meter verhoogd.
Dit is een gemiddelde snelheid van $\frac{30,4 \Delta t -4,9 \Delta t^2}{\Delta t}=30,4 -4,9 \Delta t$ m/s.\\

\noindent Voor $\Delta t=1$; $\Delta t=0,5$ en $\Delta t=0,1$ geeft dit de reeds berekende benaderingen voor de snelheid van het object na 2 seconden.
Hoe kleiner $\Delta t$ hoe beter die benadering.

\section{Snelheid na 2 seconden}

Door het vorige resultaat $\Delta t=0$ te nemen bekom je de snelheid van het object na 2 seconden.\\
De snelheid van het object na 2 seconden is dus gelijk aan 30,4.\vspace{0,5 mm}

Deze uitkomst is het limietgeval van de gemiddelde snelheid waarbij je het tijdsverschil $\Delta t$ naar 0 laat naderen.
Dit getal geeft de afgeleide van de functie $x(t)$ voor $t=2$.\vspace{1 cm}

Dit voorbeeld gaan we nu wiskundig abstract veralgemenen.




\end{document}