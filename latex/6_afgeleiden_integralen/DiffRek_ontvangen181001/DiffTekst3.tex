\documentclass{article}
\usepackage{amsrefs}
\usepackage{amssymb}
\usepackage{amsmath}
\usepackage{amsthm}
\usepackage{graphicx}

\DeclareMathOperator{\dom}{dom}
\newtheorem*{definitie}{Definitie} \newtheorem*{notatie}{Notatie} \newtheorem*{voorbeeld}{Voorbeeld} \newtheorem*{eigenschap}{Eigenschap}

\title{Afgeleiden van veeltermfuncties}
\date { }

\begin{document}

\maketitle \noindent

\section{Afgeleide van een constante functie}

Voor $c\in \mathbb{R}$ is $y=c$ het voorschrift van een constante functie.
Intu\"itief (afgeleide geeft snelheid aan waarmee een functiewaarde verandert) is $D(c)=0$.\vspace{3 mm}

\noindent Via de definitie:

$y(a)=c$; $y(a+\Delta x)=c$

dus $\frac{y(a+ \Delta x)-y(a)}{\Delta x}=\frac {c-c}{\Delta x}=0$.\vspace{2mm}

\noindent In de limiet als $\Delta x$ nul wordt blijft dit 0.
Je bekomt $D(c)=0$.

\section{Afgeleide van de identieke functie}

Het voorschrift $y=x$ is het voorschrift van de identieke functie.
Waaraan is D(x) gelijk?\vspace{3mm}

\noindent Via de definitie:

$y(a)=a$; $y(a+\Delta x)=a+\Delta x$

dus $\frac{y(a+\Delta x)-y(a)}{\Delta x}=\frac{a+\Delta x -a}{\Delta x}=1$.\vspace{2mm}

\noindent In de limiet als als $\Delta x$ nul wordt blijft dit 1.
Je bekomt $D(x)=1$.

\section{Afgeleide van $y=x^2$}

\noindent Voor $y=x^2$ is

$y(a)=a^2$, $y(a+\Delta x)=(a+\Delta x)^2=a^2+2a\Delta x + \Delta x^2$

dus $\frac{y(a+\Delta x)-y(a)}{\Delta x}=\frac{2a \Delta x +\Delta x^2}{\Delta x}=2a+ \Delta x$.

\noindent In de limiet als $\Delta x$ nul wordt dan bekom je 2a.
Je bekomt $Dy(a)=2a$ en daarom $D(x^2)=2x$.

\section{Afgeleide van $y=x^n$}

\noindent Algemeen geldt voor $n\in \mathbb{N}$ dat $D(x^n)=nx^{n-1}$.\vspace{3mm}

Merk op, als $n=0$ dan geeft dit $D(x^0)=0x^{-1}=0$.
Omdat $x^0=1$ geeft dit $D(1)=0$.
Dat komt overeen met de afgeleide van een constante functie.

\section{afgeleiden van veeltermfuncties}

Uit de afgeleide van $y=x^n$ met $n\in \mathbb{N}$ volgt dat je door middel van de volgende twee rekenregels van alle veeltermfuncties de afgeleide kunt berekenen.

\begin{eigenschap} Afgeleide van een som : \\
$\boxed { D(f+g)=Df+Dg}$
\end{eigenschap}

\begin{eigenschap} Afgeleide van een functie vermenigvuldigd met een constante :
$\boxed { D(cf)=cD(f)}$
\end{eigenschap}

\begin{voorbeeld}
\begin{equation*}
\begin{split}
D(5x^3-&7x^2+18x-9)\\
&=D(5x^3)+D(-7x^2)+D(18x)+D(9) \text{ (rekenregel som)}\\
&=5D(x^3)-7D(x^2)+18D(x)+D(9) \text{ (rekenregel product met c)}\\
&=5.3x^2-7.2x+18.1+0 \text{ $(D(x^n)=nx^{n-1})$}\\
&=15x^2-14x+18
\end{split}
\end{equation*}
\end{voorbeeld}

\begin{voorbeeld}
Bekijk terug de functie uit het beginvoorbeeld
\[
f(x)=1,5+50t-4,9t^2
\]
Met de rekenregels en afleiden van $x^n$ bekom je
\[
Df(x)=50-9,8t
\]
Merk op dat je door invullen vindt $Df(2)=50-9,8.2=30,4$.
Dit is het resultaat dat we eerder gevonden hadden.
\end{voorbeeld}

Uit de twee rekenregels volgt ook
\begin{eigenschap} Algeleide van een verschil

$\boxed {D(f-g)=Df-Dg}$
\end{eigenschap}\vspace{5 mm}
\begin{eigenschap}
De afgeleide $D(x^n)=nx^{n-1}$ geldt algemener voor alle $n\in \mathbb{R}$.
\end{eigenschap}

\begin{voorbeeld}
$D(\sqrt[5]{x})=D(x^{1/5})=\frac{1}{5}x^{-4/5}=\frac{1}{5x^{4/5}}=\frac{1}{5\sqrt[5]{x^4}}$
\end{voorbeeld}

\begin{voorbeeld}
$D(\frac{1}{x^3})=D(x^{-3})=-3x^{-4}=-\frac{3}{x^4}$
\end{voorbeeld}

\begin{voorbeeld}
$D(\sqrt[3]{x^2}-5\sqrt{x^3})=D(\sqrt[3]{x^2})-5D(\sqrt{x^3})$

\hspace{5mm} $=D(x^{2/3})-5D(x^{3/2})$

\hspace{5mm} $=\frac{2}{3}x^{-1/3}-5\frac{3}{2}x^{1/2}=\frac{2}{3\sqrt[3]{x}}-\frac{15}{2}\sqrt{x}$
\end{voorbeeld}






\end{document}