\documentclass{article}
\usepackage{amsrefs}
\usepackage{amssymb}
\usepackage{enumerate}
\usepackage{amsmath}
\usepackage{amsthm}
\usepackage{graphicx}
\usepackage{amssymb,latexsym}
\usepackage{graphicx}

\begin{document}

\begin{enumerate}

\item

$Dy=7.3x^2-2.2x+4=21x^2-4x+4$.

Vergelijken met $Dy=ax^4+bx^3+cx^2+dx+e$ geeft

$a=b=0$; $c=21$; $d=-4$ en $e=4$.

\item

$Dy=D \left( 15x^{3/5} \right)=15.\frac{3}{5} x^{3/5-1}=9x^{-2/5}=\frac{9}{\sqrt[5]{2}}$.

Vergelijken met $Dy=\frac{a}{b\sqrt[c]{x^d}}$ geeft

$a=9$; $b=1$; $c=5$ en $d=2$.
\item

$Dy=D \left( x^{7/3}\cos x \right)=\frac{7}{3} x^{4/3} \cos x-x^{7/3} \sin x= \frac{7}{3}\sqrt[3]{x^4}\cos x+(-1) \sqrt[3]{x^7} \sin x$.

Vergelijken met $Dy=a\sqrt[b]{x^c}\sin x+d\sqrt[e]{x^f}\cos x$ geeft

$a=-1$; $b=3$; $c=7$; $d=\frac{7}{3}$; $e=3$ en $f=4$.
\item

\[
Dy=D \left(  \frac{2x^2-5x+3}{x-2}  \right)=\frac{(x-2)D \left(  2x^2-5x+3 \right) -\left(  2x^2-5x+3 \right)D(x-2)}{(x-2)^2}=
\]
\[
=\frac{(x-2)(4x-5)-\left(  Zx^2-5x+3 \right)}{x^2-4x+4}=\frac{4x^2-8x-5x+10-2x^2+5x-3}{x^2-4x+4}=
\]
\[
=\frac{2x^2-8x+7}{x^2-4x+4} 
\]

Vergelijken met $Dy=\frac{ax^3+bx^2+cx+d}{ex^3+fx^2+gx+h}$ geeft

$a=e=0$; $b=2$; $c=-8$; $d=7$; $f=1$; $g=-4$; $h=4$.

\item

De $y$-co\"ordinaat van het punt op de grafiek is $y(-2)=(-2)^3+4.(-2)-3=-8+16-3=5$.

Omdat $Dy=3x^2+8x$ is de richtingsco\"effici\"ent van de raaklijn $Dy(-2)=3(-2)^2+8.(-2)=12-16=-4$.

De vergelijkng van de raaklijn is dus $y-5=-4(x-(-2))$, dus $y=-4x-8+5=-4x-3$.

Vergelijken met $y=ax+b$ geeft

$a=-4$; $b=-3$.

\item

$D \left( 5^{3x} \right)=D \left(  \left(  5^3 \right)^x \right) =\left(  5^3 \right)^x.\ln \left(  5^3 \right)=\ln \left( 5^3 \right)5^{3x}$.

Het antwoord dat er staat is dus fout.

\item

$D \left(  Bgtan \left(  \sqrt{x} \right) \right)=\frac{1}{D Bgtan \left( \sqrt{x} \right)}.D \left( \sqrt{x} \right)=\frac{1}{1+\left( \sqrt{x}  \right)^2}.\frac{1}{2 \sqrt{x}}=\frac{1}{2(1+x)\sqrt{x}}$

Het antwoord dat er staat is dus juist.

\item

\[
D \left( \ln \left( \sin \left( \ln x \right) \right) \right)=D \ln \left( \sin \left( \ln x \right) \right).D\sin (\ln x).D \ln x=
\]
\[
=\frac{1}{\sin (\ln x)}.\cos ( \ln x).\frac{1}{x}=\frac {\cot ( \ln x)}{x}
\]

Het antwoord dat er staat is dus juist.

\item
$Dy=-3x^2-2x+5$.

Nulpunten van $-3x^2-2x+5$ zijn $-\frac{5}{3}$ en $1$ (berekenen van nulpunten van een tweedegraadsfunctie).
Je bekomt hieruit de volgende tabel voor het verloop van de functie:

\hspace{5mm}\begin{tabular}{ |  l | r | r | r | r| r| }
\hline
$x$ & \hspace{8mm}  & $-\frac{5}{3}$ & \hspace{8mm} & $1$ &  \hspace{8mm} \\  \hline
  &   &   &    &  &       \\ 
$Dy$ & $-$ &  0 & $+$ & 0 & $-$ \\ 
  &   &   &   &  &   \\ \hline
$y$ & $\downarrow$ &  & $\uparrow$ &  & $\downarrow$\\
 &  &  &  &  &     \\ \hline
\end{tabular}\vspace{5mm}

Je leest af dat de functie strikt stijgend is op het interval $]-\frac{5}{3}; 1[$.

\item
$Dy=15x^2+32x-7$.

Nulpunten van $15x^2+32x-7$ zijn $-\frac{7}{3}$ en $\frac{1}{5}$. (berekenen van nulpunten van een tweedegraadsfunctie).
Je bekomt hieruit de volgende tabel voor het verloop van de functie:

\hspace{5mm}\begin{tabular}{ |  l | r | r | r | r| r| }
\hline
$x$ & \hspace{8mm}  & $-\frac{7}{3}$ & \hspace{8mm} & $\frac{1}{5}$ &  \hspace{8mm} \\  \hline
  &   &   &    &  &       \\ 
$Dy$ & $+$ &  0 & $-$ & 0 & $+$ \\ 
  &   &   &   &  &   \\ \hline
$y$ & $\uparrow$ &  rel. Maximum & $\downarrow$ & rel. Minimum  & $\uparrow$\\
 &  &  &  &  &     \\ \hline
\end{tabular}\vspace{5mm}

Je leest af dat de functie een relatief minimum heeft bij $x=\frac{1}{5}$.

\end{enumerate}

\end{document}