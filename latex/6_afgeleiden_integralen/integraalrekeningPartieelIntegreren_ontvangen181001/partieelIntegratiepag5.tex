\documentclass{article}
\usepackage{amsrefs}
\usepackage{amssymb}
\usepackage{enumerate}
\usepackage{amsmath}
\usepackage{amsthm}
\usepackage{graphicx}
\usepackage{amssymb,latexsym}
\usepackage{graphicx}
\DeclareMathOperator{\dom}{dom}

\begin{document}

\begin{itemize}
\item Hoe gebruik je parti\"ele integratie bij het berekenen van bepaalde integralen?
\end{itemize}

Je kunt parti\"ele integratie onmiddellijk toepassen bij bepaalde integralen zoals in volgend voorbeeld.

\[
\int ^e_1 \ln x dx=x.\ln x \vert ^e_1- \int ^e_1 x d(\ln x)=x.\ln x \vert ^e_1-\int ^e_1dx=(x.\ln x -x)\vert ^e_1=
\]
\[
=e.\ln e - e - (1.\ln 1 -1)=1 \text { .}
\]

Denk er aan dat je ook in het deel $vu$ van de regel $\int udv=uv+\int vdu$ de grenzen moet invullen. Als je dat vergeet dan bekom je als uitkomst van een bepaalde integraal immers een functievoorschrift terwijl het resultaat van een bepaalde integraal een getal moet zijn.









\end{document}