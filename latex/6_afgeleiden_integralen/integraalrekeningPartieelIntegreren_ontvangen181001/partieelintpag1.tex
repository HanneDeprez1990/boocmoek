\documentclass{article}
\usepackage{amsrefs}
\usepackage{amssymb}
\usepackage{enumerate}
\usepackage{amsmath}
\usepackage{amsthm}
\usepackage{graphicx}
\usepackage{amssymb,latexsym}
\usepackage{graphicx}
\DeclareMathOperator{\dom}{dom}

\begin{document}

Aan de hand van een voorbeeld zie je hoe je de productregel voor het afleiden kunt gebruiken om onbepaalde integralen op te lossen.
Dit gebruik van de productregel is de basis voor de methode van parti\"ele integratie bij het berekenen van onbepaalde integralen.\\

De onbepaalde integraal van een som van functies is de som van de individuele onbepaalde integralen:
\[
\int (f(x)+g(x))dx=\int f(x)dx + \int g(x)dx \text { .}
\]
Vele voorbeelden tonen aan dat de onbepaalde integraal van een product van functies niet het product van de individuele onbepaalde integralen is.
Bijvoorbeeld:
\[
\int x^2dx = \frac{x^3}{3}+C
\]
maar $x^2=x.x$ en $\int xdx=\frac{x^2}{2}+C$ dus
\[
\int x^2dx \neq \left( \int xdx \right)^2 \text { .}
\]
Dit komt omdat de afgeleide van een product van functies niet gelijk is aan het product van de afgeleiden van de individuele functies.

Wel geldt
\[
D\left( f(x)g(x) \right)=Df(x).g(x)+f(x).Dg(x) \text { .}
\]
Volgend voorbeeld toont aan dat deze productregel van het afleiden wel kan helpen bij het berekenen van onbepaalde integralen.\\

\noindent (Dit voorbeeld opnemen uitgelegd in een filmpje?)

We willen de onbepaalde integraal $\int x\sin xdx$ oplossen.

Uit de productregel van het afleiden weten we dat
\[
D(x\sin x)=Dx.\sin x+x.D(\sin x)=\sin x +x \cos x \text { .}
\]
Per definitie is
\[
\int D(x \sin x)dx=x \sin x +C
\]
en dus
\[
\int \left( \sin x +x \cos x \right) dx = x \sin x +C \text { .}
\]
Omdat $x \cos x = \left( \sin x +x \cos x \right)- \sin x$ bekom je hieruit
\[
\int x \cos x dx = \int \left( \sin x + x \cos x) \right) dx - \int \sin x dx= x \sin x + \cos x +C \text { .}
\]

Je merkt dat je in plaats van de gezochte integraal de integraal met $\cos$ in plaats van $\sin$ vindt.
Dit geeft aan dat je de gezochte integraal kunt vinden door te starten met
\[
D(x \cos x)=Dx.\cos x+x.D(\cos x)=\cos x -x \sin x \text { .}
\]
Dit geeft dan
\[
\int \left( \cos x - x \sin x \right)dx=\int D(x \cos x)dx=x \cos x +C
\]
en omdat $x \sin x = \cos x - \left( \cos x - x \sin x \right)$ vind je
\[
\int x \sin x dx = \int \cos x dx- \int \left( \cos x-x \sin x \right) dx=\sin x - x \cos x +C \text { .}
\]

Je merkt dus dat de productregel voor het afleiden kan helpen om onbepaalde integralen op te lossen.
Maar in het voorbeeld werken we niet op een gericht wijze.
We gaan uit de productregel voor het afleiden de methode van parti\"ele integratie bekomen.
Dit geeft een meer systematische wijze om de productregel van het afleiden te gebruiken bij het onbepaald integreren.







\end{document}