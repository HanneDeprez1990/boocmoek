\documentclass{article}
\usepackage{amsrefs}
\usepackage{amssymb}
\usepackage{enumerate}
\usepackage{amsmath}
\usepackage{amsthm}
\usepackage{graphicx}
\usepackage{amssymb,latexsym}
\usepackage{graphicx}
\DeclareMathOperator{\dom}{dom}

\begin{document}

\begin{itemize}
\item Welke types van integralen kun je oplossen door gebruik te maken van parti\"ele integratie?
\end{itemize}

\noindent \underline{Type :} $\int x^n \sin(ax)dx$; $\int x^n \cos (ax)dx$; $\int x^n e^{ax}dx$ met $n \in \mathbb{N}$ en $a \in \mathbb{R}$.

\noindent Je neemt $u=x^n$ en $dv=\sin (ax)dx$; $dv=\cos (ax)dx$; $dv=e^{ax}dx$.\\

\noindent \underline{Voorbeeld} $\int x^2 \cos \left( \frac{2x}{5}  \right)dx$.

Neem $u=x^2$ en $dv=\cos \left( \frac{2x}{5}  \right)dx$.
Dan is $du=2xdx$ en uit $\int \cos \left( \frac{2x}{5}  \right)dx=\frac{5}{2} \sin \left(  \frac{2x}{5} \right)+C$ vind je dat je $v=\frac{5}{2} \sin \left(  \frac{2x}{5} \right)$ kunt nemen.
Je bekomt
\[
\int x^2 \cos \left( \frac{2x}{5}  \right)dx=\frac{5}{2}x^2  \sin \left(  \frac{2x}{5} \right)-5\int x  \sin \left(  \frac{2x}{5} \right)dx \text { .}
\]

Je past op de bekomen integraal opnieuw parti\"ele integratie toe.
Je stelt $u=x$ en $dv= \sin \left(  \frac{2x}{5} \right)dx$.
Je bekomt $du=dx$ en uit $\int  \sin \left(  \frac{2x}{5} \right)dx=-\frac{5}{2}\cos \left(  \frac{2x}{5} \right)+C$ bekom je dat je $v=-\frac{5}{2}\cos \left(  \frac{2x}{5} \right)$ kunt nemen.
Je bekomt
\[
\int x  \sin \left(  \frac{2x}{5} \right)dx=-\frac{5}{2} x \cos \left(  \frac{2x}{5} \right)+\frac{5}{2}\int \cos \left( \frac{2x}{5}  \right)dx=-\frac{5}{2} x \cos \left(  \frac{2x}{5} \right)+\frac{25}{4}\sin \left( \frac{2x}{5} \right)+C \text { .}
\]

Voor de op te lossen integraal bekom je
\[
\int x^2 \cos \left( \frac{2x}{5}  \right)dx=\frac{5}{2}x^2  \sin \left(  \frac{2x}{5} \right)-5\left( -\frac{5}{2} x \cos \left(  \frac{2x}{5} \right)+\frac{25}{4}\sin \left( \frac{2x}{5} \right)   \right)+C=
\]
\[
=\frac{5}{2}x^2  \sin \left(  \frac{2x}{5} \right)+\frac{25}{2} x \cos \left(  \frac{2x}{5} \right)-\frac{125}{4}\sin \left( \frac{2x}{5} \right)  +C \text { .}
\]\\

\noindent \underline{Type :} $\int x^n \ln (x)dx$; $\int x^n \arcsin(ax)dx$; $\int x^n \arctan(ax)dx$ met $n \in \mathbb{N}$ en $a \in \mathbb{R}$ (bij de integraal met $\ln$ mag $n \in \mathbb{R}$).

\noindent Je neemt $u= \ln (x)$; $u=\arcsin (ax)$; $u=\arctan (ax)$ en $dv=x^n dx$.
Je bekomt weliswaar $u=\frac {x^{n+1}}{n+1}$ maar omdat $dv=\frac{dx}{x}$; $dv=\frac{adx}{\sqrt{1-a^2x^2}}$; $dv=\frac {adx}{1+a^2x^2}$ bekom je daarna de integraal uit een eenvoudiger functie die je mogelijk kunt oplossen.\\

\noindent \underline{Voorbeeld} $\int x^2 \arctan(5x)dx$.

Neem $u=\arctan (5x)$ en dus $du=\frac{5dx}{1+25x^2}$ en $dv=x^2dx$ en dus $v=\frac{x^3}{3}$.
Je bekomt
\[
\int x^2 \arctan(5x)dx=\frac{x^3 \arctan (5x)}{3}-\frac{5}{3} \int \frac {x^3dx}{1+25x^2} \text { .}
\]

In de laatste integraal gebruik je substitutie $t=1+25x^2$.
Dan is $dt=50xdx$ en $x^2=\frac{t-1}{25}$.
Je bekomt
\[
\int \frac {x^3dx}{1+25x^2}=\frac{1}{50}\int \frac{x^2}{1+25x^2}50xdx=\frac{1}{50}\int \frac{(t-1)/25}{t}dt=
\]
\[
=\frac{1}{1250}\int \left( 1-\frac{1}{t}  \right)dt=\frac{1}{1250}\left(  t-\ln \vert t \vert  \right)+C=\frac{1}{1250} \left( 1+25x^2-\ln \left( 1+25x^2 \right) \right)+C \text { .}
\]

Voor de op te lossen integraal bekom je
\[
\int x^2 \arctan(5x)dx=\frac{x^3 \arctan (5x)}{3}-\frac{5}{3750} \left( 1+25x^2-\ln \left( 1+25x^2 \right) \right)+C \text { .}
\]\\

\noindent \underline {Nog een voorbeeld :} $\int \arcsin x dx$

Neem $u=\arcsin x$ dus $du=\frac{dx}{\sqrt {1-x^2}}$ en $dv=dx$ dus $v=x$.
Je bekomt
\[
\int \arcsin x dx=x \arcsin x-\int \frac{xdx}{\sqrt{1-x^2}} \text { .}
\]

In deze laatste integraal gebruik je substitutie $t=1-x^2$.
Dan is $dt=-2xdx$ en je bekomt
\[
\int \frac{xdx}{\sqrt{1-x^2}}=-\frac{1}{2}\int \frac{dt}{\sqrt{t}}=-\frac{1}{2}\frac{\sqrt{t}}{1/2}+C=-\sqrt{t}+C=-\sqrt{1-x^2}+C \text { .}
\]

Voor de op te lossen integraal bekom je
\[
\int \arcsin x dx=x \arcsin x-\left( -\sqrt{1-x^2}  \right)+C=x \arcsin x+\sqrt{1-x^2} +C \text { .}
\]\\

\noindent \underline{Type :} $\int \sin (ax)e^{bx}dx$ met $a; b \in \mathbb{R}$ (en in plaats van $\sin$ dan er ook $\cos$ staan).

\noindent Door tweemaal na elkaar dezelfde rollen voor $u$ en $v$ te gebruiken bekom je de som van een functie en een getal vermenigvuldigd met de te zoeken integraal.
Hieruit vind je de te zoeken integraal door het oplossen van een vergelijking.\\

\noindent \underline{Voorbeeld :} $\int e^{3x}\cos (5x)dx$

Stel $u=e^{3x}$ en $dv=\cos (5x)dx$.
Je bekomt $du=3e^{3x}dx$ en uit $\int \cos (5x)dx=\frac{1}{5} \sin (5x)+C$ vind je dat je $v=\frac{1}{5} \sin (5x)$ kunt nemen.
Je bekomt
\[
\int e^{3x}\cos (5x)dx=\frac{1}{5}\sin (5x)e^{3x}-\frac{3}{5}\int \sin (5x)e^{3x}dx \text { .}
\]

Stel in de nieuwe integraal opnieuw $u=e^{3x}$ en $dv=\sin (5x)dx$.
Je bekomt $du=3e^{3x}dx$ en uit $\int \sin (5x)dx=-\frac {\cos (5x)}{5}+C$ vind je dat je $v=-\frac {\cos (5x)}{5}$ kunt nemen.
Je bekomt
\[
\int \sin (5x)e^{3x}dx=-\frac {\cos (5x)e^{3x}}{5}+\frac{3}{5}\int e^{3x}\cos (5x)dx \text { .}
\]
Voor de op te lossen integraal bekom je dan
\[
\int e^{3x}\cos (5x)dx=\frac{1}{5}\sin (5x)e^{3x}-\frac{3}{5} \left(  -\frac {\cos (5x)e^{3x}}{5}+\frac{3}{5}\int e^{3x}\cos (5x)dx  \right)=
\]
\[
=\frac{1}{5}\sin (5x)e^{3x}+\frac{3}{25} \cos (5x)e^{3x}-\frac{9}{25}\int e^{3x}\cos (5x)dx \text { .}
\]

Breng je de laatste integraal over naar het linkerlid dan bekom je
\[
\left( 1+\frac{9}{25}  \right) \int e^{3x}\cos (5x)dx = \left(  \frac{1}{5}\sin (5x)+\frac{3}{25} \cos (5x)  \right)e^{3x}+C \text { .}
\]
Hieruit vind je
\[
\int e^{3x}\cos (5x)dx=\frac{25}{34} \left( \frac{1}{5}\sin (5x)+\frac{3}{25} \cos (5x)   \right)e^{3x}+C \text { .}
\]







\end{document}