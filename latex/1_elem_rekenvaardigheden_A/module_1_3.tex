\subsection{Basisprincipes}

Een wiskundige uitdrukking ontbinden is belangrijk om deze uitdrukking eenvoudiger voor te stellen en om ze beter te kunnen analyseren. Ze kan ook gebruikt worden om een vergelijking op te lossen.

Welke uitdrukking ziet er gemakkelijker uit?

\begin{equation*}
x(x+1)(x+2)(x-\sqrt{2}) \text{ of } x^4+(3-\sqrt{2})x^3+(2-3\sqrt{2})x^2-2\sqrt{2}x
\end{equation*}

Bij het ontbinden in factoren probeer je zoveel mogelijk gemeenschappelijke factoren af te zonderen, je zet een som om in een product.

\subsubsection{Afzonderen van een getal}
Het eenvoudigste is een getal afzonderen: $5x+10y=5(x+2y)$

Elke term van de som heeft een gemeenschappelijke factor 5, of met andere woorden, elke term is deelbaar door 5. Je brengt deze factor buiten door elke term daadwerkelijk te delen door 5, en het getal buiten de haakjes te zetten. Je past als het ware omgekeerde distributiviteit toe.

\begin{voorbeeld}
\begin{eqnarray*}
14x+21&=&7(2x+3)\\
5y+5&=&5(y+1)\\
2x+4y+6z&=&2(x+2y+3z)
\end{eqnarray*}
\end{voorbeeld}

\subsubsection{Afzonderen van een letter of eenterm}
Je kan natuurlijk ook letters afzonderen. 

\begin{voorbeeld}
\begin{eqnarray*}
5x+3x^2&=&x(5+3x) \\
3xy+2zy&=&y(3x+2z) \\
a^2+3ab+ac&=&a(a+3b+c)
\end{eqnarray*}
\end{voorbeeld}

Dit kan algemener: soms kan je een gehele eenterm (getal maal een macht van een letter) afzonderen:

\begin{voorbeeld}
	\begin{eqnarray*}
		3x+6x^2=3x(1+2x) \\
		x^3+2x^2=x^2(x+2)
	\end{eqnarray*}
\end{voorbeeld}

Als er bij een term en getal of letter ontbreekt, dan kan je eenvoudig compenseren met een breuk:

\begin{voorbeeld}
	\begin{equation*}
	6x^2+3x+1=6x^2(1+\frac{1}{2x}+\frac{1}{6x^2})
	\end{equation*}
\end{voorbeeld}

Controleer altijd je antwoord door het (in gedachten) terug uitrekenen van de haakjes. Veel fouten worden gemaakt door een verkeerde macht van een letter te laten staan, of door het vergeten van constante termen.

\subsubsection{Gedeeltelijke ontbinding}
Veel ontbindingen zie je niet op het zicht. Soms moet je wat proberen en wat termen groeperen om bepaalde factoren af te zonderen. 
\begin{voorbeeld}
	
\begin{equation*}
3x+6xy+y
\end{equation*}
In deze uitdrukking zie je dat de 3 termen niets met elkaar te maken hebben (m.a.w. niks gemeenschappelijks hebben). Je kan wel de eerste twee samennemen en daar  uit afzonderen:
\begin{equation*}
3x+6xy+y = (3x+6xy)+y = 3x(1+2y)+y
\end{equation*}
Maar, je had ook de twee laatste termen kunnen samennemen:
\begin{equation*}
3x+6xy+y = 3x+(6xy+y) = 3x+y(6x+1)
\end{equation*}
Geen van beide pogingen leidt echter tot een volledige ontbinding. Het zijn echter deze pogingen die je moet ondernemen!

\end{voorbeeld}
\subsubsection{Afzonderen van een veelterm}
Je kan ook volledige veeltermen afzonderen.
\begin{voorbeeld}
\begin{equation*}
3(x^2+1)+2a(x^2+1)
\end{equation*}
Hier zie je dat de twee termen de factor $x^2+1$ gemeenschappelijk hebben. Deze kan je in zijn geheel afzonderen. Bij de eerste term zal dan $3$ overblijven, en bij de tweede $2a$:
\begin{equation*}
3(x^2+1)+2a(x^2+1)=(x^2+1)(3+2a)
\end{equation*}
Gebruik altijd goed de haakjes! Nog een voorbeeld:
\begin{equation*}
7xy(y+2z)-3z(y+2z)
\end{equation*}
Hier zonder je $y+2z$ af. Voor de eerste term blijft er dan $7xy$ over, bij de tweede $-3z$. Vergeet dat minteken niet!
\begin{equation*}
7xy(y+2z)-3z(y+2z)=(y+2z)(7xy-3z)
\end{equation*}	
\end{voorbeeld}
Als je een opgave krijgt die je niet meteen op het zicht kan oplossen, moet je eerst proberen met gedeeltelijke ontbinding. Soms zie je dan meer.

\begin{voorbeeld}
	\begin{equation*}
x^2-2ax-8a+4x
\end{equation*}
Bekijk je dit als een geheel, dan zie je dat deze 4 termen niets gemeenschappelijks hebben. Je moet nu kiezen welke termen je gaat samennemen. Je ziet bijvoorbeeld dat de eerste 2 termen een $x$ gemeenschappelijk hebben, en de laatste twee een factor 4. Zo groeperen we dan ook. Gebruik bij het groeperen de haakjes:
\begin{equation*}
x^2-2ax-8a+4x=(x^2-2ax)+(-8a+4x)
\end{equation*}
Nu zonderen we af:
\begin{equation*}
(x^2-2ax)+(-8a+4x)=x(x-2a)+4(-2a+x)
\end{equation*}
Nu zie je plots dat beide termen nog steeds iets gemeenschappelijks hebben, namelijk $x-2a$ (let op, deze zijn omgewisseld in de tweede term). Die kan je nu ook afzonderen!
\begin{equation*}
x(x-2a)+4(-2a+x)=(x-2a)(x+4)
\end{equation*}

\end{voorbeeld}

\begin{voorbeeld}
	\ \\
\begin{center}
	$\begin{array}{ccll}
	6axy-6a-4xy+9a^2 &=& (6axy-6a)+(-4xy+9a^2) &\text{eerste poging tot groeperen} \\
	&=& 6a(xy-1)+(-4xy+9a^2) &\text{gelijke factoren afzonderen} \\
	\end{array}$
\end{center}

In deze opgave zit je nu vast, de tweede term kan je niet verder ontbinden. Onze eerste groepering levert niet veel op! We proberen iets anders:
\begin{center}
	$\begin{array}{ccll}
	6axy-6a-4xy+9a^2 &=& (6axy-4xy)+(-6a+9a^2) &\text{tweede poging tot groeperen} \\
	&=& 2xy(3a-2)+3a(-2+3a) &\text{gelijke factoren afzonderen} \\
	&=& (3a-2)(2xy+3a) &\text{gelijke factor afzonderen} \\
	\end{array}
	$
\end{center}	
\end{voorbeeld}

\begin{voorbeeld}
	\ \\
\begin{center}
$\begin{array}{ccll}
-4xy-4y+7z+7xz &=& (-4xy-4y)+(7z+7xz) &\text{poging tot groeperen} \\
&=& 4y(-x-1)+7z(1+x) &\text{gelijke factoren afzonderen} \\
\end{array}
$
\end{center}
We zitten nu schijnbaar vast, omdat de twee termen wel een factor bij zich hebben staan die op elkaar lijken, maar toch niet volledig gelijk zijn. Er staat een minteken teveel. De oplossing komt als je bedenkt dat een minteken eigenlijk een factor $-1$ is, en die kan je ook afzonderen:
\begin{center}
$\begin{array}{ccll}
-4xy-4y+7z+7xz &=& (-4xy-4y)+(7z+7xz) & \text{poging tot groeperen} \\
&$=$& 4y(-x-1)+7z(1+x) &\text{gelijke factoren afzonderen} \\
&$=$& 4y\mathbf{(-1)}(x+1)+7z(1+x) &\text{-1 afzonderen} \\
&$=$& (x+1)(-4y+7z) &\text{gelijke factoren afzonderen} \\
\end{array}$
\end{center}

\end{voorbeeld}

\subsection{Merkwaardige producten}
Merkwaardige producten leveren soms een snelle manier om een uitdrukking te ontbinden in factoren. We zetten de belangrijkste even op een rijtje:

		\begin{ftonthoud}
			\begin{eqnarray*}
		(A+B)^2 &=& A^2+2AB+B^2 \\
		(A-B)^2 &=& A^2-2AB+B^2 \\
		(A+B)(A-B) &=& A^2-B^2 \\
		\end{eqnarray*}
		\end{ftonthoud}

Je kan deze merkwaardige producten bij het ontbinden in factoren gebruiken door het rechterlid om te zetten in het linkerlid; dat linkerlid is immers ontbonden in factoren.

\begin{voorbeeld}
	\begin{eqnarray*}
	x^2+4x+4=(x+2)^2\\
	y^2-6y+9=(y-3)^2\\
	z^2-16=(z+4)(z-4)
	\end{eqnarray*}
\end{voorbeeld}

Vaak is het lastig om de juiste formule te ontdekken. Je moet op zoek gaan naar een herkenningspunt:

\begin{center}
	$\begin{array}{ccll}
	(A+B)^2 &=& A^2+2AB+B^2 & \text{kwadraat plus dubbel product plus kwadraat} \\
	(A-B)^2 &=& A^2-2AB+B^2 & \text{kwadraat min dubbel product plus kwadraat} \\
	(A+B)(A-B) &=& A^2-B^2 & \text{kwadraat min kwadraat}
	\end{array}$
\end{center}

Het moment dat je dit herkent, moet je de gegeven getallen hierin proberen te passen. Dat betekent concreet dat A en B alles kunnen zijn, letters, getallen, eentermen of zelfs veeltermen.

\begin{voorbeeld}
	\begin{equation*}
	9z^2-16x^2
	\end{equation*}
	Hier zie je duidelijk een verschil van twee kwadraten. Je probeert nu de juiste waarden te vinden voor A en B. In dit geval zijn de kwadraten
	\begin{equation*}
	9z^2=(3z)^2 \text{ en } 16x^2=(4x)^2
	\end{equation*}
Dat betekent dus dat $A=3z$ en $B=4x$. Invullen geeft dan
\begin{equation*}
9z^2-16x^2=(3z-4x)(3z+4x)
\end{equation*}
\end{voorbeeld}
\begin{voorbeeld}
	\begin{equation*}
	4a^2x^2+9y^2+12axy
	\end{equation*}
	Je herkent hierin 2 kwadraten en een derde term. We gokken dus dat die derde term een dubbel product is. We controleren dit. Het eerste kwadraat is $4a^2x^2=(2ax)^2$, we stellen dus $A$ gelijk aan $2ax$. Het tweede kwadraat is $9y^2=(3y)^2$, dus $B=3y$. Zou dan $2AB=12axy$? Ja! Dus we hebben een merkwaardig product.
	\begin{equation*}
	4a^2x^2+9y^2+12axy = (2ax+3y)^2
	\end{equation*}
\end{voorbeeld}
\begin{voorbeeld}
	\begin{equation*}
	x^4-1
	\end{equation*}
	Dit lijkt geen merkwaardig product te zijn, maar is het wel. Het is belangrijk op te merken dat ook een vierde macht een kwadraat is:
	\begin{equation*}
	x^4=(x^2)^2 \text{ en } 1=1^2
	\end{equation*}
We vinden dus $A=x^2$ en $B=1$, we vullen in in de formule van het verschil van twee kwadraten:
\begin{equation*}
x^4-1 = (x^2-1)(x^2+1)
\end{equation*}
We zijn nog niet helemaal klaar. We kunnen immers misschien een of beide van deze factoren nog verder ontbinden. De factor $(x^2+1)$ is geen verschil van kwadraten, en heeft ook geen dubbele productterm, deze gaan we dus niet verder kunnen ontbinden. De factor $(x^2-1)$ is echter wel een verschil van twee kwadraten, dus:
\begin{equation*}
x^4-1 = (x^2-1)(x^2+1) = (x-1)(x+1)(x^2+1)
\end{equation*}
\end{voorbeeld}

\subsection{Discriminant}
Kwadratische uitdrukkingen kan je snel ontbinden met behulp van de abc-formule. Even herhalen:

\begin{definitie}
	Om een tweedegraadsvergelijking van de vorm $ax^2+bx+c=0$ te ontbinden zijn er drie gevallen:

\begin{itemize}
	\item $D>0$, de ontbinding is dan $a(x-x_1)(x-x_2)$
	\item $D=0$, de ontbinding is dan $a(x-x_1)(x-x_1)=a(x-x_1)^2$
	\item $D<0$, er is geen ontbinding
\end{itemize}

We zetten dus de co\"effici\"ent $a$ bij $x^2$ vooraan, en vullen de oplossingen  $x_1$ en $x_2$ in.

De discriminant wordt berekend als volgt:
\begin{equation*}
D=b^2-4ac
\end{equation*}
en de oplossingen zijn (als $D>0$)
\begin{equation*}
x_1=\frac{-b+\sqrt{D}}{2a} \text{ en } x_1=\frac{-b-\sqrt{D}}{2a}
\end{equation*}

Als $D=0$, dan is $x_1=x_2=\frac{-b}{2a}$.

\end{definitie}

\begin{voorbeeld}
	\begin{equation*}
	x^2-3x+2
	\end{equation*}
	De discriminant is $D=2^2-4ac=(-3)^2-4\cdot1\cdot2=1>0$. De oplossingen zijn
	\begin{equation*}
	x_1=\frac{-b+\sqrt{D}}{2a} \text{ en } x_1=\frac{-b-\sqrt{D}}{2a}
	\end{equation*}
De ontbinding is dus (de co\"effici\"ent $a$ bij $x^2$ is 1):
\begin{equation*}
x^2-3x+2=a(x-x_1)(x-x_2)=1(x-2)(x-1)
\end{equation*}
\end{voorbeeld}
\begin{voorbeeld}
\begin{equation*}
2x^2+x-1
\end{equation*}
De discriminant is $D=b^2-4ac=1^2-4.2.(-1)=9>0$. De oplossingen zijn
\begin{equation*}
x_1 = \frac{-b + \sqrt{D}}{2a}=\frac{-1+\sqrt{9}}{2\cdot2}=\frac{1}{2} \text{ en } x_2 = \frac{-b - \sqrt{D}}{2a}=\frac{-1-\sqrt{9}}{2.2}=-1
\end{equation*}
De ontbinding is dus (de co\"effici\"ent $a$ bij $x^2$ is $2$):
\begin{equation*}
2x^2+x-1=a(x-x_1)(x-x_2)=2(x-\frac{1}{2})(x+1)
\end{equation*}
\end{voorbeeld}
\begin{voorbeeld}
\begin{equation*}
3x^2-6x+3
\end{equation*}
	De discriminant is $D=b^2-4ac=(-6)^2-4\cdot3\cdot3=0>0$. Er is 1 oplossing en die is
	\begin{equation*}
	x_1 = \frac{-b + \sqrt{D}}{2a}=\frac{-(-6)\pm\sqrt{90}}{2.3}=1
	\end{equation*}
De ontbinding is dus (de co\"effici\"ent $a$ bij $x^2$ is 3):
\begin{equation*}
3x^2-6x+3 = a(x-x_1)^2=3(x-1)^2
\end{equation*}
\end{voorbeeld}
\begin{voorbeeld}
	\begin{equation*}
	x^2+x+1
	\end{equation*}
	De discriminant is $D=b^2-4ac=1^2-4\cdot1\cdot1=-3$. De discriminant is negatief, dus is er geen ontbinding mogelijk.
\end{voorbeeld}


\subsection{Discriminant - voorbeeld}
\begin{minipage}{.25\linewidth}
	\raggedright
	\includegraphics[width=4cm]{1_elem_rekenvaardigheden_A/inputs/QR_Code_DISCRIMINANT_module1}
\end{minipage}
\begin{minipage}{.7\linewidth}
	Zie filmpje MOOC.
\end{minipage}

\subsection{Regel van Horner}

\begin{eigenschap}
	Het rekenschema van Horner geeft vaak een manier om te ontbinden in factoren. Bij het rekenschema van Horner zonder je altijd een factor van de vorm $(x-a)$ af, waar $a$ een getal is.
\end{eigenschap}

We leggen het rekenschema van Horner uit aan de hand van een voorbeeld.

\begin{voorbeeld}
We proberen te ontbinden:
\begin{equation*}
x^3-4x+x^2-4
\end{equation*}

\textbf{Stap 1. Orden de machten van $x$ van groot naar klein}

De grootste voorkomende macht van  is hier , en die staat vooraan. De volgende is een kwadraat, dat moet op de tweede plaats, enzoverder: 
\begin{equation*}
x^3+x^2-4x-4
\end{equation*}

\textbf{Stap 2. Zorg dat elke macht van $x$ voorkomt, anders vul je aan}

In dit voorbeeld zijn alle machten van $x$ aanwezig, er ontbreekt niets. In een volgend voorbeeld bekijken we wat er zou moeten gebeuren als er wel een macht ontbreekt.

\textbf{Stap 3. Kies een waarde voor $a$}

In dit geval gaan we Horner toepassen met $a=2$.

\textbf{Stap 4. Teken het Hornerschema}

\begin{itemize}
	\item Teken een tabel met 3 rijen. 
	\item In de bovenste rij zet je alle co\"effici\"enten van de veelterm van $x$ in de juiste volgorde (Als er geen co\"effici\"ent bij  staat, dan is dit eigenlijk de co\"effici\"ent 1) 
	\item In de eerste kolom zet je het getal $a$ (in ons geval is $a=2$).
\end{itemize}

\begin{center}
	\begin{tabular}{ccc|cccc}
	& & & $x^3$ & $x^2$ & $x$ & $c^{te}$ \\
	& & & $\downarrow$ & $\downarrow$ & $\downarrow$ & $\downarrow$\\
	& & & 1 & 1 & -4 & -4 \\
	$a$ & $\rightarrow$ & 2 & & & & \\
	\hline 
	& & & & & & 
	\end{tabular}
\end{center}


\textbf{Stap 5. Het eerste cijfer mag je gewoon overschrijven}

\begin{center}
	\begin{tabular}{c|cccc}
		 & 1 & 1 & -4 & -4 \\
		2 & $\downarrow$ & & & \\
		\hline 
		 & 1 & & & 
	\end{tabular}
\end{center}


\textbf{Stap 6. Vermenigvuldig $a$ met dit eerste getal, en schrijf het in de tweede rij}

\begin{center}
	\begin{tabular}{c|cccc}
		& 1 & 1 & -4 & -4 \\
		2 & $\downarrow$ & 2 & & \\
		\hline 
		& 1 & & & 
	\end{tabular}
\end{center}

\textbf{Stap 7. Tel dit nieuw gevonden getal op bij de bovenste rij}

\begin{center}
	\begin{tabular}{c|cccc}
		& 1 & 1 & -4 & -4 \\
		2 & $\downarrow$ & 2 & & \\
		\hline 
		& 1 & 3 & & 
	\end{tabular}
\end{center}

\textbf{Stap 8. Herhaal deze stappen tot het laatste getal}

\begin{center}
	\begin{tabular}{c|cccc}
		& 1 & 1 & -4 & -4 \\
		2 & $\downarrow$ & 2 & 6 & \\
		\hline 
		& 1 & 3 & 2 & 
	\end{tabular}
\end{center}

We duiden aan dat het schema gestopt is door voor het laatste getal op de laatste rij 2 verticale strepen te zetten.

\begin{center}
	\begin{tabular}{c|ccccc}
		& 1 & 1 & -4 & & -4 \\
		2 & $\downarrow$ & 2 & 6 & & 4\\
		\hline 
		& 1 & 3 & 2 & $||$ & 0 
	\end{tabular}
\end{center}

Het Hornerschema is nu afgerond. Wat kan je hier nu uit besluiten?
\begin{itemize}
	\item Het laatste getal op de derde rij is een $0$. Dat betekent dat we de factor $(x-a)$ gaan kunnen afzonderen, in dit geval zonderen we dus $(x-2)$ af, omdat $a=2$.
	\item De overblijvende factor kan je aflezen op de laatste rij. Hier staan immers de co\"effici\"enten van deze factor. Deze staan geordend van hoogste naar laagste, en zijn in exponent eentje minder dan de oorspronkelijke opgave (geen derdemacht, maar een tweedemacht).
\end{itemize}

\begin{center}
	\begin{tabular}{c|ccccc}
		& 1 & 1 & -4 & & -4 \\
		2 & $\downarrow$ & 2 & 6 & & 4\\
		\hline 
		& 1 & 3 & 2 & $||$ & 0 \\
		& $\uparrow$ & $\uparrow$ & $\uparrow$ & & \\
		& $x^2$ & $x$ & $c^{te}$ & &
	\end{tabular}
\end{center}

We lezen dus af dat de overblijvende factor is:

\begin{equation*}
1\cdot x^2+3\cdot x+2
\end{equation*}

De eerste ontbinding is dan
\begin{equation*}
x^3+x^2-4x-4=(x-2)(x^2+3x+2)
\end{equation*}


We kunnen deze uitdrukking nog verder ontbinden door de factor $x^2+3x+2$ te ontbinden met de discriminant. We vinden dat $D=1$ en dat $x_1=-1$ en $x_2=-2$. De factor $x^2+3x+2$  is dan gelijk aan $(x+1)(x+2)$. Dit kunnen we invullen en krijgen tenslotte:
\begin{equation*}
x^3+x^2-4x-4=(x-2)(x^2+3x+2)=(x-2)(x+1)(x+2)
\end{equation*}


\begin{opmerking}
	\ \\
	\begin{enumerate}
	\item In de laatste stap hebben we de methode van de discriminant gebruikt om de overblijvende factor $x^3+x^2-4x$ nog verder te ontbinden. Maar we hadden evengoed nog eens het schema van Horner kunnen toepassen, nu met $a=-1$ (of met $a=-2$) om de factor $(x+1)$ (of om de factor $(x+2)$) af te zonderen. Zie ook het 2de voorbeeld hieronder.
	\item In plaats van eerst $(x-2)$ af te zonderen, hadden we ook eerst $(x+2)$ en daarna $(x-2)$ kunnen afzonderen, of eerst $(x+2)$ en dan $(x+1)$ of ...
	\item Welke waarde moet je voor $a$ kiezen? Een goede tip is om de delers te proberen van de constante term. In dit eerste voorbeeld zijn de delers van $-4$: $\pm1$, $\pm2$ en $\pm4$. Ga nu na of het beeld voor $x=1$ nul is, m.a.w. is $f(a)=0$? Stel, we kiezen $a=+1$, dan is $f(1)=(1)^3+(1)^2-4(1)-4=-6\ne 0$. Dus de factor  kan niet afgezonderd worden. We proberen nu $a=+2$, dan is $f(2)=(2)^3+(2)^2-4(2)-4$=0. Dus de factor $(x-2)$ kan afgezonderd worden.
	\item In feite kan je wel de factor $(x-1)$ afzonderen, maar dan vinden we bij het schema van Horner niet als laatste getal 0, maar wel de rest. Dit is dus de rest die overblijft als je de veelterm $x^3+x^2-4x$ deelt door $(x-1)$.
\end{enumerate}

\end{opmerking}
\end{voorbeeld}


\begin{voorbeeld}
	We werken een tweede voorbeeld volledig uit:
	
	\begin{equation*}
	x^3-7x+6
	\end{equation*}

\textbf{Stap 1. Orden de machten van $x$ van groot naar klein}

Dit is reeds in orde.

\textbf{Stap 2. Zorg dat elke macht van $x$ voorkomt, anders vul je aan}

De term voor $x^2$ ontbreekt. We vullen dus aan: 

\begin{equation*}
x^3-7x+6 = x^3+0x^2-7x+6
\end{equation*}

\textbf{Stap 3. Kies een waarde voor $a$}

We proberen met $a=-1$.


\textbf{Stap 4. Teken het Hornerschema}

\begin{center}
	\begin{tabular}{c|cccc}
		& 1 & 0 & -7 & 6 \\
		-1 &  &  &  & \\
		\hline 
		&  &  &  & 
	\end{tabular}
\end{center}

Uiteindelijk krijg je als uitgewerkt schema (doe dit zelf!):

\begin{center}
	\begin{tabular}{c|ccccc}
		& 1 & 0 & -7 & & 6 \\
		-1 & $\downarrow$ & -1 & 1 & & 6\\
		\hline 
		& 1 & -1 & -6 & $||$ & 12
	\end{tabular}
\end{center}

Het laatste getal is geen 0! Dat betekent dat we deze uitdrukking niet kunnen ontbinden met $a=-1$! We zullen dus een nieuwe $a$-waarde moeten kiezen, bijvoorbeeld $a=1$. We starten opnieuw.

\textbf{Stap 5. Kies een nieuwe waarde voor $a$}


We proberen met $a=1$.

\textbf{Stap 6. Teken het Hornerschema}

\begin{center}
	\begin{tabular}{c|cccc}
		& 1 & 0 & -7 & 6 \\
		1 &  &  &  & \\
		\hline 
		&  &  &  & 
	\end{tabular}
\end{center}

Uitwerken geeft:

\begin{center}
	\begin{tabular}{c|ccccc}
		& 1 & 0 & -7 & & 6 \\
		1 & $\downarrow$ & 1 & 1 & & -6\\
		\hline 
		& 1 & 1 & -6 & $||$ & 0
	\end{tabular}
\end{center}

Nu krijgen we wel een 0 en kunnen besluiten dat we kunnen ontbinden:

\begin{equation*}
x^3-7x+6=(x-1)(x^2+x-6)
\end{equation*}

We zijn nog niet klaar. Misschien kan je $x^2+x-6$ nog verder ontbinden. Je kan deze ontbinding doen met Horner of je kan ze doen met de methode van de discriminant. We passen nog eens Horner toe op $x^2+x-6$ met $a=-3$:

\begin{center}
	\begin{tabular}{c|ccc}
		& 1 & 1 & -6  \\
		-3 &  &  &  \\
		\hline 
		&  &  &  
	\end{tabular}
\end{center}

Uitwerken geeft:

\begin{center}
	\begin{tabular}{c|cccc}
		& 1 & 1 & & -6 \\
		-3 & $\downarrow$ & -3 & & 6 \\
		\hline 
		& 1 & -2 & $||$  & 0
	\end{tabular}
\end{center}


We vinden een 0, dus is het ontbindbaar met $a=-3$. Dat betekent dat we de factor $(x+3)$ kunnen afzonderen. De overblijvende factor kan je aflezen op de onderste rij. Deze is in graad eentje lager dan de oorspronkelijke, dus van de eerste graad.

\begin{equation*}
x^2+x-6=(x+3)(x-2)
\end{equation*}

Vergeet niet dat dit niet de oorspronkelijke opgave was! We kunnen dit wel gebruiken door onze bevindingen in te vullen:

\begin{equation*}
x^3-7x+6=(x-1)(x^2+x-6)=(x-1)(x+3)(x-2)
\end{equation*}

\subsubsection{Opmerking:} we hadden dus ook 2 maal na elkaar Horner kunnen toepassen; dit ziet er dan zo uit:


\begin{center}
	\begin{tabular}{c|cccccc}
		& 1 & 0 & &-7 & & 6 \\
		1 & $\downarrow$ & 1 & & 1 & & -6\\
		\hline 
		& 1 & 1 & & -6 & $||$ & 0 \\
		-3 & $\downarrow$ & -3 & & 6 & &\\
		\hline
		& 1 & -2 & $||$ & 0 &
	\end{tabular}
\end{center}


We kunnen dus de factor $(x-1)$ en de factor $(x-(-3))=(x+3)$ afzonderen en krijgen terug als resultaat:

\begin{equation*}
x^3-7x+6=(x-1)(x+3)(x-2)
\end{equation*}
\end{voorbeeld}

\subsection{Regel van Horner - voorbeeld}
\begin{minipage}{.25\linewidth}
	\raggedright
	\includegraphics[width=4cm]{1_elem_rekenvaardigheden_A/inputs/QR_Code_HORNER_module1}
\end{minipage}
\begin{minipage}{.7\linewidth}
	Zie filmpje MOOC.
\end{minipage}

\subsection{Test - ontbinden in factoren}
TODO