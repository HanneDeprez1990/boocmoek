\section{Bewerkingen}

\subsection{Volgorde van bewerkingen}
TODO

\subsection{Volgorde van bewerkingen - voorbeeld 1}
Zie filmpje MOOC.

\subsection{Volgorde van bewerkingen - voorbeeld 2}
Zie filmpje MOOC.

\subsection{Rekenen met machten of exponenten}
TODO

\subsection{Werken met haakjes}
TODO

\subsection{Begrippen tegengestelde en omgekeerde van een getal}

\subsubsection{Tegengestelde}

Het \emph{tegengestelde} van een getal $n$ is het getal dat opgeteld
bij $n$, nul oplevert. Het tegengestelde van n wordt genoteerd met
$-n$. Het tegengestelde van een getal heeft dus dezelfde absolute
waarde als het gegeven getal maar met een tegengesteld teken. De som
van een getal met zijn tegengestelde is dus steeds 0: $n+(-n)=0$.

\noindent Zo is het tegengestelde van 12 gelijk aan \textminus 12
omdat 12 + (\textminus 12) = 0, en het tegengestelde van $-\sqrt{3}$
is $\sqrt{3}$ omdat $-\sqrt{3}+\sqrt{3}=0$.

\medskip{}


\noindent Het tegengestelde van nul is nul. Dit is het enige getal
waarvan het tegengestelde gelijk is aan zichzelf. 

\noindent We zeggen dat 0 het \emph{neutraal element} is met betrekking
tot optellen.

\noindent (In de abstracte algebra is het tegengestelde het inverse
element voor een bewerking die met een plusteken genoteerd wordt).

\bigskip{}

\subsubsection{Omgekeerde}

\noindent Het \emph{omgekeerde} of de \emph{reciproque} (vaker: de
reciproke) van een getal of grootheid is 1 gedeeld door dat getal
of die grootheid. Het omgekeerde van een breuk ontstaat door teller
en noemer te verwisselen. Het omgekeerde van 7 is 1/7 en het omgekeerde
van 2/3 is 3/2. Passen we dit toe op enkele grootheden: de hertz is
het omgekeerde van de seconde: 1 Hz = 1/s en de siemens is het omgekeerde
van de ohm: 1 S = 1/$\Omega$.

\noindent Het product van een getal met zijn omgekeerde levert 1 op:
${\displaystyle n.\frac{1}{n}=1}$. 

\noindent We zeggen dat 1 het \emph{neutraal element} is voor de vermenigvuldiging.
We zien ook dat nul geen omgekeerde heeft.

\noindent (In de abstracte algebra is het omgekeerde het inverse element
voor een bewerking die met een vermenigvuldigingsteken genoteerd wordt).

\medskip{}


\noindent Een interessante toepassing van het omgekeerde vinden we
bij de deling waarbij de deler een breuk is. De deling kan dan ook
uitgevoerd worden door het deeltal te vermenigvuldigingen met het
omgekeerde van de deler. 

\noindent Populair gezegd: delen door een breuk is vermenigvuldigen
met het omgekeerde.

\noindent Voorbeeld: $5:1/3=5\times3=15$ 


\subsection{Rekenen met wortels}
TODO


\subsection{Rekenen met breuken}
TODO

\subsection{Rekenen met logaritmen}
TODO

\subsection{Algebra: het rekenen met letters}


\subsubsection{Gebruik van letters in de wiskunde}

De algebra is de kunst van het rekenen met letters. Die letters stellen
meestal getallen voor, en met getallen weet je hoe je kan rekenen:
optellen, aftrekken, vermenigvuldigen en delen. Bij algebra voeren
we dat soort operaties ook uit, alleen gebruiken we daarbij niet alleen
getallen, maar ook \emph{variabelen}. Dat zijn als het ware 'dingen'
waarvan we de waarde (nog) niet kennen of waarbij zo'n 'ding' meerdere
waarden kan aannemen. Voor de variabelen gebruiken we letters.

\noindent In de algebra worden vervolgens allerlei verbanden en structuren
onderzocht. Welke regels en eigenschappen gelden er, wat mag wel en
wat mag niet? Verzamelingen spelen hierbij een grote rol, maar ook
afbeeldingen. Door gebruik te maken van variabelen, vergelijkingen,
en dergelijke meer is het mogelijk 'algemene' uitspraken te doen over
verzamelingen en operaties. Bekende eigenschappen van rekenen met
getallen zijn de commutatieve eigenschap en de distributieve eigenschap.

\medskip{}

Voorbeeld: In het algemeen geldt voor het optellen en vermenigvuldigen
van $a$ en $b$ (natuurlijke getallen) dat:

\begin{equation*}
a + b = b + a \text{ en } a \times{} b = b \times{} a
\end{equation*}

Dat lijkt vanzelfsprekend, maar dat is het niet. Bij de operatie delen
geldt het bijvoorbeeld niet! ($12:4$ is niet hetzelfde als $4:12$)

\medskip{}

Een andere bekende eigenschap is de distributiviteit:

\begin{equation*}
a \texttimes{} (b + c) = a \texttimes{} b + a \texttimes{} c
\end{equation*}

Dat gebruiken we heel vaak, maar mag dat zomaar, wanneer wel, wanneer
niet?

Geldt dit bijvoorbeeld?

\begin{equation*}
a + (b \times{} c) = (a + b) \times{} (a + c) ???
\end{equation*}

\medskip{}


\noindent Met variabelen kunnen we ook \emph{formules} opstellen.
Stel dat we de straal van een cirkel voorstellen door de letter $r$.
Dan is de $\mathrm{omtrek}=2\pi r$ en de $\mathrm{oppervlakte}=\pi r^{2}$.
Nu kunnen we voor eender welke cirkel zijn omtrek en oppervlakte berekenen
door de waarde van $r$ te vervangen (we zeggen te \emph{substitueren})
door een getal.

\medskip{}


\noindent Een \emph{uitdrukking} is een geheel van termen bestaande
uit getallen, variabelen, bewerkingstekens (zoals $+$, $-$, $x$, $:$) en andere
wiskundige tekens (zoals haakjes): $2\pi r$, $3-8$, $2x+5$, ...

\noindent Vervolgens kunnen we twee uitdrukkingen aan elkaar gelijk
stellen. We spreken dan van een \emph{vergelijking}. Vergelijkingen
kunnen vaak worden afgeleid uit een stukje tekst (de klassieke vraagstukjes).
Als een vergelijking een variabele bevat, dan kunnen we trachten de
waarde (of alle waarden) van de variabele te vinden waardoor er een
ware bewering ontstaat als we de variabele vervangen door de gevonden
waarde(n). Dit wordt \emph{oplossen van de vergelijking} genoemd.
De waarde van de variabele heet in dit geval \emph{de wortel} of \emph{de
oplossing} van de vergelijking.

\noindent Een formule kan (en mag) meerdere variabelen en onbekenden
bevatten. \medskip{}


\uline{Voorbeeld 1}:

De oppervlakte van een rechthoek is $b.h$ waarbij $b$ de basis is,
en $h$ de hoogte.

Als je nu gegeven krijgt dat $b$ gelijk is aan 4 en $h$ gelijk is
aan 2 (we gebruiken even geen eenheden ter vereenvoudiging), dan kan
je de oppervlakte berekenen door de letters te substitueren, door
ze te vervangen door de echte waarden of echte getallen:

\begin{equation*}
\mathrm{oppervlakte}=b\cdot h=(4)\cdot(2)=8
\end{equation*}

Hoewel het hier niet echt nodig was, hebben we toch de gesubstitueerde
getallen tussen haken gezet. Je doet dit om aan te duiden dat je de
letter in zijn geheel vervangt door het getal. Bij moeilijke formules
maakt dit wel degelijk veel uit! Stel, je wil de oppervlakte van een
andere rechthoek berekenen, en je hebt gegeven dat de hoogte 2 meer
moet zijn dan de basis, of met andere woorden, $h=b+2$. Over de basis
weet je nog niets. Je kan dus enkel $h$ vervangen:

\begin{equation*}
\mathrm{oppervlakte}=b\cdot h=b\cdot(b+2)
\end{equation*}

Je ziet dus dat de letter $h$ volledig vervangen is door $b+2$,
gesymboliseerd door de haakjes.

\medskip{}


\uline{Voorbeeld 2}:

Ik heb een pot verf waarmee een oppervlakte van $5m^2$
kan geschilderd worden. Hoeveel ronde tafeltjes kan ik hiermee een
likje verf geven? De diameter van de tafeltjes is $1~m$.

Oplossing: 

We zoeken het aantal tafeltjes die volledig geschilderd kunnen worden.
Stel deze onbekende variabele voor door bijvoorbeeld $x$ (gevraagd).

We hebben voldoende verf om $~5~m^2$
te schilderen (gegeven).

De diameter $d$ van een tafeltje is 1 meter: $d=1~m$ (gegeven).

De straal $r$ van een cirkel is de helft van de diameter: $2r=d$

zodat de oppervlakte $s$ van 1 tafeltje gelijk is aan: $s=\pi r^{2}=\pi\left(\frac{d}{2}\right)^{2}=\frac{\pi d^{2}}{4}=\frac{\pi(1)^{2}}{4}=0,79~m^2$

We stellen de vergelijking op: $5=x.s$ 

en lossen deze tenslotte op naar de onbekende variabele: $x=\frac{5}{s}=\frac{5}{0,79}=6,37$

Besluit: we kunnen 6 tafeltjes schilderen (en dan blijft er nog een
klein beetje verf over).


\subsubsection{Manier van rekenen}

\begin{itemize}
	\item{Notaties}
	\begin{itemize}
		\item In het rekenen met letters wordt het puntje van de vermenigvuldiging
		vaak niet geschreven. In plaats van $a\cdot b\cdot c$ schrijf je
		$abc$. Of, in plaats van $2\cdot b$ schrijf je $2b$. Het puntje
		schrijven is niet fout, maar hoeft niet.
		\item Letters schrijf je achteraan in een uitdrukking, dus $b\cdot2\cdot c$
		schrijf je best als $2bc$. (Dit mag, want in een product mag je factoren
		van plaats verwisselen.)
		\item Vaak worden de letters die voorkomen ook in alfabetische volgorde
		geschreven. Bijvoorbeeld $yxz$ schrijf je beter als $xyz$. 
	\end{itemize}
	\noindent Deze notaties maken het meestal gemakkelijker om verder
	te rekenen, en als je je aan deze manier van schrijven houdt, ziet
	alles er meer netjes uit. Het is niet absoluut verplicht om te doen,
	maar wel aan te raden.
	
	
	\item{Eentermen en veeltermen}
	
	Een eenterm is een uitdrukking die uit 1 term bestaat, een veelterm
	bestaat uit meerdere termen, logisch toch? I.p.v. een veelterm spreken
	we ook over een polynoom.
	
	\textbullet{} $a$ en $4abcx$ zijn eentermen
	
	\textbullet{} $x^{2}+3x+1$ en $b-x$ zijn veeltermen
	
	
	\item{Som en verschil}
	
	Je kan de som of het verschil van eentermen maken, maar enkel als
	de letters of lettercombinatie in de eenterm gelijk is. Lijkt een
	cryptische regel, maar dat is het niet. 
	
	\noindent Bijvoorbeeld:
	
	\textbullet{} $a+3a=4a$ Dit gaat perfect, want de letters zijn gelijk.
	
	\textbullet{} $a+2b=?$ Dit gaat niet, $a$ en $b$ zijn verschillende
	letters. 
	
	\textbullet{} $2ab+bc=?$ Dit gaat niet want $ab$ en $bc$ zijn verschillende
	lettercombinaties. 
	
	\textbullet{} $3ab-ab=2ab$ Dit gaat perfect, want de lettercombinaties
	zijn gelijk.
	
	\textbullet{} $ab+ba=ab+ab=2ab$ Ook dit gaat, al is het met een omweg.
	Door te sorteren zie je dat de lettercombinaties gelijk zijn.
	\begin{itemize}
		\item $a+a^{2}=?$ Dit gaat niet, want de lettercombinaties zijn niet gelijk,
		eentje is een $a$ en de andere is $a^{2}$.
	\end{itemize}
	Let dus goed op bij het optellen van letters of combinaties, en sorteer
	de letters om gelijkaardige combinaties te zien. 
	
	\noindent Het optellen van veeltermen is een uitbreiding van deze
	regel. Gelijksoortige eentermen (dus met dezelfde lettercombinaties)
	tel je op: 
	\begin{itemize}
		\item \noindent $(3x+y)+(4x+z)=7x+y+z$ Enkel de eentermen van de $x$ kan
		je optellen, de rest moet je laten staan. 
		\item \noindent $(3x+y)-(a+b)=3x+y-a-b$ Jammer, hier kan je niets optellen.
	\end{itemize}
	
	\item{Producten}
	
	Producten van verschillende letters vorm je door de letters achter
	elkaar te schrijven. De bijhorende getallen vermenigvuldig je ook,
	en plaats je voorop. 
	
	\noindent Voor eentermen is dit:
	
	\textbullet{} $a\cdot b\cdot x=abx$ 
	
	\textbullet{} $x\cdot2a\cdot2b=4abx$ \medskip{}
	
	
	\noindent Als de letters gelijk zijn, kan je machten vormen. Je gebruikt
	hier eigenlijk de rekenregel: ``machten met hetzelfde grondtal vermenigvuldigen
	is de exponenten optellen''.
	\begin{itemize}
		\item \noindent $a\cdot a=a^{2}$
		\item \noindent $x^{2}\cdot x^{3}=x^{2+3}=x^{5}$
		\item \noindent $ab\cdot ab=(ab)^{2}=a^{2}\cdot b^{2}=a^{2}b^{2}$
	\end{itemize}
	\noindent Veeltermen vermenigvuldigen is lastiger, maar maakt gebruik
	van rekenregels die je al kent. De belangrijkste is de distributiviteit.
	Op die manier herleid je het probleem naar het product van eentermen:
	\begin{equation*}
	x\cdot(a+b)=x\cdot a+x\cdot b=ax+bx
	\end{equation*}
%	\begin{itemize}
%		\item \noindent $x\cdot(a+b)=x\cdot a+x\cdot b=ax+bx$
%	\end{itemize}
	\noindent Iets complexer: nog enkele voorbeelden:
	
	\begin{math}
	\begin{array}{ccc|r}
	(a+2x)\cdot(3x) & = & a\cdot(3x)+(2x)\cdot(3x) & \text{ distributiviteit}\\
	& = & 3ax+6x^{2} & \text{uitrekenen en ordenen}\\
	& & & \\
	(a+b)\cdot(a+b) & = & a\cdot a+b\cdot a+a\cdot b+b\cdot b &  \text{distributiviteit}\\
	& = & a^{2}+ba+ab+b^{2} &  \text{uitrekenen}\\
	& = & a^{2}+2ab+b^{2} &  \text{ordenen en optellen}\\
	& &  & \\
	(a+b)\cdot(x+y) & = & a\cdot x+a\cdot y+b\cdot x+b\cdot y=ax+ay+bx+by\\
	\end{array}
	\end{math}
	

	\noindent In module {*}{*}{*}{*} gaan we kijken naar de omgekeerde
	bewerkingen: ontbinden in factoren (afzonderen van eentermen en veeltermen),
	en merkwaardige producten.
	
	
	\item{Quoti\"enten}
	
	Quoti\"enten, en dus daarmee samenhorend breuken, bereken je vaak door
	vereenvoudigingen. Je doet een vereenvoudiging net op dezelfde manier
	als bij het vereenvoudigen van een breuk met gewone getallen:
	
	\begin{eqnarray*}
		\frac{6}{15} & = & \frac{2.3}{3.5} = \frac{2}{5}\\
		\frac{ab}{bc} & = & \frac{a}{c} \\
	\end{eqnarray*}
	

	\noindent Als er in de teller een letter staat die ook in de noemer
	staat, kan je de breuk vereenvoudigen. Maar de volgende breuk kan
	je NIET vereenvoudigen:
	
	\begin{equation*}
		{\displaystyle \frac{ab}{bc+1}}
	\end{equation*}
	
	\noindent In de noemer staat immers niet bij elke term een $b$, dus
	is vereenvoudiging hier niet mogelijk. Wat je echter wel kan (en mag
	doen) is zowel in teller als noemer de letter $b$ buiten de haakjes
	brengen; daarna kan je $b$ schrappen, maar of je daarmee de breuk
	vereenvoudigd hebt, laten we in het midden. De variabele $b$ mag
	nu immers niet meer nul worden.
	
	\begin{equation*}
	{\displaystyle \frac{ab}{bc+1}={\displaystyle \frac{ab}{b(c+\frac{1}{b})}=\frac{a}{c+\frac{1}{b}}}}
	\end{equation*}
	
	\medskip{}
	
	
	\noindent Het gebruik van rekenregels heb je echt nodig bij machten:
	
	\noindent ${\displaystyle \frac{a^{3}}{a^{2}}=a^{3-2}=a^{1}=a}$
	
	\noindent \medskip{}
	
	
	\noindent Nog een voorbeeld:
	
	\begin{math}
	\begin{array}{ccc|r}
	{\displaystyle \frac{p\sqrt{p}}{p^{2}}} & = & {\displaystyle \frac{p.p^{\frac{1}{2}}}{p^{2}}} &   \text{vierkantswortel omzetten in macht}\\
	& = & {\displaystyle \frac{p^{1+\frac{1}{2}}}{p^{2}}} &  \text{product van machten met gelijk grondtal is exponenten optellen}\\
	& = & {\displaystyle \frac{p^{\frac{3}{2}}}{p^{2}}} &  \\
	& = & {\displaystyle p^{\frac{3}{2}-2}} &  \text{quoti\"ent van machten met gelijk grondtal is exponenten aftrekken}\\
	& = & {\displaystyle p^{-\frac{1}{2}}} &  \text{Verder dan dit kan je de opgave niet vereenvoudigen.}\\
	\end{array}
	\end{math}
	
	\noindent Als je iets meer ervaring hebt: ${\displaystyle \frac{p\sqrt{p}}{p^{2}}=p^{1+\frac{1}{2}-2}={\displaystyle p^{-\frac{1}{2}}}}$.
	
\end{itemize}


\subsubsection{Voorbeelden}

\noindent Rekenen met letters kan dan wel op dezelfde manier gaan
zoals rekenen met getallen, toch is er vaak een moeilijkheid bij vereenvoudigingen
en dergelijke. Vandaar dat we enkele voorbeelden bekijken.

\begin{itemize}
	\item{Breuken optellen en aftrekken}
	
	\noindent Net zoals bij de gewone breuken, is de sleutel hier het
	op dezelfde noemer brengen van de breuken:
	
	\begin{equation*}
	{\displaystyle \frac{4}{a}+\frac{7}{a}=\frac{11}{a}}
	\end{equation*}
	
	\noindent Volgend voorbeeld is iets lastiger:
	
	\begin{equation*}
		{\displaystyle \frac{a}{b}+\frac{c}{d}}
	\end{equation*}
	
	\noindent Je zoekt hier een gelijke noemer. Omdat de noemers niets
	met mekaar te maken hebben, neem je gewoon het product van de noemers
	als gemeenschappelijke noemer, namelijk $b\cdot d$. Dat zou je ook
	doen als je de som $\frac{1}{7}+\frac{2}{3}$ zou moeten oplossen,
	dan zou je ook als gemeenschappelijke noemer $3\cdot7=21$ kiezen.
	
	\begin{equation*}
		{\displaystyle \frac{a}{b}+\frac{c}{d}=\frac{ad}{bd}+\frac{cb}{bd}=\frac{ad+bc}{bd}}
	\end{equation*}
		
	\noindent Een iets moeilijker voorbeeld:
	
	\begin{eqnarray*}
		{\displaystyle \frac{y}{x}+\frac{y}{x+1}} & = & {\displaystyle \frac{y.(x+1)}{x.(x+1)}+\frac{y.x}{(x+1).x}} \\
		& = & {\displaystyle \frac{xy+y}{x.(x+1)}+\frac{xy}{(x+1).x}} \\
		& = & {\displaystyle \frac{xy+y+xy}{x.(x+1)}} \\
		& = & {\displaystyle \frac{2xy+y}{x(x+1)}} \\
		& = & {\displaystyle \frac{2xy+y}{x^{2}+x}} 
	\end{eqnarray*}
	We werken hier ook de noemer uit.
	
	
	\noindent \medskip{}
	Nog een laatste voorbeeld:
	
	\begin{equation*}
	{\displaystyle \frac{x+3+a}{a+1}-\frac{2x+5-b}{2a+2}}
	\end{equation*}
	
	\noindent In dit voorbeeld is de gemeenschappelijke noemer gelijk
	aan $2a+2$, en niet meteen het product van de twee noemers! Immers,
	de twee noemers hebben een gemeenschappelijk factor, namelijk $a+1$.
	Daar moet je gebruik van maken. Concreet betekent dit dat we de eerste
	breuk in teller en noemer moeten vermenigvuldigen met 2, en de tweede
	noemer kunnen we gewoon laten staan:
	
	\begin{math}
	\begin{array}{ccl|r}
	{\displaystyle \frac{x+3+a}{a+1}-\frac{2x+5-b}{2a+2}} & = & {\displaystyle \frac{(x+3+a).2}{(a+1).2}-\frac{2x+5-b}{2a+2}}  & \text{op gelijke noemer zetten}\\
	& = & {\displaystyle \frac{2x+6+2a}{2a+2}-\frac{2x+5-b}{2a+2}} &\text{uitrekenen}\\
	& = & {\displaystyle \frac{2x+6+2a-(2x+5-b)}{2a+2}} & \text{ verschil van tellers, vergeet geen haken!}\\
	& = & {\displaystyle \frac{2x+6+2a-2x-5+b}{2a+2}} & \text{minteken verdelen over termen}\\
	& = & {\displaystyle \frac{1+2a+b}{2a+2}} & \text{gelijkaardige termen optellen}\\
	\end{array}
	\end{math}
	
	
	\item{Breuken vermenigvuldigen en delen}
	
	Breuken vermenigvuldigen en delen is eenvoudiger dan optellen en aftrekken;
	opnieuw gebruik je dezelfde rekenregels als voordien. 
	
	\noindent Vermenigvuldigen van breuken is tellers met tellers en noemers
	met noemers vermenigvuldigen:
	
	\begin{equation*}
		{\displaystyle \frac{a}{b}\cdot\frac{2}{x}=\frac{a.2}{b.x}=\frac{2a}{bx}} 
	\end{equation*}
	
	\noindent Een getal delen door een breuk is vermenigvuldigen met het
	omgekeerde van die breuk:
	
	\begin{equation*}
		{\displaystyle \frac{a}{b}:\frac{2}{x}=\frac{a}{b}.\frac{x}{2}=\frac{a.x}{b.2}=\frac{ax}{2b}}
	\end{equation*}	
	
	\noindent Ietsje moeilijker:
	
	\begin{math}
		\centering
	\begin{array}{ccc|r}
	{\displaystyle \frac{a+b}{c+d}.\frac{x+1}{y+1}} & = & {\displaystyle \frac{(a+b)}{(c+d)}.\frac{(x+1)}{(y+1)}} & \text{breuken vermenigvuldigen}\\
	& = & {\displaystyle \frac{ax+a+bx+b}{cy+c+dy+d}} & \text{uitrekenen}\\
	\end{array}
	\end{math}
	
%	\begin{tabular}{|c|c|c|c|r|}
%		\hline 
%		{\displaystyle \frac{a+b}{c+d}.\frac{x+1}{y+1}}$ & $=$ & ${\displaystyle \frac{(a+b)}{(c+d)}.\frac{(x+1)}{(y+1)}}$ &  & breuken vermenigvuldigen\\
%		\hline 
%		& $=$ & ${\displaystyle \frac{ax+a+bx+b}{cy+c+dy+d}}$ &  & uitrekenen\\
%		\hline 
%	\end{tabular}\medskip{}
	
	
	\noindent Vergeet ook niet (indien mogelijk) om nadien te vereenvoudigen:
	
	\begin{math}
	\centering
	\begin{array}{ccc|r}
	{\displaystyle \frac{2a+4x}{4b}:\frac{b}{2y}} & = & {\displaystyle \frac{2a+4x}{4b}.\frac{2y}{b}} & \text{ vermenigvuldigen met omgekeerde}\\
	& = & {\displaystyle \frac{(2a+4x).(2y)}{(4b).(b)}} & \text{breuken vermenigvuldigen}\\
	& = & {\displaystyle \frac{4ay+8xy}{4b^{2}}} &  \text{uitrekenen}\\
	& = & {\displaystyle \frac{ay+2xy}{b^{2}}} & \text{vereenvoudigen}\\
	\end{array}
	\end{math}
	
	\item{Machten en wortels}
	
	\begin{math}
	\begin{array}{ccccc}
	(2a)^{4} & = & 2^{4}.a^{4} &=& 16a^{4} \\
	\sqrt{16a} & = & \sqrt{16}.\sqrt{a}&=&4\sqrt{a} \\
	\left(\frac{a+2}{3}\right)^{2} & = &  \frac{(a+2)^{2}}{3^{2}}&=&\frac{a^{2}+4a+4}{9} \\
	\left(\frac{b}{1-a}\right)^{-2} & = &  \left(\frac{1-a}{b}\right)^{2}&=&\frac{(1-a)^{2}}{b^{2}}=\frac{1-2a+a^{2}}{b^{2}} \\
	\end{array}
	\end{math}
	
\end{itemize}


\subsection{Evenredigheden en de regel van drie}


\subsubsection{De regel van drie}

In een aantal vraagstukken worden er twee grootheden met elkaar vergeleken.
Deze twee grootheden houden dikwijls verband met elkaar. Dit wil zeggen
als de ene grootheid groter wordt, vermeerdert ook de andere. En als
de ene grootheid kleiner wordt, vermindert de andere grootheid eveneens.
We zeggen dat de twee grootheden zich \emph{evenredig} verhouden tot
elkaar (symbooltje $\sim$).\medskip{}


\uline{Voorbeeld 1}: 

Een doos ballonnen bevat 75 ballonnen en kost \EUR{10}. Hoeveel kosten
dan 90 ballonnen?

Er is een verband, want als de ene grootheid (= het aantal ballonnen)
vermeerdert, vermeerdert hier ook de andere grootheid (= de prijs
van de ballonnen). Hoe meer ballonnen je wil, hoe meer je zal moeten
betalen. Zulke vraagstukken kan je oplossen met de zogenaamde \emph{regel
van drie}. Bij dit soort opgaven ken je altijd 3 getallen en moet
je het vierde getal berekenen, vandaar...

\medskip{}

$\begin{array}{ccc}
75 ~\text{ballonnen} & \sim & \text{\EUR{10}} \\
1 ~\text{ballon} & \sim & \frac{\text{\EUR{10}}}{75 \text{ ballonnen}} \\ 
90 ~\text{ballonnen} & \sim & \frac{\text{\EUR{10}}}{75 \text{ ballonnen}}.90 \text{ ballonnen} = \text{\EUR{12}}\\ 
\end{array}$

Antwoord : Voor 90 ballonnen betaal je dan \EUR{12}.

Infeite ga je eerst op zoek naar de ``eenheidsprijs'' voor \'e\'en ballon:
\EUR{10} voor 75 ballonnen komt overeen met $10/75=0,133\frac{\text{EUR}}{\text{ballon}}$.
We kunnen ons ook afvragen hoeveel ballonnen je kan kopen voor \'e\'en
euro: $75/10=7,5\frac{\text{ballonnen}}{\text{EUR}}$ (praktisch zou dit betekenen dat je met \'e\'en euro 7 ballonnen kan kopen; je betaalt daarvoor $7.0,133=\text{\EUR{0.93}}$
en je houdt nog 6 eurocent over). \bigskip{}


We zeggen dat een verhouding \emph{omgekeerd evenredig} is wanneer
een vermeerdering langs de ene kant, een even grote vermindering aan
de andere kant veroorzaakt. 

\medskip{}


\uline{Voorbeeld 2}: 

Stel, ik heb voldoende veevoeder om 35 varkens gedurende 22 dagen
te voeren. Hoeveel dagen kom ik toe met dezelfde hoeveelheid veevoeder
als ik plots 70 varkens zou hebben?

Er is een verband, want als de ene grootheid (= het aantal varkens)
vermeerdert, vermindert hier de andere grootheid (= het aantal dagen
voederen). Zulke vraagstukken kan je eveneens oplossen met de regel
van drie. \medskip{}


%$\begin{array}{ccc}
%\text{veevoeder voor 35 varkens} & \sim & 22 \text{dagen} \\
%1 ~\text{ballon} & \sim & \frac{\text{\EUR{10}}}{75 \text{ ballonnen}} \\ 
%90 ~\text{ballonnen} & \sim & \frac{\text{\EUR{10}}}{75 \text{ ballonnen}}.90 \text{ ballonnen} = \text{\EUR{12}}\\ 
%\end{array}$

\begin{tabular}{lclcc}
veevoeder voor 35 varkens & $\sim$ & 22 dagen &  & \\
veevoeder voor 1 varken & $\sim$ & 22 dagen.veevoeder voor 35 varkens &  & \\
veevoeder voor 70 varkens & $\sim$ & $\frac{22\:\mathrm{dagen}.\mathrm{veevoeder\:voor}\:35\:\mathrm{varkens}}{\mathrm{veevoeder\:voor}\:70\:\mathrm{varkens}}$ & = & 11 dagen\\
\end{tabular}

Antwoord : Met dezelfde hoeveelheid veevoeder kan je $70$ varkens $11$
dagen lang voederen. \bigskip{}


Dit is dus niet hetzelfde als: Stel, ik heb veevoeder om $35$ varkens
gedurende $22$ dagen te voeren. Hoeveel veevoeder heb ik nodig om $70$
varkens te voederen gedurende diezelfde $22$ dagen?

Antwoord: de exacte hoeveelheid veevoeder in kg (dat een varken per
dag nodig heeft) kennen we niet, dus kunnen we ook niet in ``zoveel
kg'' antwoorden, maar als het aantal varkens verdubbelt, dan zal
de hoeveelheid veevoeder ook moeten verdubbelen. De verhouding ``hoeveelheid
veevoeder per varken'' verandert niet; we zeggen dat de verhouding
constant blijft. In symbolen:

\begin{equation*}
\frac{\mathrm{hoeveelheid\:veevoeder\:x}}{35\:\mathrm{varkens}}=\frac{\mathrm{hoeveelheid\:veevoeder\:y}}{70\:\mathrm{varkens}}=\text{constant}
\end{equation*}

Schrijven we dit iets anders: 
\begin{equation*}
\frac{70\:\mathrm{varkens}}{35\:\mathrm{varkens}}=\frac{\mathrm{hoeveelheid\:veevoeder\:y}}{\mathrm{hoeveelheid\:veevoeder\:x}}=\text{constant}=2
\end{equation*}

Besluit: de hoeveelheid veevoeder voor $70$ varkens = $2$ . hoeveelheid
veevoeder voor $35$ varkens.\medskip{}

Dit vraagstukje laat zien wat we bedoelen met ``kruiselings vermenigvuldigen''.

\subsubsection{Kruiselings vermenigvuldigen}

Kruiselings vermenigvuldigen is de benaming voor een rekenkundige handeling om een vergelijking tussen twee verhoudingen (evenredigheid) te vereenvoudigen. Daarbij wordt de noemer van het linkerlid vermenigvuldigd met de teller van het rechterlid, en de teller van het linkerlid vermenigvuldigd met de noemer van het rechterlid. Beide vermenigvuldigingen stelt men dan aan elkaar gelijk. De vergelijking  wordt door kruislings vermenigvuldigen vereenvoudigd tot $20y=40$, waaruit weer volgt dat $y=2$.

In formulevorm:

\begin{equation*}
\frac{a}{b} = \frac{c}{d} \Rightarrow ad = bc
\end{equation*}

Indien $bc \neq 0$ geldt ook het omgekeerde:

\begin{equation}
ad=bc \Rightarrow \frac{a}{b}=\frac{c}{d}
\end{equation}

De achtergrond van dit ''trucje'' is dat beide zijden van de vergelijking met hetzelfde re\"ele getal vermenigvuldigd kunnen worden, zonder dat de vergelijking verandert. In bovenstaande formulering kunnen beide zijden met het getal ''$bd$'' vermenigvuldigd worden, waarna volgt $ad=bc$.

\subsection{Rekenen met percentages en promillages}

$x$ percent of $x\%$ van een getal $y$ betekent: ${\displaystyle \left(\frac{x}{100}\right).y}$

\noindent $x$ promille of $x$ \textpertenthousand van een getal $y$ betekent:
${\displaystyle \left(\frac{x}{1000}\right).y}$

\noindent Een percentage of promillage heeft dus altijd betrekking
op een getal. $10\%$ op zich heeft m.a.w. eigenlijk geen betekenis.

\medskip{}


\uline{Voorbeelden}:

Hoeveel is $25\%$ van 50? Dit is: ${\displaystyle \left(\frac{25}{100}\right).50=0,25.50=12,5}$

Stel, de basisprijs van een product is $\EUR{82}$. Er komt
echter nog $21\%$ BTW bij. Aan welke prijs wordt dit product te koop
aangeboden?

Antwoord: ${\displaystyle 82+21\%\:\mathrm{van\:}82=82+17,22=\EUR{99,22}}$.

Dit soort berekeningen kan je vlotter maken via: $1,21*82=\EUR{99,22}$.
Dus als een hoeveelheid met $21\%$ toeneemt hoort daar de factor
$1,21$ bij.

Hoeveel moeten we betalen als we $30\%$ korting krijgen op een product
dat $200\EUR$ kost? We moeten dan van de prijs $30\%$
aftrekken: ${\displaystyle 200-30\%\:\mathrm{van\:}200=200-60=\EUR{140}}$.
Ook dit gaat eenvoudiger via $0,70*200=\EUR{140}$. Met een afname
van $30\%$ komt de factor $0,70$ overeen.

\medskip{}


We hebben geluk: bovenop de $30\%$ korting krijgen we nog een extra
$5\%$ studentenkorting. Nu betalen we $0,70*0,95*200=\EUR{133}$.

Procenten van procenten tel je dus niet op, maar je vermenigvuldigt
ze met elkaar. De klant heeft dus geen $35\%$ korting gekregen, maar
slechts $1-0,7*0,95=33,5\%$.

\medskip{}


\noindent Ook een gecombineerde toename en afname worden via een vermenigvuldiging
samengevoegd.

\noindent Met hoeveel procent neemt het aantal toe als het eerst $20\%$
vermeerdert en daarna met $45\%$ vermindert?

\noindent Antwoord: bij een toename van $20\%$ hoort de factor $1,2$
en bij een afname van $45\%$ hoort de factor $0,55$. Aangezien $1,2*0,55=0,66$
zal het aantal afnemen met $34\%$ .

\medskip{}


\noindent Tijdens een garageverkoop doet een verkoper ons een aantrekkelijk
voorstel. I.p.v. 15\% korting op alle spullen, krijgen wij een korting
van $\EUR{125}$ op het nog nieuwe televisietoestel dat $\EUR{1000}$
kost. We happen niet meteen toe, maar rekenen uit met hoeveel procent
125 van 1000 overeenkomt:

als $x\%\:\mathrm{van}\:y=z$

maw ${\displaystyle \left(\frac{x}{100}\right).y=z}$

dan is ${\displaystyle \frac{x}{100}}$ (of $x\%$) ${\displaystyle =\frac{z}{y}}$

In ons geval is: ${\displaystyle x=\frac{125}{1000}=0,125\:\mathrm{of}\:12,5\%}$.

We kiezen dus beter voor de $15\%$ korting!
