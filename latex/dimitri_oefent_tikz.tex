\begin{center}
	\begin{tikzpicture}
	%grid
	%\draw[step=1cm,gray,very thin,dotted] (-5,-5) grid 5,5);
	
	\draw[style=help lines,step=0.5cm,gray, very thin, dotted] (-5,-5) grid (5,5);
	%x-as 
	\draw[->] (-5,0)--(5,0) node[anchor=south,left,yshift=0.2cm]{$x$};
	
	%y-as
	\draw[->] (0,-5)--(0,5) node[anchor=south,left]{$y$};
		
	
	\tkzDefPoint(0,0){S}
	\tkzDefPoint(1,0){x1}
	\tkzDefPoint(0,1){y1}
	
	\tkzDefPoint(-3,0){A1}
	\tkzDefPoint(0,-1){A2}
	
	\tkzDefPoint(-3,-1){A}
	
	\tkzDefPoint(1,0){B1}
	\tkzDefPoint(0,4){B2}
	
	\tkzDefPoint(1,4){B}
	
	\tkzDefPoint(2,0){C1}
	\tkzDefPoint(0,-3){C2}
	
	\tkzDefPoint(2,-3){C}
	
	\tkzLabelPoint[below,xshift=-0.1cm](S){$0$}
	\tkzLabelPoint[right,yshift=-0.3cm](S){$O$}
	
	\tkzLabelPoint[below](x1){$1$}
	\tkzLabelPoint[left](y1){$1$}
	
	\tkzLabelPoint[left,yshift=0.2cm](A){$A$}
	\tkzLabelPoint[right,yshift=0.2cm](B){$B$}
	\tkzLabelPoint[right,yshift=0.2cm](C){$C$}
	
	\tkzLabelPoint[above](A1){$-3$}
	\tkzLabelPoint[right](A2){$-1$}
	
 	\tkzLabelPoint[below](B1){$-1$}
	\tkzLabelPoint[left](B2){$4$}
	
	\tkzLabelPoint[above](C1){$2$}
	\tkzLabelPoint[left](C2){$-3$}	
	
	\tkzDrawSegment[black!60!black,dotted](A1,A)
	\tkzDrawSegment[black!60!black,dotted](A2,A)
	\tkzDrawSegment[black!60!black,dotted](B1,B)
	\tkzDrawSegment[black!60!black,dotted](B2,B)
	\tkzDrawSegment[black!60!black,dotted](C1,C)
	\tkzDrawSegment[black!60!black,dotted](C2,C)
	
	\tkzDrawSegment[black!60!black](A,C)
	\tkzDrawSegment[black!60!black](A,B)
	\tkzDrawSegment[black!60!black](B,C)
	
	\foreach \n in {S,x1,y1,A1,A2,A,B1,B2,B,C1,C2,C}
	\node at (\n)[circle,fill,inner sep=1.5pt]{};

	\end{tikzpicture}
\end{center}


\newpage
\begin{center}
\begin{tikzpicture}[scale=2,cap=round]

% Styles
\tikzstyle{axes}=[]
\tikzstyle help lines=[color=blue!50,very thin,dotted]

% grid
\draw[style=help lines,step=0.5cm] (-2.9,-2.9) grid (2.9,2.9);

%\draw (0,0) circle (2cm);

\draw[->] (-3,0) -- (3,0) node[right] {$x$};
\draw[->] (0,-3) -- (0,3) node[above] {$y$};

%getallen op de x-as én lijntje?  
\foreach \x/\xtext in {-3,-2,-1, -.5/-\frac{1}{2}, 1,2,2.5}
\draw[xshift=\x cm] (0pt,1pt) -- (0pt,-1pt) node[below,fill=white]
{$\xtext$};
%getallen op de y-as én lijntje? 
\foreach \y/\ytext in {-3,-2,-1, -.5/-\frac{1}{2}, .5/\frac{1}{2}, 1,2}
\draw[yshift=\y cm] (1pt,0pt) -- (-1pt,0pt) node[left,fill=white]
{$\ytext$};



\end{tikzpicture}
\end{center}
\newpage

\begin{tikzpicture}[scale=3,cap=round]
% Local definitions
\def\costhirty{0.8660256}

% Colors
\colorlet{anglecolor}{green!50!black}
\colorlet{sincolor}{red}
\colorlet{tancolor}{orange!80!black}
\colorlet{coscolor}{blue}

% Styles
\tikzstyle{axes}=[]
\tikzstyle{important line}=[very thick]
\tikzstyle{information text}=[rounded corners,fill=red!10,inner sep=1ex]

% The graphic
\draw[style=help lines,step=0.5cm,very thin, gray, dotted] (-1.4,-1.4) grid (1.4,1.4);

\draw (0,0) circle (1cm);

\begin{scope}[style=axes]
\draw[->] (-1.5,0) -- (1.5,0) node[right] {$x$};
\draw[->] (0,-1.5) -- (0,1.5) node[above] {$y$};

\foreach \x/\xtext in {-1, -.5/-\frac{1}{2}, 1}
\draw[xshift=\x cm] (0pt,1pt) -- (0pt,-1pt) node[below,fill=white]
{$\xtext$};

\foreach \y/\ytext in {-1, -.5/-\frac{1}{2}, .5/\frac{1}{2}, 1}
\draw[yshift=\y cm] (1pt,0pt) -- (-1pt,0pt) node[left,fill=white]
{$\ytext$};
\end{scope}

\filldraw[fill=green!20,draw=anglecolor] (0,0) -- (3mm,0pt) arc(0:30:3mm);
\draw (15:2mm) node[anglecolor] {$\alpha$};

\draw[style=important line,sincolor]
(30:1cm) -- node[left=1pt,fill=white] {$\sin \alpha$} +(0,-.5);

\draw[style=important line,coscolor]
(0,0) -- node[below=2pt,fill=white] {$\cos \alpha$} (\costhirty,0);

\draw[style=important line,tancolor] (1,0) --
node [right=1pt,fill=white]
{
	$\displaystyle \tan \alpha \color{black}=
	\frac{{\color{sincolor}\sin \alpha}}{\color{coscolor}\cos \alpha}$
} (intersection of 0,0--30:1cm and 1,0--1,1) coordinate (t);

\draw (0,0) -- (t);

%\draw[xshift=1.85cm] node [right,text width=6cm,style=information text]
%{
%	The {\color{anglecolor} angle $\alpha$} is $30^\circ$ in the
%	example ($\pi/6$ in radians). The {\color{sincolor}sine of
%		$\alpha$}, which is the height of the red line, is
%	\[
%	{\color{sincolor} \sin \alpha} = 1/2.
%	\]
%	By the Theorem of Pythagoras we have ${\color{coscolor}\cos^2 \alpha} +
%	{\color{sincolor}\sin^2\alpha} =1$. Thus the length of the blue
%	line, which is the {\color{coscolor}cosine of $\alpha$}, must be
%	\[
%	{\color{coscolor}\cos\alpha} = \sqrt{1 - 1/4} = \textstyle
%	\frac{1}{2} \sqrt 3.
%	\]%
%	This shows that {\color{tancolor}$\tan \alpha$}, which is the
%	height of the orange line, is
%	\[
%	{\color{tancolor}\tan\alpha} = \frac{{\color{sincolor}\sin
%			\alpha}}{\color{coscolor}\cos \alpha} = 1/\sqrt 3.
%	\]%
%};
\end{tikzpicture}
