
		\tikzsetfigurename{Fig_module_3_2_8_derdemachtswortels-voorbeeld}


\def\centerarc[#1](#2)(#3:#4:#5)% Syntax: [draw options] (center) (initial angle:final angle:radius)
{ \draw[#1] ($(#2)+({#5*cos(#3)},{#5*sin(#3)})$) arc (#3:#4:#5); }


\begin{center}
	\begin{tikzpicture}[yscale=1,xscale=1]
	
	\colorlet{anglecolor}{green!50!black}
	%	\colorlet{sincolor}{red}
	%	\colorlet{tancolor}{orange!80!black}
	%	\colorlet{coscolor}{blue}
	% Styles
	\tikzstyle{axes}=[]
	\tikzstyle help lines=[color=blue!50,very thin,dotted]
	
	% grid
	%\draw[style=help lines,step=1cm] (-2,-2) grid (2,2);
	
	\draw[-] (-7,0) -- (0,0); 
	\draw[->] (5,0) -- (7,0) node[below,yshift=-0.3cm] {$Re$};
	\draw[->] (0,-7) -- (0,7) node[left,yshift=-0.3cm] {$Im$};

%	\draw[-] (1,-0.1)--(1,0.1);		
%	\draw[-] (-0.1,1)--(0.1,1);		

%	\draw[-] (2,-0.1)--(2,0.1);		
%	\draw[-] (3,-0.1)--(3,0.1);		

%	\draw[-] (6,-0.1)--(6,0.1);		


%\draw (-.1,1) node[below,right,xshift=+0.2cm] {$i$};
%\draw (1,-.1) node[below] {$1$};	
%\draw (2,-.1) node[below] {$2$};	
%\draw (3,-.1) node[below] {$3$};	
%\draw (6,-.1) node[below] {$6$};	

%	\draw (4,-.3) node[below,left,xshift=+.2cm] {$ $};
%	\draw (-.1,3) node[below,left] {$ $};
% ************************************************************* % 
\draw[dotted](0,0)circle [radius=5];

%\draw[cyan](0,0)circle [radius=1];
	
\draw[->,very thick,blue,rotate=120] (0,0) -- (5,0) node[left,xshift=-1.8cm,yshift=-0.5cm] {};	
\draw[rotate= 120,olive]  (5.5,0) 	node[blue]{$Z_{3,1} $};

\draw[->,very thick,blue,rotate=0] (0,0) -- (5,0);
\draw[rotate= 0]  (5,0) 	node[blue,above,right,yshift=+0.2cm]{$Z_{3,0} $};
	
\draw[->,very thick,blue,rotate=240] (0,0) -- (5,0);
\draw[rotate= 240]  (5,0) 	node[below,blue]{$Z_{3,2} $};

%\draw[->,thick,orange,rotate=15] (2,0) -- (6,0);
%\draw[rotate= 15]  (6,0) 	node[right,orange]{$c $};	
%\draw[]  (3.9,1) 	node[below,orange,rotate=15,yshift=0cm]{$|c|=|a| |b| $};	

%\centerarc[blue](0,0)(0:15:1.5);% node[]{$\alpha$};
%\centerarc[red](0,0)(0:105:2.5);% node[]{$\beta$};
%\centerarc[olive](0,0)(0:120:5); %node[]{$\gamma$}; 	
%\centerarc[red](0,0)(15:120:4);% node[]{$\alpha$};


%\draw[]  (1.7,0.2) 	node[blue]{$\alpha$};
%\draw[]  (2,2) 	node[red]{$\beta$};
%\draw[]  (1.5,5.2) 	node[olive]{$\theta=\alpha+\beta$};
%\draw[]  (1.0,4.1) 	node[red]{$\beta$};




%	\draw[orange,fill,rotate=15](6,0)circle [radius=0.05];	

%	\draw[->] (0,0) -- (5,1) node[below,xshift=0cm,yshift=-0.2cm] {$c$};
%	\draw[cyan,fill](5,1)circle [radius=0.05];	
	
%	\draw[->] (0,0) -- (1,2) node[above,xshift=0cm,yshift=0.2cm] {$a$};
%	\draw[cyan,fill](4,3)circle [radius=0.05];
	
%	\draw[->,thick,cyan] (0,0) -- (4,3) node[left,xshift=-1.8cm,yshift=-0.5cm] {};
%	\draw[cyan,fill](4,3)circle [radius=0.05];
	

	
	

%	\draw[] (0.4,0.7)	node[right,anglecolor]{$ \alpha$}; 	 
%	\draw[] (1.2,0.25)	node[right,blue]{$\beta$}; 	 
%	\draw[] (1.8,1)	node[right,red]{$\theta = \alpha + \beta $}; 	 
%	\draw[] (4.2,2.1)	node[right,orange]{$\alpha-\beta$}; 	 
%	\draw[] (2,1.5)	node[right,olive]{$\pi-(\alpha-\beta)$}; 	 
		
%	\draw[] (4,3) node[right,xshift=0cm,yshift=0.1cm] {$a+b$};
	
	
	
	
	\draw[dotted] (5,0) -- (-2.5,4.33) ;
	\draw[dotted] (-2.5,4.33) -- (-2.5,-4.33);
	\draw[dotted] (-2.5,-4.33) -- (5,0);


	
	
%	\draw[anglecolor] (1,0) arc [radius=1, start angle =0, end angle =30]; 	
%	\draw[anglecolor]  (0.5,0.2) node[below,right] {$\alpha$};
	
	
	%FUNCTIEVOORSCHRIFTEN	
	%\draw[teal,cap=rect,line width=1, opacity=1, domain=-0.5:1.8] plot (\x, {
	%	2*pow(\x,4)-3*pow(\x,3)-pow(\x,2)  		% <- plaats het functievoorschrift hier
	%	}) node[left,opacity=1]{$f(x)=2x^4-3x^3-x^2$};
	
	
	
	%\draw[cyan,cap=rect,ultra thick, domain=1:2] plot (\x, {\x*\x-1}) node[above, right]{};
	%\draw[red,cap=rect, loosely dashed, ultra thick, domain=-2:2] plot (\x, {(\x*\x-1)+0.05}) node[above,yshift=-.7cm, right]{};
	
	%legende
	%getallen op de x-as en lijntjes   
	%\foreach \x/\xtext in {-1,1}
	%\draw[xshift=\x cm] (0pt,1pt) -- (0pt,0pt) node[below,fill=white]
	%{$\xtext$};,3
	
	%getallen op de y-as en lijntjes  
	%BEGIN LUS
	%\foreach \y/\ytext in {-4,-2,2}
	%\draw[yshift=\y cm] (1pt,0pt) -- (0pt,0pt) node[left,fill=white]
	%{$\ytext$}; %EINDE LUS
	
	
	
	\end{tikzpicture}
\end{center}


