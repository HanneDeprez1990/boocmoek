		\tikzsetfigurename{Fig_module_3_1_1_radiaal}
\begin{center}
	\begin{tikzpicture}[yscale=1,xscale=1]
	
	\colorlet{anglecolor}{green!50!black}
	%	\colorlet{sincolor}{red}
	%	\colorlet{tancolor}{orange!80!black}
	%	\colorlet{coscolor}{blue}
	% Styles
	\tikzstyle{axes}=[]
	\tikzstyle help lines=[color=blue!50,very thin,dotted]
	
	% grid
	%\draw[style=help lines,step=1cm] (-2,-2) grid (2,2);
	
		\draw[->] (-2.5,0) -- (2.5,0) node[right] {$x$};
		\draw[-,ultra thick] (0,0) -- (2,0) node[right] {};
		
		\draw[->] (0,-2.5) -- (0,2.5) node[above] {$y$};
		
		
		\draw[cyan](0,0)circle [radius=2];
	
	
%	\draw[-,thick, rotate=0] (0,0) -- (2.2,0) node[above,yshift=+.2cm,right] {\tiny 1};
%	\draw[-,thick, rotate=10] (0,0) -- (2.2,0) node[above,yshift=+.2cm,right] {\tiny 2};	
%		\draw[-,thick, rotate=20] (0,0) -- (2.2,0) node[above,yshift=+.2cm,right] {\tiny 3};	
			\draw[-, rotate=60] (0,0) -- (2.2,0) node[above,yshift=-.5cm,right,xshift=0.5cm] {\tiny afstand langs cirkel s};	

%		\draw[black](2,2.2)circle [radius=0] node[left] {\tiny  enzovoort...};
	
	
		\draw[black] (1,0) arc [radius=1, start angle =0, end angle =60] node[right,xshift =0.1cm] {\tiny  hoek}; 	
	
	
	\draw[black,ultra thick] (2,0) arc [radius=2, start angle =0, end angle =60] node[left] {}; 	
	
	
	
	
	%FUNCTIEVOORSCHRIFTEN	
	%\draw[teal,cap=rect,line width=1, opacity=1, domain=-0.5:1.8] plot (\x, {
	%	2*pow(\x,4)-3*pow(\x,3)-pow(\x,2)  		% <- plaats het functievoorschrift hier
	%	}) node[left,opacity=1]{$f(x)=2x^4-3x^3-x^2$};
	
	
	
	%\draw[cyan,cap=rect,ultra thick, domain=1:2] plot (\x, {\x*\x-1}) node[above, right]{};
	%\draw[red,cap=rect, loosely dashed, ultra thick, domain=-2:2] plot (\x, {(\x*\x-1)+0.05}) node[above,yshift=-.7cm, right]{};
	
	%legende
	
	
	
	%getallen op de x-as en lijntjes   
	%\foreach \x/\xtext in {-1,1}
	%\draw[xshift=\x cm] (0pt,1pt) -- (0pt,0pt) node[below,fill=white]
	%{$\xtext$};,3
	
	%getallen op de y-as en lijntjes  
	%BEGIN LUS
	%\foreach \y/\ytext in {-4,-2,2}
	%\draw[yshift=\y cm] (1pt,0pt) -- (0pt,0pt) node[left,fill=white]
	%{$\ytext$}; %EINDE LUS
	
	
	
	\end{tikzpicture}
\end{center}


