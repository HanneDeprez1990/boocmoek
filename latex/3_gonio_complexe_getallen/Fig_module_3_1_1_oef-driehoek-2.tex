		\tikzsetfigurename{Fig_module_3_1_1_oef-driehoek-2}

\def\centerarc[#1](#2)(#3:#4:#5)% Syntax: [draw options] (center) (initial angle:final angle:radius)
{ \draw[#1] ($(#2)+({#5*cos(#3)},{#5*sin(#3)})$) arc (#3:#4:#5); }

\begin{center}
	\begin{tikzpicture}[yscale=1.5,xscale=1.5, rotate=0]
	
	\colorlet{anglecolor}{green!50!black}
	%	\colorlet{sincolor}{red}
	%	\colorlet{tancolor}{orange!80!black}
	%	\colorlet{coscolor}{blue}
	% Styles
	\tikzstyle{axes}=[]
	\tikzstyle help lines=[color=blue!50,very thin,dotted]
	
	% grid
	%\draw[style=help lines,step=1cm] (-2,-2) grid (2,2);
	
	\draw[-] (0,0) -- (3,1) node[right,yshift=-0.3cm] {};
%	\centerarc[](4,0)(150:171:0.7); 	 	
%	\draw[fill=black](-2,1) circle [radius=0.02] node[right, xshift=1cm,yshift=0.2cm]{$\gamma$};

	%node[left,xshift=0cm,yshift =+0.3cm] {$\beta$}

	
	\draw[-] (3,1) -- (3.5,6) node[above,yshift=+0.2cm] {};

	\centerarc[](3.5,6)(240:265:1); 	 
	
				
			
	\draw[-] (3.5,6) -- (0,0) node[right] {};
	
	
	
	\draw[](3.5,6)  node[left, xshift=-0.4cm,yshift=-1.7cm]{$\alpha$};;
	
	
	\draw[] (-2,1)node[left,yshift=-0.3cm]{};
	
	
	
		
	\draw[] (2,0.5)node[below]{a};
	\draw[] (3.5,3)node[below]{c};
	\draw[] (1,2)node[left]{b};
	
 		
 		
	%*** symbooltje rechte hoek ***%
%	\draw[-] (0,0)-- ++(0,5pt) -- ++(5pt,0pt) -- ++(0pt,-5pt)node[]{};
	%*** symbooltje rechte hoek ***%		
	
	
%\draw [red,thick,domain=0:90] plot ({cos(\x)}, {sin(\x)});
%\draw [blue,thick,domain=180:270] plot ({cos(\x)}, {sin(\x)});

%\draw[black] (0,2) arc [radius=0.5, start angle =170, end angle =328] node[below,xshift=0cm,yshift =-0.3cm] {$\alpha$}; 	 

%	\draw[black] (3.5,0.3) arc [radius=0.5, start angle =135, end angle =178] 
	
	
	
%\draw[cyan](0,0)circle [radius=2];	
%\draw[-,thick, rotate=0] (0,0) -- (2.2,0) node[above,yshift=+.2cm,right] {\tiny 1};
%\draw[-,thick, rotate=10] (0,0) -- (2.2,0) node[above,yshift=+.2cm,right] {\tiny 2};	
%\draw[-,thick, rotate=20] (0,0) -- (2.2,0) node[above,yshift=+.2cm,right] {\tiny 3};	

%\draw[black](2,2.2)circle [radius=0] node[left] {\tiny  enzovoort...};
%\draw[black] (,0) arc [radius=1, start angle =0, end angle =90] node[right,xshift=0.3cm,yshift =0.3cm] {}; 	 

%\draw[] (1,1) node{\tiny  90 $^o$};
%\draw[black](0,2) circle [radius=0.05];
%\draw[black,ultra thick] (2,0) arc [radius=2, start angle =0, end angle =90] node[left] {}; 	
	
	
	
	
	%FUNCTIEVOORSCHRIFTEN	
	%\draw[teal,cap=rect,line width=1, opacity=1, domain=-0.5:1.8] plot (\x, {
	%	2*pow(\x,4)-3*pow(\x,3)-pow(\x,2)  		% <- plaats het functievoorschrift hier
	%	}) node[left,opacity=1]{$f(x)=2x^4-3x^3-x^2$};
	
	
	
	%\draw[cyan,cap=rect,ultra thick, domain=1:2] plot (\x, {\x*\x-1}) node[above, right]{};
	%\draw[red,cap=rect, loosely dashed, ultra thick, domain=-2:2] plot (\x, {(\x*\x-1)+0.05}) node[above,yshift=-.7cm, right]{};
	
	%legende
	
	
	
	%getallen op de x-as en lijntjes   
	%\foreach \x/\xtext in {-1,1}
	%\draw[xshift=\x cm] (0pt,1pt) -- (0pt,0pt) node[below,fill=white]
	%{$\xtext$};,3
	
	%getallen op de y-as en lijntjes  
	%BEGIN LUS
	%\foreach \y/\ytext in {-4,-2,2}
	%\draw[yshift=\y cm] (1pt,0pt) -- (0pt,0pt) node[left,fill=white]
	%{$\ytext$}; %EINDE LUS
	
	
	
	\end{tikzpicture}
\end{center}


