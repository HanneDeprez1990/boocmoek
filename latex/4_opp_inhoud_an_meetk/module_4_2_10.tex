\subsection{Onderlinge ligging van twee rechten}
\noindent

Twee rechten $l_1$ en $l_2$ in het vlak zijn evenwijdig als en alleen als de positieve richting van de $x$-as even grote geori\"enteerde hoeken maakt met $l_1$ en met $l_2$.
Hieruit volgen twee mogelijkheden:
\begin{itemize}
\item $l_1$ en $l_2$ zijn allebei evenwijdig met de $y$-as.
\item $\rico (l_1)=\rico (l_2)$ (en dan zijn $l_1$ en $l_2$ niet evenwijdig met de $y$-as).
\end{itemize}

%Aan de makers: mijn voorstel is volgend voorbeeld te maken met een topcamera.


\begin{voorbeeld}
Gegeven zijn 3 punten $A(2:-3)$, $B(-1;4)$ en $C(3;4)$.
Geef een vergelijking van de rechte $l$ door $C$ evenwijdig aan de rechte $AB$.

\begin{center}
	\tikzsetfigurename{module4_2_10_onderlingeLiggingVb}
\begin{tikzpicture}
\draw[->] (-5,0)--(5,0) node[anchor=south,left,yshift=0.2cm]{$x$};
\draw[->] (0,-6)--(0,5) node[anchor=south,left]{$y$};

%	\draw[‎smooth,‎samples=100‎,domain=0:‎‎2‎‎, variable=\x] ‎plot({\x},‎‎‎‎‎‎‎‎{0.577 \x  - 6.155});‎‎
\draw[domain=-1.5:3,smooth,variable=\x] plot ({\x},{-7/3*\x+5/3}) node[anchor=south,left,yshift=0.2cm]{$l$};	

\tkzDefPoint(0,0){S}
\tkzDefPoint(1,0){x1}
\tkzDefPoint(0,1){y1}

\tkzDefPoint(2,0){A1}
\tkzDefPoint(0,-3){A2}

\tkzDefPoint(-1,0){B1}
\tkzDefPoint(0,4){B2}

\tkzDefPoint(3,0){C1}
\tkzDefPoint(0,4){C2}

\tkzDefPoint(2,-3){A}
\tkzDefPoint(-1,4){B}
\tkzDefPoint(3,4){C}

\tkzLabelPoint[below,xshift=-0.1cm](S){$0$}
%	\tkzLabelPoint[right,yshift=-0.3cm](S){$O$}

\tkzLabelPoint[below](x1){$1$}
\tkzLabelPoint[left](y1){$1$}

\tkzLabelPoint[right](A){$A$}
\tkzLabelPoint[left,yshift=0.2cm](B){$B$}
\tkzLabelPoint[left,yshift=0.2cm](C){$C$}
%	
\tkzLabelPoint[above](A1){$2$}
\tkzLabelPoint[left](A2){$-3$}

\tkzLabelPoint[below](B1){$-1$}

\tkzLabelPoint[below](C1){$3$}
\tkzLabelPoint[left,yshift=0.2cm](C2){$4$}

\tkzDrawSegment[black!60!black,dotted](A1,A)
\tkzDrawSegment[black!60!black,dotted](A2,A)
\tkzDrawSegment[black!60!black,dotted](B1,B)
\tkzDrawSegment[black!60!black,dotted](B2,B)
\tkzDrawSegment[black!60!black,dotted](C1,C)
\tkzDrawSegment[black!60!black,dotted](C2,C)

\foreach \n in {S,B1,B2,B,A1,A2,A,C1,C2,C,x1,y1}
\node at (\n)[circle,fill,inner sep=1.5pt]{};

\end{tikzpicture}
\end{center}

%\gewonefiguur{height=7cm}{4_opp_inhoud_an_meetk/inputs/AMTekst5Fig1}

%\begin{figure}[!htb]
%\begin{center}
%\includegraphics[height=7 cm]{4_opp_inhoud_an_meetk/inputs/AMTekst5Fig1}
%\caption{Voorbeeld 1.}
%\label{fig4.2.10_fig1}
%\end{center}
%\end{figure} 
De richtingsco\"effici\"ent $m$ van rechte $AB$ is
\[
m=\frac {4-(-3)}{-1-2}=-\frac {7}{3} \text { .}
\]
De richtingsc\"effici\"ent van $l$ moet dus ook -4 zijn.
Omdat $l$ ook door het punt $C$ moet gaan bekomen we volgende vergelijking van $l$:
\[
y-4=(-\frac {7}{3})(x-3) \text { dus } 7x+3y-33=0 \text { .}
\]\\

Een loodlijn op een rechte evenwijdig met de $x$-as (vergelijking van de vorm $y=b$) is een rechte evenwijdig met de $y$-as (vergelijking van de vorm $x=a$).
\end{voorbeeld}

\begin{voorbeeld}
	$l$ is de rechte met vergelijking $y=5$.
Geef een vergelijking van de loodlijn $l'$ op $l$ door $P(3;-4)$.

\begin{center}
	\tikzsetfigurename{module4_2_10_onderlingeLiggingVb2}
	\begin{tikzpicture}
\draw[->] (-5,0)--(5,0) node[anchor=south,left,yshift=0.2cm]{$x$};
\draw[->] (0,-5)--(0,6) node[anchor=south,left]{$y$};

%	\draw[‎smooth,‎samples=100‎,domain=0:‎‎2‎‎, variable=\x] ‎plot({\x},‎‎‎‎‎‎‎‎{0.577 \x  - 6.155});‎‎
\draw[domain=-5:5,smooth,variable=\x] plot ({\x},5);	

\tkzDefPoint(0,0){S}
\tkzDefPoint(1,0){x1}
\tkzDefPoint(0,1){y1}

\tkzDefPoint(0,5){y2}

\tkzDefPoint(3,0){A1}
\tkzDefPoint(0,-4){A2}

\tkzDefPoint(3,-4){A}
\tkzLabelPoint[below,xshift=-0.1cm](S){$0$}
%	\tkzLabelPoint[right,yshift=-0.3cm](S){$O$}

\tkzLabelPoint[below](x1){$1$}
\tkzLabelPoint[left](y1){$1$}
\tkzLabelPoint[left,yshift=0.2cm](y2){$5$}

\tkzLabelPoint[right](A){$P$}
%	
\tkzLabelPoint[above](A1){$3$}
\tkzLabelPoint[left](A2){$-4$}

\tkzDrawSegment[black!60!black,dotted](A1,A)
\tkzDrawSegment[black!60!black,dotted](A2,A)

\foreach \n in {S,A1,A2,A,x1,y1}
\node at (\n)[circle,fill,inner sep=1.5pt]{};

\end{tikzpicture}
\end{center}

%\gewonefiguur{height=7cm}{4_opp_inhoud_an_meetk/inputs/AMTekst5Fig2}

%\begin{figure}[!htb]
%\begin{center}
%\includegraphics[height=7 cm]{4_opp_inhoud_an_meetk/inputs/AMTekst5Fig2}
%\caption{Voorbeeld 2}
%\label{fig4.2.10_fig2}
%\end{center}
%\end{figure} 
Omdat $l'$ een rechte is evenwijdig met de $y$-as heeft $l'$ een vergelijking van de vorm $x=a$.
Omdat $(P3;-4)$ op $l'$ moet liggen moet $a=3$.
Een vergelijking van $l'$ is dus $x=3$.

Een loodlijn op een rechte evenwijdig met de $y$-as (vergelijking van de vorm $x=a$) is een rechte evenwijdig met de $x$-as (vergelijking van de vorm $y=b$).

Stel nu dat $l$ aan geen van beide assen evenwijdig is.
Stel dat $\alpha$ een ge\"ori\"enteerde hoek is tussen de positieve richting van de $x$-as en de rechte $l$.
In dat geval is de grootte van de hoek $\alpha$ geen veelvoud van $90^o$.
Dit impliceert dat de richtingsco\"effici\"ent $m=\tan (\alpha$ van de rechte $l$ bestaat en verschillend is van 0.
Als $l'$ een loodlijn is op $l$ dan is $\alpha + 90^o$ een geori\"enteerde hoek tussen de positieve richting van de $x$-as en $l'$.
Omdat $\tan (\alpha + 90^o)=-\cot (\alpha)=-\frac{1}{\tan (\alpha)}$  bekom je dat $\rico (l')=-\frac{1}{m}=-\frac{1}{\rico (l)}$.

\underline {Besluit:} Als $l$ en $l'$ loodrechte op elkaar staan en geen van beide is evenwijdig met een as van het assenstelsel dan is
\[
\rico (l).\rico (l')=-1 \text { .}
\]

\end{voorbeeld}
%Aan de makers: mijn voorstel is volgend voorbeeld te maken met een topcamera.


\begin{voorbeeld}
	Gegeven zijn de rechte $l$ met vergelijking $y=2x-3$ en het punt $P(-2;3)$.
Geef een vergelijking voor de loodlijn $l'$ door $P$ op $l$.
Geef eveneens de co\"ordinaten van de loodrechte projectie van $P$ op $l$.

\begin{center}
	\tikzsetfigurename{module4_2_10_onderlingeLiggingVb3}
	\begin{tikzpicture}
\draw[->] (-5,0)--(5,0) node[anchor=south,left,yshift=0.2cm]{$x$};
\draw[->] (0,-4)--(0,6) node[anchor=south,left]{$y$};

%	\draw[‎smooth,‎samples=100‎,domain=0:‎‎2‎‎, variable=\x] ‎plot({\x},‎‎‎‎‎‎‎‎{0.577 \x  - 6.155});‎‎
\draw[domain=-0.2:4.5,smooth,variable=\x] plot ({\x},{2*\x-3}) node[anchor=south,left] {$l$};	
\draw[domain=-5:4.5,smooth,variable=\x] plot ({\x},{-1/2*\x+2}) node[anchor=south,left,yshift=-0.2cm] {$l'$};

\tkzDefPoint(0,0){S}
\tkzDefPoint(1,0){x1}
\tkzDefPoint(0,1){y1}

\tkzDefPoint(0,5){y2}

\tkzDefPoint(-2,0){P1}
\tkzDefPoint(0,3){P2}

\tkzDefPoint(-2,3){P}

\tkzDefPoint(2,0){P'1}
\tkzDefPoint(0,1){P'2}

\tkzDefPoint(2,1){P'}

\tkzLabelPoint[below,xshift=-0.1cm](S){$0$}
%	\tkzLabelPoint[right,yshift=-0.3cm](S){$O$}

\tkzLabelPoint[below](x1){$1$}
\tkzLabelPoint[left](y1){$1$}

\tkzLabelPoint[right,yshift=0.2cm](P){$P$}
%	
\tkzLabelPoint[below](P1){$-2$}
\tkzLabelPoint[right](P2){$3$}

\tkzLabelPoint[right,yshift=0.2cm](P){$P'$}
%	
%	\tkzLabelPoint[below](P'1){$2$}
%	\tkzLabelPoint[right](P'2){$1$}

\tkzDrawSegment[black!60!black,dotted](P1,P)
\tkzDrawSegment[black!60!black,dotted](P2,P)

\foreach \n in {S,P1,P2,P,x1,y1,P'}
\node at (\n)[circle,fill,inner sep=1.5pt]{};

\end{tikzpicture}
\end{center}

%\gewonefiguur{height=7cm}{4_opp_inhoud_an_meetk/inputs/AMTekst5Fig3}

%\begin{figure}[!htb]
%\begin{center}
%\includegraphics[height=7 cm]{4_opp_inhoud_an_meetk/inputs/AMTekst5Fig3}
%\caption{Voorbeeld 3}
%\label{fig4.2.10_fig3}
%\end{center}
%\end{figure} 

Omdat $\rico (l)=2$ is $\rico (l')=-\frac{1}{2}$.
Omdat $l'$ ook door $P(-2;3)$ moet gaan is de vergelijking van $l'$
\[
y-3=-\frac{1}{2}(x+2) \text { dus } y=-\frac{1}{2}+2 \text { .}
\]

De loodrechte projectie van $P$ op $l$ is het snijpunt $P'$ van de rechten $l$ en $l'$.
De co\"ordinaten $(x,y)$ van het punt $P'$ moeten dus voldoen aan de vergelijkingen van beide rechten:
\[
l \leftrightarrow y=2x-3 \text { en } l' \leftrightarrow y=-\frac{1}{2}x+2 \text { .}
\]
Voor de $x$-co\"ordinaat van $P'$ bekom je hieruit de vergelijking
\[
2x-3=-\frac{1}{2}x+2
\]
met als oplossing $x=2$.
Vul je dit in de vergelijking van $l$ (of $l'$) in dan bekom je de $y$-co\"ordinaat van $P'$:
\[
y=2.2-3=1 \text { .}
\]
De co\"ordinaten van $P'$ zijn $(2;1)$.\\

Twee verschillende punten $A$ en $B$ in het vlak bepalen een lijnstuk $[AB]$.
De middelloodlijn van dat lijnstuk $[AB]$ is de rechte $l$ loodrecht op de rechte $AB$ door het midden $M$ van het lijnstuk $[AB]$.
Deze middelloodlijn is eveneens de verzameling van alle punten $P$ in het vlak die evenver van $A$ als van $B$ liggen.
In volgend voorbeeld zie je dit nagerekend.\\

\end{voorbeeld}

\begin{voorbeeld}
Gegeven zijn de punten $A(3;4)$ en $B(-2;1)$.
\begin{itemize}
\item We berekenen eerst een vergelijking van de middelloodlijn $l$ op $[AB]$ door de definitie te gebruiken.
Het midden $M$ van het lijnstuk $[AB]$ is het punt waarvan de co\"ordinaten de gemiddelden zijn van de co\"ordinaten van $A$ en $B$, dus 
\[
M(\frac{3+(-2)}{2};\frac{4+1}{2})=(\frac{1}{2};\frac{5}{2}) \text { .}
\]
De richtingsco\"effici\"ent van de rechte $AB$ is $\rico (AB)=\frac {1-4}{-2-3}=\frac{3}{5}$.
De richtingsco\"effici\"ent van de middelloodlijn $l$ is dus $\rico (l)=-\frac{5}{3}$.
Je bekomt daardoor als vergelijking van de rechte $l$:
\[
y-\frac{5}{2}=-\frac{5}{3}(x-\frac{1}{2}) \text { dus } y=-\frac{5}{3}x+\frac{10}{3} \text { .}
\]
\item We stellen nu een vergelijking op voor de verzameling van de punten $P$ in het vlak die evenver van $A$ als van $B$ liggen.
Een punt $P(x,y)$ ligt evenver van $A$ als van $B$ als en alleen als
\[
\sqrt{(x-3)^2+(y-4)^2}=\sqrt{(x+2)^2+(y-1)^2} \text { .}
\]
Kwadrateren van beide positieve leden in voorgaande gelijkheid en uitrekenen van de kwadraten geeft volgende nodige en voldoende voorwaarde
\[
x^2-6x+9+y^2-8y+16=x^2+4x+4+y^2-2y+1 \text { .}
\]
Merk op dat de kwadraten verdwijnen en als je alles oplost naar de veranderlijke $y$ dan bekom je
\[
y=-\frac{5}{3}x+\frac{10}{3} \text { .}
\]
\end{itemize}
Je merkt op dat je in beide delen dezelfde vergelijking bekomt.
In dit voorbeeld bekomen we dus dat de verzameling van de punten $P$ evenver gelegen van $A$ en $B$ hetzelfde is als de middelloodlijn op het lijnstuk $[AB]$.
Dit kan met behulp van elementaire meetkunde in alle gevallen bewezen worden (dat doen we in deze cursus niet).\\

\end{voorbeeld}

Gegeven zijn twee rechten $l$ en $l'$ met vergelijkingen
\[
l \leftrightarrow ax+by+c=0 \text { en } l' \leftrightarrow a'x+b'y+c'=0 \text { .}
\]
Hoe vind je de scherpe hoek $\angle (l,l')$ tussen de rechten $l$ en $l'$?

Uit de vergelijkingen kun je direct zien of een rechte evenwijdig is met de $y$-as.
Van een rechte die niet evenwijdig is met de $y$-as kun je de richtingsco\"effici\"ent uit de vergelijking vinden.
De richtingsco\"effici\"ent (als deze bestaat) van $l$ duiden we aan met $m$ en van $l'$ met $m'$.
Met deze richtingsco\"effici\"enten bereken je dan de geori\"enteerde hoek $\alpha$ (resp $\alpha '$) tussen de positieve richting van de $x$-as en de rechte $l$ (resp. $l'$) en genomen met grootte in het interval $]-90^o;90^o]$ (voor een verticale rechte nemen we dus $90^o$).

Door eventueel $l$ en $l'$ te verwisselen kunnen we aannemen dat $\alpha \geq \alpha '$.
Dan is $\angle (l,l')$ gelijk aan $\alpha - \alpha '$ als dit hoogstens $90^o$ is en anders is $\angle (l,l')$ gelijk aan $180^o-(\alpha - \alpha ')$.
Als $l$ evenwijdig is met de $y$-as dan bekomen we dat $\angle (l,l')=90^o-\vert \alpha \vert$.
We veronderstellen nu dat $l$ niet evenwijdig is met de $y$-as (en $l'$ dus ook niet).

Uit de vergelijkingen bekom je
\[
\rico (l)=m=-\frac {a}{b} \text { en dus } \tan (\alpha )=m=-\frac{a}{b}
\]
\[
\rico (l')=m'=-\frac{a'}{b'} \text { en dus } \tan(\alpha ')=m'=-\frac{a'}{b'}
\]
Omdat $\tan (\alpha -\alpha ')=\frac{\tan (\alpha)-\tan (\alpha ')}{1+\tan (\alpha).\tan (\alpha ')}$; $\tan (180^o-(\alpha -\alpha '))=-\tan (\alpha -\alpha ')$ en de tangens van een scherpe hoek steeds positief is bekom je
\[
\tan (\angle (l,l'))=\vert \frac{m-m'}{1+m.m'} \vert
\]
(als $1+m.m'=0$ dan staan $l$ en $l'$ loodrecht op elkaar) of nog
\[
\tan (\angle (l,l'))=\vert \frac{-\frac{a}{b}-(-\frac{a'}{b'})}{1+(-\frac{a}{b})(-\frac{a'}{b'})} \vert = \vert \frac{a'b-ab'}{aa'+bb'} \vert \text { .}
\]
Vanwege de absolute waarden is het niet meer nodig dat in deze formules $\alpha \geq \alpha$.\\


%\noindent \underline{Voorbeeld 1:} Gegeven zijn de punten $A(3;-2)$, $B(-4;1$ en $C(1;3)$.
%Bereken de scherpe hoek tussen de rechten $BA$ en $BC$.
%\begin{figure}[!htb]
%\begin{center}
%\includegraphics[height=7 cm]{4_opp_inhoud_an_meetk/inputs/AMTekst5Fig5}
%\end{center}
%\end{figure} 
%Je berekent eerst de richtingsco\"effici\"enten van de rechten $AB$ en $BC$.
%\[
%\rico (BC)=m=\frac{3-1}{1-(-4)}=\frac{2}{5}
%\]
%\[
%\rico (BA)=m'=\frac{1-(-2)}{-4-3}=-\frac{3}{7}
%\]
%Invullen in de formule met richtingsco\"effici\"enten geeft
%\[
%\tan (\angle (BC;BA) )=\frac{m-m'}{1+m.m'}=\frac{\frac{2}{5}-(-\frac{3}{7})}{1+(\frac{2}{5}).(-\frac{3}{7})}=1 \text { .}
%\]
%Je bekomt $\angle (BC;BA) = \arctan (1)=45^o$.\\
%
%Voorstel aan de makers is volgend voorbeeld als een lightboard filmpje toe te voegen.\\
%
%\noindent \underline{Voorbeeld 2:} Gegeven is de rechte $l$ met vergelijking $6x-3y+2=0$ en $P(-1;2)$.
%Stel een vergelijking op voor iedere rechte $l'$ door $P$ waarvoor $\angle (l.l')=30^o$.
%
%Er geldt $\rico (l)=m=2$.
%We noteren $\rico (l')=m'$.
%We moeten bekomen dat $\tan (\angle (l,l'))=\tan (30^o)=\frac {\sqrt {3}}{3}$.
%We gaan de formule gebruiken waarbij $\tan (\angle (l,l'))$ wordt uitgedrukt met de richtingsco\"effici\"enten van $l$ en $l'$.
%Omdat de hoek van $30^o$ tussen $l$ en $l'$ in twee richtingen kan voorkomen bekomen we twee mogelijke vergelijkingen om $m'$ uit te vinden:
%\[
%\frac {\sqrt{3}}{3}=\frac {m-m'}{1+m.m'}=\frac{2-m'}{1+2m'}
%\]
%\[
%\frac {\sqrt{3}}{3}=\frac {m'-m}{1+m.m'}=\frac {m'-2}{1+2m'} \text { .}
%\]
%De eerste vergelijking geeft
%\[
%(1+2m')\sqrt {3}=3(2-m') \text { waaruit } m'=\frac {6-\sqrt{3}}{3+2\sqrt {3}}=0,66 \text { .}
%\]
%De vergelijking van de bijbehorende rechte $l'$ is
%\[
%y-2=0,66(x+1) \text { dus } y=0,66x+2,66 \text { .}
%\]
%De tweede vergelijking geeft
%\[
%(1+2m')\sqrt {3}=3(m'-2) \text { waaruit } m'=\frac {-6+-\sqrt {3}}{2 \sqrt {3}-3}=-16,66 \text { .}
%\]
%De vergelijking van de bijbehorende rechte $l'$ is
%\[
%y-2=-16,66(x+1) \text { dus } y=-16,66x-14,66 \text { .}
%\]
%
