\subsection{Afstand van een punt tot een rechte}
\noindent

De afstand $d(P;l)$ van een punt $P$ tot een rechte $l$ is de afstand van $P$ tot de loodrechte projectie $P'$ van $P$ op $l$.\\

\noindent \underline{Voorbeeld 1:} Gegeven zijn een punt $P(2;-4)$ en de rechte $l$ met vergelijking $2x-5y+7=0$.
De rechte $l'=PP'$ zal loodrecht staan op $l$.
Omdat $\rico (l)=\frac {2}{5}$ moet $\rico (l')=-\frac {5}{2}$.
De vergelijking van de rechte $PP'=l'$ is daarom
\[
y-(-4)=-\frac {5}{2} (x-2) \text { dus } y=-\frac {5}{2}x+1 \text { .}
\]
Het punt $P'$ is het snijpunt van $l$ en $l'$.
De co\"ordinaten van $P'$ zijn daarom de oplossing van het stelsel
\[
\begin{cases}
y=\frac {2}{5} x +\frac {7}{5} \\
y=-\frac {5}{2} x +1
\end{cases}
\] 
De $x$-co\"ordinaat van $P'$ is daardoor de oplossing van
\[
\frac {2}{5} x+\frac {7}{5} = -\frac {5}{2} x +1 \text { dus } x=-\frac {4}{29} \text { .}
\]
Door deze waarde van $x$ in te vullen in $y=-\frac {5}{2}+1$ (vergelijking van $l'$) bekom je de $y$-co\"ordinaat van $P'$.
\[
y=-\frac {5}{2}.(-\frac{4}{29})+1=\frac {39}{29}
\]
De co\"ordinaten van $P'$ zijn $(-\frac {4}{29}; \frac {39}{29 })$.

De afstand van $P$ tot de rechte $l$ is de afstand van $P$ tot $P'$.
Je bekomt als afstand
\[
\sqrt { \left( 2+\frac {4}{29}  \right)^2 + \left( -4-\frac {39}{29}  \right)^2  } =\frac {\sqrt { 62^2+155^2}}{29}=5,76 \text { .}
\]\\

Er is een heel eenvoudige formule waarmee je de afstand van een punt $P(x_0;y_0)$ tot een rechte $l$ met vergelijking $ax+by+c=0$ kunt uitrekenen.
Je hoeft dan de vorige werkwijze in het voorbeeld niet telkens uit te voeren.
Deze formule is
\[
d(P;l)=\frac { \vert ax_0+by_0+c \vert }{\sqrt {a^2+b^2}} \text { .}
\]\\

Pas je voorgaande formule toe op het voorbeeld dan bekom je
\[
d(P;l)=\frac { \vert 2.2-5.(-4)+7 \vert }{\sqrt {4+25}}=\frac {31}{\sqrt {29}}=5,76 \text { .}
\]
Je merkt dat je inderdaad dezelfde uitkomst bekomt.\\

\noindent \underline{Voorbeeld 2:} Gegeven zijn de punten $A(-5;2)$, $B(-1;4)$ en $C(3;-2)$, zie Figuur \ref{fig4.2.14_fig1}.
Bereken de oppervlakte van driehoek $ABC$.

\begin{figure}[h]
\begin{center}
\includegraphics[height=7 cm]{4_opp_inhoud_an_meetk/inputs/AMTekst6Fig1}
\caption{Voorbeeld 2.}
\label{fig4.2.14_fig1}
\end{center}
\end{figure} 

We vatten het lijnstuk $[A;B]$ op als basis van de driehoek.
Dan is $d(A;B)$ de lengte van de basis.
De hoogte van de driehoek is dan de afstand $d(C;AB)$ van het punt $C$ tot de rechte $AB$.
Je bekomt
\[
d(A;B)=\sqrt { (-1-(-5))^2+(4-2)^2}=\sqrt {20} \text { .}
\]
Een vergelijking van de rechte $AB$ is
\[
y-2=\frac {4-2}{-1-(-5)}(x-(-5)) \text { dus } 2y-x-9=0 \text { .}
\]
De afstand van $C$ tot de rechte $AB$ is dan
\[
d(C;AB)=\frac {\vert 2.(-2)-3-9 \vert}{\sqrt{4+1}}=\frac {16}{\sqrt {5}} \text { .}
\]
De oppervlakte van driehoek $ABC$ is dus gelijk aan
\[
\frac {d(A;B).d(C;AB)}{2}=\frac { \sqrt {20}.\frac {16}{\sqrt {5}}}{2}=16 \text { .}
\]\\

De bissectrices van twee niet evenwijdige rechten $l$ en $l'$ zijn de rechten die een hoek tussen $l$ en $l'$ in twee gelijke delen verdelen.
Twee niet evenwijdige lijnen $l$ en $l'$ hebben twee bissectrices $b$ en $b'$ die loodrecht op elkaar staan, zie Figuur \ref{fig4.2.14_fig2}.

\begin{figure}[h]
\begin{center}
\includegraphics[height=7 cm]{4_opp_inhoud_an_meetk/inputs/AMTekst6Fig2}
\caption{Bisectrice.}
\label{fig4.2.14_fig2}
\end{center}
\end{figure} 

Er kan aangetoond worden dat deze twee rechten $b$ en $b'$ samen de verzameling is van alle punten $P$ in het vlak die evenver van de rechten $l$ en $l'$ liggen.
Door middel van de formule voor de afstand van een punt tot een rechte kun je met die formule vergelijkingen voor die bissectrices vinden.\\

\noindent \underline{Voorbeeld 3:} Stel vergelijkingen op van de bissectrices van de rechten $l$ en $l'$ gegeven door de volgende vergelijkingen.
\[
l \leftrightarrow 2x-5y+4=0 \text { en } l' \leftrightarrow3x+y-2=0 
\]
Een punt $P(x;y)$ behoort tot \'e\'en van de bissectrices $b$ en $b'$ als en alleen als $d(P;l)=d(P;l')$.
Omdat
\[
d(P;l)=\frac { \vert 2x-5y+4 \vert }{\sqrt {4 + 25}} \text { en } d(P;l')=\frac { \vert 3x+y-2 \vert }{\sqrt {9+1}}
\]
bekom je dat $P$ op een bissectrice ligt als en alleen als
\[
\frac { \vert 2x-5y+4 \vert }{\sqrt {29}}=\frac { \vert 3x+y-2 \vert }{\sqrt {10}} \text { .}
\]
Laat je hierin de absolute waarde weg dan zijn de uitdrukkingen aan weerszijden van de gelijkheid gelijk of tegengesteld.
Dit geeft aanleiding tot vergelijkingen van twee rechten: de bissectrices $b$ en $b'$.
Deze vergelijkingen zijn:
\begin{itemize}
\item  voor $b$
\[
\frac {2x-5y+4}{\sqrt {29}}=\frac {3x+y-2}{\sqrt {10 }}
\]
\[
(2\sqrt {10}-3\sqrt {29})x+(-5\sqrt {10}-\sqrt {29})y+(4\sqrt {10}+2\sqrt {29})=0
\]
\[
-9,83x-21,20y+23,42
\]
\item voor $b'$
\[
\frac {2x-5y+4}{\sqrt {29}}=-\frac {3x+y-2}{\sqrt {10}}
\]
\[
(2\sqrt {10}+3\sqrt {29})x+(-5\sqrt{10}+\sqrt{29})y+(4\sqrt{10}-2\sqrt{29})=0
\]
\end{itemize}
\[
22,48x-10,43y+1,88=0 \text { .}
\]
